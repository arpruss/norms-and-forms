\def\mychapter{XII}
\ifdefined\book
\else
\documentclass[11pt,oneside]{amsbook}
\usepackage[backend=biber, citestyle=authoryear]{biblatex}
\usepackage{mathpazo}
\usepackage{graphicx}
\usepackage{amsmath}
\usepackage{tikz}
\usetikzlibrary{arrows}
%\usepackage{titlesec}
\addbibresource{bibliography.bib}
\newcommand\posscite[1]{\citeauthor{#1}'s (\citeyear{#1})}
\newcommand\plural[1]{#1\mathrm{s}}
%\def\posscitewithextra[#1]#2{\citename{#2}'s (\citeyear{#2}, #1)}

%\newcounter{subsubsubsection}[subsubsection]
%\renewcommand\thesubsubsubsection{\thesubsubsection.\arabic{subsubsubsection}}
%\titleformat{\subsubsubsection}
%  {\normalfont\normalsize\bfseries}{\thesubsubsubsection}{1em}{}
%\titlespacing*{\subsubsubsection}
%{0pt}{3.25ex plus 1ex minus .2ex}{1.5ex plus .2ex}

\ifdefined\book
\renewcommand{\thechapter}{\Roman{chapter}}
\else
\renewcommand{\thechapter}{\mychapter}
\fi

\linespread{1.7}
\usepackage[margin=1.25in]{geometry}
\sloppy
\makeatletter
%% TODO: This is a cheat. It's supposed to be {paragraph}{4}, and that's 
%% what it is in amsbook.cls, but then it fails.
\def\paragraph{\@startsection{paragraph}{3}%
  \normalparindent\z@{-\fontdimen2\font}%
  \normalfont}
\def\subsubsubsection{\paragraph}
\makeatother

\def\smalltick{0.15cm}
\def\bigtick{0.3cm}
\def\pointcircle{0.08cm}
\def\causalnode{0.35cm}

\hyphenation{dia-chro-nic}

%\usepackage[utf8]{inputenc} % set input encoding (not needed with XeLaTeX)
\usepackage{amssymb}
\usepackage{mathtools}
\usepackage{enumitem}
\usepackage{amsthm}
\usepackage{physics}
%\usepackage{ntheorem}
\usepackage{chngcntr}
\counterwithin{figure}{section}

\makeatletter
% \def\@endtheorem{\endtrivlist\@endpefalse }% OLD
\def\@endtheorem{\endtrivlist}%

\setlist[description]{font=\normalfont\scshape}

\catcode`\|=\active\def|{\mid}
\DeclarePairedDelimiter{\ceil}{\lceil}{\rceil}
\DeclarePairedDelimiter{\floor}{\lfloor}{\rfloor}
\newcommand{\Subj}{\mathbin{\raisebox{.15ex}{$\scriptscriptstyle{\Box}$}\kern-.425em\rightarrow}}
\def\Existence{E!}
\def\Believes{\operatorname{Believes}}
\def\True{\operatorname{True}}
\def\Perfection{\operatorname{Perfection}}
\def\ext{\operatorname{Ext}}
\def\Iff{\leftrightarrow}
\def\Implies{\rightarrow}
\def\Entails{\Rightarrow}
\def\Cov{\operatorname{Cov}}
\def\Equiv{\Leftrightarrow}
\def\Form{\operatorname{Form}}
\def\Informs{\operatorname{Informs}}
\def\technical{$\star$}
\def\vtechnical{$\star\star$}
\def\power{\wp}
\def\Nec{\Box}
\def\Poss{\Diamond}
\def\Prop#1{$\langle$#1$\rangle$}
\def\R{\mathbb R}
\def\N{\mathbb N}
\def\tele{tel\={e}}
\makeatletter
\newtheoremstyle{indented}{3pt}{3pt}{\addtolength{\leftskip}{4.5em}}{-2.5em}{\sc}{.}{.5em}{}
\def\Principle#1#2#3{\theoremstyle{indented}\newtheorem*{principle}{#2}\begin{principle}\def\@currentlabel{#2}\label{#1}#3\end{principle}\let\principle\undefined}
\makeatother
\def\pref#1{{\sc\ref{#1}}}
\def\enum#1{\resume{enumerate}\item #1\end{enumerate}}
\def\ditem#1#2{\begin{enumerate}[resume]\item \label{\mychapter:#1} #2\end{enumerate}}
\def\nitem#1#2{\begin{description}\item[#1\label{\mychapter:#1}] #2\end{description}}
\def\bref#1{\ref{\mychapter:#1}}
\def\dref#1{(\ref{\mychapter:#1})}
\def\drefglobal#1{(\ref{#1})}
\usepackage{graphicx} % support the \includegraphics command and options
\usepackage{array} % for better arrays (eg matrices) in maths
\def\Not{\operatorname{\sim}}
\def\St{\operatorname{St}}
\def\num{\operatorname{num}}
\def\And{\mathrel{\&}}
\def\Or{\vee}
\def\BigOr{\bigvee}
\def\<{\langle}
\def\>{\rangle}
\def\union{\cup}
\def\nleq{\not\le}
\def\N{\mathbb N}
\def\R{\mathbb R}
\def\C{\mathbb C}
\def\Powerset{\mathcal P}
\def\star#1{{}^*#1}
\def\hN{\star{\N}}
\def\hR{\star{\R}}
\def\Z{\mathbb Z}
\def\Power{\mathcal P}
\def\Implies{\rightarrow}
\def\True{\operatorname{True}}
\def\Socrates{\mathrm{Socrates}}
\def\actual{@}
\def\Law{\operatorname{Law}}
\def\Chance{\operatorname{Chance}}
\def\Var{\operatorname{Var}}

\def\H2O{H${}_2$O}

\def\scr{\mathcal}
\def\e{\varepsilon}
\def\eps{\varepsilon}
\newtheorem{lem}{Lemma}
\newtheorem{prp}{Proposition}
\newtheorem*{theorem}{Theorem}
\newtheorem{corollary}{Corollary}
\newtheorem{cond}{Condition}

\renewcommand\thechapter{\Roman{chapter}}

\def\chaptertail{\ifdefined\book\else\end{document}\fi}
 

\title{Infinity, Causation and Paradox}
\author{Alexander R. Pruss}
%\date{} % Activate to display a given date or no date (if empty),
         % otherwise the current date is printed

\begin{document}
\setcounter{secnumdepth}{3}
\setcounter{tocdepth}{4}

\end{document}
\fi

\restartlist{enumerate}

\chapter{Aristotelian Details}\label{ch:details}
\section{Introduction}
\section{More on flourishing}
\subsection{Supernormality}
If Bob is facing unjust torture, and Alice finds a way to substitute herself for Bob, 
Alice has done something morally good. But at the same time, barring special circumstances
(such as Alice having made some promise to Bob, or bearing some responsibility for why
Bob is facing torture), Alice has no obligation to take Bob's place. Her action is morally
excellent, but not obligatory. We call such actions supererogatory.

On the view I am defending, what makes Alice's action morally good is that she 
flourishes volitionally in her action. If, on the other hand, Alice were to encourage Bob's 
torturers, her action would be bad, and she would volitionally languish in it. However if Alice
merely discourages Bob's torturers, but does not offer to take Bob's place, she flourishes
volitionally less than if she offers to take Bob's place, but she flourishes nonetheless.

We find in the will, thus, a distinction (a)~between languishing and flourishing, or 
better malfunction and proper function, as well as (b)~between lesser and greater flourishing,
or a lower and higher degree of proper function. This distinction lines between the moral concepts of
the impermissible and permissible, and the merely permissible and the supererogatory. 

It is likely that a similar distinction is found in areas other than morality. A typical
physicist presumably has a properly functioning physical intuition. But Albert Einstein's physical 
intuition was not merely properly functioning: it was uncannily supernormal, so that thinking through
thought experiments, such as his famous one about riding a lightbeam, led to groundbreaking
progress in physics. And, on the other hand, there are people whose physical intuition is
abnormal. We thus have a distinction (a)~between the abnormal and the at-least-normal, and then
(b)~between the merely normal and the supernormal. The supererogatory is then a species of the
supernormal.

??medical ethics

\subsection{Parts and aspects}
When my arm is functioning poorly due to an injury, I am functioning poorly insofar
as I have an arm. It it tempting to reduce evaluations of the function or flourishing
of a part of a substance to the function or flourishing of the whole in respect of the
part. But while this is tempting, there is good reason to resist this reduction.

Of course, everyone agrees that there are cases where the flourishing or languishing of a part or aspect is
instrumental to the opposite state of the whole. Xenophon has Socrates give the example of a 
person who is harmed by their wisdom because a tyrant hears about the wisdom and has the
wise person kidnapped to serve as an adviser.??ref And many a person would have escaped a broken
leg sustained if they had a minor sprain that kept them from skiing. 

But there are more interesting cases where the flourishing state of a part seems not merely instrumental to 
the opposite flourishing state of the whole, but is constitutive of it. Muscles are torn down by exercise
and regrow stronger. The process of tearing down is harmful to the muscles themselves, but is a constitutive 
part of the proper functioning of the body's system of adaptation to particular activities. And, more
generally, the death of cells is part and parcel of the normal self-renewing persistence of a multi-cellular
organism.

These cases are perhaps not entirely convincing. One might insist that when cells die as part of the self-renewal
of the organism, the death is itself a part of the proper functioning of the cell, and hence both the cell and the
whole are flourishing. Historically, Aristotelians have tended to insist that a thing's destruction is bad
for it.???refs However, this may be mistaken. If we think of substances as four-dimensional objects---spatiotemporal
entities---then a thing is destroyed just in case it has an upper temporal boundary. Now, having \textit{spatial} 
boundaries is not bad for a thing---indeed, having spatial boundaries of the right sort is constitutive of a thing's
having the correct shape and size. A dog so bloated as to be boundless, taking up the whole universe would not be a 
healthy dog! Similarly, a thing could have proper temporal boundaries, and if so, it might be harmed not just by
living too short a time, but also by living too long a time. Whether human beings are of this sort is a difficult
question beyond the scope of this book. I think the way human flourishing seems to always call for more than we have
suggests that humans are not like that. But most human \textit{cells} seem to have a proper lifetime. 

On the other hand, one might think that a muscle's being torn down is beneficial in the medium term to the organism,
but harmful in the short term. The organism does become weaker. Thus the harm to the muscles is matched by the 
short-term harm to the muscular organism.

But cases of redundancy may provide a more convincing example of where the flourishing of a part comes apart from
the partial flourishing of the whole. Suppose that an organism for its basic functioning needs $n_1$ functioning parts of some sort---say,
cells of a particular kind, or legs, or teeth---and for the sake of redundancy it needs some large number $n_2$.
Suppose that having more than $n_2$ is supernormal for the organism (as per our discussion in ??backref), until we
reach some large number $n_3$ at which point the organism has too much.  

For the sake of definiteness, suppose that the parts are teeth of some organism type, while $n_1$ is 30, $n_2$ is 35, and $n_3$ is 40. With fewer
than 30 teeth, the organism fails to chew well. With 35 to 39, it has a healthy level of redundancy. And at 40 or more,
it has too many teeth. Suppose now that Sally is an organism of this sort and she has 38 teeth. One of the teeth, however,
is getting worn down quite a bit. That tooth is no longer fully functional, and hence that tooth is failing to flourish.
However, this does not constitute any failure of flourishing in Sally. Even if the tooth stopped functioning entirely,
Sally would still have sufficient dentation both for first-order purposes and for redundancy. Sally can still be fully
functioning as a whole with respect to her teeth, even though that tooth is not itself fully functioning, and even though
she would be fully functioning to a higher degree if that tooth were to fully function. In this case, the direction of
flourishing of the part and of the whole seems to be the same, but nonetheless one can have full flourishing of the whole
without every part fully flourishing. In such a case, it would not be correct to say that Sally is languishing with respect
to that tooth. For that tooth doesn't make her languish---it just makes flourish less.

Interestingly, redundancy can be between parts or aspects of very different sorts. We might suppose that a flourishing human 
being has a sufficient number of abilities of various sorts. These abilities can be moral, intellectual, emotional, or physical, and
within each category, they can differ quite significantly from each other. Then full flourishing could turn out to be compatible with a severe
impairment within certain abilities---for instance, one's mathematical aptitudes might be dysfunctional, and yet the \textit{person} 
might fully flourish. This would allow an Aristotelian to accept the thesis that some persons with significant disabilities can nonetheless
be fully flourishing.??refs For the disability can constitute the failure of a part or aspect of the person to flourish, without
thereby constituing a failure of flourishing of the whole. 

We could, in principle, suppose that there are some cases where the failure of flourishing of a part might not even make the 
whole flourish less. Going back to Sally, perhaps 37 teeth is better than 38 (perhaps 37 makes for better fit within the jaw),
even though any number from 36 to 39 is fully normal. In that case, if Sally's 38th tooth is starting to deteriorate, this could
be moving her to an even better state. 

At the same time, there are proper parts and aspects such that the flourishing of the part or aspect always lines up with the
flourishing of the whole. In the case of a person, to have one's will flourish in an action is to flourish with respect to the 
will, and makes one better off with that respect, and hence we do not need to correct looser discussion earlier in this book where
no distinction was made between flourishing with respect to the will and having one's will flourish. Plausibly, rationality is similar. 
One is always better off for being more moral and more rational, and a failure of flourishing of one's morality is a failure of the 
person as a whole. 

How much of the above logically possible differentiation between flourishing of the part and partial flourishing of the whole is 
realized in humans is a question for further investigation. However,
we see that the teleological structure of a substance and its parts and aspects can be quite conceptually complex. If we take
the form to be the ground of the essentials of that structure (???what of contingent forms), then this makes for even more work 
for form.

\section{Teleological reductions}
\subsection{A multiplicity of concepts}
The applications above, and the Aristotelian tradition, make use of various normative concepts that are said to be grounded in forms,
such as proper function, teleology, and flourishing. It would be good to investigate if these 
can be further unified, under either one of these concepts, or some further unifying concept.

First, as we have seen in the discussion of the supererogatory and supernormal, we have both binary and comparative normative 
concepts, often in the same context. Suppose a stranger is about to be hit by a train, and the only four options are:

\ditem{kick}{give them a kick to ensure that they have no chance of survival}
\ditem{fun}{make fun of them}
\ditem{idle}{stand by idly}
\ditem{jump}{jump in and push them out of the way likely at the cost of one's own life.} 

The binary distinction is that
the first two options are impermissible, and the other two are permissible. But there is a comparative distinction
as well: \dref{kick} is worse than \dref{fun}, and \dref{jump} is better than \dref{idle}. The binary distinction does not reduce to
degree of comparison: while \dref{fun} is much worse than \dref{idle}, \dref{idle} is much worse than \dref{jump} (though 
it is more natural to say that \dref{jump} is much better than \dref{idle}). And the comparisons do not reduce to the binary 
characterizations.

Proper function and teleology appear to be primarily binary concepts: a thing functions properly or improperly, and a thing either 
does or does not achieve its end. Flourishing, on the other hand, seems to comprise both the binary and the comparative. The mildly
vicious person unjustly suffering horrendous pain appears not to be flourishing \textit{simpliciter}, but if the pain were increased,
they would languish more. Flourishing thus appears the best candidate for a foundational concept for our norms.

But because the notion of ends and teleology has been so important in the Aristotelian tradition, both in the case of voluntary 
action and involuntary activity, it is worth thinking some more about ends.

\subsection{Ends}
Many activities seem to occur for an end. The activity then counts as successful provided that the end occurs and occurs as a
fulfillment of the activity. An organism produces gametes in order to reproduce. A cat chases birds in order to catch them,
and eats them for nourishment. And I put on shoes to keep my feet comfortable when I walk. The end-directedness of much voluntary
activity is obvious, but whether there really is teleology in the involuntary cases is more controversial, though highly intuitive.

The Aristotelian tradition tends to analyze voluntary action as always end-directed, but also tends to see involuntary activity as 
often, if not always, directed at an end. I will argue, starting with the case of voluntary action, that many interesting phenomena 
would be misclassified as end-directed. The actual structure can be more complex, and while it has a directional structure, it is
misleading to think of that structure as teleological in the sense of possessing a \textit{telos}, an end that fulfills it.

Consider a sprinter who is running a hundred meters all out against a clock, rather than against other opponents. The runner has an end, 
namely to sprint 100 meters. But sprinting 100 meters does not explain the intense effort the runner puts in. Less than half of the effort 
could have been put in, and 100 meters would still have been sprinted. The bulk of the runner's effort is explained by being directed 
not at completing the sprint but at completing it in minimum time.

But what state of affairs does the runner's speed-directed effort have as its end? A runner might have a particular target time in mind.
However, we are imagining a runner who runs all out. A runner who is just aiming at a particular time could slow down if it became obvious
that a slower run would still achieve the target time, but not so our all-out runner. Our sprinter may have some specific time in mind
to motivate themselves, but interpreting their action as merely aiming at that time does not capture all of the directional structure 
of the performance. Any shortening of the time of the run is welcome given the sprinter's aims. 

We would normally describe the runner as trying to run 100 meters ``as fast as possible'', and that seems to be a coherent description
of an end. However, the language of ``as fast possible'' should not be taken literally. First, we have the question of what the relevant
comparison class is. Is the runner trying to run as fast as any human being can on any track? As fast as they themselves can run on this
track on this occasion? Or something in between? ??? unlikely success!

Second, suppose we fix a particular sense of ``as fast as possible'', and then after ten meters the runner realizes that they have
been slightly slower than is possible. At this point, it is no longer possible for the runner to achieve the goal of literally running
the run as fast as possible. But we do not expect the runner to stop. We expect the runner to resume running all-out, as part of the
same directed activity.

\section{Individual forms}
Recall the long-standing debate whether forms are individual---numerically different ones for different members of the same kind---or shared
by all members of the same kind.

In ??backref, we saw that there is some advantage to an individual form account of ethics: individual forms intuitively do 
a little more justice to the personal nature of ethical obligation.??[but conjoint twins] ??add 
But are there any other arguments for taking forms to be individual?

I believe so. An initial attempt might be to argue that then the numerically same entity---the form---is present in multiple
places at once, since a form counts as present where its matter is. I do not find this argument compelling, however, as I do not 
think multilocation is absurd.??ref But if you do, that is one argument. Let us consider some others.

\subsection{Individual unity}
My nose and my heart are parts of the same human, while my nose and your stomach are not. What constitutes the 
difference? On an individual form view, there is an elegant Aristotelian answer: my nose and my heart are informed by the same
form, while my nose and your heart are not. This answer will obviously not do on a joint-form view.

Nor will it do to say that the difference is constituted by the fact that there is continuous human matter joing my nose with my
heart, since if you and I were conjoined twins, but conjoined neither by heart nor nose, my nose would be connected by continuous
matter with your heart, even though those would still not be the nose and heart of one human. 

One might try for a teleological account of the unity of the human being: my nose and my heart function together (the nose 
allows the entry of oxygen which is distributed by the heart). However, we are social animals, and teleological cooperation
crosses individual boundaries.

It could be that some other account friendly to joint-form Aristotelianism is available. But the simplicity and elegance of the
individual-form account of individual human unity is a reason to opt for the joint-form view.

\subsection{Distant conspecifics}
Suppose a shared form theory is true. Now, imagine that in our galaxy there is only one human being, Adam, and imagine that in 
a galaxy far, far away, God creates a humanoid comes into existence, with no genetic connection to Adam, but with a form that 
is just like Adam's: this form unifies matter in the same way as
Adam's form does, it imposes exactly the same norms on the form's owner as the human form does on Adam, and it causes the same
structure and behavior as the human form does for Adam. 

At this point we have a dilemma: either the form of this humanoid must be numerically the same as Adam's or not. Suppose it 
must be numerically the same as ours. Then somehow simply by creating something in a galaxy far, far away, God causes an
entity in \textit{our} galaxy---Adam's form---to become multilocated. This seems counterintuitive. 

Suppose that the form does not need to be numerically the same as Adam's. In that case, we have admitted that there can 
be numerically different forms with the same broadly functional features (including the normative functions). This 
means that the question of whether you and I have the  numerically same form is not settled by noting that the forms have 
the same functional features. Indeed, now the question whether your and my form is numerically different or the same becomes
a metaphysical question that no empirical data is relevant to the settling of. There is nothing absurd about there being
such metaphysical questions. But it is some advantage to a theory if raises fewer such questions, having fewer degrees of 
freedom. And if one does accept a theory where it is possible but not logically necessary that different individual substances
have numerically different forms, then one really shouldn't be accepting that in practice you and I share a form. At best
one should be agnostic on this question.

\section{Accidental normative forms}
If you have promised to $\phi$, you should $\phi$. Consider two Aristotelian metaphysical explanations of what is going on here.
On the conditional-norm explanation, the human form contains the conditional norm that you should $\phi$ if you have promised 
to $\phi$, which when combined with the fact that you have promised to $\phi$ grounds an obligation to $\phi$. But there is
another possible explanation. Perhaps promising to $\phi$ causes, perhaps in virtue of a power contained in the human form, 
the normative accident of being such that you ought to $\phi$ to come into existence. One might think of the second story as 
a very robust normative power theory: a theory on which normative powers are a type of causal power that brings into existence
an irreducible and new normative entity. (Most normative power theories do not consider normative powers to be a type of 
causal power.??refs) 

The conditional-norm account posits fewer entities, and insofar as this is the case, Ockham's razor favors it. And intuitively
it seems to be the right account of promissory obligations. If this account holds for all norms, then the human substantial form could be 
the only normative property of the human being---there would be no normative accidents.

Now, while the conditional-norm account for humans
posits fewer entities (i.e., forms), it makes the human form contain more information in the form of conditionals.
For instance, on the conditional-norm view, humans without a Y chromosome have a human nature that specifies what
range of physical developments that are normal expressions of the genes that are unique to the Y chromosome. On the 
normative accident view, it may be that humans with a Y chromosome also have an additional non-physical property governing
the expression of Y-based genes.\footnote{It's worth noting that both views are compatible with a broad variety of views on gender and 
transsexuality, since choosing between the metaphysics of the two views does not settle the question of what the range
of normal physical developments is. One might think that the normative accident view allows for a more conservative
theory on which there is a metaphysical component---an accidental form---that determines whether one is really male or female, 
an accident of maleness or an accident of femaleness.  But at the same time, the normative accident view enables one to have 
a metaphysical basis for the claim that one's real sex and/or gender fails to match one's biological constitution at birth.
It is also worth noting that for norms relevant to sex and/or gender, the conditional norm view could make the antecedents 
of the conditionals be facts about DNA (such as whether one has a Y chromosome) but could also make them involve facts about 
psychology and society. The metaphysics does not by itself settle the normative questions here.}

At the same time, the normative accident view of human beings involves significantly more in the way of \textit{causal} laws
presumably grounded in human causal powers, such as that when one makes a promise, a promissory-obligation
accident comes into existence, or when one has such-and-such DNA, then such-and-such an accident governing norms of gene expression
coms into existence. Moreover, since normative accidents are in large part defined by the norms they embody, the informational
complexity of the normative accident is presumably present in the causal power for its production---it is a power to produce an
accident of, say, being required to $\phi$.  Thus while we have reduced the \textit{normative} complexity in human nature, we have done 
so at the cost of \textit{causal} complexity, \textit{and} a multiplication of entities.

This last point does not apply to normative accidents that have a cause outside of the human being. But it is difficult to think
of examples, with the exception of one theological possibility. According to many Christians/, by God's grace human beings
can be directed at the ``beatific vision'', a direct vision of God. This beatific vision exceeds the power of human
nature, and it is usually taken that natural human fulfillment does not require it. One way to make sense of this is to suppose
that God by grace gives all or some humans a normative accident of teleological directedness at the beatific vision, together
with the supernatural powers that lead in the direction of fulfillment of this telos. 

\chaptertail
