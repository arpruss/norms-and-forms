% \section{Infinity}
% We saw in ??backref that population ethics raises Mersenne questions. But \textit{infinite} population 
% ethics not only raises questions, but creates serious paradoxes. For instance, suppose there is an 
% infinite line stretching to infinity both to the left and the right, with tickmarks every meter 
% labeled by an integer (bigger numbers being to the right), and one person standing at each tickmark. 
% All the people are on par. Suppose you now have two 
% choices:
% \ditem{benefit-even}{Benefit the people at $2,4,6,...$}
% \ditem{benefit-odd-1}{Benefit the people at $1,3,5,...$}
% where all the benefits are the same.

% Intuitively, we should be indifferent between these. It makes no difference whether we should benefit the 
% people at the positive even- or positive odd-numbered locations. The options are on par. 

% But now add a new option:
% \ditem{benefit-odd-3}{Benefit the people at $3,5,7,...$}
% with the very same benefits. 
% Observe now that \dref{benefit-odd-3} benefits the people standing immediately to the right of 
% the beneficiaries of \dref{benefit-even}, while \dref{benefit-even} benefits the people standing 
% immediately to the right of the beneficiaries of \dref{benefit-odd-1}. Thus,
% the moral relationship between \dref{benefit-odd-3} and \dref{benefit-even} should be the same as that
% between \dref{benefit-even} and \dref{benefit-odd-1}, respectively. 
% But the latter two, as already noted, are intuitively on par. 
% Thus, likewise, \dref{benefit-odd-3} and \dref{benefit-even} are on par. 

% We can argue for the parity of \dref{benefit-odd-3} and \dref{benefit-even} as follows. If we re-label 
% tickmark $n$ as $(n-1)^*$, then options \dref{benefit-even} and \dref{benefit-odd-3} are equivalent to:
% \begin{itemize}
	% \item[(\bref{benefit-even}$^*$)]{Benefit the people at $1^*,3^*,5^*,...$.}
	% \item[(\bref{benefit-odd-3}$^*$)]{Benefit the people at $2^*,4^*,6^*,...$.}
% \end{itemize}
% If benefiting those at positive odd-numbered locations is on par with benefiting those at positive 
% even-numbered locations, then surely this should not depend on whether we used the original or 
% the asterisked numbering. Thus (\bref{benefit-even}$^*$) and (\bref{benefit-odd-3}$^*$) are on par.
% But they are logically equivalent to \dref{benefit-even} and \dref{benefit-odd-3} respectively, so these
% are on par as well.

% But being on par morally is transitive. So, if \dref{benefit-odd-3} and \dref{benefit-even} are on par,
% and \dref{benefit-even} and \dref{benefit-odd-1} are on par, it follows that \dref{benefit-odd-3}
% and \dref{benefit-odd-1} are on par. But that conclusion is clearly false, since if we can benefit a 
% person without anybody else losing anything, we have moral reason to do so barring some deontological consideration,
% and if we are set to do \dref{benefit-odd-3} then switching to \dref{benefit-odd-1} benefits the person at
% location $1$ without anybody losing anything.

% Perhaps, however, we should deny that \dref{benefit-even} and \dref{benefit-odd-1} are on par. There are two
% ways of doing that. One is to say that the two cases are incomparable. The other is to say that \dref{benefit-odd-1}
% is better than \dref{benefit-even}.\footnote{Saying that \dref{benefit-even} is superior to 
% \dref{benefit-odd-1} is not tenable. The relationship of \dref{benefit-even} to \dref{benefit-odd-1} is the 
% same as that of \dref{benefit-odd-3} to \dref{benefit-even}, and if we say that \dref{benefit-odd-3} is superior
% to \dref{benefit-even}, we will then by transitivity have to say that \dref{benefit-odd-3} is superior to 
% \dref{benefit-odd-1}, which is absurd.}

% Neither option is particularly appealing. Suppose that the benefit is the saving of a life. Then if
% \dref{benefit-odd-1} is better than \dref{benefit-even}, by the above reasoning \dref{benefit-even}
% will be better than \dref{benefit-odd-3}. So we will have this preference ordering:
% \ditem{benefit-order}{\dref{benefit-odd-1} $>$ \dref{benefit-even} $>$ \dref{benefit-odd-3}.}
% Now \dref{benefit-odd-1} is better than \dref{benefit-odd-3} by exactly one life saved. So it seems that
% \dref{benefit-odd-1} will have to be better than \dref{benefit-even} by less than saving a life---presumably,
% by half a life-saving---and \dref{benefit-even} will have to be better than \dref{benefit-odd-3} by less than saving 
% a life---again, presumably by half a life-saving. But this is very implausible. When the scenarios differ in 
% whose lives are saved, and there are no probabilities involved, surely any two scenarios that differ must do 
% so by one or more lives. 

% On the other hand, suppose that we have incomparability between \dref{benefit-even} and \dref{benefit-odd-1}
% and by the same token between \dref{benefit-odd-3} and \dref{benefit-even}. Now suppose that you have 
% a button pressing which saves the lives of the people in positions $1,3,5,...$ and a button that saves the 
% lives of the people in positions $2,4,6,...$ but where the first button has a side-effect: it causes a 
% migraine to a perfect stranger, Alice. It seems very plausible that pressing the second button is the 
% morally better choice. But 
% given the incomparability claims, there is a conclusive argument against this moral preference. For saving the
% lives of the people in positions $1,3,5,...$ while triggering a migraine for Alice as a side-effect is 
% better than saving the lives of the people in positions $3,5,7,...$, since saving the life of the person 
% in position $1$ is well-worth the migraine to Alice. If saving the people at $2,4,6,...$ were better than
% saving the people at $1,3,5,...$ plus triggering a migraine, then saving the people at $2,4,6,...$ would be
% even better than saving the people at $3,5,7,...$. But saving the people at $2,4,6,...$ is not better than
% saving the people at $3,5,7,...$, as we have assumed \dref{benefit-odd-3} and \dref{benefit-even} are incomparable.

% Here is another variant of the above problem. Let $L_n$ be the action of benefiting all the infinitely many people to the left of 
% position $n$, and let $R_n$ be the actions of benefiting all the infinitely many people to the right of position $n$. Write $A\le B$ to say that action $B$ 
% is at least as good morally as action $A$, and $A<B$ to say that $A\le B$ but not $B\le A$. Say that $A$ and 
% $B$ are comparable provided that $A\le B$ or $B\le A$. Here are some assumptions about the 
% moral preferability relation:
% \ditem{pref-trans}{Transitivity: If $A\le B$ and $B\le C$ then $A\le C$.}
% \ditem{pref-mono}{Strict monotonicity: For any $m$, we have $L_m < L_{m+1}$ and $R_m < R_{m-1}$.}
% \ditem{pref-inv}{Weak translation invariance: For any $m$ and $n$, we have $L_m \le R_n$ if 
% and only if $L_{m+1} \le R_{n+1}$, and $L_m \ge R_n$ if and only if $L_{m+1} \ge R_{n+1}$.}
% Transitivity is very plausible. Next, by switching from $L_m$ to $L_{m+1}$ or from $R_m$ to $R_{m-1}$, 
% one benefits the person at location $m$, without taking benefits away from anyone, and this is surely 
% better, thereby yielding strict monotonicity. Finally, weak translation invariance is based on the observation 
% that the relationship between $L_m$ and $R_n$ is exactly the same as that between $L_{m+1}$ and $R_{n+1}$.\footnote{Strong
% translation invariance would be the thesis that all the $L_m$ are morally equivalent and that all the $R_m$ are 
% morally equivalent, since the $L_m$ are all translations of one another and the $R_m$ are all translations of 
% one another. But strong translation invariance would be incompatible with strict monotonicity. For a discussion 
% of weak and strong invariance conditions, see ??Pruss:nonclassical.}

% In Appendix??forwardref, I prove that given \dref{pref-trans}, \dref{pref-mono} and \dref{pref-inv}, exactly
% one of the following conditions holds:
% \ditem{pref-incompar}{For all $n$ and $m$, actions $L_n$ and $R_m$ are incomparable with each other.}
% \ditem{pref-right}{For all $n$ and $m$, we have $L_n<R_m$.}
% \ditem{pref-left}{For all $n$ and $m$, we have $L_n>R_m$.}
% In other words, we either have complete incomparability between any left- and any right-benefit action, or else
% we have a radical skew where all the right-benefit actions beat all the left-benefit actions or all the 
% left-benefit actions beat all the right-benefit actions. 

% We could imagine a kind of agent whose morality exhibits the radical directional preference 
% of \dref{pref-right} or \dref{pref-left}. Perhaps this would be a kind that lives in a spacetime 
% without the symmetries that our spacetime exhibits, or it is a kind without requirements of 
% egalitarianism. But we are not that kind. This kind of radical skew seems deeply implausible 
% \textit{to us}, and so it seems we would need to have the radical incomparability of \dref{pref-incompar}. 

% But the radical incomparability of option \dref{pref-incompar} is also implausible. One could adapt
% the Alice argument given before. Intuitively, $L_0$ is morally preferable to $R_0$-plus-migraine for
% Alice: you shouldn't cause a migraine to a stranger to make sure that the people you save are to 
% the left of zero. But $R_0$-plus-migraine-for-Alice clearly beats $R_{1000}$, since 
% $R_{0}$ saves a thousand additional people that are not saved by $R_{1000}$ (namely the people at 
% locations $1$ through $1000$), and a side-effect of a migraine to a stranger is definitely worth 
% tolerating to save a thousand lives. But $R_{1000}$ is not worse than $L_0$ by \dref{pref-incompar},
% and and hence $R_0$-plus-migraine-for-Alice cannot be worse than $L_0$. So the incomparability view 
% undercuts a plausible judgment about avoiding side-effects.

% It seems clear that something morally paradoxical happens in these kinds of infinite cases. But an 
% Aristotelian has a neat way out. These kinds of choices are outside the human ecological niche. If 
% morality were kind-independent and necessary, morality would have to extend to such cases. But 
% it is quite reasonable to suppose human nature either does not contain principles that apply to 
% such cases, or contains principles that do apply to such cases, but end up contradicting each other 
% in those cases---with morality glitching (cf. ??forwardref to ch 10)---or end up applying to such 
% cases but generating conclusions that don't fit with some of the moral intuitions built-into that 
% nature.\footnote{It is worth noting that we cannot entirely escape the need to address such cases
% by saying that they are outside our sphere of activity. For, adapting the Pascal's Mugger story, 
% imagine you are approached by a strange person who tells you that she is a magician from another
% universe where there are infinitely many people arranged a meter apart on a line, and they are all
% drowning, and she can by a spell effect, say, $L_0$, $R_{1000}$ or $R_{0}$-plus-migraine-for-Alice. 
% She can't decide which to do and wants your advice. Obviously, you wouldn't \textit{believe} her story. But if you are a good Bayesian,
% you would assign it a non-zero probability, and the question would indeed become one of moral 
% relevance. That said, it is not surprising if morality behaves strangely once you are in an odd 
% epistemic state. What should you do, we might ask, if you come to think that dialethism is true and 
% you should do the wrong thing? Or what should you do if you become convinced of solipsism or its 
% opposite, alterism (the view that you don't exist but other people do). We should not be surprised if 
% either there are no answers to such questions or the answers are strange.}

% A different kind of being could have different norms from us and, if Aristotelian optimism applied 
% to them as well, correspondingly different intuitions. As already mentioned, they might be less 
% egalitarian than us and tolerate more arbitrariness in preferences between infinite groups. Or they
% might have different norms regarding side-effects and thus be able to morally embrace a greater
% degree of incomparability than us. 

% It should be noted that the paradoxes of infinity here only scratch the surface of the range of 
% oddities imaginable.??refs


% \section*{$^*$Appendix: Skew in benefiting infinitely many people} 
% In ??backref, it was claimed that a preference ordering on certain actions 
% that benefit an infinite number of people satisfying certain axioms either
% suffers from massive incomparability or has a massive left-right bias.
% That result follows from the following.

% ??number??
% \begin{theorem} Let $L_n$ be the set of integers less than $n$
% and $R_n$ the set of integers greater than $n$. Let $\scr A$ be the set of all
% the $L_n$ and $R_n$. Suppose $\le$ is a transitive relation on $\scr A$ such that 
% $A<B$ whenever $A\subset B$ and $n+A\le n+B$ if and only if $A\le B$ for any integer
% $n$, for all $A$ and $B$ in $\scr A$. Then exactly one of the following holds:
% \begin{itemize}
% \item[{(i)}] for all $m$ and $n$, we have neither $L_m\le R_n$ nor $R_n\le L_m$,
% \item[{(ii)}] for all $m$ and $n$, we have $L_m<R_n$
% \item[{(iii)}] for all $m$ and $n$, we have $R_n<L_m$.
% \end{itemize}
% \end{theorem}
% Here, $A<B$ provided that $A\le B$ but not $B\le A$, and $n+A=\{n+m:m\in A\}$ is the 
% translation of $A$ by $n$.

% For we can identify the actions in ??backref with the sets of people benefited by them.
% Then note that if $\le$ is transitive, so is $<$.\footnote{\label{fn:trans}More generally, if $A\le B\le C$, and 
% at least  one of the inequalities is strict, it follows that $A<C$. For by transitivity of 
% $\le$ we have $A\le C$, and if we don't have $A<C$, then it must be because $C\le A$. Then
% $A\le B\le C\le A\le B$. Hence $B\le A$ and $C\le B$ by transitivity of $\le$, which contradicts
% the claim that $A<B$ or $B<C$.}
% The condition that $A<B$ whenever $A\subset B$ follows by induction and transitivity of $<$ 
% from \dref{pref-mono} since the only way
% $A\subset B$ can hold is if $A=L_m$ and $B=L_n$ with $m<n$ or $A=R_m$ and $B=L_n$ with $m>n$.
% The translation invariance condition then follows by induction from \dref{pref-inv} since $n+L_m=L_{m+n}$
% and $n+R_m=R_{m+n}$.

% \begin{proof}[Proof of Theorem]
% Suppose (i) does not hold, so for some $m$ and $n$ we have $L_m\le R_n$ or $R_n\le L_m$.

% First suppose $L_m\ge R_n$. We will now prove (iii). 
% We have two cases. First suppose $m<n$. Then 
% $L_n>L_m\ge R_n$, so $L_n>R_n$.\footnote{See note~\ref{fn:trans}.}
% By our translation invariance condition, we have $L_k>R_k$ for all $k$.
% Now fix any $j$ and $k$. If $j\ge k$, then 
% $L_k>R_k\ge R_j$ by monotonicity so $L_k>R_j$. if 
% $j<k$, then $L_k>L_j>R_j$. So we have (iii).

% Next suppose $R_n\ge L_m$. Let $-A = \{ -x : x \in A \}$.
% Define $A \le^* B$ provided $-A \le -B$. It is easy 
% to see that $\le^*$ also satisfies all of the 
% assumptions of the Theorem. Moreover, since we have
% $R_n\ge L_m$, we have $-R_n\ge^* -L_m$. But 
% $-R_n = L_{-n}$ and $-L_m=L_{-m}$. Thus $L_{-n}\ge^* R_{-m}$.
% Applying the previous paragraph to $\le^*$ with 
% $-n$ and $-m$ in place of $m$ and $n$, we get, for 
% all $j$ and $k$, the inequality $L_k>^*R_j$.
% Hence $-L_k>-R_j$, and so $R_{-k}>R_{-j}$ for all
% $j$ and $k$, which implies (ii).
% \end{proof}

