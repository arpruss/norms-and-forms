\def\mychapter{IV}
\chapter{Applied ethics}\label{ch:applied-ethics}
\section{Introduction}
Thinking that ethical duties are grounded in norms innate to human nature does not by itself logically entail answers to controversial 
questions of applied ethics. One can think that our nature requires us to kill those whose suffering we cannot stop,
and hence that euthanasia is required, one can think that our nature prohibits the killing of the innocent even if that killing
would be in their interest, and one can have an in-between view. 

But the nature-based approach provides at least two benefits for applied ethics. First, because of the Aristotelian harmony principle, it 
allows facts about our natural behaviors and needs as the kinds of organisms we are to to provide us with defeasible but often strong
evidence about what we should do. Second, it makes it more plausible than it would be on a number of competing theories that the answers 
to applied ethics questions might be irreducibly intricate---not reducible to a small number of simple principles---and might include 
domain-specific ethical rules for the various areas of our natural lives, such as family relationships or sexuality or (if it's natural)
property rights. 

We will thus explore some things that we can say on nature-based ethics. The plausibility of what we will say will serve as indirect
evidence for the underlying Aristotelian metaphysics.

\section{Natural relationships}
\subsection{Siblings and cousins}
An interesting test case for an ethical theory is whether it can make good sense of our duties to our siblings and cousins. Duties to friends and spouses plausibly
arise from commitments we make. Duties to parents have traditionally been grounded in our obligation of gratitude for our life. Duties to children can
typically be grounded in the decision to perform actions that have a non-negligible probability of producing a person dependent on us. Duties to strangers
might be grounded in our shared rationality. But we owe more to our siblings and cousins than we do to strangers, even though typically we had no say in whether
we were to have siblings and cousins, and even when we have no favors to return.

On utilitarianism, our duties to siblings and cousins come mainly from the contingent fact that we tend to be better positioned to do good to them, say because we know
their needs better, are likely to be physically closer, and help from us is likely to be more welcome. But if such contingencies are all that is involved, then we also
have to accept an error theory about our intuitions when they go beyond these contingencies. If a sibling or a stranger is drowning, other things being equal one should
try to rescue the sibling, even if the stranger is slightly easier to pull out, or is likely to have a slightly better future life. If one finds out that a local
homeless person is a cousin one has not seen since early childhood, it is more vicious to ignore their needs than to ignore similar need in a random stranger.
Murder of a stranger is evil, but fratricide is worse.

In general, utilitarianism, contractarianism and Kantianism focus on the agent's rationality, taking the details of the agent's humanity to provide no direct normative
input into ethical decisions. The fact that most humans hate eating mud gives one reason not to feed mud to them, and the fact that we are unable to instantly
teleport ensures we do not have the same obligation to those on other continents as to those nearby. But these are non-normative facts, and the normativity of the
conclusions here comes from general normative considerations applicable to all rational beings. There is some \textit{prima facie} plausibility to the idea
that the non-normative facts about the relationships between parents and children, together with normative facts applicable to all rational beings, could explain
distinctively filial and parental duties. But this is not plausible for the cases of siblings and cousins.

However, if we see ethics as based on the norms written into our \textit{human} nature, given a harmony between the rational and animal aspects of this humanity,
will very plausibly allow for distinctive ethical norms tied to particular kinds of natural human relationships, including perhaps in the first instance familial ones.
There is no need on our Natural Law ethics to derive the duties to cousins from non-normative facts about cousinhood and norms for all rational beings: such rules
can be fundamental. And the laws can, in principle, be at any level of precision, be it to simply consider one's siblings at a higher weight in one's moral calculus
than more distant relatives (we are all relatives, after all, as we learn from evolutionary theory) or to prefer one's siblings over one's cousins to such-and-such
a specific degree. The laws could even have social construction built in: they could require us to respect our relatives in the ways that our society prescribes,
and require us to establish societies that institute ways for us to respect our relatives.

Divine command theory has a similar advantage: God's commands can be at any level of generality or precision, be it to love one's neighbor or to telephone one's cousins
at least twice a year if one can. In principle, rule utilitarianism can do this as well: it is plausible that having rules concerning special relationships like
fraternal ones could maximize utility. But Natural Law arguably gives a better explanation of the duties tied to these special relationships. For the nature of these
special relationships is very plausibly tied to our humanity, and hence it makes sense that the special obligations attached to them should flow from that humanity
rather than the commands of a God or the results of an abstract hypothetical optimization procedure.

Indeed, on Aristotelian natural law, we can say that having these kinds of special obligations is an important aspect of what makes us human---for it is an important
aspect of our form, which is precisely what makes us human.

\subsection{Less natural relationships}
We have a broad variety of socially-instituted and culturally-variable relationships which are very unlikely to have norms encoded
for them in human nature.  In English-speaking countries the relationship to the  parent of one's godchild or the godparent of one's child
tends not to have sufficient importance to even have a name, while in other cultures it is important and specifically named. The relationship
between an employer and employee varies so broadly with legal and social structures that it is probably best seen as an umbrella for a number
of different relationships, none of which is likely to be encoded in human nature.

Admittedly, a relationship could fail to be culturally widespread and yet could have norms encoded for it in human nature, but there is a more elegant approach to
analyzing such relationship: we can see them as cultural determinations of a more fundamental relationship type, with some of the norms coming from human nature's
rules for the more fundamental relationship and others from the culture. Moreover, human nature may prescribe the scope for cultures to establish the rules.
Such relationships can be thought of as ``less natural''. At the same time, the difference between these relationships and the ``more natural'' ones like siblinghood
are likely to be largely of degree. For while there may be a fundamental normative relationship of siblinghood, it has further culturally-determined
norms.

\subsection{Marriage}
A particularly interesting question, of significant relevance to controversies in our society over the past century, is where \textit{marriage}
lies on the naturalness spectrum. I shall argue that it is likely to be quite natural, with a number of fundamental norms grounded in our human
nature by arguing against two main alternative theories and combinations of them.

The first theory holds that marriage is an institution defined by many human societies. Like other such social institutions, such as judgeship, parliament membership,
monarchical sovereignty, exchequer chancellorship, and presidency, it is defined by the rights and obligations conferred by society on those who enter into the institution.
While we use the same words ``judge'' and ``monarch'' across societies, there is only a family resemblance between the institutions these terms refer to, since the actual
rights and obligations defining the institutions are often very different indeed. The resemblance may be very weak: the rights and obligations of the monarch of England
in the 13th century are about as different from the rights and obligations of the current monarch as the rights and obligations of modern day judge are from a modern day
executioner. Nonetheless, for historical reasons we may use the same word ``monarch'', sometimes clarifying with adjectives like ``absolute'' or ``constitutional''.

The second theory has it that couples choose to undertake certain obligations with respect to each other, which obligations give rise to rights,
and this complex of rights and obligations defines the marriage. In more traditional societies, a couple may not choose the obligations specifically
but rather will simply opt for the ``customary'' obligations and their consequent rights. In modern Western society, many couples write their own
wedding vows, specifying general obligations. But even in those cases, it is likely that these vows are not typically thought of as a precise and exhaustive
legal contract, but rather as a way of customizing one of the prevalent packages of obligations. Again, on the individual theory, we use the same word
``marriage'' for all these different packages of rights of obligations due to some sort of vague family resemblance between them.

A more sophisticated theory??refs may combine aspects of the social and individual theories, holding that not only do couples undertake obligations to each other and gain
rights with respect to each other, they also undertake obligations to society and gain rights with respect to society.

But the individual and social theories are unsatisfactory for multiple reasons.

\subsubsection{Discovery}
People in good marriages come to discover new normative aspects to marriage as they go through life together. ??add-specifics?
If the norms of marriage were simply whatever it was that the parties to the marriage chose, there would be nothing to discover.
And if the norms of marriage were simply set by society, it would be odd to think that it is particularly by living the married life that
one discovers the norms. Rather, the norms would be discovered by study of the history of the social institution of marriage,
the laws surrounding it, the intentions of the legislators, and so on.

We might, admittedly, in individual and social institution cases discover
new normative facts by logical derivation from previously known ones, but that is not actually the primary way in which we learn about marriage:
we learn about it by observing it from the inside.
And we discover new facts, including normative ones, about natural kinds of entities precisely by observing these entities. By observing water,
we come to see that it is H$_2$O and by observing mammals, we come to see that their middle ear should have the malleus, incus and stapes bones??check. And
it is in our own case that we are best positioned to observe marriage at work, so it is unsurprising that such observation produces knowledge of normative
aspects of the relationship.

Central to this growth is the Aristotelian harmony between different norms. Living according to the norms of marriage tends to fulfill us in other respects,
while living contrary to the norms of marriage tends to be bad for us in other respects, and these are things we can often see. A happy marriage makes for
happy spouses and an unhappy marriage for unhappy spouses.

\subsubsection{Travel}
Generally speaking, when a married couple emigrates to or visits another society, they are deemed married in their new place, unless there is some general reason that
precludes them from counting as married, such as when they are of the same sex and move to a jurisdiction that does not recognize same-sex marriage.

Moreover, this recognition of them as married is not just an honorific indexed to their country of origin. When the Queen of Denmark visited the United States in 1991,
she was referred to as a ``queen''??check, but obviously she did not have rights and obligations of a monarch with respect to the United States, and so ``queen''
here was indexed with respect to the Kingdom of Denmark, and similarly for the title ``prince'' held by her husband. However, if someone referred to Henrik
as Margrethe's \textit{husband} or to Margreth as a \textit{married woman} during the visit, these terms would not be merely indexed to Denmark. Rather, they would
have the rights and obligations of an American married couple, as modified by their special immunity to persecution, and an ordinary non-diplomatic visitor from
Denmark would not even have that modification.

Should we say that by the mere fact of entering a country, a couple that was married in their country of origin enters into a new marriage institution?
On the social theory that is exactly what happens: the couple receives a new package of rights and obligations, definitive of marriage in a new society.
But this would be quite surprising: it would mean that a couple going for a honeymoon in another country would have had two weddings (one might tongue-in-cheek wonder if
theyn they shouldn't then be entitled to a second honeymoon?), and globetrotting couples would rack up marriage after marriage. Moreover, relinquishing one's
citizenship in a country one no longer lives in would be tantamount to a divorce.

Or perhaps instead of the new institution being entered into upon entry into a country, a couple by marrying enters into the marriage institution of every
jurisdiction that is willing to recognize them as married, but the rights and obligations of these institutions are merely conditional on their being in
those countries. While this would alleviate the problem of multiple weddings, such automatic entry into institutions in states that have no jurisdiction over one seems implausible.
Moreover, the problem of multiple weddings is not solved. When Margrethe and Henrik married in 1967, there was no state of East Timor. Then in
1975 it declared independence. By that declaration, did they impose a new marriage institution on Margrethe and Henrik, a marriage institution that disappeared
next year when East Timor was annexed by Indonesia, and then reappeared in 2002?

On the purely individual theory, the travel problem disappears. Different states may add rights and obligations, but what defines the marriage is the
complex of rights and obligations that the couple entered into on their own, and it counts as a ``marriage'' in their travel destination because of the
family resemblance between these rights and obligations and those that members of that society take on when they enter into an analogous relationship.

\subsubsection{Cross-cultural criticism}
Andronia is an especially sexist society, and Bob and Alice is an Andronian married couple. You've never interacted with Bob in a context that made his sexist
views clear, but one day you find out that Alice is sick, and Bob is not showing any consideration for Alice besides the minimum needed to be shown to any human being. You
call Bob out on this, and he tells you that in Andronia it is the wife's job always to show consideration for her husband while the husband need only keep the wife alive
and show her the kind of consideration one owes every human being when she is sick. He adds: ``This is how my parents behaved, how Alice's parents behaved, and Alice knew that this is what she was getting into when we got married.''

If marriage were a natural relationship, we could say that Bob and the rest of Andronian society is just wrong about what marriage requires, and we could
say to Bob: ``That may all be, but it's not how husbands \textit{should} behave!'' We could then show Bob examples of virtuous, caring egalitarian couples
in the hope that these examples would open his mind to what marriage really entails. Or we might say to Bob: ``If that's all you've committed to, then
you're not really married, and so you are reaping the benefits of marriage from Alice under false pretenses.''

But if the complex of obligations in marriage is either socially or individually defined, and if neither Andronian society nor the couple included
any special obligation of husband to wife in sickness beyond that which we owe any other human being, then Bob could well be simply right in his understanding
of his marital duties. This is an unattractive position.

Granted, if Bob and Alice are now living in a less sexist society, we could tell Bob that by moving to this society they have accepted the additional
duties of husbands to wives. This is, however, dubious. It may be that be immigrating to a country we take on the legal obligations of that
country, but it could well be the case that Bob is meeting these legal obligations, as they tend to be fairly minimal. What Bob is failing to do is
to meet the customary obligations attached to marriage in less sexist societies, but it is implausible that by moving to a country one becomes obligated
by the customs of the country. No moral criticism would necessarily attach to an American couple if after moving to Canada they failed to celebrate
Thanksgiving in October. Furthermore, nothing of significance is changed in the above story if we specify that Bob and Alice are \textit{still} living in
Andronia. Be they in Andronia or elsewhere, a husband owes more to his wife than Bob thinks.

Perhaps we could tell Bob: ``If that's all you committed to, then you aren't married in \textit{our} sense of the word.''
But that isn't a criticism of Bob's behavior with regard to Alice. Bob could just say: ``So what?'' At most it is a criticism of his misuse of words if Bob
claimed to be  married. Moreover, even as a linguistic criticism it is unlikely to hold water. For we do in fact use ``marriage'' and related words for relationships in
a vast array of historical and present societies, many of which are quite sexist indeed.

Admittedly, on both the social and the individual view, we could criticize Bob for the relationship that he is in. We could tell him: ``If that's what marriage
in your context is, it's a corrupt institution, and you shouldn't be married to Alice.'' But this is unacceptably weak tea. It allows that Bob is married to Alice but does
not owe her consideration in sickness beyond that owed a stranger.

\subsubsection{Fulfillment of a natural desire}
Plausibly, apart from reasonable moral and practical restrictions, people should be able to marry those whom they wish to. A society that did not make this
possible would be failing its members.

Now, society has no obligation to make possible the fulfillment of every desire people have. Rather, it is reasonable to make a distinction between natural
desires and more contingent desires, and hold that society should support the fulfillment of natural desires, such as for food, drink, shelter, useful
employment, and knoweldge. Given the plausibility that marriage is one of those things society ought to make available to its members, it is plausible that
the desire for marriage is a natural human desire. But if it is a natural human desire, then it is plausible that marriage itself is natural rather than
constructed.

This is perhaps the weakest of the arguments for marriage being a natural relationship, however. First, not everyone shares the intuition that a society ought
to make marriage possible. Second, it is not clear that we couldn't have a natural desire to construct---individually or socially---an institution of a certain
type.

\subsubsection{Same-sex marriage}
\subsubsubsection{An argument for liberals}
Let us assume that egalitarian justice requires one to advocate for same-sex marriage in jurisdictions where same-sex marriage is not available.

But suppose that marriage is socially constructed, and that we are in a locality in which one of the norms of marriage is that it be a relationship
between a man and a woman. Then, if we understand ``marriage'' as the word is locally understood, it makes no more sense to advocate for same-sex marriage
than to advocate for chess without pawns: these are simply contradictions in terms. Granted, we may choose to advocate for social recognition of another,
more egalitarian institution than marriage. But that will be a different institution.

If we advocate for this different institution, we have two choices. Either, we propose to maintain the institution currently called marriage, whether for everyone
who wishes to enter into it or just for those grandfathered into it, or not. If we propose to maintain the current non-egalitarian institution, then we are not
really advocating for same-sex marriage. We are advocating for a two-institution model, closely akin to marriage plus civil unions compromises that have generally been
seen as unacceptable by advocates of same-sex marriage.

On the other hand, if it is proposed not to maintain the current institution of marriage, then the common and plausible
arguments that extending marriage to same-sex couples does no harm to currently married opposite-sex will ring hollow. For it is a part of the proposal that the
institution they are a part of be annihilated. Furthermore, in practice, in jurisdictions where marriage has been extended to same-sex couples, generally those who were
previously married still count as married. Therefore, on the assumption that marriage is socially constructed, not only is there annihilation of the institution
that couples used to be a part of, but these couples are, without their express consent, inducted into the new institution. Such automatic induction into a new
relationship does not seem consistent with the ideals of a free society, and yet generally defenders of same-sex marriage have not been bothered by this.

If, however, one holds that marriage is a natural human relationship, then one can argue for marriage equality without arguing for a two-institution model or for
the annihilation of the existing institution. Instead, one can hold that marriage is a natural human relationship which non-defectively can be instantiated by
couples of the same sex as well as couples of the opposite sex. Given that marriage, understood univocally, can be entered in by both same-sex and opposite-sex couples, it is clear why
it is discriminatory for a state to limit recognition of it to opposite-sex couples. And in advocating the end of this inequality, one isn't advocating for an end of
an existing institution, but simply for the state's recognition of the fact that this natural institution can equally well include same-sex and opposite-sex couples.

It is worth noting that defenders of marriage equality who  hold that marriage is constructed by individual couples can also avoid the above problematic consequences
of social construction. On the individual construction view, in recognizing a marriage, the state is recognizing is a certain type of contract, where the type is
defined by a kind of family resemblance. But recognition of opposite-sex contracts
of a certain type without recognition of same-sex contracts of a relevantly similar type would be unreasonable. Imagine if one could only sell a house to someone of
the opposite sex, after all. On the individual construction view, the claim that no harm is done to opposite-sex couples by state recognition of same-sex marriage is
easily defensible. So, the above argument from same-sex marriage advocacy supports the natural relationship view and the individual construction view, but not the social
construction view.

\subsubsubsection{An argument for conservatives}
Here is a plausible principle: If we limit access to an institution on the grounds of gender or sex, absent very strong reason we should strive to make an equivalent
available.??ref For instance, perhaps there is some reason for colleges to limit certain sports to one gender, but then they should make other sports available
to the other gender.
But many conservatives have not only object to same-sex marriage but also to the availability of civil-union institutions for same-sex couples. I will argue that
such conservatives should embrace a view of marriage as a natural relationship.

For if marriage is constructed, either individually or socially, then even if the norms of that construction limit marriage to persons of the opposite sex, an
equivalent institution without that limitation could be constructed, and by the principle at the top of this argument, it ought to be. In fact, it seems that
the best way to resist this argument would be for the conservative to hold that marriage is a natural relationship, and that this relationship is only possible
or only normatively possible for opposite-sex couples, while any superficially similar relationship between persons of the same sex is not a natural relationship.
Because no merely social institution would be a natural relationship, it would not be an equivalent to marriage. Therefore, the conservative can respond to the
original argument by saying that there is very strong reason not strive to make an equivalent available, namely that no equivalent is possible.

In response, as per our previous argument, the defender of same-sex marriage should say that marriage is a natural relationship that \textit{can} legitimately
hold between persons of the same sex. So this conservative response does not close the debate. But it provides the conservative with a way forward. Indeed,
it seems that both sides on the same-sex marriage debate will be better served by moving to a natural relationship view of marriage, and then discussing whether
this natural relationship has norms that make it possible and permissible for persons of the same sex to instantiate it.

\section{Double Effect}
Consider these two cases, where all the people other than the dictator are assumed to be innocent.

\ditem{trolley}{\textsc{Trolley:} A runaway trolley is heading for a fork in the tracks. If nothing is done, it will
    turn left and kill five people who are on the track. You can redirect the trolley onto the right track, 
    where there is one person on the track, who will be killed by the trolley.}
    
\ditem{dictator}{\textsc{Dictator:} A dictator tells you to kill one person. If you fail to do so, the dictator will kill
    five others.}
    
Assume there are no further relevant consequences.  Deontologists tend to have the intuition that redirecting 
the trolley is permissible but obeying the dictator's orders is murder??refs, while consequentialists assume 
that both are permissible, and indeed obligatory. Let us assume that deontologists are right.

The usual explanation of the difference is that in \textsc{Trolley} one merely foresees, without intending, the 
harm to the person on the right track if one redirects, while in \textsc{Dictator} one intends the death of the person
if one obeys the order, even if one does so merely as a means to saving the five. 

Often, this difference is formalized into a Principle of Double Effect, which says that an action with a foreseen
evil effect and an intended good effect is not ruled out simply on account of the evil effect just in case the evil effect is 
intended neither as an end nor as a means, and the bad consequences are not disproportionate to the good ones.

But now consider the following variant:

\ditem{gunshot}{\textsc{Gunshot:} If you do not cause the death of one person, the dictator will kill five others.
    The one person is tied up, and a loaded gun is permanently affixed pointing at that person. The dictator does not
    care about your intention. You are curious what a gunshot close-up sounds like, so you consider pulling the trigger.}
    
In \textsc{Gunshot}, if you pull the trigger out of curiosity, you don't intend to kill the one. Moreover, the consequences
on balance are slightly better than in \textsc{Trolley}: five are saved and one dies, but also your curiosity about gunshots is
satisfied. But if we allow \textsc{Gunshot}, then with a bit of cleverness one can come up with a way of non-intentionally
killing the one person in \textsc{Dictator}. Perhaps one is curious what size of hole a bullet makes in a shirt pocket. 

How can one rule out the non-intentional killing in \textsc{Gunshot} without allowing one to accede to the dictator's demand? 

Here are four lines of response. The first is that proportionality needs to hold not just between the good consequences and
the bad ones, but between the \textit{intended} good consequences and the foreseen bad consequences. In \textsc{Gunshot},
while the foreseen goods are proportionate to the death of the person at the other end of the gun, the the intended good
of the satisfaction of curiosity is not proportionate to that evil. Effectively, this approach forbids one from counting
goods that are caused by the foreseen evil towards proportionality, since if one were to intend these goods, one would be
intending the evil as a means to them.

But there are other cases where counting goods causally downstream of evils towards proportionality seems exactly right. 

\ditem{bear}{\textsc{Bear:} You have a leg caught in a trap. If you open the trap now, you can save the leg. If you wait
    to open the trap later, your leg will become infected and will need to be amputated. However, nearby two other people 
    are trapped in a way that they cannot escape (even with your help) until rescue comes tomorrow. A hungry bear is about 
    to eat one of them, Alice. You foresee that if you open your leg-trap now, the noise of the opening will distract the 
    bear from Alice so that it will eat Bob instead. Once the bear has eaten one person, it won't eat until after rescue comes.}
    
The problem here is that a foreseeable consequence of your opening your trap is Bob's getting eaten, and saving a leg is not
proportionate to Bob's death. Now, it is true that Bob's getting eaten keeps Alice from getting eaten. But if we do not count
goods causally downstream of evils towards proportionality, then we cannot count Alice's being saved among the goods in 
our proportionality calculation, and all we get to compare is your leg and Bob's life, and hence there is no proportionality.
But, intuitively, it is permissible to escape the trap even if this leads to a different person getting eaten.

A second line of response is that intentionally firing a gun pointed in someone's direction is very ``close'' to intentionally
killing them, and we should forbid not only intentionally killing innocent people, but also actions that are very close to such
killing.??refs But it is difficult to see why intentionally firing a gun that happens to be pointed in someone's direction should 
count as too close to killing, but redirecting a trolley at someone does not. After all, nothing of moral significance should
hang on the means by which one redirects the trolley. Now imagine that the trolley is heading by inertia towards the left 
track, but a carefully placed barrel of gunpowder near the junction can push the trolley in the direciton of the right track.
And why should it matter whether it is a trolley or a bullet that is being powered by the gunpowder? (We can imagine the
trolley is shaped like a bullet train!) 

That said, there is something intuitive about closeness theories. However, on such theories we have a seemingly arbitrary 
parameter defining the forbidden degrees of closeness to intentional killing, and we have a Mersenne question about what
grounds the parameter's value.

??ref(Koons) have, however, offered a view of Double Effect on which it \textit{does} matter whether we are dealing with 
a trolley or a bullet, and where Mersenne questions are less apparent. The reason is that trolleys and bullets are artifacts with different socially-assigned ends, and when 
one employs an artifact according to its usual operating instructions, the usual ends count morally. The end of a bullet is
killing, and when one propels a bullet in that direction, we might say that one's action counts as close enough to intentional 
killing.\footnote{??refs taken literally claim that one \textit{intends} killing in such a case. This seems wrong. After all,
after the bullet lands, one can rationally, and with no change of one's ends, try to save the person hit by it. 
But it is not rational to act against one's former ends if one has not repudiated them. But one does not need the implausible
claim about intention: all one needs is that the action is close enough to intentional killing for moral purposes.} One
difficulty with this view is that it claims a perhaps implausible moral difference between firing a gun and detonating explosives 
packed in a steel pipe sealed at one end and with a ball-bearing inserted on the side of the explosive facing the opening,
when both are pointed at the same innocent person.  

Additionally, the ??ref account does not entirely escape Mersenne questions. An artifact has a particular
end and particular set of operating instructions in virtue of patterns of social behavior. However, there are infinitely many functions 
from social behavior to ends and operating instructions that roughly match our intuitions but disagree on edge cases, such as exactly
how often must forks be used to scratch one's back for them to acquire the end of scratching the back. Again, we need an explanation
of why this function rather than another defines the artifact for moral purposes.

A third line of response depends on contingencies of psychology. It seems unlikely that one could fire the gun pointed at Bob's heart
without intending Bob's death, unless one has done something to protect Bob (say, putting a shield between the gun and Bob). 
But note that a really callous person who does not care about human life probably \textit{could} fire a gun that is 
pointed at Bob simply in order to hear the gunshot. Our actions have
plenty of consequences we know about but that we are completely indifferent to, and hence do not intend. I know that moving
my fingers to type this sentence disturbs air molecules. But I have no intention to disturb air molecules. If it turned out
that due to some random quantum oddity the air molecules were unmoved by my fingers, my prediction that I will disturb the
air would be falsified, but my action would not be unsuccessful in any way. But an action that fails to achieve one of its
ends is at least partly unsuccessful. 

We might, however, make an Aristotelian move here. Perhaps it is abnormal to fire a gun pointed at Bob's heart without intending
death, and we are morally responsible for the \textit{normal} ends of an action in addition to its actual ends. And perhaps
our human nature defines what ends an action \textit{should} have, given the agent's knowledge of the circumstances.
This is a version of the closeness view, perhaps similar to the ??ref:Koons version. 

A fourth response notes that there is something callous about aiming at a minor end when there are evident great evils at stake.
Suppose a variant of the original trolley problem where there are equal numbers of people on each track, but when you press the
button to redirect the trolley, a candy pops out. It is callous to redirect the trolley for the sake of the candy. Similarly,
it is callous to fire a shot in the direction of a person in order to see the hole the bullet makes in their clothing. The intended 
end of the action is absurdly small in comparison to the foreseen harm. The only way for the action not to be callous would be if 
one intended to save the five other people as in \textsc{Dictator}, but if one intended to save the five other people, one would 
have to intend the means to that, namely the death of the one at whom the gun is pointed, and I have assumed that it is impermissible
to do that. 

To expand on the fourth response, imagine that Carl is the agent in \textsc{Trolley}, but Carl is someone who does not have much
in the way of moral feelings, though he does intellectually desire to the right thing. Carl also enjoys redirecting trolleys, 
though like most people he rarely has a chance to do so. So when he finds himself in \textsc{Trolley}, he rejoices, and he
redirects the trolley solely for the fun of it, while reasoning that by Double Effect he is acting permissibly, because the
death of the one person on the left track is not a means to the pleasure of redirection, and proportionality holds, because
on balance the effects are good. 

What is wrong with Carl's action? It seems that he is being callous: the one person on the left track dies for Carl's pleasure.
If Carl were intending to save the five on the right track, in the way that we normally expect someone in a trolley case to do, 
there would be no callousness. This suggests that in checking proportionality, we need to compare the \textit{intended} good 
against the \textit{foreseen} evils. 

We could simply specify that the intended goods are proportionate to the foreseen evils. However, \textsc{Bear} points away 
from that. In \textsc{Bear}, your intended good, the saving of the leg, is not proportionate to the foreseen evil of Bob's 
death. Moreover, let's modify the case of Carl and the trolley. Suppose instead that the agent is Dave, and he is pinned down
in such a way that his leg lies across the left track, behind the fork, and yet he can redirect the trolley. Dave is terrified
of his leg getting cut off and is about to redirect the trolley when he sees that there is a person on the right track. Despite
his fear, he now thinks it is wrong to redirect. But then he notices that on the left track there are five people. At this point,
he concludes that it is permissible to redirect because the overall consequences are positive. But his end in redirecting the
trolley is simply to save his leg. The overall consequences function in Bob's reasoning as a defeater to the 
observation that that redirecting the trolley will result in the death of the person on the right track, rather than as 
an end. 

While Carl is callous and acting wrongly in having a person die as a side-effect of a trivial end, Dave is acting for a quite
serious end. It would be better if Dave adopted the lives of the five people on the track as an end as well, but given his
terror at the amputation, it is not reasonable to \textit{require} that (though it would be wrong for him to redirect the
trolley if the only other relevant effect he saw was the death of the person on the right track).

Our fourth response thus suggests a conjunctive proportionality condition in the Principle of Double Effect:
 \ditem{prop2}{First, the intended good is not trivial in comparison to the foreseen evil, and, second, the foreseen goods
    are proportionate to the foreseen evils.}

Now, the triviality condition clearly involves an apparently arbitrary parameter measuring relative triviality, and raises serious 
Mersenne questions. 

We have seen that of the available solutions to the problem presented by \textsc{Gunshot}, the two most tenable ones involve 
parameters calibrating closeness or triviality, and hence raise Mersenne questions.

Additionally, we should think a little about the condition that the goods---whether specifies as foreseen or as intended---are 
proportionate to the foreseen evils. To a first approximation, one might opt for a utilitarian analysis: the overall
utility is positive. We have already seen serious problems with this in ??backref in regard to uncertainty, expected value
and risk, as well as incommensurability. Further, there is probably a significant overlap between deontology and a friendliness
to partiality in ethics. Imagine a trolley problem where on the right track there is a stranger and a cat and on the left track
there is one's child. The utilitarian consequences of redirection are negative, but the fact that the person on the left track
is one's child seems to make it permissible (some will even say required) to redirect. 

But we have also seen that taking relationships into account has many apparently arbitrary free parameters. 

???intentions and Thomson?


\section{The task of medicine}
The realism about teleology and normalcy provided by the Aristotelian framework
allows for an elegant solution to the problem of what the task of medicine is.??ref:Lennox

The medical professional is a \textit{professional}. Of course, everyone should refuse to act immorally on behalf of a client. But a professional has norms
and pursues goals that go beyond general morality, and has reason to refuse to further the client's aims even when there is nothing generally immoral about
these aims but the aims nonetheless violate the professional goals. Thus, while it is not immoral to create kitsch, a professional artist nonetheless has
to refuse a commission that would be unavoidably kitschy.

In the case of some professions, the goals are very much socially defined, and apart from legal minutiae, the delineation of these goals is of relatively
small importance. For instance, we have at least three professions that deal with the directing of water: gutter installers, sewer maintainers and plumbers.
All three professions are important, but the division of labor between them is not of great importance. It would do little harm to society if we had a single
profession for all three tasks, or if we divided up the tasks in some other way, say in terms of dealing with potable and non-potable water, or incoming
and outgoing water relative to a house.

However, the division of labor between the medical professions and other professions does seem to cut nature at its joints. The medical professions directly
aim at the goods of bodily health, a very natural subdivision of the space of human goods.

Moreover, there is a special value in the medical professional
having a very sharp focus on health. Medical considerations are of great importance to everyone's life. But in
the end, the patient (or their representative; I will simplify by talking just of patients) needs to be able to make a prudent decision about the recommendations
from a medical professional, weighing this recommendation against non-medical considerations such as ones of economics, interpersonal relationships, personal
pleasure and convenience, and so on. The patient is typically not an expert in biological matters, but tends to have a good grasp of other relevant goods:
for instance, they will know what effect giving up alcohol would have on their social life, or what goods their children would have to give up if a medical
procedure is to be paid for. It is important, however, that a medical recommendation be primarily concerned with the good of the patient's health, so that
the weighing between medical goals and other goods be delegated to the patient as much as possible, and that the non-medical goods not be double-counted (once by the
medical professional and again by the patient) in figuring out the prudent course of action.

At the same time, it is also important for guiding patients to prudent
decisions that medical professionals understand health holistically, rather than narrowly thinking only of the kidneys or the feet. Thus, a focus on health in general
is important for the medical professional, or at least the medical professional who has an advising relationship with a patient.

But what is health? Health is not \textit{simply} the good of the body. There are many goods of the body besides health, such as
athletic prowess, beauty, and reproduction. These goods depend on health, but are not a part of health: for instance, a relatively healthy reproductive system is needed for
reproduction, but one can have such a system without using it.

Apparently, physicians see their task as the return of the body to normal function, and then further claim to understand normalcy in a statistical way, as average
function.??ref Tying health to normal function seems quite plausible indeed. But the normal cannot be understood merely statistically. ??xref? If it were merely
statistical, then the adult who can deadlift 400 kilograms would be as abnormal
as the adult who cannot deadlift one kilogram. Rather, normalcy often has a directionality that it inherits from a teleology towards some good. Adults who
can deadlift 400 kilograms exceeds the norm, and might be said to be supernormal, but are not thereby abnormal, nor do they need medical treatment to reduce their
strength.??Vonnegut

Moreover, among our goods, it is perhaps health that is most clearly species-relative. As a result, an Aristotelian metaphysics of forms is perfectly fitted to
grounding the norms of health as the norms of sufficient capacity to function bodily in accordance with our human teleology.????more, better definition

\section{Our animal nature}
\subsection{The moral significance of our animal nature}
On the theory being defended, we have a will whose proper function defines the right for us. At the same time, 
we are biological organisms, and our will plays a certain functional role in driving our activities and organizing our 
lives. Among those higher animals, say cats, that are not agents, that functional role is still filled by 
\textit{something}---some kind of activity driver which we may or may not wish to call a will. It is plausible when 
we reflect on these higher animals that their activity driver will be malfunctioning if they are not driven to live the kind
of animal lives that are proper to their kind. By analogy, it is plausible that if we are not driven to live the kind of animal lives
that are proper to our kind, our activity driver is malfunctioning. But our activity driver is the will, and on the Natural Law
metaethics I am defending, right action is defined by the proper functioning of the will. Thus the argument from analogy combined
with the metaethics yields a case that we \textit{ought} to pursue lives proper to the kinds of animals we are. Indeed, we expect
a certain degree of harmony between our animal lives and our lives as agents.

This harmony has some plausible implications for environmental ethics and our relationship to other animals. 

\subsection{Living naturally}
\subsubsection{Transhumanism}
Various animal activities, such as breathing, drinking, eating and walking are part of a flourishing life for the kind of mammal 
that we are. To the extent that we should be seeking our own flourishing, we have reason to pursue animal activities. However,
this may not yield a very strong norm of pursuit of such activities for two reasons. First, our own flourishing does not appear
to be the only thing the norms of our will call us to---the flourishing of others seems just as important.??add-earlier-to-egoism-discussion
Second, flourishing as animals is likely only a small part of our flourishing---the bulk of our flourishing consists in specifically
human forms of flourishing such as understanding and love. 

However, in addition to seeing the animal activities as part of a flourishing life, there is another consideration in favor of 
living a life consistent with our animal nature. Aristotelian optimism gives us reason to think there is a harmony between 
the norms of the will and the norms of the rest of a rational animal's organism, and so a rational animal typically ought to act in 
accord with, and not contrary to, its broader animal nature. 

This rules out more radical forms of transhumanism. It would not accord with our broader animal nature to upload ourselves to a computer,
even if (which is dubious) we could survive such an upload. The virtual life would not include 
activities such as breathing, drinking, eating and walking that are a part of a natural human life.

Exactly how much modification of the human body is compatible with the norms of our will is not immediately clear. But the above
lines of thought suggest a significant degree of caution. Again, this is an area where we expect many ethical parameters specifying
permissible and impermissible modifications.

??what are some relevant considerations?

\subsubsection{Ecology}
It is in the nature of any organism to interact in certain ways with its natural environment. While that interaction may 
involve out-competing some other organisms, it is plausible to think that the natural behavior of an organism tends to 
harmonize with its environment to a significant degree (from an evolutionary point of view, organisms and their environment tend 
to co-evolve). Given a harmony between the norms of the will and the norms of the rest of the rational animal, we would expect 
a rational animal to have norms that aim it at a certain degree of integration with a natural environment. These norms will then
constitute moral norms to harmonize with the environment.

It could be that the reason why an intelligent organism has such norms is because harm to the environment harms the organism.
But even if so, it is likely that the norms are not simply derivative from the organism's need to do well in its own 
niche. Imagine that biological warfare has wiped out all humans beyond you and a handful of friends, leaving the ecosystem 
intact but lethal to humans. You prepare to flee earth for another solar system, expecting humanity never to return. 
The spaceship has a good library, but the last novel in a series of fantasy novels that you like has not been released.
Instead, it is stored in a computer system so set up by a crazy copyright owner that if you download the novel, all the 
nuclear weapons on earth will be launched. Your safety is assured if you wait to download the novel once your spaceship has 
left the earth's atmosphere, but the ecosystem will never recover from the damage then. While under more normal circumstances,
severe damage to the ecosystem has evident repercussions for humans, in this case it does not (except maybe for psychological
ones---but we can suppose that these are outweighed by the delights of the novel). 

The evident wrongfulness of destroying the ecosystem for the sake of one enjoyable novel is not simply grounded in our duties to 
protect our species. It may, nonetheless, be \textit{explained} by the need to protect our species. For it might be better for 
us to have coarser-grained duties to the environment, that forbid large-scale destruction for the sake of minor goods, than to 
have the finer-grained duties that would allow such destruction in extremely rare circumstances. After all, we know that it is 
all too easy to fool ourselves as to where the boundaries of the ``rare'' circumstances lie. The rule utilitarian has room for 
such an account, but so does an Aristotelian who thinks that there is an inner harmony in our nature such that the various 
aspects of our nature are good for us in other respects. This pattern of explanation will be discussed further in Section~\ref{sec:moral-explanation}
of Chapter~\ref{ch:X}.

\section{The definition of life}??move??
Here is an intuition that until fairly recently would have been widely shared: There are deep metaphysical divides between non-living and living things,
and between merely living things and persons, and these divides mark a hierarchy of value, a chain of being. If we could defend such a divide, it would
dovetail with the idea that persons are in an important way \textit{sacred}, having rights while other things have mere interests, if that.

I want to offer a highly speculative Aristotelian reconstruction of this intuition. To introduce the reconstruction, start with a puzzle for
Aristotelian views. It seems that on such views:
\ditem{2-pursue-perfect}{Each thing naturally strives for its own perfections.}
\ditem{2-natural-activity}{The natural activity of a thing is a perfection of it.}
But this generates a regress. Let's say that reproduction is an oak tree's perfection. Then by \dref{2-pursue-perfect}, the oak tree naturally strives for
reproduction. This natural activity of striving for reproduction, by \dref{2-natural-activity}, is then itself a perfection of the oak tree. Therefore,
by \dref{2-pursue-perfect}, the oak tree must naturally strive for it: hence the oak tree naturally strives for striving for reproduction. And so on,
\textit{ad infinitum}. But surely an oak tree does not pursue infinitely many things. And even after a few level of meta-striving we exhaust plausibility.

I suggest that we can deny \dref{2-pursue-perfect}. Some perfections of a thing are not actually naturally striven for by the thing.\footnote{An interesting
theological example may be the idea in the Thomistic tradition that both the beatific vision and our striving for it are gifts of God's grace, rather than
natural for us, even though the beatific vision perfects us.??} The oak tree does strive for reproduction with its reproductive organs. Moreover, it has a
second order striving: it strives to strive for reproduction, by growing the reproductive organs with which it strives for reproduction. There may be one or
two more meta-levels, but at some level we can say: it just does this, without striving to do it.

Non-living things, on an Aristotelian metaphysics, also have form and also strive for ends. But plausibly they don't strive to strive: they just strive.
We thus have a hierarchical division between inorganic things which do not strive to strive and living things which have second order teleological strivings.

The problem of the definition of life is a thorny conceptual problem in biology or its philosophy. Different authors give different lists of features such
as homeostasis, growth and reproduction as part of the definition of life. The multiplicity of features listed makes the concept of life seem arbitrary.
Moreover, it is philosophically problematic to tie the the concept of life too tightly to the physical forms of life around us. For it is very plausible
that if there are immaterial agents such as deities, spirits or angels, they should also count as alive.\footnote{It is worth noting that not everyone who
believes in deities, spirits or angels believes them to be immaterial. The ancient Greeks did not think their deities immaterial. And a minority opinion
among Christian theologians held angels to be made of ``subtle matter''.??ref But the argument only needs that some do believe them to be immaterial.}
 After all, those who believe in such beings sometimes
hold them to be immortal. But if they were not alive, their immortality would be a trivial claim: a being that is not alive in the first place cannot die.
However, these beings are conceptualized as alive, even when they cannot engage in homeostasis, growth or reproduction. And yet while a particular existence
claim about the existence of immortal immaterial agents might be false, it does not seem to be fundamentally conceptually confused. Thus, a good account
of life should include the kind of life that is attributed to immaterial agents, and none??check of the accounts in the philosophy of biology do that.

Furthermore, it is a merit of a definition that when applied to cases where we do not know how to classify a thing, the definition does not trivially
decide the issue, but it points to the question we need to answer if we are to decide the issue. To that end, consider two borderline cases: viruses
and sophisticated robots, like Star Trek's Data. In neither case are we confident whether we have life. Viruses are famously a borderline case.
And while Data is described as a ``synthetic life-form''??ref, and the Star Trek canon clearly favors his being actually alive, the question is
not so philosophically clear. Data obviously fails typical biological definitions of life: while he engages in self-maintenance, he doesn't grow or
reproduce in the biological sense of the word (though he does make other androids), in a way that does not match typical viewers' intuitions.\footnote{Though,
admittedly, there may be some static due to the show confusing the question of consciousness with that of life.??check} And
whether a virus qualifies as alive varies from definition
to definition??ref in a way that makes it sound like the question of viruses being alive is merely verbal. Yet given the strong intuition that there
is something of great value about life, even something sacred, the question of what is and is not alive should not be merely verbal.

On the other hand, an account on which what it is to be alive is to have a second order teleological striving---to strive to strive for a perfection---will nicely
include any immaterial agents. It will include any entity that prepares itself for future teleological activity, say by growth,
and hence will include all the physical forms of life we know about. It will exclude elementary particles. And whether it includes viruses or sophisticated
robots is unclear---as it should be. For it is unclear whether viruses and sophisticated robots have form at all. If viruses have form, then it is likely
that their activity of attaching to hosts for purposes of future replication is a striving for replicative striving, and hence they are alive. But it is
not clear whether they have form. If sophisticated robots have form, they also exhibit meta-striving, and hence are alive. But in both cases we do not
know whether there is form, or whether we are dealing with a mere agglomeration of particles.
Aristotle himself seems to have thought that
artifacts only had form in the analogical sense of a blueprint in the mind of the designer??ref, but he could have been wrong in the case of artifacts like Data.
(For more on the epistemic issues here, see Section~\ref{sec:epist-of-form}
in Chapter~\ref{ch:God}.)

We thus have two levels in a chain of being: things that strive but don't meta-strive, and things that meta-strive. Now, among the things that meta-strive,
we can describe a higher kind of thing: a thing that strives for all of its perfections. The premises of the regress argument with which we started this
section apply to such a being. Thus, this is a being that strives for striving for ... for perfection, at any number of levels. While this is implausible
for an oak tree or even a dog, we do actually know of one kind of being that does that: humans. Human beings not only conceptualize particular perfections, such as friendship
or striving for striving for health, but they conceptual perfection as such, and strive for it as such. If a trustworthy being offered you to increase
some perfection or other, and assured you that you would in no way be harmed, it would be rational for you to accept the offer, because perfection as
such is one of the things you and I pursue.

At the same time, in a minded being, the infinite chain that results from striving for all one's perfections need not be a chain of separate desires
and hence does not require a being that is actually infinite. Rather, all that's needed is for the being to be such that it has or teleologically strives
to have the concept of a perfection as such and a desire for perfection as such. This desire then can manifest in a striving to figure out what the perfections are---a striving that is central to
the search for happiness (\textit{eudaimonia}) that was so characteristic of Socratic and post-Socratic Greek philosophy---and a striving to be ready
to accept whatever one finds. In fact, it might be that for reasons having to do with the nature of infinity \textit{only} a minded being can pursue
an infinite number of ends---for any non-minded being that did that would need to have infinitely many distinct causal sources of its activity in
a way that might well violate causal finitism, the thesis that it is impossible for an infinite number of causes to work together (for a defense
of causal finitism, see ??ref). And among minded beings, perhaps it is definitive of \textit{persons} that they pursue all good.

We thus have a qualitative hierarchy of being between the mere strivers, the mere meta-strivers and the universal strivers. The first division in
the hierarchy may well correspond to that between the non-living and the living, and the second might---depending on speculative questions about
infinity---align with the division between mere life and personhood. And it is very natural to see qualitative divisions of value here as well.

\section{Infinity}
We saw in ??backref that population ethics raises Mersenne questions. But \textit{infinite} population 
ethics not only raises questions, but creates serious paradoxes. For instance, suppose there is an 
infinite line stretching to infinity both to the left and the right, with tickmarks every meter 
labeled by an integer (bigger numbers being to the right), and one person standing at each tickmark. 
All the people are on par. Suppose you now have two 
choices:
\ditem{benefit-even}{Benefit the people at $2,4,6,...$}
\ditem{benefit-odd-1}{Benefit the people at $1,3,5,...$}
where all the benefits are the same.

Intuitively, we should be indifferent between these. It makes no difference whether we should benefit the 
people at the positive even- or positive odd-numbered locations. The options are on par. 

But now add a new option:
\ditem{benefit-odd-3}{Benefit the people at $3,5,7,...$}
with the very same benefits. 
Observe now that \dref{benefit-odd-3} benefits the people standing immediately to the right of 
the beneficiaries of \dref{benefit-even}, while \dref{benefit-even} benefits the people standing 
immedaitely to the right of the beneficiaries of \dref{benefit-odd}. Thus,
the moral relationship between \dref{benefit-odd-3} and \dref{benefit-even} should be the same as that
between \dref{benefit-even} and \dref{benefit-odd}. 
But the latter two, as already noted, are intuitively on par. 
Thus, likewise, \dref{benefit-odd-3} and \dref{benefit-even} are on par. 

We can argue for the parity of \dref{benefit-odd-3} and \dref{benefit-even} as follows. If we re-label 
tickmark $n$ as $(n-1)^*$, then options \dref{benefit-even} and \dref{benefit-odd-3} are equivalent to:
\begin{itemize}
	\item[\ref{benefit-even}$^*$]{Benefit the people at $1^*,3^*,5^*,...$.}
	\item[\ref{benefit-odd-3}$^*$]{Benefit the people at $2^*,4^*,6^*,...$.}
\end{itemize}
If benefiting those at positive odd-numbered locations is on par with benefiting those at positive 
even-numbered locations, then surely this should not depend on whether we used the original or 
the asterisked numbering. Thus (\ref{benefit-even}$^*$) and (\ref{benefit-odd-3}$^*$) are on par.
But they are logically equivalent to \dref{benefit-even} and \dref{benefit-odd-3} respectively, so these
are on par as well.

But being on par morally is transitive. So, if \dref{benefit-odd-3} and \dref{benefit-even} are on par,
and \dref{benefit-even} and \dref{benefit-odd} are on par, it follows that \dref{benefit-odd-3}
and \dref{benefit-odd-1} are on par. But that conclusion is clearly false, since if we can benefit a 
person without anybody else losing anything, we have moral reason to do so barring some deontological consideration,
and if we are set to do \dref{benefit-odd-3} then switching to \dref{benefit-odd-1} benefits the holder of 
ticket~$1$ without anybody losing anything.

Perhaps, however, we should deny that \dref{benefit-even} and \dref{benefit-odd-1} are on par. There are two
ways of doing that. One is to say that the two cases are incomparable. The other is to say that \dref{benefit-odd-1}
is better than \dref{benefit-even}.\footnote{Saying that \dref{benefit-even} is superior to 
\dref{benefit-odd-1} is not tenable. The relationship of \dref{benefit-even} to \dref{benefit-odd-1} is the 
same as that of \dref{benefit-odd-3} to \dref{benefit-even}, and if we say that \dref{benefit-odd-3} is superior
to \dref{benefit-even}, we will then by transitivity have to say that \dref{benefit-odd-3} is superior to 
\dref{benefit-odd-1}, which is absurd.}

Neither option is particularly appealing. Suppose that the benefit is the saving of a life. Then if
\dref{benefit-odd-1} is better than \dref{benefit-even}, then by the above reasoning \dref{benefit-even}
will be better than \dref{benefit-odd-3}. So we will have this preference ordering:
\ditem{benefit-order}{\dref{benefit-odd-1} $>$ \dref{benefit-even} $>$ \dref{benefit-odd-3}.}
Now \dref{benefit-odd-1} is better than \dref{benefit-odd-3} by exactly one life saved. So it seems that
\dref{benefit-odd-1} will have to be better than \dref{benefit-even} by less than saving a life---presumably,
by half a life-saving---and \dref{benefit-even} will have to be better than \dref{benefit-odd-3} by less than saving 
a life---again, presumably by half a life-saving. But this is very implausible. When the scenarios differ in 
whose lives are saved, and there are no probabilities involved, surely any two scenarios that differ must do 
so by one or more lives. 

On the other hand, suppose that we have incomparability between \dref{benefit-even} and \dref{benefit-odd-1}.
Then it seems difficult to see how we will get moral guidance as to which we should perform when one of them
is a little bit more costly than the other. 

Here is another way to see the issue. Let $L_n$ be the action of benefiting all the people to the left of 
position $n$, and let $R_n$ be the actions of benefiting all the people to the right of position $n$, and 
for simplicitly consider only such left- and right-benefit actions. Write $A\le B$ to say that action $B$ 
is at least as good morally as action $A$, and $A<B$ to say that $A\le B$ but not $B\le A$. Say that $A$ and 
$B$ are comparable provided that $A\le B$ or $B\le A$. Here are some assumptions about the 
moral preferability relation:
\ditem{pref-trans}{Transitivity: If $A\le B$ and $B\le C$ then $A\le C$.}
\ditem{pref-mono}{Strict monotonicity: For any $m$, we have $L_m < L_{m+1}$ and $R_m < R_{m-1}$.}
\ditem{pref-inv}{Weak translation invariance: For any $m$ and $n$, we have $L_m \le R_n$ if 
and only if $L_{m+1} \le R_{n+1}$, and $L_m \ge R_n$ if and only if $L_{m+1} \ge R_{n+1}$.}
Transitivity is very plausible. Next, by switching from $L_m$ to $L_{m+1}$ or from $R_m$ to $R_{m-1}$, 
one benefits the person at location $m$, without taking benefits away from anyone, and this is surely 
better, thereby yielding strict monotonicity. Finally, weak translation invariance is based on the observation 
that the relationship between $L_m$ and $R_n$ is exactly the same as that between $L_{m+1}$ and $R_{n+1}$.\footnote{Strong
translation invariance would be the thesis that all the $L_m$ are morally equivalent and that all the $R_m$ are 
morally equivalent, since the $L_m$ are all translations of one another and the $R_m$ are all translations of 
one another. But strong translation invariance would be incompatible with strict monotonicity. For a discussion 
of weak and strong invariance conditions, see ??Pruss:nonclassical.}

In Appendix??forwardref, I prove that given \dref{pref-trans}, \dref{pref-mono} and \dref{pref-inv}, exactly
one of the following conditions holds:
\ditem{pref-incompar}{For all $n$ and $m$, actions $L_n$ and $R_m$ are incomparable with each other.}
\ditem{pref-right}{For all $n$ and $m$, we have $L_n<R_m$.}
\ditem{pref-left}{For all $n$ and $m$, we have $L_n>R_m$.}
In other words, we either have complete incomparability between any left- and any right-benefit action, or else
we have a radical skew where all the right-benefit actions beat all the left-benefit actions or all the 
left-benefit actions beat all the right-benefit actions. 

We could, perhaps, imagine a species whose morality exhibits the radical directional preference 
of \dref{pref-right} or \dref{pref-left}. Perhaps this would be a species that lives in a spacetime 
without the symmetries that our spacetime exhibits. But we are not that species. This kind of radical 
skew seems deeply implausible, and so it seems we need to have the radical incomparability of 
\dref{pref-incompar}. 

But the radical incomparability of option \dref{pref-incompar} is also somewhat implausible. It seems that, say, saving the lives of the
people to the left of zero and saving the lives of the people to the right of zero are on par, in such a 
way that if it is significantly more costly to save the people on one side of zero than 
on the other, one has significant reason to do that. For instance, suppose that there is a button
that saves the lives of the people to the left of zero and a button that saves the lives of the people
to the right of zero. However, the button that saves the lives of the people to the right of zero will 
have a side-effect of causing a severe migraine to Alice, a stranger is not one of the people on the line. It 
seems very intuitive that one should press the button that saves the people to the left of zero. 

But on the incomparability view, it does not follow that we should press the button that saves the 
people to the left of zero. For $R_0$-plus-migraine-for-Alice clearly beats $R_{1000}$, since 
$R_{0}$ saves a thousand additional people that are not saved by $R_{1000}$ (namely the people at 
locations $1$ through $1000$), and a side-effect of a migraine to a stranger is definitely worth 
tolerating to save a thousand lives. But $R_{1000}$ is not worse than $L_0$ by \dref{pref-incompar},
and and hence $R_0$-plus-migraine-for-Alice cannot be worse than $L_0$. So the incomparability view 
undercuts a plausible judgment about avoiding side-effects.

% Imagine that the benefit in question
% is saving the person's life, and by pressing one of three buttons,
% respectively labeled $L_{-1000}$, $L_0$, and $R_0$, you confer the indicated benefit. 
% You can only press one of the four buttons. Let's add that the buttons have a satisfying tactile feel, 
% so you would want to press one of the buttons even if you didn't have any other reason for it. 

% It would be extremely wicked to refuse to press a button: an infinite number of additional people would die if you 
% did not press the button, and not only is pressing the button cost-free, it is itself a minor benefit to you.
% Next, note that it is quite wicked to press the $L_{-1000}$ button. For if you select $L_{-1000}$, you could 
% have with no additional effort selected $L_0$ instead and saved the people at locations $-1000$ through $-1$.
% In a selection between finitely many options, when you could have chosen an option which saved a thousand 
% additional lives at no cost and you didn't, you acted wickedly. The qualifier that there are only
% finitely many options rules out scenarios where for every option there is another one that saves a thousand 
% more lives. That would be the case if you were, say, given a choice between all the infinitely $L_n$, since
% for any $L_n$ you chose, $L_{n+1000}$ would save infinitely many lives.

% Now, $L_0$ and $R_0$ each have the property that they are not beaten by any other option available to you.
% Indeed, intuitively, pressing the $L_0$ or $R_0$ button is a good action. But now we have the following 
% paradoxical conclusion:
% \ditem{pref-paradox}{Pressing the $L_{-1000}$ button is wicked, pressing the $R_0$ button is good, but pressing 
	% the $R_0$ button is not morally better than pressing the $L_{-1000}$ button.}
% In other words, the comparative and absolute moral evaluations come apart in a very odd way. It seems obvious
% that a good action is always better than a wicked one!

% One might object that pressing the $L_{-1000}$ button is not wicked but good since on the whole it saves 
% infinitely many people. But given that a person who chose $L_{-1000}$ would be subject to the devastating
% moral critique that among the finite number of options if they had picked $L_0$ instead they would 
% have saved a thousand lives at no cost to themselves, it is hard not to call the person wicked. 

% In fact, we can modify the example to make the wickedness intuition clearer. Suppose that when you press
% $L_{-1000}$ you get a moderate electric shock, though no permanent harm, and you have to trudge ten 
% kilometers over muddy ground to get to the button, while the $L_0$ button has no side-effects and is nearby.
% To go to that effort to press the button that saves $1000$ fewer people than the button nearby surely is 
% wicked. (Imagine that one of the $1000$ fewer people saved by $L_{-1000}$ is someone you know. They would feel it's 
% very pointed for you to put in the effort.) 

% Now denote the $L_{-1000}$ option together with the shock and trek by $L'_{-1000}$. Assuming \dref{pref-incompar}, 
% we can still argue that $L'_{-1000}$ is incomparable with $R_0$. For $L'_{-1000} > L_{-2000}$, since
% $L'_{-1000}$ saves a thousand lives that $L_{-2000}$ does not, merely at the cost of some pain, time and mud. 
% On the other hand, obviously, $L'_{-1000} \le L_{-1000}$ (with equality if harm to self doesn't count for moral
% evaluation, and otherwise strict inequality).
% Suppose now that $L'_{-1000}$ is comparable with $R_0$. Then either $L'_{-1000} \le R_0$
% or $R_0 \le L'_{-1000}$ (or both). If $L'_{-1000} \le R_0$, then since $L_{-2000} < L'_{-1000}$, by transitivity
% we have $L_{-2000} < R_0$, which contradicts the assumed incomparability of the $L_n$ and $R_m$ in \dref{pref-incompar}.
% If $R_0 \le L'_{-1000}$, then since $L'_{-1000} \le L_{-1000}$, so $R_0 \le L_{-1000}$ by transitivity, once 
% again contradicting \dref{pref-incompar}.

% Hence we once again have a paradox if the options are $L'_{-1000}$, $L_0$ and $R_0$. Choosing $L'_{-1000}$ is wicked.
% Choosing $R_0$ is not wicked. But $R_0$ is not a better choice than $L'_{-1000}$.

It seems clear that something morally paradoxical happens in these kinds of infinite cases. But an 
Aristotelian has a neat way out. These kinds of choices are outside the human ecological niche. If 
morality were species-independent and necessary, morality would have to extend to such cases. But 
it is quite reasonable to suppose the human nature either does not contain principles that apply to 
such cases, or contains principles that do apply to such cases, but end up contradicting each other 
in those cases---with morality glitching (cf. ??forwardref to ch 10)---or end up applying to such 
cases but generating conclusions that don't fit with some of the moral intuitions built-into that 
nature.\footnote{It is worth noting that we cannot entirely escape the need to address such cases
by saying that they are outside our sphere of activity. For, adapting the Pascal's Mugger story, 
imagine you are approached by a strange person who tells you that she is a magician from another
universe where there are infinitely many people arranged a meter apart on a line, and they are all
drowning, and she can by a spell perform $L_{-1000}$, $L_0$ or $R_0$. She can't decide which to do 
and wants your advice. Obviously, you wouldn't believe her story. But if you are a good Bayesian,
you would assign it a non-zero probability, and the question would indeed become one of moral 
relevance. That said, it is not surprising if morality behaves strangely once you are in an odd 
epistemic state. What should you do, we might ask, if you come to think that dialethism is true and 
you should do the wrong thing? Or what should you do if you become convinced of solipsism or its 
opposite, alterism (the view that you don't exist but other people do). We should not be surprised if 
either there are no answers to such questions or the answers are strange.}

It should be noted that the paradoxes of infinity here only scratch the surface of the range of 
oddities imaginable.??refs

\section*{$^*$Appendix: Skew in benefiting infinitely many people} 
In ??backref, it was claimed that a preference ordering on certain actions 
that benefit an infinite number of people satisfying certain axioms either
suffers from massive incomparability or has a massive left-right bias.
That result follows from the following.

??number??
\begin{theorem} Let $L_n$ be the set of integers less than $n$
and $R_n$ the set of integers greater than $n$. Let $\scr A$ be the set of all
the $L_n$ and $R_n$. Suppose $\le$ is a transitive relation on $\scr A$ such that 
$A<B$ whenever $A\subset B$ and $n+A\le n+B$ if and only if $A\le B$ for any integer
$n$, for all $A$ and $B$ in $\scr A$. Then exactly one of the following holds:
\begin{itemize}
\item[{(i)}] for all $m$ and $n$, we have neither $L_m\le R_n$ nor $R_n\le L_m$,
\item[{(ii)}] for all $m$ and $n$, we have $L_m<R_n$
\item[{(iii)}] for all $m$ and $n$, we have $R_n<L_m$.
\end{itemize}
\end{theorem}
Here, $A<B$ provided that $A\le B$ but not $B\le A$, and $n+A=\{n+m:m\in A\}$ is the 
translation of $A$ by $n$.

For we can identify the actions in ??backref with the sets of people benefited by them.
Then note that if $\le$ is transitive, so is $<$.\footnote{\label{fn:trans}More generally, if $A\le B\le C$, and 
at least  one of the inequalities is strict, it follows that $A<C$. For by transitivity of 
$\le$ we have $A\le C$, and if we don't have $A<C$, then it must be because $C\le A$. Then
$A\le B\le C\le A\le B$. Hence $B\le A$ and $C\le B$ by transitivity of $\le$, which contradicts
the claim that $A<B$ or $B<C$.}
The condition that $A<B$ whenever $A\subset B$ follows by induction and transitivity of $<$ 
from \dref{pref-mono} since the only way
$A\subset B$ can hold is if $A=L_m$ and $B=L_n$ with $m<n$ or $A=R_m$ and $B=L_n$ with $m>n$.
The translation invariance condition then follows by induction from \dref{pref-inv} since $n+L_m=L_{m+n}$
and $n+R_m=R_{m+n}$.

\begin{proof}[Proof of Theorem]
Suppose (i) does not hold, so for some $m$ and $n$ we have $L_m\le R_n$ or $R_n\le L_m$.

First suppose $L_m\ge R_n$. We will now prove (iii). 
We have two cases. First suppose $m<n$. Then 
$L_n>L_m\ge R_n$, so $L_n>R_n$.\footnote{See note~\ref{fn:trans}.}
By our translation invariance condition, we have $L_k>R_k$ for all $k$.
Now fix any $j$ and $k$. If $j\ge k$, then 
$L_k>R_k\ge R_j$ by monotonicity so $L_k>R_j$. if 
$j<k$, then $L_k>L_j>R_j$. So we have (iii).

Next suppose $R_n\ge L_m$. Let $-A = \{ -x : x \in A \}$.
Define $A \le^* B$ provided $-A \le -B$. It is easy 
to see that $\le^*$ also satisfies all of the 
assumptions of the Theorem. Moreover, since we have
$R_n\ge L_m$, we have $-R_n\ge^* -L_m$. But 
$-R_n = L_{-n}$ and $-L_m=L_{-m}$. Thus $L_{-n}\ge^* R_{-m}$.
Applying the previous paragraph to $\le^*$ with 
$-n$ and $-m$ in place of $m$ and $n$, we get, for 
all $j$ and $k$, the inequality $L_k>^*R_j$.
Hence $-L_k>-R_j$, and so $R_{-k}>R_{-j}$ for all
$j$ and $k$, which implies (ii).
\end{proof}

\chaptertail

