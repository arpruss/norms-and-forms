\def\mychapter{IV}
\chapter{Applied ethics}\label{ch:applied-ethics}
\section{Introduction}
Thinking that ethical duties are grounded in norms innate to human nature does not by itself logically entail answers to controversial 
questions of applied ethics. One can think that our nature requires us to kill those whose suffering we cannot stop,
and hence that euthanasia is required, one can think that our nature prohibits the killing of the innocent even if that killing
would be in their interest, and one can have an in-between view. 

But the nature-based approach provides at least two benefits for applied ethics. First, because of the Aristotelian harmony principle, it 
allows facts about our natural behaviors and needs as the kinds of organisms we are to to provide us with defeasible but often strong
evidence about what we should do. Second, it makes it more plausible than it would be on a number of competing theories that the answers 
to applied ethics questions might be irreducibly intricate---not reducible to a small number of simple principles---and might include 
domain-specific ethical rules for the various areas of our natural lives, such as family relationships or sexuality or (if it's natural)
property rights. 

We will thus explore some things that we can say on nature-based ethics. The plausibility of what we will say will serve as indirect
evidence for the underlying Aristotelian metaphysics.

\section{Natural relationships}
\subsection{Siblings and cousins}
An interesting test case for an ethical theory is whether it can make good sense of our duties to our siblings and cousins. Duties to friends and spouses plausibly
arise from commitments we make. Duties to parents have traditionally been grounded in our obligation of gratitude for our life. Duties to children can
typically be grounded in the decision to perform actions that have a non-negligible probability of producing a person dependent on us. Duties to strangers
might be grounded in our shared rationality. But we owe more to our siblings and cousins than we do to strangers, even though typically we had no say in whether
we were to have siblings and cousins, and even when we have no favors to return.

On utilitarianism, our duties to siblings and cousins come mainly from the contingent fact that we tend to be better positioned to do good to them, say because we know
their needs better, are likely to be physically closer, and help from us is likely to be more welcome. But if such contingencies are all that is involved, then we also
have to accept an error theory about our intuitions when they go beyond these contingencies. If a sibling or a stranger is drowning, other things being equal one should
try to rescue the sibling, even if the stranger is slightly easier to pull out, or is likely to have a slightly better future life. If one finds out that a local
homeless person is a cousin one has not seen since early childhood, it is more vicious to ignore their needs than to ignore similar need in a random stranger.
Murder of a stranger is evil, but fratricide is worse.

In general, utilitarianism, contractarianism and Kantianism focus on the agent's rationality, taking the details of the agent's humanity to provide no direct normative
input into ethical decisions. The fact that most humans hate eating mud gives one reason not to feed mud to them, and the fact that we are unable to instantly
teleport ensures we do not have the same obligation to those on other continents as to those nearby. But these are non-normative facts, and the normativity of the
conclusions here comes from general normative considerations applicable to all rational beings. There is some \textit{prima facie} plausibility to the idea
that the non-normative facts about the relationships between parents and children, together with normative facts applicable to all rational beings, could explain
distinctively filial and parental duties. But this is not plausible for the cases of siblings and cousins.

However, if we see ethics as based on the norms written into our \textit{human} nature, given a harmony between the rational and animal aspects of this humanity,
will very plausibly allow for distinctive ethical norms tied to particular kinds of natural human relationships, including perhaps in the first instance familial ones.
There is no need on our Natural Law ethics to derive the duties to cousins from non-normative facts about cousinhood and norms for all rational beings: such rules
can be fundamental. And the laws can, in principle, be at any level of precision, be it to simply consider one's siblings at a higher weight in one's moral calculus
than more distant relatives (we are all relatives, after all, as we learn from evolutionary theory) or to prefer one's siblings over one's cousins to such-and-such
a specific degree. The laws could even have social construction built in: they could require us to respect our relatives in the ways that our society prescribes,
and require us to establish societies that institute ways for us to respect our relatives.

Divine command theory has a similar advantage: God's commands can be at any level of generality or precision, be it to love one's neighbor or to telephone one's cousins
at least twice a year if one can. In principle, rule utilitarianism can do this as well: it is plausible that having rules concerning special relationships like
fraternal ones could maximize utility. But Natural Law arguably gives a better explanation of the duties tied to these special relationships. For the nature of these
special relationships is very plausibly tied to our humanity, and hence it makes sense that the special obligations attached to them should flow from that humanity
rather than the commands of a God or the results of an abstract hypothetical optimization procedure.

Indeed, on Aristotelian natural law, we can say that having these kinds of special obligations is an important aspect of what makes us human---for it is an important
aspect of our form, which is precisely what makes us human.

\subsection{Less natural relationships}
We have a broad variety of socially-instituted and culturally-variable relationships which are very unlikely to have norms encoded
for them in human nature.  In English-speaking countries the relationship to the  parent of one's godchild or the godparent of one's child
tends not to have sufficient importance to even have a name, while in other cultures it is important and specifically named. The relationship
between an employer and employee varies so broadly with legal and social structures that it is probably best seen as an umbrella for a number
of different relationships, none of which is likely to be encoded in human nature.

Admittedly, a relationship could fail to be culturally widespread and yet could have norms encoded for it in human nature, but there is a more elegant approach to
analyzing such relationship: we can see them as cultural determinations of a more fundamental relationship type, with some of the norms coming from human nature's
rules for the more fundamental relationship and others from the culture. Moreover, human nature may prescribe the scope for cultures to establish the rules.
Such relationships can be thought of as ``less natural''. At the same time, the difference between these relationships and the ``more natural'' ones like siblinghood
are likely to be largely of degree. For while there may be a fundamental normative relationship of siblinghood, it has further culturally-determined
norms.

\subsection{Marriage}
A particularly interesting question, of significant relevance to controversies in our society over the past century, is where \textit{marriage}
lies on the naturalness spectrum. I shall argue that it is likely to be quite natural, with a number of fundamental norms grounded in our human
nature by arguing against two main alternative theories and combinations of them.

The first theory holds that marriage is an institution defined by many human societies. Like other such social institutions, such as judgeship, parliament membership,
monarchical sovereignty, exchequer chancellorship, and presidency, it is defined by the rights and obligations conferred by society on those who enter into the institution.
While we use the same words ``judge'' and ``monarch'' across societies, there is only a family resemblance between the institutions these terms refer to, since the actual
rights and obligations defining the institutions are often very different indeed. The resemblance may be very weak: the rights and obligations of the monarch of England
in the 13th century are about as different from the rights and obligations of the current monarch as the rights and obligations of modern day judge are from a modern day
executioner. Nonetheless, for historical reasons we may use the same word ``monarch'', sometimes clarifying with adjectives like ``absolute'' or ``constitutional''.

The second theory has it that couples choose to undertake certain obligations with respect to each other, which obligations give rise to rights,
and this complex of rights and obligations defines the marriage. In more traditional societies, a couple may not choose the obligations specifically
but rather will simply opt for the ``customary'' obligations and their consequent rights. In modern Western society, many couples write their own
wedding vows, specifying general obligations. But even in those cases, it is likely that these vows are not typically thought of as a precise and exhaustive
legal contract, but rather as a way of customizing one of the prevalent packages of obligations. Again, on the individual theory, we use the same word
``marriage'' for all these different packages of rights of obligations due to some sort of vague family resemblance between them.

A more sophisticated theory??refs may combine aspects of the social and individual theories, holding that not only do couples undertake obligations to each other and gain
rights with respect to each other, they also undertake obligations to society and gain rights with respect to society.

But the individual and social theories are unsatisfactory for multiple reasons.

\subsubsection{Discovery}
People in good marriages come to discover new normative aspects to marriage as they go through life together. ??add-specifics?
If the norms of marriage were simply whatever it was that the parties to the marriage chose, there would be nothing to discover.
And if the norms of marriage were simply set by society, it would be odd to think that it is particularly by living the married life that
one discovers the norms. Rather, the norms would be discovered by study of the history of the social institution of marriage,
the laws surrounding it, the intentions of the legislators, and so on.

We might, admittedly, in individual and social institution cases discover
new normative facts by logical derivation from previously known ones, but that is not actually the primary way in which we learn about marriage:
we learn about it by observing it from the inside.
And we discover new facts, including normative ones, about natural kinds of entities precisely by observing these entities. By observing water,
we come to see that it is H$_2$O and by observing mammals, we come to see that their middle ear should have the malleus, incus and stapes bones??check. And
it is in our own case that we are best positioned to observe marriage at work, so it is unsurprising that such observation produces knowledge of normative
aspects of the relationship.

Central to this growth is the Aristotelian harmony between different norms. Living according to the norms of marriage tends to fulfill us in other respects,
while living contrary to the norms of marriage tends to be bad for us in other respects, and these are things we can often see. A happy marriage makes for
happy spouses and an unhappy marriage for unhappy spouses.

\subsubsection{Travel}
Generally speaking, when a married couple emigrates to or visits another society, they are deemed married in their new place, unless there is some general reason that
precludes them from counting as married, such as when they are of the same sex and move to a jurisdiction that does not recognize same-sex marriage.

Moreover, this recognition of them as married is not just an honorific indexed to their country of origin. When the Queen of Denmark visited the United States in 1991,
she was referred to as a ``queen''??check, but obviously she did not have rights and obligations of a monarch with respect to the United States, and so ``queen''
here was indexed with respect to the Kingdom of Denmark, and similarly for the title ``prince'' held by her husband. However, if someone referred to Henrik
as Margrethe's \textit{husband} or to Margreth as a \textit{married woman} during the visit, these terms would not be merely indexed to Denmark. Rather, they would
have the rights and obligations of an American married couple, as modified by their special immunity to persecution, and an ordinary non-diplomatic visitor from
Denmark would not even have that modification.

Should we say that by the mere fact of entering a country, a couple that was married in their country of origin enters into a new marriage institution?
On the social theory that is exactly what happens: the couple receives a new package of rights and obligations, definitive of marriage in a new society.
But this would be quite surprising: it would mean that a couple going for a honeymoon in another country would have had two weddings (one might tongue-in-cheek wonder if
theyn they shouldn't then be entitled to a second honeymoon?), and globetrotting couples would rack up marriage after marriage. Moreover, relinquishing one's
citizenship in a country one no longer lives in would be tantamount to a divorce.

Or perhaps instead of the new institution being entered into upon entry into a country, a couple by marrying enters into the marriage institution of every
jurisdiction that is willing to recognize them as married, but the rights and obligations of these institutions are merely conditional on their being in
those countries. While this would alleviate the problem of multiple weddings, such automatic entry into institutions in states that have no jurisdiction over one seems implausible.
Moreover, the problem of multiple weddings is not solved. When Margrethe and Henrik married in 1967, there was no state of East Timor. Then in
1975 it declared independence. By that declaration, did they impose a new marriage institution on Margrethe and Henrik, a marriage institution that disappeared
next year when East Timor was annexed by Indonesia, and then reappeared in 2002?

On the purely individual theory, the travel problem disappears. Different states may add rights and obligations, but what defines the marriage is the
complex of rights and obligations that the couple entered into on their own, and it counts as a ``marriage'' in their travel destination because of the
family resemblance between these rights and obligations and those that members of that society take on when they enter into an analogous relationship.

\subsubsection{Cross-cultural criticism}
Andronia is an especially sexist society, and Bob and Alice is an Andronian married couple. You've never interacted with Bob in a context that made his sexist
views clear, but one day you find out that Alice is sick, and Bob is not showing any consideration for Alice besides the minimum needed to be shown to any human being. You
call Bob out on this, and he tells you that in Andronia it is the wife's job always to show consideration for her husband while the husband need only keep the wife alive
and show her the kind of consideration one owes every human being when she is sick. He adds: ``This is how my parents behaved, how Alice's parents behaved, and Alice knew that this is what she was getting into when we got married.''

If marriage were a natural relationship, we could say that Bob and the rest of Andronian society is just wrong about what marriage requires, and we could
say to Bob: ``That may all be, but it's not how husbands \textit{should} behave!'' We could then show Bob examples of virtuous, caring egalitarian couples
in the hope that these examples would open his mind to what marriage really entails. Or we might say to Bob: ``If that's all you've committed to, then
you're not really married, and so you are reaping the benefits of marriage from Alice under false pretenses.''

But if the complex of obligations in marriage is either socially or individually defined, and if neither Andronian society nor the couple included
any special obligation of husband to wife in sickness beyond that which we owe any other human being, then Bob could well be simply right in his understanding
of his marital duties. This is an unattractive position.

Granted, if Bob and Alice are now living in a less sexist society, we could tell Bob that by moving to this society they have accepted the additional
duties of husbands to wives. This is, however, dubious. It may be that be immigrating to a country we take on the legal obligations of that
country, but it could well be the case that Bob is meeting these legal obligations, as they tend to be fairly minimal. What Bob is failing to do is
to meet the customary obligations attached to marriage in less sexist societies, but it is implausible that by moving to a country one becomes obligated
by the customs of the country. No moral criticism would necessarily attach to an American couple if after moving to Canada they failed to celebrate
Thanksgiving in October. Furthermore, nothing of significance is changed in the above story if we specify that Bob and Alice are \textit{still} living in
Andronia. Be they in Andronia or elsewhere, a husband owes more to his wife than Bob thinks.

Perhaps we could tell Bob: ``If that's all you committed to, then you aren't married in \textit{our} sense of the word.''
But that isn't a criticism of Bob's behavior with regard to Alice. Bob could just say: ``So what?'' At most it is a criticism of his misuse of words if Bob
claimed to be  married. Moreover, even as a linguistic criticism it is unlikely to hold water. For we do in fact use ``marriage'' and related words for relationships in
a vast array of historical and present societies, many of which are quite sexist indeed.

Admittedly, on both the social and the individual view, we could criticize Bob for the relationship that he is in. We could tell him: ``If that's what marriage
in your context is, it's a corrupt institution, and you shouldn't be married to Alice.'' But this is unacceptably weak tea. It allows that Bob is married to Alice but does
not owe her consideration in sickness beyond that owed a stranger.

\subsubsection{Fulfillment of a natural desire}
Plausibly, apart from reasonable moral and practical restrictions, people should be able to marry those whom they wish to. A society that did not make this
possible would be failing its members.

Now, society has no obligation to make possible the fulfillment of every desire people have. Rather, it is reasonable to make a distinction between natural
desires and more contingent desires, and hold that society should support the fulfillment of natural desires, such as for food, drink, shelter, useful
employment, and knoweldge. Given the plausibility that marriage is one of those things society ought to make available to its members, it is plausible that
the desire for marriage is a natural human desire. But if it is a natural human desire, then it is plausible that marriage itself is natural rather than
constructed.

This is perhaps the weakest of the arguments for marriage being a natural relationship, however. First, not everyone shares the intuition that a society ought
to make marriage possible. Second, it is not clear that we couldn't have a natural desire to construct---individually or socially---an institution of a certain
type.

\subsubsection{Same-sex marriage}
\subsubsubsection{An argument for liberals}
Let us assume that egalitarian justice requires one to advocate for same-sex marriage in jurisdictions where same-sex marriage is not available.

But suppose that marriage is socially constructed, and that we are in a locality in which one of the norms of marriage is that it be a relationship
between a man and a woman. Then, if we understand ``marriage'' as the word is locally understood, it makes no more sense to advocate for same-sex marriage
than to advocate for chess without pawns: these are simply contradictions in terms. Granted, we may choose to advocate for social recognition of another,
more egalitarian institution than marriage. But that will be a different institution.

If we advocate for this different institution, we have two choices. Either, we propose to maintain the institution currently called marriage, whether for everyone
who wishes to enter into it or just for those grandfathered into it, or not. If we propose to maintain the current non-egalitarian institution, then we are not
really advocating for same-sex marriage. We are advocating for a two-institution model, closely akin to marriage plus civil unions compromises that have generally been
seen as unacceptable by advocates of same-sex marriage.

On the other hand, if it is proposed not to maintain the current institution of marriage, then the common and plausible
arguments that extending marriage to same-sex couples does no harm to currently married opposite-sex will ring hollow. For it is a part of the proposal that the
institution they are a part of be annihilated. Furthermore, in practice, in jurisdictions where marriage has been extended to same-sex couples, generally those who were
previously married still count as married. Therefore, on the assumption that marriage is socially constructed, not only is there annihilation of the institution
that couples used to be a part of, but these couples are, without their express consent, inducted into the new institution. Such automatic induction into a new
relationship does not seem consistent with the ideals of a free society, and yet generally defenders of same-sex marriage have not been bothered by this.

If, however, one holds that marriage is a natural human relationship, then one can argue for marriage equality without arguing for a two-institution model or for
the annihilation of the existing institution. Instead, one can hold that marriage is a natural human relationship which non-defectively can be instantiated by
couples of the same sex as well as couples of the opposite sex. Given that marriage, understood univocally, can be entered in by both same-sex and opposite-sex couples, it is clear why
it is discriminatory for a state to limit recognition of it to opposite-sex couples. And in advocating the end of this inequality, one isn't advocating for an end of
an existing institution, but simply for the state's recognition of the fact that this natural institution can equally well include same-sex and opposite-sex couples.

It is worth noting that defenders of marriage equality who  hold that marriage is constructed by individual couples can also avoid the above problematic consequences
of social construction. On the individual construction view, in recognizing a marriage, the state is recognizing is a certain type of contract, where the type is
defined by a kind of family resemblance. But recognition of opposite-sex contracts
of a certain type without recognition of same-sex contracts of a relevantly similar type would be unreasonable. Imagine if one could only sell a house to someone of
the opposite sex, after all. On the individual construction view, the claim that no harm is done to opposite-sex couples by state recognition of same-sex marriage is
easily defensible. So, the above argument from same-sex marriage advocacy supports the natural relationship view and the individual construction view, but not the social
construction view.

\subsubsubsection{An argument for conservatives}
Here is a plausible principle: If we limit access to an institution on the grounds of gender or sex, absent very strong reason we should strive to make an equivalent
available.??ref For instance, perhaps there is some reason for colleges to limit certain sports to one gender, but then they should make other sports available
to the other gender.
But many conservatives have not only object to same-sex marriage but also to the availability of civil-union institutions for same-sex couples. I will argue that
such conservatives should embrace a view of marriage as a natural relationship.

For if marriage is constructed, either individually or socially, then even if the norms of that construction limit marriage to persons of the opposite sex, an
equivalent institution without that limitation could be constructed, and by the principle at the top of this argument, it ought to be. In fact, it seems that
the best way to resist this argument would be for the conservative to hold that marriage is a natural relationship, and that this relationship is only possible
or only normatively possible for opposite-sex couples, while any superficially similar relationship between persons of the same sex is not a natural relationship.
Because no merely social institution would be a natural relationship, it would not be an equivalent to marriage. Therefore, the conservative can respond to the
original argument by saying that there is very strong reason not strive to make an equivalent available, namely that no equivalent is possible.

In response, as per our previous argument, the defender of same-sex marriage should say that marriage is a natural relationship that \textit{can} legitimately
hold between persons of the same sex. So this conservative response does not close the debate. But it provides the conservative with a way forward. Indeed,
it seems that both sides on the same-sex marriage debate will be better served by moving to a natural relationship view of marriage, and then discussing whether
this natural relationship has norms that make it possible and permissible for persons of the same sex to instantiate it.

\section{Double Effect}
... intentions

... proportionality

... partiality

\section{The task of medicine}
The realism about teleology and normalcy provided by the Aristotelian framework
allows for an elegant solution to the problem of what the task of medicine is.??ref:Lennox

The medical professional is a \textit{professional}. Of course, everyone should refuse to act immorally on behalf of a client. But a professional has norms
and pursues goals that go beyond general morality, and has reason to refuse to further the client's aims even when there is nothing generally immoral about
these aims but the aims nonetheless violate the professional goals. Thus, while it is not immoral to create kitsch, a professional artist nonetheless has
to refuse a commission that would be unavoidably kitschy.

In the case of some professions, the goals are very much socially defined, and apart from legal minutiae, the delineation of these goals is of relatively
small importance. For instance, we have at least three professions that deal with the directing of water: gutter installers, sewer maintainers and plumbers.
All three professions are important, but the division of labor between them is not of great importance. It would do little harm to society if we had a single
profession for all three tasks, or if we divided up the tasks in some other way, say in terms of dealing with potable and non-potable water, or incoming
and outgoing water relative to a house.

However, the division of labor between the medical professions and other professions does seem to cut nature at its joints. The medical professions directly
aim at the goods of bodily health, a very natural subdivision of the space of human goods.

Moreover, there is a special value in the medical professional
having a very sharp focus on health. Medical considerations are of great importance to everyone's life. But in
the end, the patient (or their representative; I will simplify by talking just of patients) needs to be able to make a prudent decision about the recommendations
from a medical professional, weighing this recommendation against non-medical considerations such as ones of economics, interpersonal relationships, personal
pleasure and convenience, and so on. The patient is typically not an expert in biological matters, but tends to have a good grasp of other relevant goods:
for instance, they will know what effect giving up alcohol would have on their social life, or what goods their children would have to give up if a medical
procedure is to be paid for. It is important, however, that a medical recommendation be primarily concerned with the good of the patient's health, so that
the weighing between medical goals and other goods be delegated to the patient as much as possible, and that the non-medical goods not be double-counted (once by the
medical professional and again by the patient) in figuring out the prudent course of action.

At the same time, it is also important for guiding patients to prudent
decisions that medical professionals understand health holistically, rather than narrowly thinking only of the kidneys or the feet. Thus, a focus on health in general
is important for the medical professional, or at least the medical professional who has an advising relationship with a patient.

But what is health? Health is not \textit{simply} the good of the body. There are many goods of the body besides health, such as
athletic prowess, beauty, and reproduction. These goods depend on health, but are not a part of health: for instance, a relatively healthy reproductive system is needed for
reproduction, but one can have such a system without using it.

Apparently, physicians see their task as the return of the body to normal function, and then further claim to understand normalcy in a statistical way, as average
function.??ref Tying health to normal function seems quite plausible indeed. But the normal cannot be understood merely statistically. ??xref? If it were merely
statistical, then the adult who can deadlift 400 kilograms would be as abnormal
as the adult who cannot deadlift one kilogram. Rather, normalcy often has a directionality that it inherits from a teleology towards some good. Adults who
can deadlift 400 kilograms exceeds the norm, and might be said to be supernormal, but are not thereby abnormal, nor do they need medical treatment to reduce their
strength.??Vonnegut

Moreover, among our goods, it is perhaps health that is most clearly species-relative. As a result, an Aristotelian metaphysics of forms is perfectly fitted to
grounding the norms of health as the norms of sufficient capacity to function bodily in accordance with our human teleology.????more, better definition


\section{Consent}
?? http://alexanderpruss.blogspot.com/2022/08/a-tale-of-two-membranes.html
\section{Environmental ethics}
\section{Relationship to other animals}
\section{The definition of life}??move??
Here is an intuition that until fairly recently would have been widely shared: There are deep metaphysical divides between non-living and living things,
and between merely living things and persons, and these divides mark a hierarchy of value, a chain of being. If we could defend such a divide, it would
dovetail with the idea that persons are in an important way \textit{sacred}, having rights while other things have mere interests, if that.

I want to offer a highly speculative Aristotelian reconstruction of this intuition. To introduce the reconstruction, start with a puzzle for
Aristotelian views. It seems that on such views:
\ditem{2-pursue-perfect}{Each thing naturally strives for its own perfections.}
\ditem{2-natural-activity}{The natural activity of a thing is a perfection of it.}
But this generates a regress. Let's say that reproduction is an oak tree's perfection. Then by \dref{2-pursue-perfect}, the oak tree naturally strives for
reproduction. This natural activity of striving for reproduction, by \dref{2-natural-activity}, is then itself a perfection of the oak tree. Therefore,
by \dref{2-pursue-perfect}, the oak tree must naturally strive for it: hence the oak tree naturally strives for striving for reproduction. And so on,
\textit{ad infinitum}. But surely an oak tree does not pursue infinitely many things. And even after a few level of meta-striving we exhaust plausibility.

I suggest that we can deny \dref{2-pursue-perfect}. Some perfections of a thing are not actually naturally striven for by the thing.\footnote{An interesting
theological example may be the idea in the Thomistic tradition that both the beatific vision and our striving for it are gifts of God's grace, rather than
natural for us, even though the beatific vision perfects us.??} The oak tree does strive for reproduction with its reproductive organs. Moreover, it has a
second order striving: it strives to strive for reproduction, by growing the reproductive organs with which it strives for reproduction. There may be one or
two more meta-levels, but at some level we can say: it just does this, without striving to do it.

Non-living things, on an Aristotelian metaphysics, also have form and also strive for ends. But plausibly they don't strive to strive: they just strive.
We thus have a hierarchical division between inorganic things which do not strive to strive and living things which have second order teleological strivings.

The problem of the definition of life is a thorny conceptual problem in biology or its philosophy. Different authors give different lists of features such
as homeostasis, growth and reproduction as part of the definition of life. The multiplicity of features listed makes the concept of life seem arbitrary.
Moreover, it is philosophically problematic to tie the the concept of life too tightly to the physical forms of life around us. For it is very plausible
that if there are immaterial agents such as deities, spirits or angels, they should also count as alive.\footnote{It is worth noting that not everyone who
believes in deities, spirits or angels believes them to be immaterial. The ancient Greeks did not think their deities immaterial. And a minority opinion
among Christian theologians held angels to be made of ``subtle matter''.??ref But the argument only needs that some do believe them to be immaterial.}
 After all, those who believe in such beings sometimes
hold them to be immortal. But if they were not alive, their immortality would be a trivial claim: a being that is not alive in the first place cannot die.
However, these beings are conceptualized as alive, even when they cannot engage in homeostasis, growth or reproduction. And yet while a particular existence
claim about the existence of immortal immaterial agents might be false, it does not seem to be fundamentally conceptually confused. Thus, a good account
of life should include the kind of life that is attributed to immaterial agents, and none??check of the accounts in the philosophy of biology do that.

Furthermore, it is a merit of a definition that when applied to cases where we do not know how to classify a thing, the definition does not trivially
decide the issue, but it points to the question we need to answer if we are to decide the issue. To that end, consider two borderline cases: viruses
and sophisticated robots, like Star Trek's Data. In neither case are we confident whether we have life. Viruses are famously a borderline case.
And while Data is described as a ``synthetic life-form''??ref, and the Star Trek canon clearly favors his being actually alive, the question is
not so philosophically clear. Data obviously fails typical biological definitions of life: while he engages in self-maintenance, he doesn't grow or
reproduce in the biological sense of the word (though he does make other androids), in a way that does not match typical viewers' intuitions.\footnote{Though,
admittedly, there may be some static due to the show confusing the question of consciousness with that of life.??check} And
whether a virus qualifies as alive varies from definition
to definition??ref in a way that makes it sound like the question of viruses being alive is merely verbal. Yet given the strong intuition that there
is something of great value about life, even something sacred, the question of what is and is not alive should not be merely verbal.

On the other hand, an account on which what it is to be alive is to have a second order teleological striving---to strive to strive for a perfection---will nicely
include any immaterial agents. It will include any entity that prepares itself for future teleological activity, say by growth,
and hence will include all the physical forms of life we know about. It will exclude elementary particles. And whether it includes viruses or sophisticated
robots is unclear---as it should be. For it is unclear whether viruses and sophisticated robots have form at all. If viruses have form, then it is likely
that their activity of attaching to hosts for purposes of future replication is a striving for replicative striving, and hence they are alive. But it is
not clear whether they have form. If sophisticated robots have form, they also exhibit meta-striving, and hence are alive. But in both cases we do not
know whether there is form, or whether we are dealing with a mere agglomeration of particles.
Aristotle himself seems to have thought that
artifacts only had form in the analogical sense of a blueprint in the mind of the designer??ref, but he could have been wrong in the case of artifacts like Data.
(For more on the epistemic issues here, see Section~\ref{sec:epist-of-form}
in Chapter~\ref{ch:God}.)

We thus have two levels in a chain of being: things that strive but don't meta-strive, and things that meta-strive. Now, among the things that meta-strive,
we can describe a higher kind of thing: a thing that strives for all of its perfections. The premises of the regress argument with which we started this
section apply to such a being. Thus, this is a being that strives for striving for ... for perfection, at any number of levels. While this is implausible
for an oak tree or even a dog, we do actually know of one kind of being that does that: humans. Human beings not only conceptualize particular perfections, such as friendship
or striving for striving for health, but they conceptual perfection as such, and strive for it as such. If a trustworthy being offered you to increase
some perfection or other, and assured you that you would in no way be harmed, it would be rational for you to accept the offer, because perfection as
such is one of the things you and I pursue.

At the same time, in a minded being, the infinite chain that results from striving for all one's perfections need not be a chain of separate desires
and hence does not require a being that is actually infinite. Rather, all that's needed is for the being to be such that it has or teleologically strives
to have the concept of a perfection as such and a desire for perfection as such. This desire then can manifest in a striving to figure out what the perfections are---a striving that is central to
the search for happiness (\textit{eudaimonia}) that was so characteristic of Socratic and post-Socratic Greek philosophy---and a striving to be ready
to accept whatever one finds. In fact, it might be that for reasons having to do with the nature of infinity \textit{only} a minded being can pursue
an infinite number of ends---for any non-minded being that did that would need to have infinitely many distinct causal sources of its activity in
a way that might well violate causal finitism, the thesis that it is impossible for an infinite number of causes to work together (for a defense
of causal finitism, see ??ref). And among minded beings, perhaps it is definitive of \textit{persons} that they pursue all good.

We thus have a qualitative hierarchy of being between the mere strivers, the mere meta-strivers and the universal strivers. The first division in
the hierarchy may well correspond to that between the non-living and the living, and the second might---depending on speculative questions about
infinity---align with the division between mere life and personhood. And it is very natural to see qualitative divisions of value here as well.

\chaptertail

