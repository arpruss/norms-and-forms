\def\mychapter{IV}
\chapter{Applications}\label{ch:applied-ethics}
\section{Double Effect}
\section{Medical ethics}
\section{Environmental ethics}
\section{Marriage and other natural relationships}
\section{A great chain of being and the definition of life}??move??
Here is an intuition that until fairly recently would have been widely shared: There are deep metaphysical divides between non-living and living things, 
and between merely living things and persons, and these divides mark a hierarchy of value, a chain of being. If we could defend such a divide, it would 
dovetail with the idea that persons are in an important way \textit{sacred}, having rights while other things have mere interests, if that. 

I want to offer a highly speculative Aristotelian reconstruction of this intuition. To introduce the reconstruction, start with a puzzle for
Aristotelian views. It seems that on such views:
\ditem{2-pursue-perfect}{Each thing naturally strives for its own perfections.}
\ditem{2-natural-activity}{The natural activity of a thing is a perfection of it.}
But this generates a regress. Let's say that reproduction is an oak tree's perfection. Then by \dref{2-pursue-perfect}, the oak tree naturally strives for 
reproduction. This natural activity of striving for reproduction, by \dref{2-natural-activity}, is then itself a perfection of the oak tree. Therefore,
by \dref{2-pursue-perfect}, the oak tree must naturally strive for it: hence the oak tree naturally strives for striving for reproduction. And so on,
\textit{ad infinitum}. But surely an oak tree does not pursue infinitely many things. And even after a few level of meta-striving we exhaust plausibility.

I suggest that we can deny \dref{2-pursue-perfect}. Some perfections of a thing are not actually naturally striven for by the thing.\footnote{An interesting
theological example may be the idea in the Thomistic tradition that both the beatific vision and our striving for it are gifts of God's grace, rather than
natural for us, even though the beatific vision perfects us.??} The oak tree does strive for reproduction with its reproductive organs. Moreover, it has a 
second order striving: it strives to strive for reproduction, by growing the reproductive organs with which it strives for reproduction. There may be one or
two more meta-levels, but at some level we can say: it just does this, without striving to do it.

Non-living things, on an Aristotelian metaphysics, also have form and also strive for ends. But plausibly they don't strive to strive: they just strive. 
We thus have a hierarchical division between inorganic things which do not strive to strive and living things which have second order teleological strivings.

The problem of the definition of life is a thorny conceptual problem in biology or its philosophy. Different authors give different lists of features such 
as homeostasis, growth and reproduction as part of the definition of life. The multiplicity of features listed makes the concept of life seem arbitrary.
Moreover, it is philosophically problematic to tie the the concept of life too tightly to the physical forms of life around us. For it is very plausible
that if there are immaterial agents such as deities, spirits or angels, they should also count as alive.\footnote{It is worth noting that not everyone who
believes in deities, spirits or angels believes them to be immaterial. The ancient Greeks did not think their deities immaterial. And a minority opinion
among Christian theologians held angels to be made of ``subtle matter''.??ref But the argument only needs that some do believe them to be immaterial.}
 After all, those who believe in such beings sometimes 
hold them to be immortal. But if they were not alive, their immortality would be a trivial claim: a being that is not alive in the first place cannot die. 
However, these beings are conceptualized as alive, even when they cannot engage in homeostasis, growth or reproduction. And yet while a particular existence
claim about the existence of immortal immaterial agents might be false, it does not seem to be fundamentally conceptually confused. Thus, a good account
of life should include the kind of life that is attributed to immaterial agents, and none??check of the accounts in the philosophy of biology do that.

Furthermore, it is a merit of a definition that when applied to cases where we do not know how to classify a thing, the definition does not trivially
decide the issue, but it points to the question we need to answer if we are to decide the issue. To that end, consider two borderline cases: viruses
and sophisticated robots, like Star Trek's Data. In neither case are we confident whether we have life. Viruses are famously a borderline case.
And while Data is described as a ``synthetic life-form''??ref, and the Star Trek canon clearly favors his being actually alive, the question is
not so philosophically clear. Data obviously fails typical biological definitions of life: while he engages in self-maintenance, he doesn't grow or
reproduce in the biological sense of the word (though he does make other androids), in a way that does not match typical viewers' intuitions.\footnote{Though,
admittedly, there may be some static due to the show confusing the question of consciousness with that of life.??check} And 
whether a virus qualifies as alive varies from definition
to definition??ref in a way that makes it sound like the question of viruses being alive is merely verbal. Yet given the strong intuition that there
is something of great value about life, even something sacred, the question of what is and is not alive should not be merely verbal. 

On the other hand, an account on which what it is to be alive is to have a second order teleological striving---to strive to strive for a perfection---will nicely 
include any immaterial agents. It will include any entity that prepares itself for future teleological activity, say by growth,
and hence will include all the physical forms of life we know about. It will exclude elementary particles. And whether it includes viruses or sophisticated
robots is unclear---as it should be. For it is unclear whether viruses and sophisticated robots have form at all. If viruses have form, then it is likely
that their activity of attaching to hosts for purposes of future replication is a striving for replicative striving, and hence they are alive. But it is
not clear whether they have form. If sophisticated robots have form, they also exhibit meta-striving, and hence are alive. But in both cases we do not
know whether there is form, or whether we are dealing with a mere agglomeration of particles. 
Aristotle himself seems to have thought that
artifacts only had form in the analogical sense of a blueprint in the mind of the designer??ref, but he could have been wrong in the case of artifacts like Data.
(For more on the epistemic issues here, see Section~\ref{sec:epist-of-form}
in Chapter~\ref{ch:God}.) 

We thus have two levels in a chain of being: things that strive but don't meta-strive, and things that meta-strive. Now, among the things that meta-strive,
we can describe a higher kind of thing: a thing that strives for all of its perfections. The premises of the regress argument with which we started this
section apply to such a being. Thus, this is a being that strives for striving for ... for perfection, at any number of levels. While this is implausible
for an oak tree or even a dog, we do actually know of one kind of being that does that: humans. Human beings not only conceptualize particular perfections, such as friendship
or striving for striving for health, but they conceptual perfection as such, and strive for it as such. If a trustworthy being offered you to increase
some perfection or other, and assured you that you would in no way be harmed, it would be rational for you to accept the offer, because perfection as
such is one of the things you and I pursue. 

At the same time, in a minded being, the infinite chain that results from striving for all one's perfections need not be a chain of separate desires
and hence does not require a being that is actually infinite. Rather, all that's needed is for the being to be such that it has or teleologically strives 
to have the concept of a perfection as such and a desire for perfection as such. This desire then can manifest in a striving to figure out what the perfections are---a striving that is central to
the search for happiness (\textit{eudaimonia}) that was so characteristic of Socratic and post-Socratic Greek philosophy---and a striving to be ready
to accept whatever one finds. In fact, it might be that for reasons having to do with the nature of infinity \textit{only} a minded being can pursue
an infinite number of ends---for any non-minded being that did that would need to have infinitely many distinct causal sources of its activity in 
a way that might well violate causal finitism, the thesis that it is impossible for an infinite number of causes to work together (for a defense
of causal finitism, see ??ref). And among minded beings, perhaps it is definitive of \textit{persons} that they pursue all good.

We thus have a qualitative hierarchy of being between the mere strivers, the mere meta-strivers and the universal strivers. The first division in
the hierarchy may well correspond to that between the non-living and the living, and the second might---depending on speculative questions about
infinity---align with the division between mere life and personhood. And it is very natural to see qualitative divisions of value here as well.

\chaptertail

