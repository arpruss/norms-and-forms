\def\mychapter{VIII}
\ifdefined\book
\else
\documentclass[11pt,oneside]{amsbook}
\usepackage[backend=biber, citestyle=authoryear]{biblatex}
\usepackage{mathpazo}
\usepackage{graphicx}
\usepackage{amsmath}
\usepackage{tikz}
\usetikzlibrary{arrows}
%\usepackage{titlesec}
\addbibresource{bibliography.bib}
\newcommand\posscite[1]{\citeauthor{#1}'s (\citeyear{#1})}
\newcommand\plural[1]{#1\mathrm{s}}
%\def\posscitewithextra[#1]#2{\citename{#2}'s (\citeyear{#2}, #1)}

%\newcounter{subsubsubsection}[subsubsection]
%\renewcommand\thesubsubsubsection{\thesubsubsection.\arabic{subsubsubsection}}
%\titleformat{\subsubsubsection}
%  {\normalfont\normalsize\bfseries}{\thesubsubsubsection}{1em}{}
%\titlespacing*{\subsubsubsection}
%{0pt}{3.25ex plus 1ex minus .2ex}{1.5ex plus .2ex}

\ifdefined\book
\renewcommand{\thechapter}{\Roman{chapter}}
\else
\renewcommand{\thechapter}{\mychapter}
\fi

\linespread{1.7}
\usepackage[margin=1.25in]{geometry}
\sloppy
\makeatletter
%% TODO: This is a cheat. It's supposed to be {paragraph}{4}, and that's 
%% what it is in amsbook.cls, but then it fails.
\def\paragraph{\@startsection{paragraph}{3}%
  \normalparindent\z@{-\fontdimen2\font}%
  \normalfont}
\def\subsubsubsection{\paragraph}
\makeatother

\def\smalltick{0.15cm}
\def\bigtick{0.3cm}
\def\pointcircle{0.08cm}
\def\causalnode{0.35cm}

\hyphenation{dia-chro-nic}

%\usepackage[utf8]{inputenc} % set input encoding (not needed with XeLaTeX)
\usepackage{amssymb}
\usepackage{mathtools}
\usepackage{enumitem}
\usepackage{amsthm}
\usepackage{physics}
%\usepackage{ntheorem}

\makeatletter
% \def\@endtheorem{\endtrivlist\@endpefalse }% OLD
\def\@endtheorem{\endtrivlist}%

\catcode`\|=\active\def|{\mid}
\DeclarePairedDelimiter{\ceil}{\lceil}{\rceil}
\DeclarePairedDelimiter{\floor}{\lfloor}{\rfloor}
\newcommand{\Subj}{\mathbin{\raisebox{.15ex}{$\scriptscriptstyle{\Box}$}\kern-.425em\rightarrow}}
\def\Existence{E!}
\def\Believes{\operatorname{Believes}}
\def\True{\operatorname{True}}
\def\Perfection{\operatorname{Perfection}}
\def\ext{\operatorname{Ext}}
\def\Iff{\leftrightarrow}
\def\Implies{\rightarrow}
\def\Entails{\Rightarrow}
\def\Equiv{\Leftrightarrow}
\def\Form{operatorname{Form}}
\def\Informs{operatorname{Informs}}
\def\technical{$\star$}
\def\vtechnical{$\star\star$}
\def\power{\wp}
\def\Nec{\Box}
\def\Poss{\Diamond}
\def\Prop#1{$\langle$#1$\rangle$}
\def\R{\mathbb R}
\def\N{\mathbb N}
\def\tele{tel\={e}}
\makeatletter
\newtheoremstyle{indented}{3pt}{3pt}{\addtolength{\leftskip}{4.5em}}{-2.5em}{\sc}{.}{.5em}{}
\def\Principle#1#2#3{\theoremstyle{indented}\newtheorem*{principle}{#2}\begin{principle}\def\@currentlabel{#2}\label{#1}#3\end{principle}\let\principle\undefined}
\makeatother
\def\pref#1{{\sc\ref{#1}}}
\def\enum#1{\resume{enumerate}\item #1\end{enumerate}}
\def\ditem#1#2{\begin{enumerate}[resume]\item \label{\mychapter:#1} #2\end{enumerate}}
\def\dref#1{(\ref{\mychapter:#1})}
\def\drefglobal#1{(\ref{#1})}
\usepackage{graphicx} % support the \includegraphics command and options
\usepackage{array} % for better arrays (eg matrices) in maths
\def\Not{\operatorname{\sim}}
\def\St{\operatorname{St}}
\def\num{\operatorname{num}}
\def\And{\mathrel{\&}}
\def\Or{\vee}
\def\BigOr{\bigvee}
\def\<{\langle}
\def\>{\rangle}
\def\union{\cup}
\def\nleq{\not\le}
\def\N{\mathbb N}
\def\R{\mathbb R}
\def\C{\mathbb C}
\def\Powerset{\mathcal P}
\def\star#1{{}^*#1}
\def\hN{\star{\N}}
\def\hR{\star{\R}}
\def\Z{\mathbb Z}
\def\Power{\mathcal P}
\def\Implies{\rightarrow}
\def\True{\operatorname{True}}
\def\Socrates{\mathrm{Socrates}}
\def\actual{@}

\def\H2O{H${}_2$O}

\def\scr{\mathcal}
\def\e{\varepsilon}
\def\eps{\varepsilon}
\newtheorem{lem}{Lemma}
\newtheorem*{theorem}{Theorem}
\newtheorem{corollary}{Corollary}
\newtheorem{cond}{Condition}

\renewcommand\thechapter{\Roman{chapter}}

\def\chaptertail{\ifdefined\book\else\end{document}\fi}
 

\title{Infinity, Causation and Paradox}
\author{Alexander R. Pruss}
%\date{} % Activate to display a given date or no date (if empty),
         % otherwise the current date is printed

\begin{document}
\setcounter{secnumdepth}{3}
\setcounter{tocdepth}{4}

\end{document}
\fi

\restartlist{enumerate}

\chapter{Metaphysics}\label{ch:metaphysics}
\section{Composition}
\section{Ill-matched matter, rearrangement, the power to continue existing and immortality}
\section{Diachronic identity}
\section{Naturalism}
Is the Aristotelian hylomorphic account of humans compatible with naturalism? This depends on how naturalism
is defined and whether we take the theistic version of the account or not.

Since theism implies the existence of a non-natural causally efficacious substance, namely God, it will be incompatible
with most versions of naturalism. And I have argued theat the Aristotelian account is unsatisfactory without
theism. 

Still, it is an interesting question whether the what the Aristotelian account says about the forms of finite
substances is compatible with naturalism. If so, then one could combine the Aristotelian account with 
a naturalism restricted to finite objects, and save some naturalistic intuitions. Furthermore, it would mean
that an Aristotelian not convined by the arguments that theism is needed to make the theory satisfactory could
be a full-blown naturalist.

If we take naturalism to say that the only entities that exist are the ones that would figure in a completed science,
then it is unlikely that Aristotelian metaphysics would be compatible with naturalism about finite objects. However,
such a strong naturalism would likely also conflict with many other metaphysical theories that are rarely taken to 
contradict naturalism. 

For instance, consider theories of time. Of the theories of time, four dimensionalism, on which
ordinary objects like ourselves are extended in time as well as space, is what seems to fit best with Relativity Theory. 
But the most common four-dimensionalist view of changing properties is perdurantism: changing objects are made of temporal parts or
slices to which the properties are primarily attributed, so that a tomato that once was green and now is red has a green temporal part
and a red temporal part. But slices are unlikely to figure in a completed science if
our current science is a good guide to that. Consider that our current physics does not consider particles
like electrons and quarks to be fundamental, and hence not made up of smaller parts. But electrons and quarks change over
time (e.g., with respect to spin and flavor), and so they would need tos have temporal parts.  But these parts are not found
in our physics. 

Or consider that our science may quantify over physical objects like particles and fields, and applies predicates to them, but 
does not quantify over properties. Thus, properties, whether understood as universals or as tropes, go beyond current science.
Yet it seems implausible to understand naturalism as implying nominalism.

One may weaken naturalism to say that the only \textit{causally relevant} entities are those of a completed science. But 
this would still rule out perdurantism, since objects change with respect to causally relevant properties, and hence their
temporal parts have these properties, and have them more fundamentally. It would also likely rule out many versions of trope
theory, since the causal efficacy of tropes is supposed to explain the causal efficacy of the objects made of them.

Let's step back. Naturalism denies the existence of causally efficacious ``supernatural'' entities like 
ghosts, but it is neutral on temporal parts and tropes of ``natural'' entities like electrons. 
What is the relevant difference between ghosts and temporal parts of electrons? It seems to be this. While
neither entity is posited by science, if ghosts exist, then certain phenomena that it belongs to science 
to investigate have no scientific explanation, barring systematic overdetermination. If there are ghosts,
they are presumably responsible for at least some of the appearances of ghosts, some of the chills people
feel in graveyards, etc. These phenomena then either have no physical explanation at all, or by a massive
coincidence are overdetermined by a physical and a spectral cause. There is a competition, thus, between 
scientific and spiritual explanations when we are concerned with ghosts. 

On the other hand, there is neither
competition nor overdetermination in the case of the objects studied by science and their ontological
constituents such as temporal parts or tropes. When a temporal part or a trope of an ice cube makes 
your hand cold, the ice cube also makes your hand cold. The ice cube's cooling causal influence on your hand
does not compete with the ice cube's temporal part's causal influence on you or on the causal influence of 
a trope of coldness (if there are temporal parts of ice cubes or tropes of coldness). Nor is this overdetermination,
since it is not overdetermination when an object $C$ causes an effect $E$ by means of $C$'s part. After all,
it is not overdetermination when the ice cube cools the hand it is lying on \textit{and} the lower half of 
the ice cube cools the hand. 

We might thus want to say that naturalism holds that any phenomenon that it falls within the purview of science 
to seek for causal explanations of either has no causal explanation at all, or else has a scientific causal explanation 
and is not overdetermined by a scientific and a non-scientific one. On this account, theism is incompatible with
naturalism if there is a first state of physical reality. For if there is such a state, it belongs
to science to seek for its causal explanation. However, there is a theistic explanation of that state and no 
scientific explanation. We thus have the kind of competition that naturalism rules out.\footnote{Matters are a little more
complicated if there is no initial physical state, but we can apply the above argument to an initial
segment of physical states: science cannot explain that segment but theism claims to do so, and yet it lies
in the scope of science to ask for such an explanation.} One might object that science knows that it cannot, on pain
of vicious circularity, provide a scientific explanation of the first physical state, and hence it does not belong to
science to seek that explanation. But this objection confuses the first physical state \textit{qua} first and the
first physical state as it intrinsically is. It does not belong to science to seek the explanation of the first physical
state \textit{qua} first. But if we just consider it as a specific physical state---particles and fields arranged thus-and-so---then
it certainly belongs to science to seek for its explanation, though that search will be rightly abandoned if sufficient evidence
is gathered that the state is in fact the first one.

Now, let's go back to Aristotelianism. Bracketing theism, the entity that might trouble the naturalist is form.
But the forms of things are components of substances---humans, dogs, oaks, etc.---that are also found in our current
science (e.g., biology) and are likely to be found in the completed version of that science. And there is neither
competition(??what's that? remove here and earlier?) nor overdetermination between causal explanations provided by
forms and their substances. As long as the substances do not have spooky causal powers beyond the scope of science,
adding forms to the ontology no more contradicts a plausibly defined naturalism than perdurantism or trope theory
does.

It is, of course, open to the Aristotelian to suppose that some finite substances, whether natural like humans and oaks 
or  supernatural like angels or ghosts, have causal powers of a sort that goes beyond the scope of science, though
specifying what those powers would have to be like is difficult. But nothing in the applications given for Aristotelianism
posited such powers. Thus as far as the argument of this book goes, there is no contradiction with a carefully
specified naturalism with respect to finite things.\footnote{I think the most plausible place to look for a tension 
would be in examining human free will, but that goes beyond the scope of this book.}

\chaptertail 

\def\mychapter{IX}
\ifdefined\book
\else
\documentclass[11pt,oneside]{amsbook}
\usepackage[backend=biber, citestyle=authoryear]{biblatex}
\usepackage{mathpazo}
\usepackage{graphicx}
\usepackage{amsmath}
\usepackage{tikz}
\usetikzlibrary{arrows}
%\usepackage{titlesec}
\addbibresource{bibliography.bib}
\newcommand\posscite[1]{\citeauthor{#1}'s (\citeyear{#1})}
\newcommand\plural[1]{#1\mathrm{s}}
%\def\posscitewithextra[#1]#2{\citename{#2}'s (\citeyear{#2}, #1)}

%\newcounter{subsubsubsection}[subsubsection]
%\renewcommand\thesubsubsubsection{\thesubsubsection.\arabic{subsubsubsection}}
%\titleformat{\subsubsubsection}
%  {\normalfont\normalsize\bfseries}{\thesubsubsubsection}{1em}{}
%\titlespacing*{\subsubsubsection}
%{0pt}{3.25ex plus 1ex minus .2ex}{1.5ex plus .2ex}

\ifdefined\book
\renewcommand{\thechapter}{\Roman{chapter}}
\else
\renewcommand{\thechapter}{\mychapter}
\fi

\linespread{1.7}
\usepackage[margin=1.25in]{geometry}
\sloppy
\makeatletter
%% TODO: This is a cheat. It's supposed to be {paragraph}{4}, and that's 
%% what it is in amsbook.cls, but then it fails.
\def\paragraph{\@startsection{paragraph}{3}%
  \normalparindent\z@{-\fontdimen2\font}%
  \normalfont}
\def\subsubsubsection{\paragraph}
\makeatother

\def\smalltick{0.15cm}
\def\bigtick{0.3cm}
\def\pointcircle{0.08cm}
\def\causalnode{0.35cm}

\hyphenation{dia-chro-nic}

%\usepackage[utf8]{inputenc} % set input encoding (not needed with XeLaTeX)
\usepackage{amssymb}
\usepackage{mathtools}
\usepackage{enumitem}
\usepackage{amsthm}
\usepackage{physics}
%\usepackage{ntheorem}

\makeatletter
% \def\@endtheorem{\endtrivlist\@endpefalse }% OLD
\def\@endtheorem{\endtrivlist}%

\catcode`\|=\active\def|{\mid}
\DeclarePairedDelimiter{\ceil}{\lceil}{\rceil}
\DeclarePairedDelimiter{\floor}{\lfloor}{\rfloor}
\newcommand{\Subj}{\mathbin{\raisebox{.15ex}{$\scriptscriptstyle{\Box}$}\kern-.425em\rightarrow}}
\def\Existence{E!}
\def\Believes{\operatorname{Believes}}
\def\True{\operatorname{True}}
\def\Perfection{\operatorname{Perfection}}
\def\ext{\operatorname{Ext}}
\def\Iff{\leftrightarrow}
\def\Implies{\rightarrow}
\def\Entails{\Rightarrow}
\def\Equiv{\Leftrightarrow}
\def\Form{operatorname{Form}}
\def\Informs{operatorname{Informs}}
\def\technical{$\star$}
\def\vtechnical{$\star\star$}
\def\power{\wp}
\def\Nec{\Box}
\def\Poss{\Diamond}
\def\Prop#1{$\langle$#1$\rangle$}
\def\R{\mathbb R}
\def\N{\mathbb N}
\def\tele{tel\={e}}
\makeatletter
\newtheoremstyle{indented}{3pt}{3pt}{\addtolength{\leftskip}{4.5em}}{-2.5em}{\sc}{.}{.5em}{}
\def\Principle#1#2#3{\theoremstyle{indented}\newtheorem*{principle}{#2}\begin{principle}\def\@currentlabel{#2}\label{#1}#3\end{principle}\let\principle\undefined}
\makeatother
\def\pref#1{{\sc\ref{#1}}}
\def\enum#1{\resume{enumerate}\item #1\end{enumerate}}
\def\ditem#1#2{\begin{enumerate}[resume]\item \label{\mychapter:#1} #2\end{enumerate}}
\def\dref#1{(\ref{\mychapter:#1})}
\def\drefglobal#1{(\ref{#1})}
\usepackage{graphicx} % support the \includegraphics command and options
\usepackage{array} % for better arrays (eg matrices) in maths
\def\Not{\operatorname{\sim}}
\def\St{\operatorname{St}}
\def\num{\operatorname{num}}
\def\And{\mathrel{\&}}
\def\Or{\vee}
\def\BigOr{\bigvee}
\def\<{\langle}
\def\>{\rangle}
\def\union{\cup}
\def\nleq{\not\le}
\def\N{\mathbb N}
\def\R{\mathbb R}
\def\C{\mathbb C}
\def\Powerset{\mathcal P}
\def\star#1{{}^*#1}
\def\hN{\star{\N}}
\def\hR{\star{\R}}
\def\Z{\mathbb Z}
\def\Power{\mathcal P}
\def\Implies{\rightarrow}
\def\True{\operatorname{True}}
\def\Socrates{\mathrm{Socrates}}
\def\actual{@}

\def\H2O{H${}_2$O}

\def\scr{\mathcal}
\def\e{\varepsilon}
\def\eps{\varepsilon}
\newtheorem{lem}{Lemma}
\newtheorem*{theorem}{Theorem}
\newtheorem{corollary}{Corollary}
\newtheorem{cond}{Condition}

\renewcommand\thechapter{\Roman{chapter}}

\def\chaptertail{\ifdefined\book\else\end{document}\fi}
 

\title{Infinity, Causation and Paradox}
\author{Alexander R. Pruss}
%\date{} % Activate to display a given date or no date (if empty),
         % otherwise the current date is printed

\begin{document}
\setcounter{secnumdepth}{3}
\setcounter{tocdepth}{4}

\end{document}
\fi

\restartlist{enumerate}

\chapter{Laws of nature and causal powers}\label{ch:laws}
\section{Humean and pushy laws}
There are two main views of laws of nature. On Humean views, we have laws simply in virtue of non-causal 
regularities of the behavior of objects in nature, and causation is then grounded in the laws. On pushy views, 
we have laws in virtue of metaphysical components of reality that affect or constrain the behavior of objects. 

The debate between Humean and pushy views is one of the biggest debates in the philosophy of science. In this 
section, I will argue that pushiness is the right view.  ??

\section{Laws and causation}

\section{Laws, natures and a partial Humeanism}
\chaptertail

