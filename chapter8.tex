\def\mychapter{VIII}
\ifdefined\book
\else
\documentclass[11pt,oneside]{amsbook}
\usepackage[backend=biber, citestyle=authoryear]{biblatex}
\usepackage{mathpazo}
\usepackage{graphicx}
\usepackage{amsmath}
\usepackage{tikz}
\usetikzlibrary{arrows}
%\usepackage{titlesec}
\addbibresource{bibliography.bib}
\newcommand\posscite[1]{\citeauthor{#1}'s (\citeyear{#1})}
\newcommand\plural[1]{#1\mathrm{s}}
%\def\posscitewithextra[#1]#2{\citename{#2}'s (\citeyear{#2}, #1)}

%\newcounter{subsubsubsection}[subsubsection]
%\renewcommand\thesubsubsubsection{\thesubsubsection.\arabic{subsubsubsection}}
%\titleformat{\subsubsubsection}
%  {\normalfont\normalsize\bfseries}{\thesubsubsubsection}{1em}{}
%\titlespacing*{\subsubsubsection}
%{0pt}{3.25ex plus 1ex minus .2ex}{1.5ex plus .2ex}

\ifdefined\book
\renewcommand{\thechapter}{\Roman{chapter}}
\else
\renewcommand{\thechapter}{\mychapter}
\fi

\linespread{1.7}
\usepackage[margin=1.25in]{geometry}
\sloppy
\makeatletter
%% TODO: This is a cheat. It's supposed to be {paragraph}{4}, and that's 
%% what it is in amsbook.cls, but then it fails.
\def\paragraph{\@startsection{paragraph}{3}%
  \normalparindent\z@{-\fontdimen2\font}%
  \normalfont}
\def\subsubsubsection{\paragraph}
\makeatother

\def\smalltick{0.15cm}
\def\bigtick{0.3cm}
\def\pointcircle{0.08cm}
\def\causalnode{0.35cm}

\hyphenation{dia-chro-nic}

%\usepackage[utf8]{inputenc} % set input encoding (not needed with XeLaTeX)
\usepackage{amssymb}
\usepackage{mathtools}
\usepackage{enumitem}
\usepackage{amsthm}
\usepackage{physics}
%\usepackage{ntheorem}
\usepackage{chngcntr}
\counterwithin{figure}{section}

\makeatletter
% \def\@endtheorem{\endtrivlist\@endpefalse }% OLD
\def\@endtheorem{\endtrivlist}%

\setlist[description]{font=\normalfont\scshape}

\catcode`\|=\active\def|{\mid}
\DeclarePairedDelimiter{\ceil}{\lceil}{\rceil}
\DeclarePairedDelimiter{\floor}{\lfloor}{\rfloor}
\newcommand{\Subj}{\mathbin{\raisebox{.15ex}{$\scriptscriptstyle{\Box}$}\kern-.425em\rightarrow}}
\def\Existence{E!}
\def\Believes{\operatorname{Believes}}
\def\True{\operatorname{True}}
\def\Perfection{\operatorname{Perfection}}
\def\ext{\operatorname{Ext}}
\def\Iff{\leftrightarrow}
\def\Implies{\rightarrow}
\def\Entails{\Rightarrow}
\def\Cov{\operatorname{Cov}}
\def\Equiv{\Leftrightarrow}
\def\Form{\operatorname{Form}}
\def\Informs{\operatorname{Informs}}
\def\technical{$\star$}
\def\vtechnical{$\star\star$}
\def\power{\wp}
\def\Nec{\Box}
\def\Poss{\Diamond}
\def\Prop#1{$\langle$#1$\rangle$}
\def\R{\mathbb R}
\def\N{\mathbb N}
\def\tele{tel\={e}}
\makeatletter
\newtheoremstyle{indented}{3pt}{3pt}{\addtolength{\leftskip}{4.5em}}{-2.5em}{\sc}{.}{.5em}{}
\def\Principle#1#2#3{\theoremstyle{indented}\newtheorem*{principle}{#2}\begin{principle}\def\@currentlabel{#2}\label{#1}#3\end{principle}\let\principle\undefined}
\makeatother
\def\pref#1{{\sc\ref{#1}}}
\def\enum#1{\resume{enumerate}\item #1\end{enumerate}}
\def\ditem#1#2{\begin{enumerate}[resume]\item \label{\mychapter:#1} #2\end{enumerate}}
\def\nitem#1#2{\begin{description}\item[#1\label{\mychapter:#1}] #2\end{description}}
\def\bref#1{\ref{\mychapter:#1}}
\def\dref#1{(\ref{\mychapter:#1})}
\def\drefglobal#1{(\ref{#1})}
\usepackage{graphicx} % support the \includegraphics command and options
\usepackage{array} % for better arrays (eg matrices) in maths
\def\Not{\operatorname{\sim}}
\def\St{\operatorname{St}}
\def\num{\operatorname{num}}
\def\And{\mathrel{\&}}
\def\Or{\vee}
\def\BigOr{\bigvee}
\def\<{\langle}
\def\>{\rangle}
\def\union{\cup}
\def\nleq{\not\le}
\def\N{\mathbb N}
\def\R{\mathbb R}
\def\C{\mathbb C}
\def\Powerset{\mathcal P}
\def\star#1{{}^*#1}
\def\hN{\star{\N}}
\def\hR{\star{\R}}
\def\Z{\mathbb Z}
\def\Power{\mathcal P}
\def\Implies{\rightarrow}
\def\True{\operatorname{True}}
\def\Socrates{\mathrm{Socrates}}
\def\actual{@}
\def\Law{\operatorname{Law}}
\def\Chance{\operatorname{Chance}}
\def\Var{\operatorname{Var}}

\def\H2O{H${}_2$O}

\def\scr{\mathcal}
\def\e{\varepsilon}
\def\eps{\varepsilon}
\newtheorem{lem}{Lemma}
\newtheorem{prp}{Proposition}
\newtheorem*{theorem}{Theorem}
\newtheorem{corollary}{Corollary}
\newtheorem{cond}{Condition}

\renewcommand\thechapter{\Roman{chapter}}

\def\chaptertail{\ifdefined\book\else\end{document}\fi}
 

\title{Infinity, Causation and Paradox}
\author{Alexander R. Pruss}
%\date{} % Activate to display a given date or no date (if empty),
         % otherwise the current date is printed

\begin{document}
\setcounter{secnumdepth}{3}
\setcounter{tocdepth}{4}

\end{document}
\fi

\restartlist{enumerate}

\chapter{Metaphysics}\label{ch:metaphysics}
\section{Composition}
Intuitively, your parts compose a whole, namely you. On the other hand,
if you choose exactly one atom from every star in every galaxy in the universe,
intuitively these scattered atoms do not make up any whole.

Yet it seems that one could produce a continuous series of worlds, adding, subtracting or moving 
one particle at a time, where the first item in the series consists of the scattered
atoms in different stars and the last item consists of you. Somehow, as one moves
along the sequence, at some point a new macroscopic object pops in (maybe it's 
you, or maybe it's something else), despite the variation between successive terms
in the sequence being extremely slight. What makes for the transition? ??ref:Sider

???vagueness, nihilism, Sider, etc.

As Tomaszewski has noted??ref, the Aristotelian, however, has a solution, namely
the invocation of form. After all, it is not possible to have a sequence where at
one end you have one atom from every star and at the other have all of your parts,
and the transition is particle-by-particle. For your parts include a human form,
while the scattered atoms contain at most the forms of particles or atoms. Adding,
substracting or moving particles is not enough: one needs to add a human form
somewhere in the sequence.

We can, further, give an account of when a plurality of objects, the $x$s, compose a 
substantial whole: namely, there is a form among the $x$s which informs all the other $x$s, 
and everything informed by that form overlaps with at least one of the $x$s:
\ditem{compose}{$\exists F[F\in xx\And \Form(F)\And \forall y((y\in xx \And \Not y=F) \rightarrow 
    \Informs(F,y))]$,}
assuming that the $\Form(x)$ predicate only applies to substantial forms.
Here, the informing relation is one where the form gives identity to the thing it informs.
\footnote{Understood this way, informing is a relation that holds both between an a substantial 
form and the material parts of the substance and between a substantial form and the accidents or 
accidental forms of the substance. One may, however, object that these are two different relations.
If so, take my $\Informs(x,y)$ relation to be a disjunction of the two.}

??why is the transition where it is?: Mersenne

\section{Identity over time}
One of the classic questions of metaphysics is about the grounds of identity over time. 
A very general way of posing the question is to ask for an explanation or ground of claims
of the form:
\ditem{identity-ground}{Object $x_1$ is identical to object $x_2$,}
where $x_1$ and $x_2$ respectively exist at times $t_1$ and $t_2$, which are presupposed to be different\footnote{One may worry that this
presupposition cannot be stated without using identity, i.e., without denying that
$t_1=t_2$. However, it is not clear that even if this is true, it affects the
significance of the question. Moreover, if time turns out to be linearly ordered,
then we can state the presupposition disjunctively: $t_1$ is earlier than $t_2$ 
or $t_2$ is earlier than $t_1$.} and where we require the explanation not to
involve identity and to involve only purely qualitative properties and 
relations.\footnote{If we allow the account to involve non-qualitative properties 
like Socrateity (the property of being Socrates), then we can offer a infinite 
account: $x_1$ is identical to $x_2$ provided that any property had by $x_1$ is 
had by $x_2$.}

Put in that very general way, it seems unlikely that we will have a solution, 
absent some extremely controversial metaphysical assumptions, such as 
Leibniz's Principle of Identity of Indiscernibles (PII).\footnote{The PII says
that two things are identical just in case they have the same purely qualitative
properties. If the PII holds, then we can say that $x_1=x_2$ if and only if 
for every purely qualitative property we have $Q(x_1)$ iff $Q(x_2)$.} 

But there is a somewhat less general way of putting the diachronic identity question.
Suppose that at time $t_1$, some proper plurality of items, the $x$s, compose an 
object and at time $t_2$ the $y$s compose an object. Then we ask for the grounds of:
\ditem{identity-comp-ground}{An object composed of the $x$s at $t_1$ is identical 
    with an object composed of the $y$s at $t_2$.\footnote{The reason for the
    indefinite pronoun is that, first, there might be more than one object composed
    of the same parts and, second, using the definite pronoun introduces another 
    instance of the identity relation given the Russellian analysis of ``The $F$ is
    $G$'' as saying that some $F$ is $G$ and has the property of being
    \textit{identical} with every $F$ that is $G$, whereas we only want an account
    of a single identity relation.}}
This is not asking for an account of diachronic identity in general, but of diachronic
identity of complex objects (note that we only require the $x$s to be a proper
plurality, i.e., for there to be more than one of them).
    
Here we \textit{can} give an Aristotelian account:
\ditem{identity-Arist}{There is a form $F$ such that at $t_1$, $F$ unites the $x$s,
    and at $t_2$, $F$ unites the $y$s.}
Here we might stipulate that a form $F$ unites the $z$s just in case $F$ is one of 
the $z$s, at least one of the $z$s is a non-form, every form among the $z$s informs 
each non-form among the $z$s, anything informed  by $F$ overlaps at least one of the $z$s:
\ditem{Unites}{$\operatorname{Unites}(F,zz) \equiv
[F\in zz\And \Form(F)\and\exists x(x\in zz\And \Not\Form(x))\And \forall G\forall x((\Form(G)\And\Not\Form(x)\And G\in zz\And x\in zz)
\rightarrow \operatorname{Informs}(G,x))\And \forall x(\operatorname{Informs}(F,x)\rightarrow 
\exists y(y\in zz \And O(y,x)))]$.}
Here $O(y,z)$ says that $y$ overlaps $z$,
i.e., $\exists w(w\le y\And w\le z)$, where $\le$ is the parhood
relation. There is no identity anywhere in \dref{Unites}.\footnote{Note that the right hand side of 
\dref{Unites} is much more complex than \dref{composes}, even though we are giving an account of a 
very similar phenomenon. A simpler formulation than \dref{Unites}, and more in line with \dref{composes}, 
would say that $F$ unites the $z$s just in case it is one of them and informs every one of the $z$s other 
than itself and everything informed by overlaps at least one of the $z$s. However, this formulation 
makes use of identity in talking of $z$ \textit{other than} $F$. To avoid the identity operator, we 
quantify over all the substantial forms among the $z$s and say that they inform all the non-forms. 
For the present account to work, we cannot have a case where all the things informed by a substantial
form $F$ are also informed by another substantial form $G$, i.e., it cannot be that all the non-substantial-form 
constituents of one substance $a$ are parts of another substance $b$. For if we had that, then the 
\dref{Unites} would make $F$ unite all the constituents of $a$ together with the form of $b$, which does not
seem right. On a natural interpretation of the metaphysics of conjoint twins (??cross-ref,??ref), there are common
parts that informed by two forms. Our assumption is compatible with that interpretation, but rejects the odd
possibility of conjoint twins where the common parts include \textit{all} the non-substantial-form constituents
of one of them. If our assumption is rejected, then we may need to make use of the identity operator in \dref{Unites}.
However, the identity operator would only need to be used in the case of where one of the relata is a form, and so 
we could still have an explanation of identity of substances in terms of identity of form, and metaphysical
progress will have been made.}

It may appear that \dref{identity-Arist} uses the concept of identity in claiming 
that the \textit{same} $F$ unites the $x$s as unites the $y$s. Even if that were so,
progress would have been made: an account of identity for complex would be given things 
in terms of identity for simple things. But in any case, we need not concede the point
as we can rewrite \dref{identity-Arist} in first order logic without any equal signs:
\ditem{identity-Arist-FOL}{$\exists F(\operatorname{Form}(F)\And\operatorname{Unites}(F,xx,t_1)
    \And\operatorname{Unites}(F,yy,t_2))$.}
Granted, \dref{identity-Arist-FOL} is logically equivalent to:
\ditem{identity-Arist-FOL}{$\exists F\exists G(\operatorname{Form}(F)\And\operatorname{Form}(G)\operatorname{Unites}(F,xx,t_1)
    \And\operatorname{Unites}(G,yy,t_2)\And F=G)$,}
but this only show that the identity is eliminable from \dref{identity-Arist-FOL2},
just as the identity can be eliminated from
\ditem{tallgreen2}{$\exists x\exists y (\operatorname{Tall}(x)\And \operatorname{Green}(y)\And x=y)$}
(``There is a tall object which is identical to a green object'') to yield:
\ditem{tallgreen}{$\exists x(\operatorname{Tall}(x)\And \operatorname{Green}(x)$}
(``There is a tall green object'').

Thus, \dref{identity-Arist} gives us an account of the identity of complex objects
that does not presuppose identity.\footnote{One might also worry that complexity
presupposes identity: an object is complex provided it has two distinct parts.
But the Aristotelian can make sense of complexity of objects by saying that an
object is complex provided that it has a part that is a form and a part that is not
a form.}

\section{Teleological animalism and cerebra}
Each premise of the following argument is very plausible.
\ditem{human}{We are humans.} 
\ditem{human-mammal}{Humans are mammals.} 
\ditem{mammal}{So, we are mammals.}
\ditem{mammal-animal}{Mammals are animals.}
\ditem{animalism}{So, we are animals.}

Nonetheless, the conclusion---labaled as ``animalism''---is denied by many philosophers. One traditional path to 
this denial is a Cartesian dualism on which we are immaterial souls that inhabit human animals. The other path 
is modern colocationist views on which we are a special kind of material object---a person---constituted by a human 
animal. The third path is brain views on which we are brains, or parts of brains (namely cerebra), which in turn are a 
proper part of a human animal. 

On all three paths, one will want to deny \dref{mammal}. Perhaps the best way to do so would be to distinguish ``humans'' 
in \dref{human} and \dref{human-mammal} into  human animals and human persons, and insist that premise \dref{human} is true 
only of human persons, while \dref{human} is true only of human animals. Thus the argument is unsound if ``humans'' is used 
consistently and otherwise invalid. 

In any case, the intermediate step \dref{mammal} gets denied. Instead, we are 
\textit{associated} with mammals, by ensoulment, constitution or parthood.   Yet \dref{mammal} is by itself extremely plausible. 
To deny that we are mammals seems akin to denying that earth
is round. Further, both the Cartesian and brain views imply that we
rarely if ever see or touch another person. One needs extremely good arguments for such counterintuitive theses. 

Probably the main candidate here is a family of arguments about the difficulties of accounting for cerebrum transplants
on animalism.??refs The simplest version is that if your cerebrum is removed from your skull and placed in a vat in such a way
that it can continue functioning, then intuitively you continue to think and come along with the cerebrum. But a cerebrum
is not an animal. On the contrary, the cerebrumless body appears to be an animal. After all, some animals (e.g., fish??) lack
cerebra, and it seems that the destruction of a cerebrum in an animal that normally has one would result in the animal 
becoming severely disabled rather than ceasing to exist. Thus, animalism points to the cerebrumless body as you---the cerebrumless
body is the same animal as you---and the cerebrum in the vat as something  or someone else, contrary to our intuitions.

Cartesians, colocationists and brain theorists who identify with the cerebrum have no such problem. They can all say that
the cerebrumless body is not you, and instead you inhabit or are colocated with or are just plain identical with the cerebrum
in the vat. 

Here I want to argue that an Aristotelian about humans can embrace a teleological variety of animalism on which it is natural to say that 
we go along with our cerebra. Now animalism is highly plausible as 
a view of the human person that does justice to the intuitions that humans are mammals, weigh about 80~kg and can be seen 
without surgery, but faces a serious cerebrum transplant problem. If Aristotelianism can help animalists overcome that problem,
that is some further evidence for Aristotelianism about humans.

There are two teleological features in the animalism I will sketch. The first teleological feature is the thesis 
that what defines somethingas an animal (and indeed an animal of a particular type) is its teleology. While it is usual to 
think of animals as things that nourish themselves, grow, reproduce, and have a certain level of autonomy from the enviroment, 
the teleological animalist instead insists that animals need not engage in these activities, but need only have a teleological
orientation towards them, need to be the sorts of things that \textit{should} engage in these activities. 

The second teleological feature is to see organisms, including humans, as having a teleological \textit{hierarchy}, with some
\tele{} subordinated to others. Sometimes the subordination is instrumental: our teeth rend food in order to nourish us. 
But there can also be a value-based subordination, where an activity, while not merely instrumental towards another activity, 
is less central to the flourishing of the organism and to the organism's identity. On the teleological animalism I am sketching,
it is postulated that in humans, activities common to all animals are subordinated to specifically personal activities, namely 
rational and moral behavior. ???can we have this in non-theistic versions??? wouldn't we have subordination to reproductive
activity???

When there is a hierarchical subordination, it is plausible to think that in cases of splitting, an organism is more apt
to come along with the organs supporting the higher-level features. If in humans it is moral and intellectual capacity 
that is at the top of the teoleological hierarchy, and the cerebrum is much more directly supportive of these capacities
than the lower brain, heart, lungs, etc., then we would expect the human organism to come along with the cerebrum. If you
cut a worm in such a way that one end contains just the head and the other contains the rest of the body, and both parts
behave as if they were alive, it is reasonable to suppose that the form of the original individual may go with the larger
piece which contains more in the way of life-supporting organs, rather than with the head, because the head is less 
teleologically central to the worm than to us. But given human teleology, we would expect us to go along with the head,
if the head were given life support, or even with the brain or cerebrum.

What about the remaining cerebrumless human body, which maintains its vital functions? If the original individual goes along
with the cerebrum, and if we take the idea of one human form in two bodies to be absurd, there are three possibilities 
for the cerebrumless body:
\begin{itemize}
\item[(a)] the cerebrumless body gains a new human form, or 
\item[(b)] it gains some non-human form, or 
\item[(c)] it becomes a non-living substance or a formless heap of matter. 
\end{itemize}

If the cerebrumless body gains a new human form, then we have the counterintuitive consequence that a temporary removal
of a cerebrum followed by reimplantation would either result in conjoined twins---two human beings joined at the edges
of the cerebrum---or would result in the death of one or more of the two human beings. None of these options seems very
plausible, but at the same time, we should not be too surprised if strange things happen when you move cerebra around!

If the cerebrumless body gains a non-human form, this is presumably an organismic form, since the entity is capable of
nourishment, reproduction, etc. But we have something moderately puzzling: a non-human organism
that would produce a human being if it were to mate with a human or with another organism of the same sort but of the
opposite sex. Moreover, it seems plausible that if the cerebrumless body has a teleology at all, that teleology impels
it to try to support the cerebrum: oxygen would be directed by the body towards the missing cerebrum, presumably. This
suggests that the being is incomplete without the cerebrum. But a being that ought to have a human cerebrum seems to be 
a human being. 

The last option is a formless heap of matter or a non-living substance (such as a body of water might be on some 
Aristotelian theories??refs). This does not 
seem particularly puzzling in the transfer case. 
If the form departs, by going along with the cerebrum, then formlessness would seem to be the obviously expected result, 
barring some special reason to the contrary.

What if instead of the cerebrum being removed and put on life-support, the cerebrum is simply destroyed? This corresponds
to a tragic real-life scenario: upper brain death. Bioethicists disagree on whether upper brain death is death.??refs
We have four moderately plausible options for cerebral destruction:
\begin{itemize}
\item[(i)] the original individual continues to live, i.e., the cerebrumless body retains the original human form??backref-to-identity, or
\item[(ii)] the original individual dies and the cerebrumless body gains a new human form, or
\item[(iii)] the original individual dies and the cerebrumless body gains a non-human form, or
\item[(iv)] the original individual dies and the cerebrumless body has no organismic form, and is either a non-living substance or a heap.
\end{itemize}

Between (a)--(c) and (i)--(iv), there are twelve combinations with various intuitive connections between them. 
However, we can intuitively reduce the number of options quite significantly. I have assumed that in the transfer
case, the original form goes along with the cerebrum in the transfer case, departing from the cerebrum. Suppose that
(i) is false. Then the original individual dies and the cerebrumless body is deprived of its original form. It seems
very plausible that what happens to the cerebrumless body upon deprivation of its original human form should not depend
on what that form does after departing the cerebrumless body---whether it continues to inform a reduced body (the cerebrum), 
or survives disembodied as many religious people think, or perishes. Thus, if (i) is false, then (a), (b) or (c) holds
respectively if and only if (ii), (iii) or (iv) holds. 

We thus have three plausible options with (i) false:
\begin{itemize}
\item[($\alpha$)] (a) and (ii)
\item[($\beta$)] (b) and (iii)
\item[($\gamma$)] (c) and (iv)
\end{itemize}

What if (i) is true, so that in the case of cerebral destruction, the original form continues to inform the cerebrumless body? 
Does this  tell us anything about what happens in the transfer case? One might initially think that the falsity of (i) fits
poorly with any of (a), (b) and (c). After all, if destruction of the cerebrum results in the form staying with the cerebrumless
body, shouldn't we expect the form remain with the cerebrumless body when the cerebrum is transfered?

But this is not clear on teleological animalism. For we might think that in division or partial destruction of the body, the form 
goes along with the part that is the best candidate for being informed by it, at least when there is a unique best candidate and that best 
candidate is ``good enough''. On the teleological account, the quality of candidacy for being informed is measured by how high in the teleological hierarchy are
the goals directly promoted by the part. If the part promotes goals at the top of the hierarchy, then the candidate is automatically
good enough. Thus, when the cerebrum survives, it is reasonable to think the form goes along with the cerebrum, because the cerebrum
is good enough as a candidate, and higher in the hierarchy than the cerebrumless body. But if the cerebrum is destroyed, the cerebrumless
body is the unique best candidate, because it is the only candidate. Whether the cerebrumless body is good enough as a candidate is not clear, but neither is it
clear that it is not good enough. It is, after all, on the next step down in the hierarchy after the cerebrum, having among its tasks the full
support of the cerebrum's functioning, as well as many important purely animal functions. Thus, all of the following are at least somewhat reasonable 
epistemic possibilities:
\begin{itemize}
\item[($\delta$)] (a) and (i)
\item[($\e$)] (b) and (i)
\item[($\zeta$)] (c) and (i)
\end{itemize}

On teleological animalism we thus have six combinations for making sense of what happens to the form and cerebrum, namely ($\alpha$)--($\zeta$), 
and none of them appear immensely problematic, though (a) and (b) have some moderately counterintuitive consequences. But even if we feel
the need to reject these, that still leaves us with ($\gamma$) and ($\zeta$): in cerebral transfer cases, the cerebrumless body is not a living
thing, while in cerebral destruction cases, the cerebrumless body may (if we have (i)) or may not (if we have (iv)) continue to be a living
human being. 

One may think ($\zeta$) is implausible, because it implies that whether the body-minus-cerebrum continues to have the original human
form depends on what happens to the cerebrum---whether it is merely removed or actually destroyed. But such dependence is, as already
noted, to be reasonably expected. For if we think of the cerebrum as a magnet for the original human form, then transfer of that ``magnet''
might be reasonably be thought to pull the form along, hence depriving the cerebrumless body of it, while destruction of the ``magnet'' might
well leave the form in place. 

Teleological animalism, thus, has multiple ways of making sense of what happens in cerebral transfer and destruction cases, while these cases
are highly problematic to other types of animalism. Given the plausibility of animalism as such, this gives us another reason to accept
teleological animalism.

\section{Ill-matched matter, rearrangement, the power to continue existing and immortality}
\section{Naturalism}
Is the Aristotelian hylomorphic account of humans compatible with naturalism? This depends on how naturalism
is defined and whether we take the theistic version of the account or not.

Since theism implies the existence of a non-natural causally efficacious substance, namely God, it will be incompatible
with most versions of naturalism. And I have argued theat the Aristotelian account is unsatisfactory without
theism. 

Still, it is an interesting question whether the what the Aristotelian account says about the forms of finite
substances is compatible with naturalism. If so, then one could combine the Aristotelian account with 
a naturalism restricted to finite objects, and save some naturalistic intuitions. Furthermore, it would mean
that an Aristotelian not convined by the arguments that theism is needed to make the theory satisfactory could
be a full-blown naturalist.

If we take naturalism to say that the only entities that exist are the ones that would figure in a completed science,
then it is unlikely that Aristotelian metaphysics would be compatible with naturalism about finite objects. However,
such a strong naturalism would likely also conflict with many other metaphysical theories that are rarely taken to 
contradict naturalism. 

For instance, consider theories of time. Of the theories of time, four dimensionalism, on which
ordinary objects like ourselves are extended in time as well as space, is what seems to fit best with Relativity Theory. 
But the most common four-dimensionalist view of changing properties is perdurantism: changing objects are made of temporal parts or
slices to which the properties are primarily attributed, so that a tomato that once was green and now is red has a green temporal part
and a red temporal part. But slices are unlikely to figure in a completed science if
our current science is a good guide to that. Consider that our current physics does not consider particles
like electrons and quarks to be fundamental, and hence not made up of smaller parts. But electrons and quarks change over
time (e.g., with respect to spin and flavor), and so they would need tos have temporal parts.  But these parts are not found
in our physics. 

Or consider that our science may quantify over physical objects like particles and fields, and applies predicates to them, but 
does not quantify over properties. Thus, properties, whether understood as universals or as tropes, go beyond current science.
Yet it seems implausible to understand naturalism as implying nominalism.

One may weaken naturalism to say that the only \textit{causally relevant} entities are those of a completed science. But 
this would still rule out perdurantism, since objects change with respect to causally relevant properties, and hence their
temporal parts have these properties, and have them more fundamentally. It would also likely rule out many versions of trope
theory, since the causal efficacy of tropes is supposed to explain the causal efficacy of the objects made of them.

Let's step back. Naturalism denies the existence of causally efficacious ``supernatural'' entities like 
ghosts, but it is neutral on temporal parts and tropes of ``natural'' entities like electrons. 
What is the relevant difference between ghosts and temporal parts of electrons? It seems to be this. While
neither entity is posited by science, if ghosts exist, then certain phenomena that it belongs to science 
to investigate have no scientific explanation, barring systematic overdetermination. If there are ghosts,
they are presumably responsible for at least some of the appearances of ghosts, some of the chills people
feel in graveyards, etc. These phenomena then either have no physical explanation at all, or by a massive
coincidence are overdetermined by a physical and a spectral cause. There is a competition, thus, between 
scientific and spiritual explanations when we are concerned with ghosts. 

On the other hand, there is neither
competition nor overdetermination in the case of the objects studied by science and their ontological
constituents such as temporal parts or tropes. When a temporal part or a trope of an ice cube makes 
your hand cold, the ice cube also makes your hand cold. The ice cube's cooling causal influence on your hand
does not compete with the ice cube's temporal part's causal influence on you or on the causal influence of 
a trope of coldness (if there are temporal parts of ice cubes or tropes of coldness). Nor is this overdetermination,
since it is not overdetermination when an object $C$ causes an effect $E$ by means of $C$'s part. After all,
it is not overdetermination when the ice cube cools the hand it is lying on \textit{and} the lower half of 
the ice cube cools the hand. 

We might thus want to say that naturalism holds that any phenomenon that it falls within the purview of science 
to seek for causal explanations of either has no causal explanation at all, or else has a scientific causal explanation 
and is not overdetermined by a scientific and a non-scientific one. On this account, theism is incompatible with
naturalism if there is a first state of physical reality. For if there is such a state, it belongs
to science to seek for its causal explanation. However, there is a theistic explanation of that state and no 
scientific explanation. We thus have the kind of competition that naturalism rules out.\footnote{Matters are a little more
complicated if there is no initial physical state, but we can apply the above argument to an initial
segment of physical states: science cannot explain that segment but theism claims to do so, and yet it lies
in the scope of science to ask for such an explanation.} One might object that science knows that it cannot, on pain
of vicious circularity, provide a scientific explanation of the first physical state, and hence it does not belong to
science to seek that explanation. But this objection confuses the first physical state \textit{qua} first and the
first physical state as it intrinsically is. It does not belong to science to seek the explanation of the first physical
state \textit{qua} first. But if we just consider it as a specific physical state---particles and fields arranged thus-and-so---then
it certainly belongs to science to seek for its explanation, though that search will be rightly abandoned if sufficient evidence
is gathered that the state is in fact the first one.

Now, let's go back to Aristotelianism. Bracketing theism, the entity that might trouble the naturalist is form.
But the forms of things are components of substances---humans, dogs, oaks, etc.---that are also found in our current
science (e.g., biology) and are likely to be found in the completed version of that science. And there is neither
competition(??what's that? remove here and earlier?) nor overdetermination between causal explanations provided by
forms and their substances. As long as the substances do not have spooky causal powers beyond the scope of science,
adding forms to the ontology no more contradicts a plausibly defined naturalism than perdurantism or trope theory
does.

It is, of course, open to the Aristotelian to suppose that some finite substances, whether natural like humans and oaks 
or  supernatural like angels or ghosts, have causal powers of a sort that goes beyond the scope of science, though
specifying what those powers would have to be like is difficult. But nothing in the applications given for Aristotelianism
posited such powers. Thus as far as the argument of this book goes, there is no contradiction with a carefully
specified naturalism with respect to finite things.\footnote{I think the most plausible place to look for a tension 
would be in examining human free will, but that goes beyond the scope of this book.}

\chaptertail 

\def\mychapter{IX}
\ifdefined\book
\else
\documentclass[11pt,oneside]{amsbook}
\usepackage[backend=biber, citestyle=authoryear]{biblatex}
\usepackage{mathpazo}
\usepackage{graphicx}
\usepackage{amsmath}
\usepackage{tikz}
\usetikzlibrary{arrows}
%\usepackage{titlesec}
\addbibresource{bibliography.bib}
\newcommand\posscite[1]{\citeauthor{#1}'s (\citeyear{#1})}
\newcommand\plural[1]{#1\mathrm{s}}
%\def\posscitewithextra[#1]#2{\citename{#2}'s (\citeyear{#2}, #1)}

%\newcounter{subsubsubsection}[subsubsection]
%\renewcommand\thesubsubsubsection{\thesubsubsection.\arabic{subsubsubsection}}
%\titleformat{\subsubsubsection}
%  {\normalfont\normalsize\bfseries}{\thesubsubsubsection}{1em}{}
%\titlespacing*{\subsubsubsection}
%{0pt}{3.25ex plus 1ex minus .2ex}{1.5ex plus .2ex}

\ifdefined\book
\renewcommand{\thechapter}{\Roman{chapter}}
\else
\renewcommand{\thechapter}{\mychapter}
\fi

\linespread{1.7}
\usepackage[margin=1.25in]{geometry}
\sloppy
\makeatletter
%% TODO: This is a cheat. It's supposed to be {paragraph}{4}, and that's 
%% what it is in amsbook.cls, but then it fails.
\def\paragraph{\@startsection{paragraph}{3}%
  \normalparindent\z@{-\fontdimen2\font}%
  \normalfont}
\def\subsubsubsection{\paragraph}
\makeatother

\def\smalltick{0.15cm}
\def\bigtick{0.3cm}
\def\pointcircle{0.08cm}
\def\causalnode{0.35cm}

\hyphenation{dia-chro-nic}

%\usepackage[utf8]{inputenc} % set input encoding (not needed with XeLaTeX)
\usepackage{amssymb}
\usepackage{mathtools}
\usepackage{enumitem}
\usepackage{amsthm}
\usepackage{physics}
%\usepackage{ntheorem}
\usepackage{chngcntr}
\counterwithin{figure}{section}

\makeatletter
% \def\@endtheorem{\endtrivlist\@endpefalse }% OLD
\def\@endtheorem{\endtrivlist}%

\setlist[description]{font=\normalfont\scshape}

\catcode`\|=\active\def|{\mid}
\DeclarePairedDelimiter{\ceil}{\lceil}{\rceil}
\DeclarePairedDelimiter{\floor}{\lfloor}{\rfloor}
\newcommand{\Subj}{\mathbin{\raisebox{.15ex}{$\scriptscriptstyle{\Box}$}\kern-.425em\rightarrow}}
\def\Existence{E!}
\def\Believes{\operatorname{Believes}}
\def\True{\operatorname{True}}
\def\Perfection{\operatorname{Perfection}}
\def\ext{\operatorname{Ext}}
\def\Iff{\leftrightarrow}
\def\Implies{\rightarrow}
\def\Entails{\Rightarrow}
\def\Cov{\operatorname{Cov}}
\def\Equiv{\Leftrightarrow}
\def\Form{\operatorname{Form}}
\def\Informs{\operatorname{Informs}}
\def\technical{$\star$}
\def\vtechnical{$\star\star$}
\def\power{\wp}
\def\Nec{\Box}
\def\Poss{\Diamond}
\def\Prop#1{$\langle$#1$\rangle$}
\def\R{\mathbb R}
\def\N{\mathbb N}
\def\tele{tel\={e}}
\makeatletter
\newtheoremstyle{indented}{3pt}{3pt}{\addtolength{\leftskip}{4.5em}}{-2.5em}{\sc}{.}{.5em}{}
\def\Principle#1#2#3{\theoremstyle{indented}\newtheorem*{principle}{#2}\begin{principle}\def\@currentlabel{#2}\label{#1}#3\end{principle}\let\principle\undefined}
\makeatother
\def\pref#1{{\sc\ref{#1}}}
\def\enum#1{\resume{enumerate}\item #1\end{enumerate}}
\def\ditem#1#2{\begin{enumerate}[resume]\item \label{\mychapter:#1} #2\end{enumerate}}
\def\nitem#1#2{\begin{description}\item[#1\label{\mychapter:#1}] #2\end{description}}
\def\bref#1{\ref{\mychapter:#1}}
\def\dref#1{(\ref{\mychapter:#1})}
\def\drefglobal#1{(\ref{#1})}
\usepackage{graphicx} % support the \includegraphics command and options
\usepackage{array} % for better arrays (eg matrices) in maths
\def\Not{\operatorname{\sim}}
\def\St{\operatorname{St}}
\def\num{\operatorname{num}}
\def\And{\mathrel{\&}}
\def\Or{\vee}
\def\BigOr{\bigvee}
\def\<{\langle}
\def\>{\rangle}
\def\union{\cup}
\def\nleq{\not\le}
\def\N{\mathbb N}
\def\R{\mathbb R}
\def\C{\mathbb C}
\def\Powerset{\mathcal P}
\def\star#1{{}^*#1}
\def\hN{\star{\N}}
\def\hR{\star{\R}}
\def\Z{\mathbb Z}
\def\Power{\mathcal P}
\def\Implies{\rightarrow}
\def\True{\operatorname{True}}
\def\Socrates{\mathrm{Socrates}}
\def\actual{@}
\def\Law{\operatorname{Law}}
\def\Chance{\operatorname{Chance}}
\def\Var{\operatorname{Var}}

\def\H2O{H${}_2$O}

\def\scr{\mathcal}
\def\e{\varepsilon}
\def\eps{\varepsilon}
\newtheorem{lem}{Lemma}
\newtheorem{prp}{Proposition}
\newtheorem*{theorem}{Theorem}
\newtheorem{corollary}{Corollary}
\newtheorem{cond}{Condition}

\renewcommand\thechapter{\Roman{chapter}}

\def\chaptertail{\ifdefined\book\else\end{document}\fi}
 

\title{Infinity, Causation and Paradox}
\author{Alexander R. Pruss}
%\date{} % Activate to display a given date or no date (if empty),
         % otherwise the current date is printed

\begin{document}
\setcounter{secnumdepth}{3}
\setcounter{tocdepth}{4}

\end{document}
\fi

\restartlist{enumerate}

\chapter{Laws of nature and causal powers}\label{ch:laws}
\section{Humean and pushy laws}
\subsection{Deterministic versions}
There are two main views of laws of nature. On Humean views, we have laws simply in virtue of non-causal 
regularities of the behavior of objects in nature, and causation is then grounded in the laws. On pushy views, 
we have laws in virtue of metaphysical components of reality that affect or constrain the behavior of objects. 
In this section, I will argue that pushiness is the right view. 

Introduce Lewis??

??causation

??argument from ideological parsimony

\subsection{Too much power}
Now imagine a universe consisting of a single quantum ``coin toss'' performed a very large number of times. Observing some 
infinite sequences of these coin tosses, such as $HHH...H$, $TTT...T$, $HTHT...HT$ or $THTH...TH$, would make us confident 
that the coin tosses are deterministic. But even though this would make us confident of determinism (e.g., $HTHT...HT$
would make us confident that there is a law that each toss is followed by its opposite), any such sequence could also
occur without determinism. Not so on our Humean story. On our Humean story, any one of these global regularities \textit{logically 
guarantees} a deterministic law in a world consisting of a single coin tossed repeatedly. This is highly counterintuitive. 
Furthermore, suppose that we in fact have a ``patternless'' sequence that to the Humean yields indeterministic law that says 
that each coin toss is independent and fair. But while the statement that each coin toss is independent and fair should allow any 
finite sequence of coin tosses, it turns out that certain sequences, such as our four examples above, are logically incompatible with 
independence and fairness on our Humeanism.

Or consider magic. Suppose you live in a multiverse consisting of two physical universes, each of which has laws of nature roughly like 
ours. Now, plausibly, any state of a physical universe like ours can be encoded as a countable sequence of real numbers (i.e., a finite
sequence, or one that can be enumerated using the natural numbers: $x_0,x_1,x_2,...$). For instance, suppose the universe is made up of 
a countable cardinality of particles, each of which has a finite number of properties naturally expressible as one or a finite number of real 
numbers (e.g., mass and charge can be expressed as one number, and position can be expressed as three given a coordinate system), which
properties either change continuously or have a countably infinite number of times of discontinuity. Then all we need in order to fully
describe the system is to specify the properties at a countable number of times (e.g., the times at which discontinuities happen and
all times that can be expressed by a rational number), and the resulting description can be expressed as a countable sequence of real 
numbers.\footnote{\label{note:product}$^*$This argument uses the fact that if $A$ and $B$ are countable sets, then the Cartesian product set $A\times B$ of pairs $(a,b)$ with
$a$ from $A$ and $b$ from $B$ is also countable. To see this, note that if we can enumerate $A$ as $\{ a_0,a_1,a_2,... \}$ and
$B$ as $\{ b_0,b_1,b_2,... \}$, then we can enumerate $A\times B$ as $(a_0,b_0), (a_1,b_0), (a_0,b_1), (a_2,b_0), (a_1,b_1), (a_0,b_2),...$.
The trick here is to first enumerate the one way where the indices add up to $0$ ($0=0+0$), then the two ways where the indices add up to
$1$ ($1=1+0=0+1$), then the three ways where the indices add up to $2$ ($2=2+0=1+1=0+2$), and so on.} But a countable sequence of real
numbers can, with a well-known trick, be expressed as a single real number.\footnote{$^*$Any real number can be remapped to the range
from $0$ to $1$, exclusive, by taking $x$ to $(1/2)+(1/\pi)\arctan x$. Any countably infinite sequence of numbers $x_0,x_1,x_2,...$ between $0$ and $1$ 
exclusive can then be expressed as a sequence of decimal numbers where $i$th number is of the form $0.x_{i,0}x_{i,1},x_{i,2},...$ (with 
the convention that an infinite terminal sequence of nines is preferred to an infinite terminal sequence of zeroes, say). One can then 
encode all these numbers into a single number of the form 
$0.x_{0,0}x_{1,0}x_{0,1}x_{2,0},x_{1,1},x_{0,2}...$ (this is basically the same pattern as in Note~\ref{note:product}), so that every
digit of every one of the numbers in our initial series occurs at a determinate position in the encoded number. And \textit{a fortiori}
if one can encode any countably infinite sequence of reals into a single real, one can encode any finite sequence of reals into a single
real.} And if instead of a classical particle system one prefers a quantum story, then note that most models of quantum mechanics make 
the wavefunction be a vector in a separable Hilbert space---and the cardinality of the set of vectors in a separable Hilbert space is the 
same as the cardinality of the set of all real numbers. Adding a countable number of discontinuities in case of collapse, we can still
encode the state of the universe as a single real number.

Now, any real number can be encoded into the tilt angle of a physical rod (e.g., encoding $x$ into a rod tilted---with respect to some
axis---at angle $\arctan x$). Thus, we can encode the complete physical state, over all of time of one of the two universes into the tilt angle of a physical rod
in the other universe. Now fix some encoding that can be specified in a relatively brief way. Suppose now that you are in one of the 
universes, and at a precisely specifiable moment (say, one that is a precise number of Planck times since the beginning of your universe)
you tilt the rod at angle that happens to match the state of the other universe. Then including the information that the 
rod angle at this moment matches the complete state of the other universe will indeed provide a vast amount of information about your 
multiverse in a fairly brief compass, and hence would be included in the optimal Lewis-Ramsey description, and will be a law. Thus, by
tilting a rod at a specific angle---and surely any rod tilt angle is physically possible for you (though maybe not possible to induce
\textit{intentionally})---you can create a law correlating the rod angle with the other universe. Thus, by waving a rod, you can make
the rod be a vastly informative dowsing rod that by law and not merely by coincidence carries complete information about another universe.
This is highly implausible magic!

\subsubsection{A plurality of bestnesses}
A problem that will be familiar from many of our earlier discussions is that there are many free parameters in the account of 
the bestness of the best system. For instance, given a measure of informativeness $I(T)$ and length $L(T)$ of a system
$T$, we will want to combine them into a measure of quality of theory $f(I(T),L(T))$ with some sort of value function $f(x,y)$ such that $f(x,y)<f(x',y)$ 
and $f(x,y)>f(x,y')$ when $x<x'$ and $y<y'$. But there are infinitely many functions satisfying these inequalities. Perhaps
with some thought we can find some more reasonable constraints, but it is very implausible to think we can reduce the space
of reasonable candidates to one.

Moreover, neither $I(T)$ nor $L(T)$ has a unique privileged candidate.

If a world has a phase space---say, defined by values of various natural determinables like charge, position and momentum at various 
times---it makes sense to think of the informativeness of a theory $T$ as inversely related to the size of the set of trajectories through 
phase space (i.e., functions from time to phase space) compatible with $T$. If we could get the set of allowed trajectories down to one,
that would be maximal informativeness. 

But how do we measure the size of the set of trajectories compatible with the theory? The set-theoretic cardinality of the set of allowed
trajectories for many theories that intuitively vary significantly in their informativeness will be the same. For suppose our determinables
are position. Now consider theory $T_1$ according to which there is a single particle which for all time is found in the same position
in three-dimensional Euclidean space, and $T_2$ according to which the the particle moves through three-dimensional Euclidean space over
a continuous trajectory. Clearly, $T_1$ is much more informative than $T_2$. But the cardinality of the space of trajectories allowed by
$T_1$ is \textit{the continuum}, the cardinality of the real numbers.\footnote{A position can be encoded as three real numbers. Any real
number can be recoded to be between $0$ and $1$ (use the function $f(x)=(2/\pi)(\pi/2 + \arctan x)$. Any three numbers between $0$ and
$1$ can be encoded in a single number. For instance, the decimal numbers with the digits $0.x_1x_2x_3...$, $0.y_1y_2y_3...$ and 
$0.z_1z_2z_3...$ can be encoded as the single number $0.x_1y_1z_1x_2y_2z_2x_3y_3z_3...$. ??check infinite nines.} However, that is also
the same as the cardinality of the space of all continuous trajectories, which is what $T_2$ allows.\footnote{To specify a continuous
trajectory, one only needs to specify its values at a countable number of times---say, all times that are represented by rational 
numbers. Thus one needs to specify a countable number of real numbers. But a somewhat more complicated interweaving allows that to be
coded in a single real number.??ref and ??backref}

We might, on the other hand, try to define the size of the set of trajectories as a volume rather than a cardinality. 
An immediate problem may seem to be that of the choice of units of volume. This is not a serious problem for comparing
degrees of informativeness,  as long as we measure the volumes of the set of allowed trajectories using the 
same units. Even completely arbitrary units like the length of Charles III's forearm and the time between his mother's
and his own coronations can be used for comparative purposes. However, the scaling issue is relevant for the combination
problem. For unless our combination function $f$ has some additional special properties, like being linear in the second variable
(i.e., $f(x,\alpha y)=\alpha f(x,y)$), the choice of scaling may still be an issue, because it might be that $f(x,y)<f(x,y')$ but 
$f(x,\alpha y')<f(x,\alpha y)$ for some choice of $x$, $y$, $y'$ and $\alpha$. On the other hand, we might use the scaling worry
to justify a linearity constraint on the second variable in $f$, thereby reducing the arbitrariness of the choice of $f$.

A more serious problem is 
cases of two theories that clearly vary in information content will both allow a set of trajectories with zero volume. For instance, 
suppose a world with only one moment of time, and one determinable: three-dimensional position of a single
particle. A theory that requires the particle to be at the coordinates $(0,0,0)$ reduces the space of trajectories to a subset of zero
volume, as does a theory that mere constrains the particle to have the first coordinate zero. The first theory requires the particle to 
be at the origin and the second to lie in the $yz$-plane. But a point and a plane both have zero volume. 

We might find a way of comparing sets of zero volume. For instance, we might try to find the Hausdorff dimension (which may fractional)
of the two sets, and deem the set with higher dimension to convey less information. And then we could use Hausdorff measure to compare
sets of the same Hausdorff dimension. But what would we do about sets both of whose Hausdorff measures are infinite? Moreover, the above
approach leads to a pair of numbers, $(\alpha,\beta)$, where $\alpha$ is the Hausdorff dimension of the allowed subset of phase space
and $\beta$ is the $\alpha$-dimensional Hausdorff measure. Thus, instead of combining length with a single number, we need to combine length
with \textit{two} numbers. And what do we do in the case of sets whose Hausdorff measure is infinite with respect to the Hausdorff dimension---these
may need to be compared as well. Furthermore, Hausdorff dimension itself
is not the only way to formalize the concept of dimension.???

Thus we should not expect a canonical measure of informativeness: there will be many free parameters.

Now, length may seem more tractable. Indeed, given a fixed language whose sentences are
finite strings of characters from a finite symbol set, there is no difficulty in making sense of the length of the briefest 
expression of a system of propositions in that language. 

But there are many languages. Just think of such decisions as the choice of grouping notation. In recent
use in logic, for instance, we have parenthesis notation, Polish notation and dot notation. Surely there are many more
reasonable grouping notation. Or think which logical primitives should be included. For truthfunctional connectives,
nand is sufficient, as is nor, as is either of the pairs and-not and or-not, but why should we limit ourselves to a minimal 
sufficient set. Perhaps we should allow all the binary connectives. Or perhaps some but not all. Or perhaps all the ternary
ones. Or perhaps just one seven-place nor. It is reasonable to think there is a privileged set of predicates---the perfectly 
natural ones. But is it reasonable to think there is a privileged set of logical operators? This is murky. Or a privileged
grouping notation? That seems even more dubious.

We thus have little reason to hope in a single distinguished measure of bestness. Now, we might hope that for a wide range $R$ 
of measures of the quality of a theory in a world, there is sufficient overlap between the best theories to ensure that
the things our best science will converge on as the laws will in fact be entailed by the theories that are best according to the 
measures in $R$. We could, then, define a law as a proposition $L$ such that for all quality measures $Q$ in $R$, the $Q$-best theory
entails $L$. But specifying the boundaries of $R$ will be subject to difficulties very similar to those in defining a single canonical 
measure $Q$.

It seems we cannot escape the idea that there is significant vagueness in the concept of a law for the Humean.
Is this a problem?

One relevant question is whether the nomicity of a law is itself a part of a scientific explanation. One might well think so.
There is nothing unnatural in hearing a scientist say that planets move in elliptical orbits because
by a law of nature objects attract gravitationally in proportion to their mass and the inverse square of distance, while the
sun's mass is much larger than the mass of any other objects in the vicinity and the kinetic energy of the planets is not 
too high, and by theorems proved by Newton an inverse square law of attraction produces elliptical orbits when the kinetic
energies are not too high. Here a part of the story is that there is a law of nature.

If this is true, and if an account of what it is to be a law of nature is a very complex statement involving a measure
of the quality of theories, and various linguistic facts about lengths of expressions, then our scientific explanations in 
terms of laws are much more complicated than we likely thought. While the concepts figuring \textit{in} the laws may be very
natural, the concept of a law is itself rather unnatural. Moreover, the plurality of bestnesses implies that there is a serious
vagueness in the concept of a law, which makes our scientific explanations, even in the most precise areas of fundamental physics,
full of vagueness. Thus the Humean should probably deny that claims about the nomicity of laws enter into scientific 
explanations.\footnote{An additional argument for this is as follows. Consider a law that says that every $x$ is $F$. If the nomicity
of laws is explanatory, then the fact that every $x$ is $F$ is intuitively explained by the fact that it's a law that every $x$ is $F$.
But what does the fact that it's a law add on the Humean account? Only some facts about brevity and informativeness. And \textit{these}
facts don't seem to contribute to explaining why every $x$ is $F$. It is better, thus, for the Humean to deny that facts about the 
laws \textit{as} laws do not enter into explanations.} There is a cost to this move, of course. As noted, it feels very natural
to give explanations in term of statements about what is law. 

On the other hand, on the old deductive nomological
model of explanation, a scientific explanation was a deductively valid argument for the explanandum with at least one of the premises
being a law. For instance:
\ditem{DN1}{All massive object attract gravitationally.}
\ditem{DN2}{The sun is a massive object.}
\ditem{DN3}{So, the sun attracts gravitationally.}
Here, the law is \dref{DN1}. But note that it is not a part of the statement in \dref{DN1} \textit{that} it is a law. 
Suppose that this is how we see laws entering into explanations: the content of the law does the explaining, but not
its nomicity. Then vagueness of the concept of law would not affect the propositions \dref{DN1} and \dref{DN2} constituting the
above explanation.  

However, vagueness of the concept of law would still affect what counts as an explanation, assuming that generalizations that
are not laws are not allowed in explanations, at least not in the place where we would put a law. We end up damaging the
objectivity of the concept of explanation on this approach: it becomes a linguistic question what is and is not an explanation. 
And if explanation is significantly relevant to justification---for instance, via inference to best explanation---we damage the 
objectivity of the concept of justification.

Our now familiar Aristotelian solution to issues of unacceptable vagueness, which is to ground boundaries in human nature,
is not plausible in the case of laws of nature, because doing so would render laws of nature too anthropocentric. (Granted,
however, we might be able to use the Aristotelian solution to resolve any infection of vagueness to the concept of justification,
since our justification is appropriately anthropocentric.)

??refs on vagueness of laws

\subsection{Indeterministic extensions}
BSA was initially formulated for deterministic laws. But what about indeterministic laws, such as that some quantum setup has a
chance $1/3$ of resulting in outcome $A$---or, for homey simplicity, that a certain indeterministic coin toss has chance $1/2$
of heads? The standard move is to go to a Probabilistic BSA (PBSA). In BSA, we required all the claims of the theory to be true.
In PBSA, we only require non-probabilistic claims to be true. However, now, in addition to optimizing informativeness and 
brevity, we also optimize \textit{fit}, where we check how well the outcomes fit probabilitic predictions. A theory which claims 
there is some large number $N$ of independent fair (i.e., the chances of heads and tails are equal??backref for first use of `fair') 
coin tosses will tend to better fit worlds where the number of heads is closer to $N/2$ than worlds where the number of heads is 
further from $N/2$. Thus, we optimize informativeness, fit and brevity over theories whose non-probabilistic content is true.\footnote{If we 
want a bit more elegance in the formulation, we can drop the requirement that the non-probabilistic content is true and instead
stipulate that a non-probabilistic statement has good fit when it is true but infinitely bad fit when false, so no theory with false
non-probabilistic statements will win the crown of being the best system.}

\subsubsection{Violations of the Principal Principle}\label{sec:principal-principle}
??chances vs probabilities

Suppose the world consists of some large number $N$ of independent fair indeterministic coin tosses, where $N$ is large
and easily mathematically expressible, e.g., $N=2^{256}$. As we would expect, very close to half of the coin tosses are heads, and 
we suppose that a probabilistic best systems view will have laws that correctly assign an independent chance of $1/2$ (or, if 
one wishes,  something close to $1/2$, a complication I will ignore) to each toss. Moreover, because $N$ is easily expressible,
and expressing it conveys a significant amount of information about the world, the laws also state that the number of tosses is
$N$. 

Now, the following version of van~Fraassen's Principal Principle is very plausible:
\ditem{LPP}{If from a law $U$ it can be proved that the chance of $E$ is $p$, then $P(E\mid \Law(U)) = p$,}
where $\Law(U)$ says that $U$ is a law.
Here is another very plausible thesis:
\ditem{INC}{If $E$ and $F$ are logically incompatible and $F$ is logically possible, then $P(E\mid F) = 0$.}
(The restriction to the case where $F$ is logically possible is to handle the intuition that perhaps $P(F \mid F)=1$
even if $F$ is logically impossible.)

Now, let $U$ be our law that says that the world consists of $N$ independent fair indeterministic coin tosses. Let $E_0$ be the event
of not getting any heads. Then it can be proved from $U$ that the chance of $E$ is $1/2^N$, so by \dref{LPP} we have:
\ditem{non-zero-prob}{$P(E_0\mid\Law(U))=1/2^N$.}
But on our probabilistic best systems analysis, it is logically impossible to have $U$ be a law when no heads have ever occurred.
Thus, $\Law(U)$ and $E_0$ are logically incompatible, and so by \dref{INC} we have:
\ditem{non-zero-prob}{$P(E_0\mid\Law(U))=0$,}
a contradiction.

The point here is quite simple. Orthodox probability reasoning shows that it is possible but unlikely that we will have no heads given
a large number of independent fair coin tosses, but our probabilistic best-systems analysis must categorically reject such a possibility.

One might think that assigning zero probability to things so incredibly unlikely is unproblematic. But we also have some other strange
probabilistic results. In defining the laws, we are maximizing a balance of fit and brevity. If in our world half of the coin tosses 
are heads, then a law assiging chance $1/2$ to heads is likely the best balance. But suppose that the frequency of heads isn't $1/2$
but something very close to $1/2$. Then the law may still be that the chance is $1/2$, because a slight decrease in fit due to having
a chance not quite matching the exact frequency might be more than offset by the gain brevity if chance $1/2$ is significantly more
briefly expressible than the actual frequency, say ``$1/2-3\cdot 2^{-254}$''. 

Let's suppose $N$ is very large but easily expressible (e.g., $N=2^{2^{2^{2^{2^2}}}}=2^{2^{65536}}$), and let $\scr W_N$ be 
the set of worlds with nothing but $N$ tosses where there are no briefly expressible and highly informative patterns other than 
those conveyed by the independence of the coin tosses, the frequencies and the number of tosses.

Let $F_N$ be the set of all fractions between $0$ and $1$ with denominator $N$. Each member of $F_N$ could be the exact frequency
of heads in some world in $\scr W_N$.

Let $A_{N,1/2}$ be the subset of frequencies in $F_N$ such that there is a world $w$ in $\scr W_N$ in which there is a Humean
law specifying the chance to be $1/2$. For simplicity, suppose $N$ is even, so $1/2$ is a member of $F_N$. For $N$ sufficiently large, we 
should expect $1/2$ to be a member of $A_{N,1/2}$. But for the reasons given above, we would expect some frequencies in $F_N$ that are very
close to $1/2$ to be in $A_{N,1/2}$ as well. If $E_\alpha$ is the event of the actual frequency being $\alpha$, then on our Humean
view we have:
\ditem{PP-zero}{If $\alpha$ is not in $A_{N,1/2}$, we have $P(E_\alpha\mid\Law(U))=0$.}
On the other hand, very plausibly, if $\alpha$ is in $A_{N,1/2}$, then we have a real possibility of having frequency $\alpha$
on the assumption that the law is $U$, and so we would expect:
\ditem{PP-non-zero}{If $\alpha$ is not in $A_{N,1/2}$, we have $P(E_\alpha\mid\Law(U))>0$.}
Frequencies that are sufficiently far from $1/2$ (such as the $0$ in our previous example) are not going to be in $A_{N,1/2}$: the fit 
of these frequencies is too bad. The case of $\alpha=0$ has already been discussed.

What is surprising, however, is that very likely there are some members $\alpha$ and $\beta$ in $F_N$ such that $\alpha<\beta<1/2$
and yet $\alpha$ is a member of $A_{N,1/2}$ but $\beta$ is not a member of $A_{N,1/2}$. For we can suppose that both $\alpha$ and $\beta$
are \textit{extremely} close to each other, and very close to $1/2$, but $\alpha$ is much more briefly expressible than $\beta$, so that 
although the fit of a chance $1/2$ law is slightly poorer for worlds with frequency $\alpha$ than for worlds with frequency $\beta$,
the greater brevity of an expression for $\alpha$ makes a law giving the chance of heads as $\alpha$ have a better balance of fit and
brevity in a world with frequency $\alpha$ than a law giving the chance as $1/2$, while $1/2$ has a better balance of fit and brevity in 
a world with frequency $\beta$ than a law giving the chance as $1/2$.

I cannot give precise examples here, because
we do not actually have the measures of fit in hand and estimating the brevity of the shortest expression of some number is often a very
difficult task. But suppose I am right. Then by \dref{PP-zero} and \dref{PP-non-zero}, some frequencies in $F_N$ that are further from
$1/2$, such as $\alpha$, will have a higher conditional probability on $U$ than some frequencies that are closer to $1/2$, such as
$\beta$. I imagined that the counterintuitive result that $E_0$ has zero conditional probability on $U$ fails might be given the ``excuse''
that $U$ entails a chance that is so tiny that we might as well take the probability to be zero. But that excuse won't work for $E_\beta$
getting zero conditional probability on $U$. For if we say that $U$ entails such a low chance for $E_\beta$ that $E_\beta$ deserves
conditional probability zero, then $E_\alpha$ should get conditional probability zero as well, since $E_\alpha$ has an even lower chance
on $U$.\footnote{For a large number of tosses, the chance of getting a particular frequency $x$ of heads has an approximately normal distribution 
centered on $1/2$ if the true chance of each independent toss is $1/2$, and hence drops off as $x$ moves away from $1/2$.}

??refs

\subsubsection{$^*$Chance and propositions}
There is a variety of slightly different accounts of chance that can be given on PBSA. Let ``the mosaic'' refer to the arrangement of
properties that are systematized in the best system. Here are a few options, depending on whether one takes conditional or unconditional
chances to be fundamental:
\ditem{uncond-chance-no-init}{Event $E$ has unconditional chance $p$ if and only if the mosaic's best system entails that $E$ has chance $p$.}
\ditem{uncond-chance-init}{Event $E$ has unconditional chance $p$ if and only if the mosaic's best system conjoined with a complete
    specification of the mosaic's initial conditions entails that $E$ has chance $p$.}
\ditem{cond-chance-init}{Event $E$ has conditional chance $p$ on $C$ if and only if the mosaic's best system entails 
    that the conditional chance of $E$ on $C$ is $p$.}
There is room for much technical discussion of the details here, but notice that each of these three accounts is viciously circular: it
is attempting to define a chance (conditional or not) by a definition that itself makes use of the concept of chance.

There is a complicated way out of this difficulty that modifies the PBSA.

Suppose that we have a language $\scr L$ whose vocabulary can express all the fundamental physical concepts acceptable to 
the Humean, plus which has one more function symbol, $\Chance(x)$. This new function symbol is uninterpreted, and I stipulate that sentences using that function symbol are 
neither true nor false. Let $M$ be some appropriate set of mathematical axioms that include the correct axioms of set theory (perhaps indeed
they include all truths of set theory) as well as the 
right axioms of probability formulated using the chance symbol (so these will be axioms of unconditional or conditional probability, depending 
on whether the chance symbol is unary or binary).  Say that a theory (a set of sentences) $T$ in $\scr L$ is \textit{non-false} provided that 
no false sentence can be formally proved from $T\cup M$. This is equivalent to saying that all the provable consequences of $T\cup M$ that do 
not use the chance symbol are true. We can now specify that the best system is the non-false theory $T$ that optimizes brevity, fit and 
informativeness. 

Defining informativeness is difficult, but we might define it by looking at two kinds of information conveyed by the 
provable consequences of $T$. First, we have sentences that do not make use of the chance symbol. The informativeness
of that part of the consequences will be measured much as in a non-stochastic BSA. The informativeness of a statement
of form $\Chance(E)=r$, where $E$ does not make use of the chance symbol, will be null if $r=1/2$, and while if $r>1/2$, it will increase in proportion with both $r-1/2$ 
and the informativeness of the non-probabilistic claim that $E$ occurs, and if $r<1/2$, it will increase in proportion 
with $1/2-r$ and the informativeness of the non-probabilistic claim that $E$ does not occur if $r<1/2$. We might for
simplicity ignore claims using the chance symbol not of the form $\Chance(E)=r$. Of course, we
will have to beware of overlap in information between the chancy and non-chancy sentences, and so on. 

Finally, we measure fit by looking at all the provable consequences of the form $\Chance(E)=r$ where $E$ does not contain another
instance of the chance symbol, and saying the fit of each is a measure of closeness between $r$ and the numerical truth value of $E$ (falsehood
being zero and truth being one)---perhaps a measure according to some standard scoring rule.??refs

The details are doubtless very difficult, but that's to be expected.

Given all this, we get a best system $T$ consisting of sentences some of which fail to express a proposition. Given that laws of nature
are not linguistic entities in a particular language, and that they are factual, and hence capable of being true, 
we surely cannot take  a set of uninterpreted sentences like $T$ in $\scr L$ to be a system of laws. 

But we can now make a further move.
Let $B_T$ be the proposition that all the sentences in $T$ are provable consequences of the best system. Then $B_T$ is indeed a proposition
and factual. We can then say that the laws of nature are all the logical consequences of $B_T$. The laws, then, are not the sentences derivable from the best system, 
since these include uninterpreted sentences, but rather the laws are the consequences of the claim that these sentences are sentences of the 
best system.\footnote{An alternative would be
to consider the logical consequences of the proposition that $T$ \textit{is} the best system. But that would be less satisfactory. For it
seems that one could have a world with all the laws we have \textit{and} more. But if it is a law that $T$ is the best system, then it seems
there couldn't be a world with additional laws beyond those in $T$.???}

One technical problem of this approach lies on the probability side. Suppose we are in a world with a fair coin sequentially tossed a large finite 
number $N$ of times so set up that if $T$ is the best system then $T\cup M$ proves a sentence saying that the coin was indeed tossed 
sequentially $N$ times and proves the $\Chance(H_1)=1/2$, where $H_1$ is the event of the first throw being heads.\footnote{This 
will require $N$ to be a number that is sufficiently briefly expressible to make it into the best system.} Then $B_T$ will be a law of nature.
In the notation of Section~\ref{sec:principal-principle}, it will be law that the frequency of heads is a member of the set $A_{1/2,N}$.
This set contains $1/2$, and contains some frequencies in $F_N$ very close to $1/2$, but also excludes frequencies just as close to
$1/2$ as some of the included ones, on the grounds of that some of the excluded ones are so briefly expressible (e.g., $1/2+1/2^{256}$
in a world with $N=2^{256}$) that the brevity cost of the greater fit is worthwhile. This kind of law of nature looks little like the 
kinds of laws scientists posit. We do not have laws of nature allowing and disallowing events based on the brevity of the expressibility 
of facts about these events.

\subsubsection{Non-Humean chances}
There is another approach to chances, however, than PBSA. Suppose that we take chances themselves to form part of the Humean
mosaic, perhaps being fundamental properties, rather that being something defined via patterns in the mosaic. Then a 
non-deterministic theory can stick with BSA. The requirement that chancy statements be \textit{true} now is cashed
out in terms of the chances in the Humean mosaic. It is now quite possible for the best system to say that the chance
of heads is $1/2$ even if the coin never comes up heads. For we no longer optimize fit, but informativeness. And the
claim that the chance of heads is $1/2$ might well be quite informative, since it could tell us about the stochastic 
properties of a vast number of coin tosses---even if it does not give us useful information about the outcomes of
these tosses.

From the Humean point of view, the down side of this is that chances are too much like causal propensities, which the
typical Humean wants to reduce to patterns in the mosaic. After all, if our account of chance is not based on frequencies,
then it seems that our best option for what it means to say that the chance of heads is $1/2$ is that the immediate
cause has a propensity of $1/2$ to yield heads.

Thus, if we allow chances into the mosaic, we probably should allow causation as well. The resulting BSA has a richer
mosaic, and is harder to refute by counterexample. The problem that coincidental correlations become rigidified into
magical laws??backref is less pressing once it is clear that the magical laws are not causal laws---for causation is 
now a part of the mosaic, and hence mere correlations do not yield causation. However, we still have the problem of 
the plurality of bestnesses. And because we have so significantly enriched the mosaic, the main reason to believe the
BSA is weakened. For that main reason is ideological parsimony: the elegant fewness of fundamental concepts. Once we 
have introduced causation and causal propensities, one of the great claims to intellectual power on the BSA side---giving
an account of causation??---is gone.

\section{Laws and causation}

\section{Laws, natures and a partial Humeanism}
\chaptertail

