\def\mychapter{VIII}
\ifdefined\book
\else
\documentclass[11pt,oneside]{amsbook}
\usepackage[backend=biber, citestyle=authoryear]{biblatex}
\usepackage{mathpazo}
\usepackage{graphicx}
\usepackage{amsmath}
\usepackage{tikz}
\usetikzlibrary{arrows}
%\usepackage{titlesec}
\addbibresource{bibliography.bib}
\newcommand\posscite[1]{\citeauthor{#1}'s (\citeyear{#1})}
\newcommand\plural[1]{#1\mathrm{s}}
%\def\posscitewithextra[#1]#2{\citename{#2}'s (\citeyear{#2}, #1)}

%\newcounter{subsubsubsection}[subsubsection]
%\renewcommand\thesubsubsubsection{\thesubsubsection.\arabic{subsubsubsection}}
%\titleformat{\subsubsubsection}
%  {\normalfont\normalsize\bfseries}{\thesubsubsubsection}{1em}{}
%\titlespacing*{\subsubsubsection}
%{0pt}{3.25ex plus 1ex minus .2ex}{1.5ex plus .2ex}

\ifdefined\book
\renewcommand{\thechapter}{\Roman{chapter}}
\else
\renewcommand{\thechapter}{\mychapter}
\fi

\linespread{1.7}
\usepackage[margin=1.25in]{geometry}
\sloppy
\makeatletter
%% TODO: This is a cheat. It's supposed to be {paragraph}{4}, and that's 
%% what it is in amsbook.cls, but then it fails.
\def\paragraph{\@startsection{paragraph}{3}%
  \normalparindent\z@{-\fontdimen2\font}%
  \normalfont}
\def\subsubsubsection{\paragraph}
\makeatother

\def\smalltick{0.15cm}
\def\bigtick{0.3cm}
\def\pointcircle{0.08cm}
\def\causalnode{0.35cm}

\hyphenation{dia-chro-nic}

%\usepackage[utf8]{inputenc} % set input encoding (not needed with XeLaTeX)
\usepackage{amssymb}
\usepackage{mathtools}
\usepackage{enumitem}
\usepackage{amsthm}
\usepackage{physics}
%\usepackage{ntheorem}

\makeatletter
% \def\@endtheorem{\endtrivlist\@endpefalse }% OLD
\def\@endtheorem{\endtrivlist}%

\catcode`\|=\active\def|{\mid}
\DeclarePairedDelimiter{\ceil}{\lceil}{\rceil}
\DeclarePairedDelimiter{\floor}{\lfloor}{\rfloor}
\newcommand{\Subj}{\mathbin{\raisebox{.15ex}{$\scriptscriptstyle{\Box}$}\kern-.425em\rightarrow}}
\def\Existence{E!}
\def\Believes{\operatorname{Believes}}
\def\True{\operatorname{True}}
\def\Perfection{\operatorname{Perfection}}
\def\ext{\operatorname{Ext}}
\def\Iff{\leftrightarrow}
\def\Implies{\rightarrow}
\def\Entails{\Rightarrow}
\def\Equiv{\Leftrightarrow}
\def\Form{operatorname{Form}}
\def\Informs{operatorname{Informs}}
\def\technical{$\star$}
\def\vtechnical{$\star\star$}
\def\power{\wp}
\def\Nec{\Box}
\def\Poss{\Diamond}
\def\Prop#1{$\langle$#1$\rangle$}
\def\R{\mathbb R}
\def\N{\mathbb N}
\def\tele{tel\={e}}
\makeatletter
\newtheoremstyle{indented}{3pt}{3pt}{\addtolength{\leftskip}{4.5em}}{-2.5em}{\sc}{.}{.5em}{}
\def\Principle#1#2#3{\theoremstyle{indented}\newtheorem*{principle}{#2}\begin{principle}\def\@currentlabel{#2}\label{#1}#3\end{principle}\let\principle\undefined}
\makeatother
\def\pref#1{{\sc\ref{#1}}}
\def\enum#1{\resume{enumerate}\item #1\end{enumerate}}
\def\ditem#1#2{\begin{enumerate}[resume]\item \label{\mychapter:#1} #2\end{enumerate}}
\def\dref#1{(\ref{\mychapter:#1})}
\def\drefglobal#1{(\ref{#1})}
\usepackage{graphicx} % support the \includegraphics command and options
\usepackage{array} % for better arrays (eg matrices) in maths
\def\Not{\operatorname{\sim}}
\def\St{\operatorname{St}}
\def\num{\operatorname{num}}
\def\And{\mathrel{\&}}
\def\Or{\vee}
\def\BigOr{\bigvee}
\def\<{\langle}
\def\>{\rangle}
\def\union{\cup}
\def\nleq{\not\le}
\def\N{\mathbb N}
\def\R{\mathbb R}
\def\C{\mathbb C}
\def\Powerset{\mathcal P}
\def\star#1{{}^*#1}
\def\hN{\star{\N}}
\def\hR{\star{\R}}
\def\Z{\mathbb Z}
\def\Power{\mathcal P}
\def\Implies{\rightarrow}
\def\True{\operatorname{True}}
\def\Socrates{\mathrm{Socrates}}
\def\actual{@}

\def\H2O{H${}_2$O}

\def\scr{\mathcal}
\def\e{\varepsilon}
\def\eps{\varepsilon}
\newtheorem{lem}{Lemma}
\newtheorem*{theorem}{Theorem}
\newtheorem{corollary}{Corollary}
\newtheorem{cond}{Condition}

\renewcommand\thechapter{\Roman{chapter}}

\def\chaptertail{\ifdefined\book\else\end{document}\fi}
 

\title{Infinity, Causation and Paradox}
\author{Alexander R. Pruss}
%\date{} % Activate to display a given date or no date (if empty),
         % otherwise the current date is printed

\begin{document}
\setcounter{secnumdepth}{3}
\setcounter{tocdepth}{4}

\end{document}
\fi

\restartlist{enumerate}

\chapter{Metaphysics}\label{ch:metaphysics}
\section{Composition}
Intuitively, your parts compose a whole, namely you. On the other hand,
if you choose exactly one atom from every star in every galaxy in the universe,
intuitively these scattered atoms do not make up any whole.

Yet it seems that one could produce a continuous series of worlds, adding, subtracting or moving 
one particle at a time, where the first item in the series consists of the scattered
atoms in different stars and the last item consists of you. Somehow, as one moves
along the sequence, at some point a new macroscopic object pops in (maybe it's 
you, or maybe it's something else), despite the variation between successive terms
in the sequence being extremely slight. What makes for the transition? ??ref:Sider

???vagueness, nihilism, Sider, etc.

As Tomaszewski has noted??ref, the Aristotelian, however, has a solution, namely
the invocation of form. After all, it is not possible to have a sequence where at
one end you have one atom from every star and at the other have all of your parts,
and the transition is particle-by-particle. For your parts include a human form,
while the scattered atoms contain at most the forms of particles or atoms. Adding,
substracting or moving particles is not enough: one needs to add a human form
somewhere in the sequence.

We can, further, give an account of when a plurality of objects, the $x$s, compose a 
substantial whole: namely, there is a form among the $x$s which informs all the other $x$s, 
and everything informed by that form overlaps with at least one of the $x$s:
\ditem{compose}{$\exists F[F\in xx\And \Form(F)\And \forall y((y\in xx \And \Not y=F) \rightarrow 
    \Informs(F,y))]$,}
assuming that the $\Form(x)$ predicate only applies to substantial forms.
Here, the informing relation is one where the form gives identity to the thing it informs.
\footnote{Understood this way, informing is a relation that holds both between an a substantial 
form and the material parts of the substance and between a substantial form and the accidents or 
accidental forms of the substance. One may, however, object that these are two different relations.
If so, take my $\Informs(x,y)$ relation to be a disjunction of the two.}

??why is the transition where it is?: Mersenne

\section{Identity over time}
One of the classic questions of metaphysics is about the grounds of identity over time. 
A very general way of posing the question is to ask for an explanation or ground of claims
of the form:
\ditem{identity-ground}{Object $x_1$ is identical to object $x_2$,}
where $x_1$ and $x_2$ respectively exist at times $t_1$ and $t_2$, which are presupposed to be different\footnote{One may worry that this
presupposition cannot be stated without using identity, i.e., without denying that
$t_1=t_2$. However, it is not clear that even if this is true, it affects the
significance of the question. Moreover, if time turns out to be linearly ordered,
then we can state the presupposition disjunctively: $t_1$ is earlier than $t_2$ 
or $t_2$ is earlier than $t_1$.} and where we require the explanation not to
involve identity and to involve only purely qualitative properties and 
relations.\footnote{If we allow the account to involve non-qualitative properties 
like Socrateity (the property of being Socrates), then we can offer a infinite 
account: $x_1$ is identical to $x_2$ provided that any property had by $x_1$ is 
had by $x_2$.}

Put in that very general way, it seems unlikely that we will have a solution, 
absent some extremely controversial metaphysical assumptions, such as 
Leibniz's Principle of Identity of Indiscernibles (PII).\footnote{The PII says
that two things are identical just in case they have the same purely qualitative
properties. If the PII holds, then we can say that $x_1=x_2$ if and only if 
for every purely qualitative property we have $Q(x_1)$ iff $Q(x_2)$.} 

But there is a somewhat less general way of putting the diachronic identity question.
Suppose that at time $t_1$, some proper plurality of items, the $x$s, compose an 
object and at time $t_2$ the $y$s compose an object. Then we ask for the grounds of:
\ditem{identity-comp-ground}{An object composed of the $x$s at $t_1$ is identical 
    with an object composed of the $y$s at $t_2$.\footnote{The reason for the
    indefinite pronoun is that, first, there might be more than one object composed
    of the same parts and, second, using the definite pronoun introduces another 
    instance of the identity relation given the Russellian analysis of ``The $F$ is
    $G$'' as saying that some $F$ is $G$ and has the property of being
    \textit{identical} with every $F$ that is $G$, whereas we only want an account
    of a single identity relation.}}
This is not asking for an account of diachronic identity in general, but of diachronic
identity of complex objects (note that we only require the $x$s to be a proper
plurality, i.e., for there to be more than one of them).
    
Here we \textit{can} give an Aristotelian account:
\ditem{identity-Arist}{There is a form $F$ such that at $t_1$, $F$ unites the $x$s,
    and at $t_2$, $F$ unites the $y$s.}
Here we might stipulate that a form $F$ unites the $z$s just in case $F$ is one of 
the $z$s, at least one of the $z$s is a non-form, every form among the $z$s informs 
each non-form among the $z$s, anything informed  by $F$ overlaps at least one of the $z$s:
\ditem{Unites}{$\operatorname{Unites}(F,zz) \equiv
[F\in zz\And \Form(F)\and\exists x(x\in zz\And \Not\Form(x))\And \forall G\forall x((\Form(G)\And\Not\Form(x)\And G\in zz\And x\in zz)
\rightarrow \operatorname{Informs}(G,x))\And \forall x(\operatorname{Informs}(F,x)\rightarrow 
\exists y(y\in zz \And O(y,x)))]$.}
Here $O(y,z)$ says that $y$ overlaps $z$,
i.e., $\exists w(w\le y\And w\le z)$, where $\le$ is the parhood
relation. There is no identity anywhere in \dref{Unites}.\footnote{Note that the right hand side of 
\dref{Unites} is much more complex than \dref{composes}, even though we are giving an account of a 
very similar phenomenon. A simpler formulation than \dref{Unites}, and more in line with \dref{composes}, 
would say that $F$ unites the $z$s just in case it is one of them and informs every one of the $z$s other 
than itself and everything informed by overlaps at least one of the $z$s. However, this formulation 
makes use of identity in talking of $z$ \textit{other than} $F$. To avoid the identity operator, we 
quantify over all the substantial forms among the $z$s and say that they inform all the non-forms. 
For the present account to work, we cannot have a case where all the things informed by a substantial
form $F$ are also informed by another substantial form $G$, i.e., it cannot be that all the non-substantial-form 
constituents of one substance $a$ are parts of another substance $b$. For if we had that, then the 
\dref{Unites} would make $F$ unite all the constituents of $a$ together with the form of $b$, which does not
seem right. On a natural interpretation of the metaphysics of conjoint twins (??cross-ref,??ref), there are common
parts that informed by two forms. Our assumption is compatible with that interpretation, but rejects the odd
possibility of conjoint twins where the common parts include \textit{all} the non-substantial-form constituents
of one of them. If our assumption is rejected, then we may need to make use of the identity operator in \dref{Unites}.
However, the identity operator would only need to be used in the case of where one of the relata is a form, and so 
we could still have an explanation of identity of substances in terms of identity of form, and metaphysical
progress will have been made.}

It may appear that \dref{identity-Arist} uses the concept of identity in claiming 
that the \textit{same} $F$ unites the $x$s as unites the $y$s. Even if that were so,
progress would have been made: an account of identity for complex would be given things 
in terms of identity for simple things. But in any case, we need not concede the point
as we can rewrite \dref{identity-Arist} in first order logic without any equal signs:
\ditem{identity-Arist-FOL}{$\exists F(\operatorname{Form}(F)\And\operatorname{Unites}(F,xx,t_1)
    \And\operatorname{Unites}(F,yy,t_2))$.}
Granted, \dref{identity-Arist-FOL} is logically equivalent to:
\ditem{identity-Arist-FOL}{$\exists F\exists G(\operatorname{Form}(F)\And\operatorname{Form}(G)\operatorname{Unites}(F,xx,t_1)
    \And\operatorname{Unites}(G,yy,t_2)\And F=G)$,}
but this only show that the identity is eliminable from \dref{identity-Arist-FOL2},
just as the identity can be eliminated from
\ditem{tallgreen2}{$\exists x\exists y (\operatorname{Tall}(x)\And \operatorname{Green}(y)\And x=y)$}
(``There is a tall object which is identical to a green object'') to yield:
\ditem{tallgreen}{$\exists x(\operatorname{Tall}(x)\And \operatorname{Green}(x)$}
(``There is a tall green object'').

Thus, \dref{identity-Arist} gives us an account of the identity of complex objects
that does not presuppose identity.\footnote{One might also worry that complexity
presupposes identity: an object is complex provided it has two distinct parts.
But the Aristotelian can make sense of complexity of objects by saying that an
object is complex provided that it has a part that is a form and a part that is not
a form.}

\section{Teleological animalism and cerebra}
Each premise of the following argument is very plausible.
\ditem{human}{We are humans.} 
\ditem{human-mammal}{Humans are mammals.} 
\ditem{mammal}{So, we are mammals.}
\ditem{mammal-animal}{Mammals are animals.}
\ditem{animalism}{So, we are animals.}

Nonetheless, the conclusion---labaled as ``animalism''---is denied by many philosophers. One traditional path to 
this denial is a Cartesian dualism on which we are immaterial souls that inhabit human animals. The other path 
is modern colocationist views on which we are a special kind of material object---a person---constituted by a human 
animal. The third path is brain views on which we are brains, or parts of brains (namely cerebra), which in turn are a 
proper part of a human animal. 

On all three paths, one will want to deny \dref{mammal}. Perhaps the best way to do so would be to distinguish ``humans'' 
in \dref{human} and \dref{human-mammal} into  human animals and human persons, and insist that premise \dref{human} is true 
only of human persons, while \dref{human} is true only of human animals. Thus the argument is unsound if ``humans'' is used 
consistently and otherwise invalid. 

In any case, the intermediate step \dref{mammal} gets denied. Instead, we are 
\textit{associated} with mammals, by ensoulment, constitution or parthood.   Yet \dref{mammal} is by itself extremely plausible. 
To deny that we are mammals seems akin to denying that earth
is round. Further, both the Cartesian and brain views imply that we
rarely if ever see or touch another person. One needs extremely good arguments for such counterintuitive theses. 

Probably the main candidate here is a family of arguments about the difficulties of accounting for cerebrum transplants
on animalism.??refs The simplest version is that if your cerebrum is removed from your skull and placed in a vat in such a way
that it can continue functioning, then intuitively you continue to think and come along with the cerebrum. But a cerebrum
is not an animal. On the contrary, the cerebrumless body appears to be an animal. After all, some animals (e.g., fish??) lack
cerebra, and it seems that the destruction of a cerebrum in an animal that normally has one would result in the animal 
becoming severely disabled rather than ceasing to exist. Thus, animalism points to the cerebrumless body as you---the cerebrumless
body is the same animal as you---and the cerebrum in the vat as something  or someone else, contrary to our intuitions.

Cartesians, colocationists and brain theorists who identify with the cerebrum have no such problem. They can all say that
the cerebrumless body is not you, and instead you inhabit or are colocated with or are just plain identical with the cerebrum
in the vat. 

Here I want to argue that an Aristotelian about humans can embrace a teleological variety of animalism on which it is natural to say that 
we go along with our cerebra. Now animalism is highly plausible as 
a view of the human person that does justice to the intuitions that humans are mammals, weigh about 80~kg and can be seen 
without surgery, but faces a serious cerebrum transplant problem. If Aristotelianism can help animalists overcome that problem,
that is some further evidence for Aristotelianism about humans.

There are two teleological features in the animalism I will sketch. The first teleological feature is the thesis 
that what defines somethingas an animal (and indeed an animal of a particular type) is its teleology. While it is usual to 
think of animals as things that nourish themselves, grow, reproduce, and have a certain level of autonomy from the enviroment, 
the teleological animalist instead insists that animals need not engage in these activities, but need only have a teleological
orientation towards them, need to be the sorts of things that \textit{should} engage in these activities. 

The second teleological feature is to see organisms, including humans, as having a teleological \textit{hierarchy}, with some
\tele{} subordinated to others. Sometimes the subordination is instrumental: our teeth rend food in order to nourish us. 
But there can also be a value-based subordination, where an activity, while not merely instrumental towards another activity, 
is less central to the flourishing of the organism and to the organism's identity. On the teleological animalism I am sketching,
it is postulated that in humans, activities common to all animals are subordinated to specifically personal activities, namely 
rational and moral behavior. ???can we have this in non-theistic versions??? wouldn't we have subordination to reproductive
activity???

When there is a hierarchical subordination, it is plausible to think that in cases of splitting, an organism is more apt
to come along with the organs supporting the higher-level features. If in humans it is moral and intellectual capacity 
that is at the top of the teoleological hierarchy, and the cerebrum is much more directly supportive of these capacities
than the lower brain, heart, lungs, etc., then we would expect the human organism to come along with the cerebrum. If you
cut a worm in such a way that one end contains just the head and the other contains the rest of the body, and both parts
behave as if they were alive, it is reasonable to suppose that the form of the original individual may go with the larger
piece which contains more in the way of life-supporting organs, rather than with the head, because the head is less 
teleologically central to the worm than to us. But given human teleology, we would expect us to go along with the head,
if the head were given life support, or even with the brain or cerebrum.

What about the remaining cerebrumless human body, which maintains its vital functions? If the original individual goes along
with the cerebrum, and if we take the idea of one human form in two bodies to be absurd, there are three possibilities 
for the cerebrumless body:
\begin{itemize}
\item[(a)] the cerebrumless body gains a new human form, or 
\item[(b)] it gains some non-human form, or 
\item[(c)] it becomes a non-living substance or a formless heap of matter. 
\end{itemize}

If the cerebrumless body gains a new human form, then we have the counterintuitive consequence that a temporary removal
of a cerebrum followed by reimplantation would either result in conjoined twins---two human beings joined at the edges
of the cerebrum---or would result in the death of one or more of the two human beings. None of these options seems very
plausible, but at the same time, we should not be too surprised if strange things happen when you move cerebra around!

If the cerebrumless body gains a non-human form, this is presumably an organismic form, since the entity is capable of
nourishment, reproduction, etc. But we have something moderately puzzling: a non-human organism
that would produce a human being if it were to mate with a human or with another organism of the same sort but of the
opposite sex. Moreover, it seems plausible that if the cerebrumless body has a teleology at all, that teleology impels
it to try to support the cerebrum: oxygen would be directed by the body towards the missing cerebrum, presumably. This
suggests that the being is incomplete without the cerebrum. But a being that ought to have a human cerebrum seems to be 
a human being. 

The last option is a formless heap of matter or a non-living substance (such as a body of water might be on some 
Aristotelian theories??refs). This does not 
seem particularly puzzling in the transfer case. 
If the form departs, by going along with the cerebrum, then formlessness would seem to be the obviously expected result, 
barring some special reason to the contrary.

What if instead of the cerebrum being removed and put on life-support, the cerebrum is simply destroyed? This corresponds
to a tragic real-life scenario: upper brain death. Bioethicists disagree on whether upper brain death is death.??refs
We have four moderately plausible options for cerebral destruction:
\begin{itemize}
\item[(i)] the original individual continues to live, i.e., the cerebrumless body retains the original human form??backref-to-identity, or
\item[(ii)] the original individual dies and the cerebrumless body gains a new human form, or
\item[(iii)] the original individual dies and the cerebrumless body gains a non-human form, or
\item[(iv)] the original individual dies and the cerebrumless body has no organismic form, and is either a non-living substance or a heap.
\end{itemize}

Between (a)--(c) and (i)--(iv), there are twelve combinations with various intuitive connections between them. 
However, we can intuitively reduce the number of options quite significantly. I have assumed that in the transfer
case, the original form goes along with the cerebrum in the transfer case, departing from the cerebrum. Suppose that
(i) is false. Then the original individual dies and the cerebrumless body is deprived of its original form. It seems
very plausible that what happens to the cerebrumless body upon deprivation of its original human form should not depend
on what that form does after departing the cerebrumless body---whether it continues to inform a reduced body (the cerebrum), 
or survives disembodied as many religious people think, or perishes. Thus, if (i) is false, then (a), (b) or (c) holds
respectively if and only if (ii), (iii) or (iv) holds. 

We thus have three plausible options with (i) false:
\begin{itemize}
\item[($\alpha$)] (a) and (ii)
\item[($\beta$)] (b) and (iii)
\item[($\gamma$)] (c) and (iv)
\end{itemize}

What if (i) is true, so that in the case of cerebral destruction, the original form continues to inform the cerebrumless body? 
Does this  tell us anything about what happens in the transfer case? One might initially think that the falsity of (i) fits
poorly with any of (a), (b) and (c). After all, if destruction of the cerebrum results in the form staying with the cerebrumless
body, shouldn't we expect the form remain with the cerebrumless body when the cerebrum is transfered?

But this is not clear on teleological animalism. For we might think that in division or partial destruction of the body, the form 
goes along with the part that is the best candidate for being informed by it, at least when there is a unique best candidate and that best 
candidate is ``good enough''. On the teleological account, the quality of candidacy for being informed is measured by how high in the teleological hierarchy are
the goals directly promoted by the part. If the part promotes goals at the top of the hierarchy, then the candidate is automatically
good enough. Thus, when the cerebrum survives, it is reasonable to think the form goes along with the cerebrum, because the cerebrum
is good enough as a candidate, and higher in the hierarchy than the cerebrumless body. But if the cerebrum is destroyed, the cerebrumless
body is the unique best candidate, because it is the only candidate. Whether the cerebrumless body is good enough as a candidate is not clear, but neither is it
clear that it is not good enough. It is, after all, on the next step down in the hierarchy after the cerebrum, having among its tasks the full
support of the cerebrum's functioning, as well as many important purely animal functions. Thus, all of the following are at least somewhat reasonable 
epistemic possibilities:
\begin{itemize}
\item[($\delta$)] (a) and (i)
\item[($\e$)] (b) and (i)
\item[($\zeta$)] (c) and (i)
\end{itemize}

On teleological animalism we thus have six combinations for making sense of what happens to the form and cerebrum, namely ($\alpha$)--($\zeta$), 
and none of them appear immensely problematic, though (a) and (b) have some moderately counterintuitive consequences. But even if we feel
the need to reject these, that still leaves us with ($\gamma$) and ($\zeta$): in cerebral transfer cases, the cerebrumless body is not a living
thing, while in cerebral destruction cases, the cerebrumless body may (if we have (i)) or may not (if we have (iv)) continue to be a living
human being. 

One may think ($\zeta$) is implausible, because it implies that whether the body-minus-cerebrum continues to have the original human
form depends on what happens to the cerebrum---whether it is merely removed or actually destroyed. But such dependence is, as already
noted, to be reasonably expected. For if we think of the cerebrum as a magnet for the original human form, then transfer of that ``magnet''
might be reasonably be thought to pull the form along, hence depriving the cerebrumless body of it, while destruction of the ``magnet'' might
well leave the form in place. 

Teleological animalism, thus, has multiple ways of making sense of what happens in cerebral transfer and destruction cases, while these cases
are highly problematic to other types of animalism. Given the plausibility of animalism as such, this gives us another reason to accept
teleological animalism.

\section{Relativistic considerations}
\subsection{Introduction}
The Special Theory of Relativity---and the General as well, but let's simplify the discussion by restricting to 
the Special---has some rather interesting implications for spatially extended objects. I will first give an argument
that a rattlesnake can survive (at least briefly) as just a part of a rattle. But a rattle is far less central to the organic
life of a snake than a cerebrum is to that of a human, and hence if a rattlesnake can survive as just a part of a rattle, a human
can survive as just a cerebrum. 

Then I will offer a geometric and probabilistic argument for the Small Endpoints Thesis 
(SET) that \textit{all} material substances on earth are \textit{subatomic} in size at the beginning and end of their lives, 
and a metaphysical causal argument for the Small Beginnings Thesis (SBT) that this is true at least of the 
beginnings of material substances on earth. These theses are hard to believe. Many already find the idea that we started 
our lives as zygotes hard to believe, since human zygotes do not seem sufficiently sophisticated to support the 
functioning characteristic of humans.??refs But that we started our life as beings much smaller than zygotes invites 
a completely incredulous stare. Nonetheless, the arguments for SET and especially SBT appear to be very strong.

The two smallness theses will then point to significant advantages for the robust Aristotelian metaphysics defended 
in this book over ``faint-hearted'' Aristotelianism and non-Aristotelian alternatives, in addition to providing
further support for the possibility of survival as a cerebrum. Furthermore, these theses require a significant 
weakening of traditional Aristotelian claims about matter having to be well-disposed to have a particular form, 
as well as explanations of the perishing of a substance in terms of the matter losing that disposition. 

\subsection{Some background}
Special Relativity presents spacetime as four-dimensional, with three spatial and one temporal dimension.
The famous fact that will be crucial to the arguments here is that many ordinary concepts, such as simultaneity,
then end up relative to a reference frame (specifically, an inertial one). For any object---real or hypothetical---constantly 
moving with a constant sub-light velocity (i.e., undergoing sub-light inertial motion), there is a reference frame according 
to which that object 
is standing still. Note, however, that the concept of ``moving with a constant sub-light velocity'' is itself absolute: 
if an object does that in one reference frame, it does it in all of them (standing still is the special case of this where
the velocity is zero). However, what is relative is simultaneity: if two distinct points of spacetime are simultaneous in one 
reference frame, there will always be another reference frame where they are not simultaneous. 

We will need some more relativistic detail for the arguments. 

Two points of spacetime are said to be ``timelike-related'' provided it's possible to travel between them 
at a speed lower than that of light. They are ``spacelike-related'' provided that to travel between them 
you need to exceed the speed of light.\footnote{The remaining case is that it is possible to travel between 
them but only at exactly the speed of light. In that case, they are ``lightlike-related''. We won't need that 
case.} If two distinct points $z$ and $w$ are spacelike-related, there will always be reference frames according to which they are 
simultaneous, and reference frames according to which they are not simultaneous. In fact, there will be a reference frame
according to which $z$ is earlier than $w$ and a reference frame according to which $w$ is earlier than $z$. 

On the other hand, if two distinct points are not spacelike-related, then they are absolutely non-simultaneous---non-simultaneous 
in every reference frame. Furthermore, one of these points will be absolutely earlier than the other.

Given a reference frame $F$, for any point $z$ in spacetime there will be a set of points in spacetime simultaneous 
with $z$ according to $F$. These points all lie on the same ``hyperplane'' (i.e., a flat three-dimensional slice 
through spacetime). Moreover, these points are all spacelike-related to each other, so we say that this is a 
spacelike hyperplane. A reference frame $F$ then divides up spacetime into a family $S_F$ of parallel spacelike 
``hyperplanes of simultaneity'' such that two points of spacetime are simultaneous according to $F$ provided that they lie on the 
same hyperplane $K$ in $S_F$. 

Conversely, given any one spacelike hyperplane, we can form the set $S$ of all the hyperplanes parallel to it, and this 
set then defines a reference frame $F$ such that $S_F=S$.

\subsection{Of snakes and rattles}
Imagine a long rattlesnake stretched out in a line from head to tail, with the snake staying still in a reference frame $F$. For concreteness and 
simplicity, let's model the rattlesnake as one meter long, including a six centimeter rattle, with has a thickness of at 
most three centimeters throughout its body. Now something terrible will happen to the rattlesnake. Science-fictional blasters 
are pointed at every centimeter of the  snake's body other than the rattle. These blasters are all timed to have their blasts 
hit the snake simultaneously, in frame $F$, at $t_1$. Once the blast hits the snake, it very quickly vaporizes the section 
of the snake's body it is pointed at. Stipulate that a ``jiffy'' is the very short amount of time it takes light to travel one 
centimeter, about 33 picoseconds,\footnote{This definition is attributed on the Internet to the chemist Gilbert N. Lewis, but 
I have not been able to track 
down an original citation.} and let us suppose it takes four jiffies to vaporize each section of the snake (the section 
is three centimeters thick, so the blast can go from one side to the other of the snake in less than four jiffies without exceeding the 
speed of light). If $t_2$ is four jiffies after $t_1$, then at $t_2$ all that will be left of the snake is the rattle. 

But according to Relativity, causal effects propagate at speeds not exceeding that of light. Now, consider the \textit{tip}
of the rattle, which I will stipulate to be the very last centimeter. At $t_1$, the nearest blast is five centimeters away from 
the tip of the rattle. Since the blasts are the cause of death, and causation propagates at a speed not exceeding that of light,
over the four jiffies between $t_1$ and $t_2$, the effect of the blasts will not yet reach the rattle's tip. But the blasts are 
the cause of the death of the snake---namely of the removal of the snake form from its matter. Thus the rattle's tip will not 
have yet lost its snake form. But as long as the snake form is found in \textit{some} matter, the snake still exists. So 
at $t_2$, the snake exists, even though everything outside the rattle has been destroyed (and presumably the vaporization of 
the section of the snake just next to the rattle has resulted in some of the rattle getting destroyed, too). Thus, as I advertised 
in the previous section, we have seen that the snake can survive as just a part of a rattle.

Likely, this survival is very short: we can reasonably speculate that the loss of snake form proceeds through the rattle 
very fast, maybe even at the speed light. A few more jiffies, then, and the snake will likely be gone.

One might object that loss of form is not the kind of effect that we have in mind when we say that causal influences do not 
exceed the speed of light. There are two supplementary arguments I can offer here. First, if the tip of the snake's rattle 
has indeed lost its snake form at $t_2$, let $z$ be a spacetime point within the tip of the snake's (no longer alive) rattle 
at $t_2$. Let $E$ be the event of the blasters beginning to blast. Then $z$ and $E$ are spacelike-related, because the spatial 
distance between $z$ and the nearest part of $E$ is five centimeters, while the temporal distance between $z$ and $E$ is four
jiffies, so one cannot travel between $z$ and $E$ without exceeding the speed of light. But if they are spacelike-related, then
there will be a reference frame $F'$ according to which $z$ is earlier than $E$. Thus, in reference frame $F'$, the tip of the 
rattle will have lost its form \textit{before} the blasters went off. This seems very strange indeed.

Second, it is a standard Aristotelian point that a body part that loses its form stops existing. A severed finger is a 
finger in name only: the finger has perished and there is just a heap of matter. Thus, loss of form is the destruction 
of the rattle's tip. But the idea that the blasters somehow managed to destroy the rattle's tip in a faster-than-light 
way is rather implausible. Destruction of objects seems like the kind of thing that shouldn't proceed in a faster-than-light
kind of way, especially since the argument of the previous paragraph shows that the destruction in some reference frames 
would have to preceed the blast. 

As already noted, if a snake can survive---even if only for a few jiffies---as only a rattle, there should be no deep 
difficulty in a human surviving as a cerebrum. 

One might object that rattles are not living and informed parts of a snake, but dead stuff. If so, then replace the 
rattlesnake with a non-rattlesnake in the argument, point blasters at all but the last six centimeters of tail, and 
use the same argument to conclude that the snake can survive as just a few centimeters of tail. This is not quite as 
amazing as surviving as part of a rattle, but it is sufficient to make the \textit{a fortiori} argument above about the 
possibility of survival as a cerebrum.

Finally, note that this snake thought experiment already points our attention towards SET, by showing that it is 
at least \textit{possible} for a large organism to shrink to an insignificant portion at the time of death.  

\subsection{A geometric argument for SET}
??acpa permission

First, imagine you are holding a perfect cube. Now imagine putting a perfectly flat plane against the cube. If the 
plane is parallel to and touching one of the six faces of the cube, it will make contact with the whole of the face, 
and hence with infinitely many points. Similarly, if the plane is parallel to and touching one of the twelve edges 
of the cube, it will make contact with the whole of the edge, again with infinitely many points of contact. But in 
every other orientation of the plane against the cube, the plane will contact the cube at exactly one point, indeed at 
one of the eight vertices of the cube. 

Moreover, there is a sense in which there are vastly more orientations of the plane relative to the cube where the 
plane contacts the cube at one point than where the plane contacts the cube at more than one point. There are only 
six orientations where the plane contacts the cube at a face---the plane has to be parallel to taht face. There are, 
admittedly, infinitely many orientations where the plane contacts the cube along an edge, but these orientations 
are constrained by the need for the plane to be parallel to the edge. Once we decide which of the twelve edges of 
the cube to align our plane with, we have only one degree of freedom with regard to the orientation between the cube 
and the plane. Imagine rocking the plane against the cube, keeping contact with the edge---there is only one degree 
of freedom to the rocking. But when the plane makes contact with one of the eight vertices of the cube, there are two 
degrees of freedom—we can rock the plane along two different angles and keep the contact. This geometric observation 
remains correct regardless of what solid object we use in place of the cube. If we place a plane against the boundary 
of the object, for ``almost all orientations'' of the plane, the plane will be contacting the object only at a single 
point. (In the case of some shapes like an ellipsoid, this will be true for all orientations.) Moreover, the same thing is 
true in $n$ dimensions for any $n>1$: if we have an $n$-dimensional object and have an $(n-1)$-dimensional hyperplane rest 
against it, for almost all orientations of the hyperplane, the object will contact the hyperplane at a single point. 
(This is made mathematically precise in the Appendix.)

Now move to the four-dimensional spacetime of Special Relativityand consider a material substance with a beginning and end to 
its life, say Rover. Let $D$ be the four-dimensional shape consisting of the points of spacetime occupied by Rover, and 
imagine a reference frame $F$. This reference frame has a simultaneity relation defined by a family $S_F$ of spacelike 
hyperplanes, where, recall, two points are simultaneous provided that they lie on the same in $S_F$. Assume $D$ is 
finite in size, both spatially and temporally. Suppose $z$ is one of the points of 
spacetime where Rover begins to exist with respect to frame $F$. Let $K$ be the hyperplane in $S_F$ that passes through 
$z$. THis will be a hyperplane resting against $D$. If Rover begins to exist at two or more points according to frame $F$, 
then $K$ will make contact with two or more points of the boundary of $D$. Now, for almost all 
orientations, a hyperplane of that orientation that rests against the object will make contact with the object’s boundary 
at a single point, as discussed earlier. Moreover, the orientation of a spacelike hyperplane defines a reference frame (since the
orientation of a hyperplane is sufficient to determine the family of parallel hyperplanes). Thus, according to almost all 
reference frames, Rover begins to exist at a single point. 

Exactly the same argument applies to Rover’s demise: at almost all reference frames, Rover ends its life at a single point. 
See Figure ?? for an illustration in $1+1$ rather than $3+1$ dimensions, where the hyperplanes are just lines.
??? Figure 1: A version of the argument with one spatial and one temporal dimension. Spacelike ``hyperplanes'' are lines whose angle with the horizontal is 
less than $45^\circ$. Dotted lines define the only two reference frames where the object begins with more than one point. Dashed lines define the only reference frame where the object ends with more than one point. The solid lines are examples of reference frames where the object begins and ends at a point.

Single points are subatomic. So Rover begins and ends its life subatomic in size. ???randomness

We might ask why I am weakening the 
conclusion to talk of the subatomic. The reason is that this is my way of finessing the issue that in non-Bohmian Quantum 
Mechanics, particles do not have precise positions. So it might turn out not to be meaningful to talk of an object as 
having a precise starting point. There may be some ``fuzzing out'' of the starting and ending positions, which I will 
assume will not go beyond the subatomic level. 

A few words are needed about what we mean by ``almost all'' reference frames. As noted, a reference frame is defined by the 
orientation of its hyperplanes. We can define an orientation of a hyperplane $K$ by a four-dimensional vector $v$ that is 
perpendicular to $K$ and has length one. The set of all such vectors is a hypersphere of radius one (the set of points in 
four-dimensional space of distance one from the center), though we are not interested in the whole hypersphere, but only in 
the part of it that corresponds to vectors that are perpendicular to spacelike hyperplanes.\footnote{$^*$It is important for 
technical reasons that this part has non-zero hyperarea.} Now let $U$ be the set of all the unit-length vectors $v$ such that 
(a)~the vector $v$ defines a reference frame (by defining a family of spacelike hyperplanes perpendicular to $v$) and (b)~in 
this reference frame Rover starts existing at two or more points. Then $U$ has zero hyperarea as a portion of the hypersphere 
of all vectors of radius one. Mathematicians refer to something that happens in a region of zero area, volume, or similar 
measure as happening ``in almost no cases'', and to what happens everywhere outside such a region as happening ``in almost 
all cases''. It follows that in almost reference frames, Rover starts existing at only one point, and ends at only one.  
A more precise version of this argument will be given in the Appendix.



Thus, if we fix a random reference frame, we can be all but certain that any specified material object with a finite lifetime 
starts off and ends up subatomic in size with respect to that frame. In fact, as long as the number of material objects in 
the universe with finite lifetime is finite or countably infinite, then for any fixed random reference frame, we can be all 
but certain that all of them start off and end up subatomic in size. (This follows from the mathematical fact that the 
union of a countable collection of sets of measure zero has measure zero.) When I initially gave the Small Endpoints Thesis, 
I didn’t specify a reference frame. The above argument is compatible with the existence of some reference frames where SET is false. However, the argument shows that for any fixed frame the reader might have in mind—say, a frame defined by some precisification of the reader’s current center of mass—it is nearly certain that SET is true in that frame, since almost all frames have SET true in them. It would require incredible luck for the reader to have in mind a reference frame where SET is false—luck akin to tossing a coin infinitely many times and getting heads each time. I assume the reader is not so lucky.
	There is one gap in the above argument. I assumed that the object is finite both spatially and temporally, i.e., its life occupies a bounded region of spacetime. If the universe counts as a material substance, and the universe is infinite in size, it is a counterexample to this assumption. However, there is some metaphysical reason to doubt that the universe is a substance. First, there is a classic Aristotelian thesis that a substance cannot have substances as proper parts, and the universe, if it exists, would have dogs as proper parts. Second, intuitively, the universe is not conscious. But, plausibly, any substance that has a conscious mind as a part is conscious. Cats have minds. Parthood is transitive, so if there is such a substance as the universe, it has not only Felix the cat but Felix’s mind as a part, and hence the universe would be conscious if Felix is. Furthermore, there is some empirical reason to think the universe will expand forever, and hence the universe does not have an end, while SET is only a thesis about material substances with a beginning and end of life.


\section{Ill-matched matter, rearrangement, the power to continue existing and immortality}


\section{Naturalism}
Is the Aristotelian hylomorphic account of humans compatible with naturalism? This depends on how naturalism
is defined and whether we take the theistic version of the account or not.

Since theism implies the existence of a non-natural causally efficacious substance, namely God, it will be incompatible
with most versions of naturalism. And I have argued theat the Aristotelian account is unsatisfactory without
theism. 

Still, it is an interesting question whether the what the Aristotelian account says about the forms of finite
substances is compatible with naturalism. If so, then one could combine the Aristotelian account with 
a naturalism restricted to finite objects, and save some naturalistic intuitions. Furthermore, it would mean
that an Aristotelian not convined by the arguments that theism is needed to make the theory satisfactory could
be a full-blown naturalist.

If we take naturalism to say that the only entities that exist are the ones that would figure in a completed science,
then it is unlikely that Aristotelian metaphysics would be compatible with naturalism about finite objects. However,
such a strong naturalism would likely also conflict with many other metaphysical theories that are rarely taken to 
contradict naturalism. 

For instance, consider theories of time. Of the theories of time, four dimensionalism, on which
ordinary objects like ourselves are extended in time as well as space, is what seems to fit best with Relativity Theory. 
But the most common four-dimensionalist view of changing properties is perdurantism: changing objects are made of temporal parts or
slices to which the properties are primarily attributed, so that a tomato that once was green and now is red has a green temporal part
and a red temporal part. But slices are unlikely to figure in a completed science if
our current science is a good guide to that. Consider that our current physics does not consider particles
like electrons and quarks to be fundamental, and hence not made up of smaller parts. But electrons and quarks change over
time (e.g., with respect to spin and flavor), and so they would need tos have temporal parts.  But these parts are not found
in our physics. 

Or consider that our science may quantify over physical objects like particles and fields, and applies predicates to them, but 
does not quantify over properties. Thus, properties, whether understood as universals or as tropes, go beyond current science.
Yet it seems implausible to understand naturalism as implying nominalism.

One may weaken naturalism to say that the only \textit{causally relevant} entities are those of a completed science. But 
this would still rule out perdurantism, since objects change with respect to causally relevant properties, and hence their
temporal parts have these properties, and have them more fundamentally. It would also likely rule out many versions of trope
theory, since the causal efficacy of tropes is supposed to explain the causal efficacy of the objects made of them.

Let's step back. Naturalism denies the existence of causally efficacious ``supernatural'' entities like 
ghosts, but it is neutral on temporal parts and tropes of ``natural'' entities like electrons. 
What is the relevant difference between ghosts and temporal parts of electrons? It seems to be this. While
neither entity is posited by science, if ghosts exist, then certain phenomena that it belongs to science 
to investigate have no scientific explanation, barring systematic overdetermination. If there are ghosts,
they are presumably responsible for at least some of the appearances of ghosts, some of the chills people
feel in graveyards, etc. These phenomena then either have no physical explanation at all, or by a massive
coincidence are overdetermined by a physical and a spectral cause. There is a competition, thus, between 
scientific and spiritual explanations when we are concerned with ghosts. 

On the other hand, there is neither
competition nor overdetermination in the case of the objects studied by science and their ontological
constituents such as temporal parts or tropes. When a temporal part or a trope of an ice cube makes 
your hand cold, the ice cube also makes your hand cold. The ice cube's cooling causal influence on your hand
does not compete with the ice cube's temporal part's causal influence on you or on the causal influence of 
a trope of coldness (if there are temporal parts of ice cubes or tropes of coldness). Nor is this overdetermination,
since it is not overdetermination when an object $C$ causes an effect $E$ by means of $C$'s part. After all,
it is not overdetermination when the ice cube cools the hand it is lying on \textit{and} the lower half of 
the ice cube cools the hand. 

We might thus want to say that naturalism holds that any phenomenon that it falls within the purview of science 
to seek for causal explanations of either has no causal explanation at all, or else has a scientific causal explanation 
and is not overdetermined by a scientific and a non-scientific one. On this account, theism is incompatible with
naturalism if there is a first state of physical reality. For if there is such a state, it belongs
to science to seek for its causal explanation. However, there is a theistic explanation of that state and no 
scientific explanation. We thus have the kind of competition that naturalism rules out.\footnote{Matters are a little more
complicated if there is no initial physical state, but we can apply the above argument to an initial
segment of physical states: science cannot explain that segment but theism claims to do so, and yet it lies
in the scope of science to ask for such an explanation.} One might object that science knows that it cannot, on pain
of vicious circularity, provide a scientific explanation of the first physical state, and hence it does not belong to
science to seek that explanation. But this objection confuses the first physical state \textit{qua} first and the
first physical state as it intrinsically is. It does not belong to science to seek the explanation of the first physical
state \textit{qua} first. But if we just consider it as a specific physical state---particles and fields arranged thus-and-so---then
it certainly belongs to science to seek for its explanation, though that search will be rightly abandoned if sufficient evidence
is gathered that the state is in fact the first one.

Now, let's go back to Aristotelianism. Bracketing theism, the entity that might trouble the naturalist is form.
But the forms of things are components of substances---humans, dogs, oaks, etc.---that are also found in our current
science (e.g., biology) and are likely to be found in the completed version of that science. And there is neither
competition(??what's that? remove here and earlier?) nor overdetermination between causal explanations provided by
forms and their substances. As long as the substances do not have spooky causal powers beyond the scope of science,
adding forms to the ontology no more contradicts a plausibly defined naturalism than perdurantism or trope theory
does.

It is, of course, open to the Aristotelian to suppose that some finite substances, whether natural like humans and oaks 
or  supernatural like angels or ghosts, have causal powers of a sort that goes beyond the scope of science, though
specifying what those powers would have to be like is difficult. But nothing in the applications given for Aristotelianism
posited such powers. Thus as far as the argument of this book goes, there is no contradiction with a carefully
specified naturalism with respect to finite things.\footnote{I think the most plausible place to look for a tension 
would be in examining human free will, but that goes beyond the scope of this book.}

\chaptertail 


