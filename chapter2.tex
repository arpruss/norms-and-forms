\def\mychapter{II}
\ifdefined\book
\else
\documentclass[11pt,oneside]{amsbook}
\usepackage[backend=biber, citestyle=authoryear]{biblatex}
\usepackage{mathpazo}
\usepackage{graphicx}
\usepackage{amsmath}
\usepackage{tikz}
\usetikzlibrary{arrows}
%\usepackage{titlesec}
\addbibresource{bibliography.bib}
\newcommand\posscite[1]{\citeauthor{#1}'s (\citeyear{#1})}
\newcommand\plural[1]{#1\mathrm{s}}
%\def\posscitewithextra[#1]#2{\citename{#2}'s (\citeyear{#2}, #1)}

%\newcounter{subsubsubsection}[subsubsection]
%\renewcommand\thesubsubsubsection{\thesubsubsection.\arabic{subsubsubsection}}
%\titleformat{\subsubsubsection}
%  {\normalfont\normalsize\bfseries}{\thesubsubsubsection}{1em}{}
%\titlespacing*{\subsubsubsection}
%{0pt}{3.25ex plus 1ex minus .2ex}{1.5ex plus .2ex}

\ifdefined\book
\renewcommand{\thechapter}{\Roman{chapter}}
\else
\renewcommand{\thechapter}{\mychapter}
\fi

\linespread{1.7}
\usepackage[margin=1.25in]{geometry}
\sloppy
\makeatletter
%% TODO: This is a cheat. It's supposed to be {paragraph}{4}, and that's 
%% what it is in amsbook.cls, but then it fails.
\def\paragraph{\@startsection{paragraph}{3}%
  \normalparindent\z@{-\fontdimen2\font}%
  \normalfont}
\def\subsubsubsection{\paragraph}
\makeatother

\def\smalltick{0.15cm}
\def\bigtick{0.3cm}
\def\pointcircle{0.08cm}
\def\causalnode{0.35cm}

\hyphenation{dia-chro-nic}

%\usepackage[utf8]{inputenc} % set input encoding (not needed with XeLaTeX)
\usepackage{amssymb}
\usepackage{mathtools}
\usepackage{enumitem}
\usepackage{amsthm}
\usepackage{physics}
%\usepackage{ntheorem}

\makeatletter
% \def\@endtheorem{\endtrivlist\@endpefalse }% OLD
\def\@endtheorem{\endtrivlist}%

\catcode`\|=\active\def|{\mid}
\DeclarePairedDelimiter{\ceil}{\lceil}{\rceil}
\DeclarePairedDelimiter{\floor}{\lfloor}{\rfloor}
\newcommand{\Subj}{\mathbin{\raisebox{.15ex}{$\scriptscriptstyle{\Box}$}\kern-.425em\rightarrow}}
\def\Existence{E!}
\def\Believes{\operatorname{Believes}}
\def\True{\operatorname{True}}
\def\Perfection{\operatorname{Perfection}}
\def\ext{\operatorname{Ext}}
\def\Iff{\leftrightarrow}
\def\Implies{\rightarrow}
\def\Entails{\Rightarrow}
\def\Equiv{\Leftrightarrow}
\def\Form{operatorname{Form}}
\def\Informs{operatorname{Informs}}
\def\technical{$\star$}
\def\vtechnical{$\star\star$}
\def\power{\wp}
\def\Nec{\Box}
\def\Poss{\Diamond}
\def\Prop#1{$\langle$#1$\rangle$}
\def\R{\mathbb R}
\def\N{\mathbb N}
\def\tele{tel\={e}}
\makeatletter
\newtheoremstyle{indented}{3pt}{3pt}{\addtolength{\leftskip}{4.5em}}{-2.5em}{\sc}{.}{.5em}{}
\def\Principle#1#2#3{\theoremstyle{indented}\newtheorem*{principle}{#2}\begin{principle}\def\@currentlabel{#2}\label{#1}#3\end{principle}\let\principle\undefined}
\makeatother
\def\pref#1{{\sc\ref{#1}}}
\def\enum#1{\resume{enumerate}\item #1\end{enumerate}}
\def\ditem#1#2{\begin{enumerate}[resume]\item \label{\mychapter:#1} #2\end{enumerate}}
\def\dref#1{(\ref{\mychapter:#1})}
\def\drefglobal#1{(\ref{#1})}
\usepackage{graphicx} % support the \includegraphics command and options
\usepackage{array} % for better arrays (eg matrices) in maths
\def\Not{\operatorname{\sim}}
\def\St{\operatorname{St}}
\def\num{\operatorname{num}}
\def\And{\mathrel{\&}}
\def\Or{\vee}
\def\BigOr{\bigvee}
\def\<{\langle}
\def\>{\rangle}
\def\union{\cup}
\def\nleq{\not\le}
\def\N{\mathbb N}
\def\R{\mathbb R}
\def\C{\mathbb C}
\def\Powerset{\mathcal P}
\def\star#1{{}^*#1}
\def\hN{\star{\N}}
\def\hR{\star{\R}}
\def\Z{\mathbb Z}
\def\Power{\mathcal P}
\def\Implies{\rightarrow}
\def\True{\operatorname{True}}
\def\Socrates{\mathrm{Socrates}}
\def\actual{@}

\def\H2O{H${}_2$O}

\def\scr{\mathcal}
\def\e{\varepsilon}
\def\eps{\varepsilon}
\newtheorem{lem}{Lemma}
\newtheorem*{theorem}{Theorem}
\newtheorem{corollary}{Corollary}
\newtheorem{cond}{Condition}

\renewcommand\thechapter{\Roman{chapter}}

\def\chaptertail{\ifdefined\book\else\end{document}\fi}
 

\title{Infinity, Causation and Paradox}
\author{Alexander R. Pruss}
%\date{} % Activate to display a given date or no date (if empty),
         % otherwise the current date is printed

\begin{document}
\setcounter{secnumdepth}{3}
\setcounter{tocdepth}{4}

\end{document}
\fi

\restartlist{enumerate}

\chapter{Ethics}\label{ch:ethics}
\section{Normative ethics and boundaries}
\subsection{The rule of preferential treatment}
Let us begin with a more detailed discussion of an example from Thomas Aquinas's discussion of the order of charity. Aquinas thinks,
along with common sense, that those who are closer to us have a greater moral call on us.
Thus, if it is a question of bestowing the same good on one of two people, where one is more closely
related to us, we should benefit the closer one. But Aquinas writes: ``The case may occur, however, that one 
ought rather to invite strangers [to eat with us], on 
account of their greater want.''??ref And then he raises the question of what one should do ``if of two, one be 
more closely connected, and the other in greater want.''??ref

We might hope that here Aquinas would give us some clever rule for weighing connection against need. But 
instead he writes very sensibly: ``it is not possible to decide, by any general rule, which of them we ought 
to help rather than the other, since there are various degrees of want as well as of connection''.??ref It is
tempting at this point to throw up one's hands and simply say that in these in-between cases there is no
fact of the matter as to what should be done, or both options are permissible, or else relativism applies
to the case. But that would not do justice to the way we agonize when we find ourselves in such a difficult 
situation, trying to discover the truth of the matter. (It is interesting to note that the most common real-life moral dilemmas
tend to be like these kinds of cases, rather than highly controversial questions about trolleys, strategic bombing or
bioethics much discussed by philosophers.)
And indeed Aquinas maintains a realist attitude to
the question while simply offering this advice for how to figure out the answer in a particular case: ``the matter 
requires the judgment of a prudent man.??https://www.newadvent.org/summa/3031.htm\#article2

We can think of this as the problem of specifying a function $f(r,a,s,b)$ of four variables, two of them, $r$ and $s$, being
degrees of relation and the other two, $a$ and $b$, being degrees of benefit, where the function takes one of three values
corresponding to whether it is obligatory, permissible but not obligatory or impermissible to bestow a benefit of degree $a$ on a person with
relation of degree $r$ to the agent in place of bestowing a benefit of degree $b$ on someone related to degree $s$. 

In fact, the problem of a rule of preferential treatment is much more complicated than the above indicates. First, the \textit{kinds} of benefit and relation also matter: ``we ought in preference 
to bestow on each one such benefits as pertain to the matter in which, speaking simply, he is most closely connected with us.''??ref
So the function will depend not merely on quantitative features but qualitative ones. Second, although Aquinas does not mention it here,
the evaluation will no doubt depend on various features of the circumstances. And, third, in practice instead of choosing between
two certain benefits, we are choosing between two probability distributions over the space of possible benefits.

Now, as Aquinas admits, we do not know what the moral evaluation function for choices between benefits to different people is.
But abstractly speaking there is some such function, even if we do not know what it is, just as there is a function that assigns to each person
alive now the number of hairs they now have, even though we cannot specify any of the values of the function.
And we have good reason to expect the moral evaluation function to be very complicated. Indeed, probably the only serious proposal for a
relatively simple function $f$ here is the utilitarian suggestion that $f(r,a,s,b)$ yields obligation when $a>b$,
mere permission when $a=b$ and prohibition when $a<b$. But this utilitarian suggestion betrays the intuition that
the degrees of relation $r$ and $s$, much less the kinds of benefit and relation, are relevant to the moral evaluation.???refs

Indeed, the function is apt to look arbitrary. Fix the degrees of relationship to be one's parent and a total stranger,
and fix a specific and certain financial benefit of \$100 to one's parent, and fix the circumstances. Then as we vary the 
financial benefit to the stranger from zero to infinity, we will presumably initially have a requirement of benefiting the parent
(it would be wrong to give \$1 to a stranger instead of \$100 to a parent in ordinary circumstances), 
then a permission either way, and then a requirement to benefit the second party. There will be boundaries between these regions
of logical space, and these boundaries will look as arbitrary and contingent as the boundaries between different tax brackets.
Like the tax brackets, some proposals for boundaries will be \textit{clearly} unreasonable, but there will be many proposals
that appear reasonable. And whatever the actual boundaries will look arbitrary.

Of course, seemingly arbitrary numbers can come out of an elegant and simple rule: it seems arbitrary that the fifth and sixth 
digits of $\pi$ are $5$ and $9$ respectively, but there is an elegant mathematical explanation. But apart from the
utilitarian proposal, we do not have any at all plausible simple proposal for $f$.

These seemingly arbitrary boundaries in the order of charity raise call out for an explanation at least as much as 
the exact distance between the earth and the moon does. Just as it seems implausible that the distance between the earth
and the moon \textit{must} be exactly what it is, it seems implausible to think that the boundaries must be exactly where
they are---unless the utilitarian is right about $f$ being very simple. 

In fact, the ethics case calls out for an explanation even more than Mersenne's scientific examples did. For we might 
be able to swallow the earth-moon distance being a contingent and brute unexplained fact. But a brute fact seems unfitting for a moral
rule. A claim that it just so happened, with no explanation at all, that you should $\phi$ undercuts the moral force
of the alleged moral obligation. We expect anything seemingly arbitrary in our moral norms to have an explanatory ground.

To further argue for this point, consider a version of Divine Command Theory on which obligations are divine commands, and
God rolled indeterministic
dice to decide which actions to command, and by chance God's commands coincided with our common-sense morality, though they
could just as well as well have commanded cruelty and dishonesty. A Divine Command Theory on which it is mere chance
that cruelty is forbidden rather than commanded provides an unacceptable answer to the Euthyphro problem.??
Intuitively, a set of injunctions that is as arbitrary as that cannot constitute morality. But this point generalizes beyond
divine command theory. Suppose that that we have some preferential treatment rules that are brute and contingent, and could
just as well have enjoined on us the anti-utilitarian rule that we should always prefer the lesser benefit. Then whatever
these rules are, they do not constitute morality, but at best happen to agree with morality in content. 

Thus, even if there is some bruteness in the rules of preferential treatment, the rules in our world must be generated in a way
that makes rules such as the anti-utilitarian rules not be among the possible outcomes. But this makes it very unlikely that
the rules would be brute. For what force would limit the brute rules to avoid unacceptable options? Such a view of limited
bruteness would be akin to a view on which banana peels can come into existence \textit{ex nihilo}, but not where we might trip
over them.

\subsection{Other examples}
But before I continue the discussion of the possible explanation for the above ethical Mersenne question, let me follow
Mersenne's lead and multiply the examples, in order to defend against potential answers that only work in some cases.
??

\subsection{Standard normative systems}
\subsection{Vagueness}
\subsection{Necessity}

\section{Metaethics}
\section{Flourishing}
\section{Supererogation}
\section{The great chain of being}
\chaptertail 
