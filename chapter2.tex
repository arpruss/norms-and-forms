
\def\mychapter{II}
\ifdefined\book
\else
\documentclass[11pt,oneside]{amsbook}
\usepackage[backend=biber, citestyle=authoryear]{biblatex}
\usepackage{mathpazo}
\usepackage{graphicx}
\usepackage{amsmath}
\usepackage{tikz}
\usetikzlibrary{arrows}
%\usepackage{titlesec}
\addbibresource{bibliography.bib}
\newcommand\posscite[1]{\citeauthor{#1}'s (\citeyear{#1})}
\newcommand\plural[1]{#1\mathrm{s}}
%\def\posscitewithextra[#1]#2{\citename{#2}'s (\citeyear{#2}, #1)}

%\newcounter{subsubsubsection}[subsubsection]
%\renewcommand\thesubsubsubsection{\thesubsubsection.\arabic{subsubsubsection}}
%\titleformat{\subsubsubsection}
%  {\normalfont\normalsize\bfseries}{\thesubsubsubsection}{1em}{}
%\titlespacing*{\subsubsubsection}
%{0pt}{3.25ex plus 1ex minus .2ex}{1.5ex plus .2ex}

\ifdefined\book
\renewcommand{\thechapter}{\Roman{chapter}}
\else
\renewcommand{\thechapter}{\mychapter}
\fi

\linespread{1.7}
\usepackage[margin=1.25in]{geometry}
\sloppy
\makeatletter
%% TODO: This is a cheat. It's supposed to be {paragraph}{4}, and that's 
%% what it is in amsbook.cls, but then it fails.
\def\paragraph{\@startsection{paragraph}{3}%
  \normalparindent\z@{-\fontdimen2\font}%
  \normalfont}
\def\subsubsubsection{\paragraph}
\makeatother

\def\smalltick{0.15cm}
\def\bigtick{0.3cm}
\def\pointcircle{0.08cm}
\def\causalnode{0.35cm}

\hyphenation{dia-chro-nic}

%\usepackage[utf8]{inputenc} % set input encoding (not needed with XeLaTeX)
\usepackage{amssymb}
\usepackage{mathtools}
\usepackage{enumitem}
\usepackage{amsthm}
\usepackage{physics}
%\usepackage{ntheorem}
\usepackage{chngcntr}
\counterwithin{figure}{section}

\makeatletter
% \def\@endtheorem{\endtrivlist\@endpefalse }% OLD
\def\@endtheorem{\endtrivlist}%

\setlist[description]{font=\normalfont\scshape}

\catcode`\|=\active\def|{\mid}
\DeclarePairedDelimiter{\ceil}{\lceil}{\rceil}
\DeclarePairedDelimiter{\floor}{\lfloor}{\rfloor}
\newcommand{\Subj}{\mathbin{\raisebox{.15ex}{$\scriptscriptstyle{\Box}$}\kern-.425em\rightarrow}}
\def\Existence{E!}
\def\Believes{\operatorname{Believes}}
\def\True{\operatorname{True}}
\def\Perfection{\operatorname{Perfection}}
\def\ext{\operatorname{Ext}}
\def\Iff{\leftrightarrow}
\def\Implies{\rightarrow}
\def\Entails{\Rightarrow}
\def\Cov{\operatorname{Cov}}
\def\Equiv{\Leftrightarrow}
\def\Form{\operatorname{Form}}
\def\Informs{\operatorname{Informs}}
\def\technical{$\star$}
\def\vtechnical{$\star\star$}
\def\power{\wp}
\def\Nec{\Box}
\def\Poss{\Diamond}
\def\Prop#1{$\langle$#1$\rangle$}
\def\R{\mathbb R}
\def\N{\mathbb N}
\def\tele{tel\={e}}
\makeatletter
\newtheoremstyle{indented}{3pt}{3pt}{\addtolength{\leftskip}{4.5em}}{-2.5em}{\sc}{.}{.5em}{}
\def\Principle#1#2#3{\theoremstyle{indented}\newtheorem*{principle}{#2}\begin{principle}\def\@currentlabel{#2}\label{#1}#3\end{principle}\let\principle\undefined}
\makeatother
\def\pref#1{{\sc\ref{#1}}}
\def\enum#1{\resume{enumerate}\item #1\end{enumerate}}
\def\ditem#1#2{\begin{enumerate}[resume]\item \label{\mychapter:#1} #2\end{enumerate}}
\def\nitem#1#2{\begin{description}\item[#1\label{\mychapter:#1}] #2\end{description}}
\def\bref#1{\ref{\mychapter:#1}}
\def\dref#1{(\ref{\mychapter:#1})}
\def\drefglobal#1{(\ref{#1})}
\usepackage{graphicx} % support the \includegraphics command and options
\usepackage{array} % for better arrays (eg matrices) in maths
\def\Not{\operatorname{\sim}}
\def\St{\operatorname{St}}
\def\num{\operatorname{num}}
\def\And{\mathrel{\&}}
\def\Or{\vee}
\def\BigOr{\bigvee}
\def\<{\langle}
\def\>{\rangle}
\def\union{\cup}
\def\nleq{\not\le}
\def\N{\mathbb N}
\def\R{\mathbb R}
\def\C{\mathbb C}
\def\Powerset{\mathcal P}
\def\star#1{{}^*#1}
\def\hN{\star{\N}}
\def\hR{\star{\R}}
\def\Z{\mathbb Z}
\def\Power{\mathcal P}
\def\Implies{\rightarrow}
\def\True{\operatorname{True}}
\def\Socrates{\mathrm{Socrates}}
\def\actual{@}
\def\Law{\operatorname{Law}}
\def\Chance{\operatorname{Chance}}
\def\Var{\operatorname{Var}}

\def\H2O{H${}_2$O}

\def\scr{\mathcal}
\def\e{\varepsilon}
\def\eps{\varepsilon}
\newtheorem{lem}{Lemma}
\newtheorem{prp}{Proposition}
\newtheorem*{theorem}{Theorem}
\newtheorem{corollary}{Corollary}
\newtheorem{cond}{Condition}

\renewcommand\thechapter{\Roman{chapter}}

\def\chaptertail{\ifdefined\book\else\end{document}\fi}
 

\title{Infinity, Causation and Paradox}
\author{Alexander R. Pruss}
%\date{} % Activate to display a given date or no date (if empty),
         % otherwise the current date is printed

\begin{document}
\setcounter{secnumdepth}{3}
\setcounter{tocdepth}{4}

\end{document}
\fi

\restartlist{enumerate}

\chapter{Mersenne questions in ethics}\label{ch:ethics}
\section{Motivating examples}
\subsection{The rule of preferential treatment}
Let us begin with a more detailed discussion of an example from Thomas Aquinas's discussion of the order of charity. Aquinas thinks,
along with common sense, that those who are closer to us have a greater moral call on us.
Thus, if it is a question of bestowing the same good on one of two people, where one is more closely
related to us, we should benefit the closer one. But Aquinas writes: ``The case may occur, however, that one 
ought rather to invite strangers [to eat with us], on 
account of their greater want.''??ref And then he raises the question of what one should do ``if of two, one be 
more closely connected, and the other in greater want.''??ref

We might hope that here Aquinas would give us some clever rule for weighing connection against need. But 
instead he writes very sensibly: ``it is not possible to decide, by any general rule, which of them we ought 
to help rather than the other, since there are various degrees of want as well as of connection''.??ref It is
tempting at this point to throw up one's hands and simply say that in these in-between cases there is no
fact of the matter as to what should be done, or both options are permissible, or else relativism applies
to the case. But that would not do justice to the way we agonize when we find ourselves in such a difficult 
situation, trying to discover the truth of the matter. (It is interesting to note that the most common real-life moral dilemmas
tend to be like these kinds of cases, rather than highly controversial questions about trolleys, strategic bombing or
bioethics much discussed by philosophers.)
And indeed Aquinas maintains a realist attitude to
the question while simply offering this advice for how to figure out the answer in a particular case: ``the matter 
requires the judgment of a prudent man.??https://www.newadvent.org/summa/3031.htm\#article2

We can think of this as the problem of specifying a function $f(r,a,s,b)$ of four variables, two of them, $r$ and $s$, being
degrees of relation and the other two, $a$ and $b$, being degrees of benefit, where the function takes one of three values
corresponding to whether it is obligatory, permissible but not obligatory or impermissible to bestow a benefit of degree $a$ on a person with
relation of degree $r$ to the agent in place of bestowing a benefit of degree $b$ on someone related to degree $s$. 

In fact, the problem of a rule of preferential treatment is much more complicated than the above indicates. First, the \textit{kinds} of benefit and relation also matter: ``we ought in preference 
to bestow on each one such benefits as pertain to the matter in which, speaking simply, he is most closely connected with us.''??ref
So the function will depend not merely on quantitative features but qualitative ones. Second, although Aquinas does not mention it here,
the evaluation will no doubt depend on various features of the circumstances. And, third, in practice instead of choosing between
two certain benefits, we are choosing between two probability distributions over the space of possible benefits.

Now, as Aquinas admits, we do not know what the moral evaluation function for choices between benefits to different people is.
But abstractly speaking there is some such function, even if we do not know what it is, just as there is a function that assigns to each person
alive now the number of hairs they now have, even though we cannot specify any of the values of the function.
And we have good reason to expect the moral evaluation function to be very complicated. Indeed, probably the only serious proposal for a
relatively simple function $f$ here is the utilitarian suggestion that $f(r,a,s,b)$ yields obligation when $a>b$,
mere permission when $a=b$ and prohibition when $a<b$. But this utilitarian suggestion betrays the intuition that
the degrees of relation $r$ and $s$, much less the kinds of benefit and relation, are relevant to the moral evaluation.???refs

Indeed, the function is apt to look arbitrary. Fix the degrees of relationship to be one's parent and a total stranger,
and fix a specific and certain financial benefit of \$1000 to one's parent, and fix the circumstances. Then as we vary the 
financial benefit to the stranger from zero to infinity, we will presumably initially have a requirement of benefiting the parent
(it would be wrong to give \$1 to a stranger instead of \$1000 to a parent in ordinary circumstances), 
then a permission either way, and then a requirement to benefit the second party. There will be boundaries between these regions
of logical space, and these boundaries will look as arbitrary and contingent as the boundaries between different tax brackets.
Like the tax brackets, some proposals for boundaries will be \textit{clearly} unreasonable, but there will be many proposals
that appear reasonable. And whatever the actual boundaries will look arbitrary.

Of course, seemingly arbitrary numbers can come out of an elegant and simple rule: it seems arbitrary that the fifth and sixth 
digits of $\pi$ are $5$ and $9$ respectively, but there is an elegant mathematical explanation. But apart from the
utilitarian proposal, we do not have any at all plausible simple proposal for $f$.

These seemingly arbitrary boundaries in the order of charity raise call out for an explanation at least as much as 
the exact distance between the earth and the moon does. Just as it seems implausible that the distance between the earth
and the moon \textit{must} be exactly what it is, it seems implausible to think that the boundaries must be exactly where
they are---unless the utilitarian is right about $f$ being very simple. 

In fact, the ethics case calls out for an explanation even more than Mersenne's scientific examples did. For we might 
be able to swallow the earth-moon distance being a contingent and brute unexplained fact. But a brute fact seems unfitting for a moral
rule. A claim that it just so happened, with no explanation at all, that you should $\phi$ undercuts the moral force
of the alleged moral obligation. We expect anything seemingly arbitrary in our moral norms to have an explanatory ground.

To further argue for this point, consider a version of Divine Command Theory on which obligations are divine commands, and
God rolled indeterministic
dice to decide which actions to command, and by chance God's commands coincided with our common-sense morality, though they
could just as well as well have commanded cruelty and dishonesty. A Divine Command Theory on which it is mere chance
that cruelty is forbidden rather than commanded provides an unacceptable answer to the Euthyphro problem.??
Intuitively, a set of injunctions that is as arbitrary as that cannot constitute morality. But this point generalizes beyond
divine command theory. Suppose that that we have some preferential treatment rules that are brute and contingent, and could
just as well have enjoined on us the anti-utilitarian rule that we should always prefer the lesser benefit. Then whatever
these rules are, they do not constitute morality, but at best happen to agree with morality in content. 

Thus, even if there is some bruteness in the rules of preferential treatment, the rules in our world must be generated in a way
that makes rules such as the anti-utilitarian rules not be among the possible outcomes. But this makes it very unlikely that
the rules would be brute. For what force would limit the brute rules to avoid unacceptable options? Such a view of limited
bruteness would be akin to a view on which banana peels can come into existence \textit{ex nihilo}, but not where we might trip
over them.

It is important to remember that the Mersenne question here is a metaphysical question: What explanatory grounds are there for why this
rule, rather than some competitor, holds? The epistemic question may well have a virtue-theoretic answer like Aquinas's: if
we acquire the requisite virtues, we will be able to judge particular cases fairly reliably, and until then our best bet is
to ask the advice of virtuous others.

But before I continue the discussion of the possible explanation for the above ethical Mersenne question, let me follow
Mersenne's lead and multiply the examples, in order to defend against potential answers that only work in some cases, and
to make clear how widespread the problem is.

\subsection{Risk and uncertainty}
Some people---perhaps you---would accept a 92\% chance of winning a thousand dollars at the cost of an 8\% chance
of losing ten thousand. I wouldn't. I say that both I and they are reasonable. On the other hand, someone who 
(in ordinary circumstances) rejects a 99.9999\% chance of winning a thousand dollars at the cost of a 0.0001\% chance
of losing ten thousand and someone someone who accepts a 10\% chance of winning a thousand dollars at the cost of a 90\% chance
of losing ten thousand are unreasonable. 
It is well known that attitudes to risk vary between people, and while there are unreasonable attitudes, it is very plausible
that there is a broad range of reasonable attitudes.??refs
So, as we vary the probabilities of wins and losses, we move between cases
where accepting the risk is unreasonable, to cases where both accepting and rejecting are reasonable, to cases where
rejecting is unreasonable.

This, once again, raises the Mersenne problem of why the transitions between the various evaluative categories lie where they
do.  And of course things are more complicated than described above. The rational evaluation function will depend not just
on the probabilities involves but also on the values of the potential gains and losses. 

While in the previous case, utilitarianism provided a neat but implausible solution, so too in this case, expected utility
maximization provides a neat but implausible solution. On expected utility maximization, you are rationally required to
accept a chance $p$ of a good of degree $\alpha$ despite a chance $q$ of a bad of degree $\beta$ against a status quo of
value zero just in case the
expected utility $p\alpha + q\beta$ is strictly positive; when it is zero, you are permitted but not required; 
and when it is negative, you are not permitted. One problem with this solution is it requires all goods to be neatly
quantifiable (cf.\ the next example for difficulties related to that). But the more serious problem is that it requires an
implausibly negatively judgmental attitude towards ordinary people's attitudes to risk.

Indeed, here is a plausible trio of theses about risk that are incompatible with expected utility maximization:
\ditem{2-nobound}{There is no upper bound on possible finite utilities.}
\ditem{2-finite}{A decade of the worst tortures the KGB could think of has a finite negative utility.}
\ditem{2-notworthit}{There is no possible good $G$ of finite utility such that one would be rationally required in
accepting a certainty of a decade of the worst tortures the KGB could thing of one for a one in billion chance of $G$.}
For as long as $(1/1000000000)\alpha + \beta>0$, where $\alpha$ is the value of $G$ and $\beta$ is the (highly negative)
value of the tortures, one would rationally required to accept the deal on expected utility maximization, and by \dref{2-nobound}
and \dref{2-finite} there exists a possible $G$ that makes $(1/1000000000)\alpha + \beta$ strictly positive.
Hence, we should reject expected utility maximization, and and absent expected utility maximization, it is likely that the rationality evaluation function for risk will be messy
and arbitrary-looking.

The most plausible thing for the apologist for expected utility maximization to reject is the no-upper-bound thesis \dref{2-nobound}.
Here is one way an argument for such a rejection might go. First, there is a maximum intensity of goods that our brain can handle.
Second, goods become significantly less valuable as they are repeated, decreasing in such a way that the sum of the values of any 
goods you could have over an arbitrarily long life has an upper bound.??refs

But the repetition thesis is only plausible when boredom and other memory-based phenomena are in play. Suppose you have 
lived for a very long time. Then you suffer from partial amnesia: you have
lost all episodic memory of your past meals and of your past pinpricks. You are offered what you are reliably informed is 
the most delicious and wholesome dessert every prepared by the best chef on earth, a dessert  which you are told you've eaten some large 
number $n$ times in the past, and you may eat the dessert at the cost of a one in ten chance of a small pinprick. It's clearly worth it,
regardless of what $n$ is. So now suppose this happens to you every day of a very long life. The marginal value of each such 
dessert (i.e., the amount it contributes to total lifelong utility), absent memories of past desserts, must  be at least one 
tenth of the marginal disvalue of the pinprick, at least given expected  utility maximization. But the disvalue of the pinpricks 
clearly does not tend to zero with forgotten repetition. Hence, the value of the desserts does not tend to zero. And hence for any
finite utility bound, enough such desserts will exceed the bound.

For a different example, not involving radically large utilities, imagine this Star Trek plot. Captain Kirk visits a planet inhabited
by two intelligent alien species, all on par with respect to moral value: there are $1,000,000$ oligons and 
$2,000,000,000$ pollakons. Unfortunately, an asteroid is heading for the planet. If nothing is done, it will hit the planet in such
a way as to wipe out all the pollakons. The only thing Kirk can do is to fire phasers at the asteroid. Spock has calculated that if this
is done, the asteroid's track will be redirected in such a way that it will wipe out all the oligons. Kirk asks whether that will help 
the pollakons? Spock's answer is that probably not: there is a $999/1000$ chance that all the pollakons will still die, but there is
a $1/1000$ chance that they will all survive. 

It seems very plausible that Kirk should not fire phasers. And it is even more plausible that Kirk is not required to fire phasers.
He should not sacrifice the oligons for a small chance of saving the pollakons. 
But the expected utility of firing phasers is $-1,000,000+(1/1000)(2,000,000,000)=+2,000,000$ lives.

We can also argue against expected utility maximization by considering the following case. Suppose that on every day $n$ of
eternity, with $n\ge 1$, you are offered the opportunity to pay half a unit of utility in exchange for playing a game with a 
$1/2^n$ chance of winning $2^n$ units of utility. By expected utility maximization, you would value the value of the game at
$(1/2^n)\cdot (2^n)=1$ units of utility, and at a price of $1/2$ units, it would be worth playing. 

But consider what will almost surely happen if you adopt the policy of following expected utility maximization and playing the game, where ``almost sure'' is 
the technical term that probabilists use to describe an event that happens with probability one (such as getting heads at least 
once if you toss a fair coin infinitely many times). The sum of the probabilities of winning on the different days is finite: 
$1/2+1/4+1/8+\dots=1<\infty$. The Borel-Cantelli Lemma??ref then says that almost surely you will win only a finite number of times.\footnote{We 
can give an elementary proof of this fact in the case at hand (the proof generalizes to the general case). Let $W_n$ be the event that you will win at least once after day $n$. 
Then $P(W_n)\le 2^{-(n+1)}+2^{-(n+2)}+\dots = 2^{-n}$. Let $W$ be the event that you win infinitely many times. Then $P(W)\le P(W_n)$ for
every $n$, since if you win infinitely many times, you must win on infinitely many days after day $n$, and so you must win on at least one
day after day $n$. Since $P(W_n) \le 2^{-n}$, we have $P(W) \le 2^{-n}$ for every $n$. But probabilities cannot be negative, and the only 
non-negative real number $x$ such that $x \le 2^{-n}$ for every $n$ is zero. So $P(W)=0$, and hence almost surely $W$ does not happen,
so that almost surely you win only finitely many times.} In other words, almost surely, there will come a day after which you will win no more. At that point, you may well be ahead,
having won more than you paid. But the sum of what you won is finite, and from then on you will just lose half a unit of utility every 
day. Eventually, there will come a day when your losses will overtake your winnings, and from then on, you will just fall further and further
behind every day.\footnote{The example above uses exponential growth. More moderate growth will work, as long as the sum of the probabilities
is finite. Thus, we could say that on day $n$ the prize is $n (\log (n+2))^2$ and the probability of winning the prize is the reciprocal
of that, since $\sum_{n=1}^\infty 1/(n (\log (n+2))^2)<\infty$.}

The very unhappy situation of playing infinitely many times and eventually starting to lose every time is the almost 
sure result of following expected utility maximization on each day. We can compare this to the neutral situation of 
refusing ever to play, and getting zero each day, or the situation of accepting the expected utility maximizing gamble
for a number of days, until the probability of winning becomes really small, and refusing from then on.

It is worth noting that this is not just a paradox involving the aggregation of infinitely many utilities, except in the trivial
sense that infinitely many zeroes make a zero (i.e., there is overall no benefit from playing the game once you stop winning). 
Almost surely, after a finite number of days, the expected utility maximizer
falls behind the consistent refuser, and every day after that, the expected utility maximizer is further and further behind,
like someone who got a subscription to a streaming service and forgot to either use or cancel it. And all these amounts are 
finite, and a finite, albeit unknown, distance into the future.

We can also consider an interpersonal version of the story. Suppose we have (countably) infinitely many people, numbered $1,2,...$, and person $n$ 
is offered the chance to pay half a unit of utility in exchange for a chance $1/2^n$ of winning $2^n$ units. As before, by expected utility
considerations it's worth it. So, if everyone is an expected utility maximizer, everyone will pay. But by the Borel-Cantelli Lemma, 
almost surely, only finitely many people will win. Thus, almost surely, we will have infinitely many people
pay a cost of half a unit each, and finitely many people win some finite amount. This is a disastrous situation, with a negative infinite
overall utility. Almost surely, it would be much better if everyone refused to play, or only those who had a ``non-negligible'' chance 
at winning played.\footnote{It is worth noting that exponential growth is not necessary for the examples to work. All we need is that
there is a chance $p_n$ of winning a prize of $1/p_n$, and that $\sum_{n=1}^\infty p_n < \infty$. While for ease of calculation above
I let $p_n=1/2^n$, one can have much more moderate shrinkage, such as $p_n=1/n^2$ or even $p_n=1/(n(\log (n+2))^2$.}

??refs:PrussBlog2011,Zhao, 
Wilkinson??ref:https://philpapers.org/rec/WILRAA-16

It appears that expected utility maximization cannot be rationally required. But it is the only clearly non-arbitrary solution
to the problem of deciding under uncertainty.

In addition to Mersenne questions about risk and prudential rationality, there will be Mersenne questions about risk and morality.
For instance, what risks we may morally impose on others in exchange for a good to ourselves depends in a complex way 
on one's relationship to these others, the probability of the risk, the degree to which these others accept the risk, the 
benefit to self, and so on. When I drive, I risk killing other drivers, their passengers, pedestrians by the side road, and so on.
But the probability of these awful outcomes is very small, and typically other people on or by the road have accepted reasonable
risks (or have had them accepted by proxies, in the case of children), so these dire but unlikely outcomes typically do not render it impermissible for
me to go to the grocery store to pick up ice cream.\footnote{I leave open the question whether concerns about global warming 
render it impermissible.} But when the risk is higher, say because I am tired and sleepy after a long day and hence less likely to be
a safe driver, the matter becomes less clear. At some point, as the risk increases, it becomes impermissible to go to the grocery
store for ice cream. 
A particularly thorny set of issues arises in the special case of balancing the risk that the innocent are punished with the risk that the guilty go free.
And we have the Mersenne question of why the switchovers happen where they do.

Expected utility utilitarians\footnote{As opposed to actual-outcome utilitarians who evaluate actions morally based on the
actual utilities that would result from an action.??refs} will have a nice answer to this problem. But utilitarianism, as already
noted??ref, has many highly counterintuitive implications. 

????add: moral risk?

\subsection{Orderings between goods}
Under ordinary circumstances, it would not be reasonable to choose to be a mediocre mathematician rather than a superb musician. 
But suppose one's choice is whether to be a superb
musician or a superb mathematician? Here we are dealing with incommensurable goods and either choice is reasonable.

But now let's ask this general question: Is it is reasonable to choose to be a mathematician of quality $\alpha$ rather than
a musician of quality $\beta$? Again, we have a function that takes a number of variables, including $\alpha$ and $\beta$
and the circumstances, and tells us whether (a)~it reasonable to opt to become a mathematician but not reasonable to opt for
music, or (b)~both are reasonable, or (c)~opting for music is reasonable but opting for mathematics is not. And, just as before,
it is very plausible that the function is extremely complex.

The problem obviously generalizes to all the many kinds of pairings of incommensurable goods there are.  In each case, there 
will be some function of many variables encoding the correct rational evaluation of the situation/, and we will have the Mersenne
question of what grounds the fact that this function, rather than one of the infinitely many others, encodes the correct
rational evaluation.

We also have Mersenne questions here that involve qualitative rather than quantitative comparisons. Other things being equal,
social pleasures are better than solitary ones. This seems rather arbitrary. What makes it be so?
 
In the preferential treatment and moral risk examples, utilitarianism offered a nice solution. But the problem of incommensurable goods is
also going to be a problem for any plausible utilitarianism. Utilitarianism comes in two varieties, depending on whether
the good is pleasure or the good is satisfaction of desire. As Mill famously noted, it is essential to the plausibility
of utilitarianism that one be able to make a distinction between lower and higher pleasures, so as to get the common-sense
conclusion that it is better to be Socrates unsatisfied than to be a satisfied pig.

But once one makes the distinction between lower and higher pleasures, or lower and higher desires, incommensurability
quickly shows up, since different kinds of pleasures and desires do not simply come in a linear ranking. Let's suppose that you get more 
enjoyment and satisfaction of the desire for truth out of mathematics and more enjoyment and satisfaction of the desire for music out of music, and let us suppose (contrary to typical situations)
that your choice of life will not affect anyone else. Then it seems right to say that the mathematical and musical lives are
incommensurable even on utilitarianism. But even if they are not incommensurable, but equal or one is better than the other, 
we still have a Mersenne problem as to what level of quality of mathematical life exceeds, equals or falls below what level of 
quality of musical life. And in fact it will be more complex than that, in that the quality of a mathematical or musical life
is clearly multidimensional.

One might try to get out of this by hoping for some precise definition of the degree of pleasure or the strength of a desire.
Perhaps there is a neural correlate of the degrees of pleasure or the strengths of desire that can be quantified in a single
number. But such an approach is likely to lead to the swinish utilitarianism that Mill wisely rejects. For presumably the
neural correlate can be manipulated directly, and the pig could be given pleasures which, in terms of neural intensity,
exceed the highest of Socrates' refined joys, and could be made to have a degree of intensity of desire for its swill far
exceeding Socrates' desire for virtue.  

Moreover, any neural approach is likely to fall prey to questions of cross-species comparison. While pig and human brains are
similar, they are not the same, and states of pleasure and desire are likely to be merely analogical. It is clear that some
comparisons between human and porcine goods are possible: a tiny human pleasure is worth less than a great porcine one. As one
increases the human pleasure and/or decreases the porcine one, there will come cases where neither of the two is to be
preferred, and then eventually cases where the human pleasure is to be preferred over the porcine one. But where exactly
the cross-over points are is not something we can just read off the neural correlates. And things get even messier when we
compare humans to possible beings that have no brains, such as intelligent robots (if these are possible) or aliens with very 
different biochemistry.

And even if one could give some such precise formulation, we would still have
the Mersenne problem of why \textit{this} formulation corresponds with true value rather than some other. 

\subsection{Intersubject aggregation of value and population ethics}
Whether or not consequentialism is true, there are some questions which need to be settled in a 
consequentialist way, for instance questions where the stakeholders are strangers one has no special obligations 
towards and where there are no deontic considerations. If we aim for the consequentialistically best outcome for 
more than one person, we will need a way of aggregating value between these subjects. A particularly difficult set
of cases comes up when the number of subjects in existence varies between the options.

Perhaps the most straightforward option is to \textit{add up} utilities across the affected population. This faces
several problems. The most famous is Parfit's repugnant conclusion??ref. Any finite scenario full of highly
fulfilled people can be beat by a scenario with a much larger number of people whose level of fulfillment is
minimal. Yet it seems implausible to think that we should aim at vastly multiplying human population at a 
minimal level of fulfillment. 

A technical problem is the following. Utilities are normally considered to be defined ``up to positive
affine invariance''??ref. They can be rescaled and shifted without changing anything. This means that for any positive 
number $\alpha$ and any number $\beta$, if we consistently replace every utility $x$ with $\alpha x + \beta$, then 
we have not changed anything. But if we add utilities across a variable number of individuals, although multiplying all
utilities by a positive factor $\alpha$ makes no difference to our aggregate decisions, the addition of 
a constant $\beta$ to every utility can make a difference. For instance, suppose we have a choice between ten individuals
each enjoying a utility of $15$ each and twelve individuals enjoying a utility of $10$ each. As it stands, on
the additive model of aggregation, the ten individual option has a total utility of $10\cdot 15 = 150$ and the twelve
individual option has a total utility of $12\cdot 10 = 120$, thereby yielding a preference for the ten individual
option. But if we take into account the affine invarience with $\beta = 25$ and $\alpha=1$, then the utilities enjoyed 
by the individuals in the two scenarios become $15+25=40$ and $10+25=35$, yielding totals of $10\cdot 40=400$ and 
$12\cdot 35=420$, flipping the preference in favor of the larger population option. 

Essentially, the technical problem is that we need a ``zero point'' for utilities. If we have such a point, then increasing
the number of people with utility above zero will always improve the outcome, while increasing the number of people below it
will make things worse. Now, while there may be clear cases---Einstein's life was above the zero point, but a life of constant
torture is below---defining a zero point precisely in terms of the vast multitude of various good- and bad-making features of 
a human life is apt to involve a large number of Mersenne questions.

The most natural alternative to adding utilities is averaging them. As Parfit has noted, this leads to the another
unpleasant conclusion: if the average of some nation's utility is even slightly below the average utility for all
human beings, then we would be better off if the people in that nation didn't exist.\footnote{The problem becomes
perhaps even more vivid if we include non-human animals as subjects. For it may well be that no non-human animal
enjoys a utility greater than that of an average human. One way to this conclusion is the intuition that no 
non-human animal is such that we should save its life instead of an average human's. Another way is to reflect on
the fact that an average human enjoys goods---say, moral or cultural ones---that are qualitatively higher than those
of any non-human animal. On any averaging view, then, it seems it would be good to allow all non-human animals on 
earth die out. (Antinatalists may of course disagree with the intuitions in this footnote.??refs)} 

Furthermore, averaging only escapes the repugnant conclusion for humans given assumptions about what the rest of reality 
is like. Suppose that it turns out that humans are outnumbered by non-human persons by a factor of ten, and the overall average
utility in reality is terrible, as most persons live lives of horrific misery, so that the average utility of a non-human 
person is $-100$ while that of a human is $10$. This makes the average utility overall be $(-100\cdot 10 + 10)/11 = -90$.
Now suppose we have a choice between two options. On the first option, we can make every human enjoy a utility of $100$, which
is a life of deep fulfillment, without changing the number of humans. On the second option, we can multiply human population
by a factor of $10$, but make each person's utility be a very miserable $-50$. On the first option, overall average utility
becomes $(-100\cdot 10 + 100)/11 = -81.8$. On the second option, it becomes $(-100+(-50))/2 = -75.0$. In other words, it is 
worthwhile to multiply human population as long as humans are even a little less miserable than the average among non-humans,
and depending on what options are available, multiplication of human misery, with sufficient increase of population, may be 
better than making all humans happy. This is even more repugnant than the original repugnant conclusion, since we are 
``improving'' things by making people not merely slightly well-off but by making them actually miserable, just not as miserable
as the average.

The last point vividly illustrates a well-known problem with averaging: what decision is right depends on epistemically
inaccessible facts about intelligent life outside earth.??ref One can, of course, solve this by restricting the averaging
to our own species and hoping that we won't meet aliens. But we can still get
some version of this repugnant conclusion simply within the human species. We can imagine a situation where three quarters of humans 
are found in a repressive nuclear state, on average have an utterly miserable flourishing level of $-100$, and there is no way for those of us outside
that state to remedy the situation. Suppose instead we have a choice between two policies for the rest of the world: keep the 
population of that part of the world constant and bring everyone to enjoying approximately a utility of $20$ or triple the population 
and bring  everyone down to a pretty miserable $-35$. The resulting averages will be $(-100\cdot 3/4)+20\cdot 1/4 = -70.0$ and $(-100+(-35))/2 = -67.5$.
As in the alien case, we have a nasty version of the repugnant conclusion: lots of people at a low level of happiness---indeed, a level 
of significant misery---outweigh a moderate amount of people at a high level of happiness. 

It is worth noting that in the case of averaging, there is a decision point about aggregating across time. Averaging the utility 
across all subjects at all times is mathematically straightforward if the the number of subjects across time is finite. If it is infinite,
there are many additional complications.\footnote{$^*$If the infinity is countable, then any averaging will have to 
involving a limiting procedure. For instance, if $u_i$ is the utility of the $i$th subject, then we can aggregate by computing
$\lim_{n\to\infty} \sum_{i=1}^n u_i$. However, the value of that limit is likely to depend on the order of the subjects. 
If all the subjects are in the same spacetime, we might try to order the subjects temporally. That itself carries some decision
points. Do we order subjects temporally by the beginnings of their existence, the middles, or the ends? And in relativistic
spacetimes, temporal ordering will be relative to a foliation, and so a privileged foliation will be needed. Though we might get
lucky and the limit might be the same for any reasonable ordering. If, however, the subjects are in different spacetimes---as in a 
multiverse scenario---then finding an appropriate for the limit is even harder.??ref} However, averaging across all
subjects at all times could seriously exacerbate some of the repugnant conclusion worries. For instance, if non-rational 
conscious animals count, then the vast majority of conscious animals on earth across all time may be relatively primitive,
say on the level of lizards and squirrels, and hence not capable of much flourishing. As a result, the average level of happiness
may easily end up being even lower than if we just average at the present time, thereby making it more practical to increase 
average flourishing by vastly multiplying subjects, human or not, with low levels flourishing.

The other intuitively natural option is to average simultaneous utilities at each time, and then integrate them across time. This runs into
at least three special problems. First, we have a difficulty of how to account for a multiverse whose universes do not share a 
common time-line. Second, we need a privileged sequence of reference frames---a privileged foliation---to define the 
simultaneity of utilities. Third, and perhaps most problematically, many aspects of the flourishing or languishing of 
humans are difficult to localize in time. Examples include having a \textit{diversity} of cultural experiences over a lifetime, having 
one's goals be posthumously fulfilled or frustrated, or failing to ever find a good friend.


It is possible that some simple function without free parameters can be used to aggregate utility across people in a way that
avoids paradoxes. But it seems very unlikely. Intuitively, we have some sort of diminishing returns as we increase the number of
people with a minimal level of happiness, but the diminishment is not the simple ``one over the population count'' multiplier of
the averaging solution. It seems very plausible that an aggregation function that does not generate some repugnant conclusion will
be rather complex and will have multiple free parameters. And then we can ask what grounds these parameters' having the values they
do. ??see if literature proves there is no function


The above assumed, for ease of modeling, that we could reduce the utility enjoyed by each individual to a number. For reasons 
discussed in Section??ref, this is itself unlikely. Our flourishing seems to consist of a number of incommensurable goods such 
as friendship, understanding, play, etc. Moreover, even if we reduce flourishing to pleasure or desire satisfaction, it is 
likely that we have incommensurable pleasures and incommensurable desires. All this greatly complicates any aggregation procedure,
multiplying the number of its free parameters.

\subsection{A miscellany of other Mersenne questions}
\subsubsection{Introduction}
There are many other cases which involve thresholds or transitions that appear to be arbitrary. We will discuss them briefly.
Some readers will disagree with a number of the examples I gave. (Double Effect, for instance, is quite controversial;  philosophical
anarchists will deny that any state is such that one morally ought obey its commands as such; I myself am not committed to 
the picture of lethal self-defense given below.) But it seems
likely that a sufficient number of the remaining examples will still compellingly raise Mersenne problems. And the list is not
exhaustive: the reader should be able to generate more items, as the following are largely given as examples.

\subsubsection{Torture}
On strict deontological views, one shouldn't torture one innocent person to save any number of lives. But of course
it would be permissible to gently prick someone with a pin to save even one life. Somewhere between the pinprick
and the torture is a transition. What makes the transition be where it is?

On threshold deontological views, it is wrong to torture one innocent to save a small number (say, one or two) of lives,
but it is permissible to do so to save a very large number (say, a billion). Again, we have a transition to be 
explained.\footnote{I am grateful to Philip Swenson for this example.} And note that even if one is a strict deontologist
about torturing the innocent, likely one is a threshold deontologist about some other things. Thus, one may think it's
permissible to save an innocent life but not permissible to lie to get a deserved (but on other grounds) salary raise,
and hence there needs to be an explanation of the grounds of the transition from permissibility to impermissibility. Or
one may think it is permissible to trespass on a neighbor's property to save a cat's life but not to save a grasshopper's.
Probably everyone who isn't a full-blown consequentialist is a threshold deontologist about some things.

\subsubsection{Proportionality in Double Effect}
The Principle of Double Effect allows one to foreseeably cause bad effects that it would
be impermissible to cause intentionally, as long as these bad effects are not intended either as ends or means. For instance, it seems permissible to bomb Hitler's headquarters even
if one finds out that an innocent prisoner is held captive there. But of course there needs to be a proportionality
condition imposed on this: the good achieved, say the end of a war, must be proportionate to the bad, say the death of the prisoner. 
It would be wrong to demolish an old building while knowing that there is a child playing inside: the good of having a lot to build
on is not proportionate to the death of the child. So there will be some function of variables including harms and benefits that
specifies when the benefit is proportional to the harm in Double Effect contexts. In fact, there will be other variables, such as
one's relationships to those harmed and those benefited. 

\subsubsection{Self-defense}
It is widely held that lethal self-defense an aggressor is permissible. There are at least two questions that come up here.
The first, similar to (or maybe even reducible to), the question of proportionality in double effect: what degree of aggression licenses lethal self-defense. For instance, if an assaulting party clearly threatens only to shave your hair, lethal self-defense is unjustified, even though such an assault constitutes battery. On the other hand, if the assaulting party is likely to kill you, then lethal self-defense appears permissible. But threatening something tantamount to death is probably not sufficient to justify lethal self-defense. If an evil surgeon is going to cut off your healthy legs without your consent, many will say that lethal self-defense is permissible even if medical care is sure to save your life. Furthermore, the chance of the aggressor succeeding in inflicting the threatened damage matters. Lethal force is not permitted against a malicious child with a foam sword, even though there is a tiny chance that the sword will trip you up, resulting in your hitting your head on a rock and dying. At the same time, certainty of the threatened damage is clearly more than is required. So there is now a Mersenne question of what explains what combinations of chance and damage justify lethality in self-defense.

The second, and more conceptually complex, question is what constitutes aggression. Again, there are clear cases of what does not 
and what does. If a terrorist hands Alice a weapon and informs her that she will be killed if Bob is still alive in fifteen 
minutes, and it is not physically possible for Alice to use the weapon against the terrorist, she cannot engage in lethal 
self-defense against Bob on the grounds that by being alive he is an aggressor. On the other hand, the terrorist's credible threat 
makes them an aggressor, and it is very plausible that Alice can legitimate engage in lethal self-defense against them if it were 
to become possible.

In between the extremes of threatening by simply being alive and threatening by culpable voluntary attack is a range of cases, 
such as threatening by breathing (e.g., there is only enough air in spaceship for one of two people), threatening
by involuntary activity similar to voluntary activity (e.g., a muscle spasm is about to happen resulting in a detonator 
being activated), threatening in non-culpable non-moral ignorance (e.g., thinking one is controlling a video game when one is controlling an armed real-life robot), threatening by voluntary and right activity (e.g., a police officer using justified lethal force), wrongly threatening in non-culpable moral ignorance (e.g., innocently not knowing that the situation at hand is one where disconnecting the ventilator is impermissible), and threatening in culpable moral ignorance (e.g., engaging in genocide because
it is more convenient to come to agree with the leader than to investigate the moral questions for oneself). 

Moreover, the lines between these are themselves more continuous than may initially strike the eye. It may seem, for instance,
that it is very natural to draw the line between cases where the threatening party is acting rightly and where they are
acting wrongly. But now consider cass where the threatening party is acting only \textit{slightly} wrongly, for instance 
because they are a police officer engaging in a use of lethal force that is only slightly disproportionate.\footnote{There may
also be cases where the wrongfulness is not of the relevant sort to justify lethal self-defense. For instance, suppose that 
as a witness to the importance of solving problems non-violently, Alice vowed non-violence for life. Bob now threatens Alice's 
life, and Alice pulls out a gun and is about to shoot Bob in self-defense. Assuming that a vow of non-violence is morally 
binding, it is wrong for Alice to shoot Bob. But the wrongness of Alice's shooting Bob does not justify Bob's switching 
to ``self-defense mode'' and pulling the trigger before Alice can pull hers.} Likewise, one might think the difference between
inculpable and culpable ignorance is sharp, but there are cases where the ignorance is only slightly culpable. For instance, 
consider the case of the doctor disconnecting someone from a respirator incorrectly thinking that the disconnection is permissible,
where the reason the doctor was mistaken about the morality of the situation was because she was five minutes late for an 
ethics lecture many years ago in medical school, and their lateness was only very slightly culpably wrong (they should have 
put slightly more effort in running to catch the bus). 

It is thus unlikely that there is a natural place where we draw the line as to the nature of the aggression. But there are
truths of the matter, and a Mersenne question as to why they are what they.

\subsubsection{Respect}
We need to show respect for rational beings. This respect includes such things as not killing them when they are
innocent and non-aggressive, not eating them (except perhaps in extreme circumstances), not acting as if they were fungible, treating them as ends rather than as
mere means, and so on. But what is an intelligent being? First, we have a distinction between an individual and a kind based
concept of intelligence: on the former, a being is intelligent to the extent that it currently has certain intellectual powers;
on the latter, a being is intelligent to the extent that it is of a kind that should have certain intellectual powers. But whichever we choose, and
plausibly there are principled reasons to choose one rather than the other\footnote{Though they will be highly controversial, since 
a significant part of the debate about the moral status of the unborn turns on this.}, 
we still have a Mersenne question as to the degree of intellectual power---whether actual or proper to the kind---that is 
needed for us to have duties of respect. Intellectual powers, after all, clearly come in degrees, and if at some point respect
is called for, we need an explanation of why that point shows up where it does.

The question of what in fact the degree of intellectual powers is needed for respect is one that we actually face with regard to our treatment
of higher mammals on earth, and that we currently only face hypothetically with regard to extraterrestrial life. 
It is an important question. But, as usual, the Mersenne puzzle isn't that of determining what the fact is, but of what makes 
an answer be an answer, especially in light of the appearance that any threshold will be arbitrary.

\subsubsection{Punishment}
Punishment should not be disproportionate to a crime. \textit{Lex talionis} provides a neat and elegant account
of this: the criminal get done to them the same thing as they did. But at the same time, \textit{lex talionis} is 
not in general plausible. It is morally abhorrent to torture torturers or rape rapists. And even if we accept such
abhorrent extremes, there are cases where punishment simply has to switch types of harm. If a thief does not have enough
honest property of their own to make possible a deprivation equal to what they stole, imprisonment is a plausible alternate currency,
but there is no simple and elegant formula providing an exchange rate between the value of stolen property (perhaps
corrected for the economic state of the victim) and length of imprisonment, unless it turns out---as is quite unlikely??back/forwardref---that 
that there is a precise way of quantifying personal utilities. Maybe there is no simple and elegant formula, but a 
complex one. If so, it is puzzling what grounds it. And even if there is no specific formula, there are bounds 
of moral acceptabiloty: a day's imprisonment for stealing and destroying a car is insufficient; 
a lifetime's imprisonment for stealing a book is excessive. And what grounds these bounds?

\subsubsection{Consent}
Standards of consent necessary to permit one's being treated a certain way vary widely depending on the treatment.
There are multiple dimensions in which we can measure the ``strength'' of a consent requirement: how well informed the 
consenting party needs to be, what age or level of intellectual development does the party need to have, what proxies if
any can offer consent on the party's behalf, how unpressured the consent needs to be, how clearly formulate the consent
needs to be, whether the consent must be specific to the case or whether prior blanket consent suffices, etc.
Under ordinary circumstances, no consent---at most, lack of refusal---is needed for a pat on the shoulder. The permissibility
of major surgery, however, has a consent requirement of significant ``strength'' along many of the above axes. On the other hand,
the permissibility of sex has a consent requiremnt of even greater ``strength'' along some of the above axes---thus, while
proxy consent and prior blanket consent can suffice for major surgery, they do not suffice for sex.\footnote{It is tempting
to explain this in terms of the fact that surgery---or at least the sort of surgery for which proxy consent suffices---benefits 
the patient regardless of the patient's consent, while sex is only beneficial when consented to. But this is arguably false.
Parents can validly consent to an organ transplant between their children, even if the donor is not expected to benefit
on balance (though generally there is a benefit from having one's sibling alive!).} The mapping between
the form of treatment and the multidimensional strength of consent is of great complexity, and has an appearance of significant
arbitrariness. What grounds it?

\subsubsection{Normative powers}
We have a number of normative powers that we exercise through communicative acts with specific illocutionary force.
With promises, we create obligations for ourselves, and with commands, we create obligations
for others. With requests, we create reasons for others, and with permissions, we remove reasons for others. 
The scope of our normative powers has limitations, though where exactly the limitations lie can be a matter of 
controversy. Perhaps the clearest case is requests, where as a society we have developed a broad spectrum of
levels of insistence, signaled by linguistic and extra-linguistic cues. Normally, a 
more insistent request creates stronger reasons and a less insistent one creates weaker ones. We do not appear
to have any lower limit on the strength of reasons we can create solely by requests. We can always add yet another
``But, please, don't let me impose on you'' to weaken the request. However, there is a contextually-variable
upper limit on the strength of reasons created by a request. One can roughly measure the stregth of these
reasons, say, by the cost to the requestee at which fulfillment becomes pretty unreasonable. The strength of
the reasons to fulfill a request is then a function of the intrinsic reason provided by the requester's needs
(if someone is starving, one has reason to offer them food even if they don't ask for it), the degree of
insistence, and the relationship between the two parties.\footnote{Note that the relationship may itself have been modified
by the fact of the request. A friendship might be damaged by an unreasonable or rude request, and strengthened by 
the vulnerability revealed in a disclosure of need.} Normally, then, the more insistent the request, the more it skews
the requestee's reasons in favor of the requested action, but there is a limit to how far one can skew the strength of
reasons away from the no-request \textit{status quo}. If I am not actually in poverty, requesting money from strangers
for my personal pleasure cannot create a very strong request-based reason, no matter how much I ask for it.\footnote{Though
of course I might create a prudential reason to give me the money to shut me up, or out of fear that I am a mugger, but
that's not a request-based reason in the sense I am talking about.}

We have similar limits on the strength of reasons coming from the exercise of other normative powers. The example of
promises is particularly interesting here in that we might think that our normative powers are strongest here since 
we use them to bind our own wills. But notice that while I can probably make a promise to defend my friend's life 
where the promise creates a reason whose strength is such that I am obligated to seriously risk my own life, by promising
to come to my friend's party I cannot create a reason strong enough to stand against serious risk to my life, unless
there is something very special about that party.\footnote{Perhaps the party is the best hope for reconciling with 
someone who is dying of cancer, and my friend cares deeply about the relationship.} Even in promises, there are 
serious limits to how far we can affect the reasonableness of our decisions. 

Social convention sets certain aspects of the mapping from communicative acts exercising normative powers to the normative
effects. It determines which normative powers are exercised in which words or gestures, and how various communicative and
extra-communicative features affect the strength of reasons. But social convention works within the limits discussed
above. Social convention can \textit{say} that if I spit on my hand and shake hands after promising to come to your
party, then I am obligated to come to the party even if it costs me my life, but in fact this action would not create
any such reason, since skewing the intrinsic reasons thus far is just as beyond our normative power as running a one minute
mile is beyond our locomotive power. But while we can find a physical explanation for the locomotive limits on an 
individual human, the normative limits raise Mersenne questions about all of their free parameters---and they have many,
since there are multiple normative powers and in each one the limit will depend on multiple contextual factors.

\subsubsection{Political: constitution problem}
Political philosophy also provides a number of examples of seemingly arbitrary parameters. Consider the constitution problem. A state has a written or unwritten
constitution specifying what must happen for legislation to be valid and hence authoritatively binding on the citizens. But 
how is a constitution instituted? One theory is that it happens by the consent of the people.??Aquinas But obviously for any state
of sizeable size it will be false that all the people have consented: some have not made their opinions heard and some have been
overruled. Requiring ``consensus'' or a supermajority raises the question of exactly how many dissenters can be tolerated, and
once that question is answered we have a Mersenne question as to what grounds that cut-off being where it is. Requiring a simple
majority or plurality involves one less free parameter: the cut-offs in having more than half of the voters or having more voters
than any alternatives seem non-arbitrary. However, even a majority or plurality based system leaves questions about other parameters.
Does one need a quorum of the governed? It would not seem right, for instance, to have a vote on a constitution on a day 
where the bulk of the population is unable to get the polls due to a hurricane or a war, and as a result only a small number of
unrepresentative citizens can express their opinion. But if a quorum is required, then of course we have a Mersenne question
as to exactly what constitutes quorum.

Voting cut-offs and quorum are fairly easy to quantify. But what about the question of who the people giving their consent are?
Presumably, small children should not be eligible. But where do we draw the line between small children and paradigmatic adult
deciders? Any age-based line raises several Mersenne question---one about the numerical age cutoff and multiple questions about 
how age is measured (from fertilization, implantation, brain development, beginning of the birth process, completion of the
birth process, etc.)? A cut-off based on mental capacity, on the other hand, involves many parameters that need to be set, because
there is no single measure of mental capacity, and so one needs to have multiple measures with their respective weights. Moreover,
we have a decision point on whether those who do not get a vote have proxies voting for them (e.g., parents) and, if so, who 
counts as whose proxy.

Or consider such details as how well-publicized the constitutional consultation needs to be, how clearly spelled-out the 
constitution needs to be for people's vote on it to be valid, and who gets to decide which options are presented to people?

Many of the above questions only make sense in the case of a formal consultation process of a sort that has occurred rather
rarely over the course of human history. If we are not to think the vast majority of polities to be illegitimate, the account
needs to allow for implicit consent, maybe of the sort involved in social customs. But there things become much less clear.
There will almost always be some citizens who regard the state as illegitimate---indeed, some will regard any state 
as illegitimate. For many people, acceptance of a political system's legitimacy is not an simple binary question, but 
something that comes in degrees and has contexts. Imagine that two thirds of the population has a credence of two thirds
that the political system is legitimate, and the remaining third of the population has a credence of ten percent in 
the system's legitimacy, and they express these credences in their actions. Should we then look at the average credence
of the relevant citizens (e.g., those of age)---$0.48$ in my example above---and see if it meets some cut-off? If so, the
cut-off will raise Mersenne questions. Moreover, people often do not just have a single credence it the all-or-nothing 
legitimacy of the political system, but rather have different credences regarding different aspects of legitimacy: is this
a state that has the right to use violence against its citizens to enforce laws, is it a state that has a right to levy taxes,
to draft citizens to defend it or do other work for it, etc.

\subsubsection{Political: dissolution problem}
A set of Mersenne questions similar to those raised by the constitution problem is raised by the dissolution problem: the question
of when it is that a political regime becomes illegitimate, and the respects in which it may be illegitimate (thus, perhaps, the
traffic regulations of the Nazi state were legitimate). One might think of dissolution as resulting from the state's failure to
keep its side of the social bargain. But there has probably never been a state that kept its side in every respect. It is only
gross failure that implies illegitimacy, but that raises a Mersenne question about the degree of grossness.

\subsubsection{Political: scope of authority}
We have a set of Mersenne questions regarding who lies within the scope of the state's authority. Typical human states
have authority largely but not entirely defined geographically---for an exception, consider ships under a country's flag that
move on the high seas.??check Geography itself raises Mersenne questions. Suppose we say that a state in the future has authority
over the same territory as it now occupies. But what counts as ``the same territory''? We live on a planet that is constantly
changing its shape, whether at a large scale due to movements of tectonic plates or a small scale due to our own digging. When
a tectonic plate shifts, how does territory shift? There are, of course, precise ways to answer these questions. We define
latitude and longitude in terms of coordinates on a mathematically idealized oblate spheroid approximating the earth. We might
then define ``same territory'' in terms of these coordinates. But there are infinitely many ways of mathematically modeling the
earth at any given time and infinitely many ways of matching up that model to the physical soil of the earth over time.
And it seems unlikely that the geography of a planet is in the end what defines a state. It may well be the case that in the future 
a significant proportion of the earth's population lives on space stations or in the asteroid belt, and we will have ship of Theseus kinds of
questions about identifying the sameness of a space station over time, and difficult questions about defining segments of the asteroid
belt. 

One might say that a state, or its people in their constitution, gets to define 
its own understanding of who counts as among the governed, and so a state can opt to define the governed as those occupying a 
certain portion of a certain specific diachronic oblate spheroid model or to define the governed as those within a certain specific
distance of a particular landmark. But there must be limits to a state's normative power to define these boundaries, since otherwise
a state could simply swallow up territory by mere fiat. Imagine, for instance, if France defined its territory in terms of all land
within five hundred kilometers of the Eiffel tower, and then French agents conquered the world by secretly taking small bits
of the tower and distributing them worldwide so that no place was more than 500~km away from one or more of them. It is very plausible
that a state has some freedom in how it defines its boundaries, but that freedom is limited. And the range of ways of defining the
scope of authority seems like the sort of thing that would have many different parameters without privileged values, and hence 
raises many Mersenne questions.

One may wonder why questions about the legitimacy and scope of a political system are being raised as part of a discussion of ethical
questions. There are two reasons. First, we have a moral duty to obey the commands of a legitimate state. Second, only those
acting on behalf of a legitimate state are morally permitted to make certain onerous demands on the population, especially ones backed up 
by threats of violence.  


\section{Arbitrariness}
Whatever the values of the parameters in the ethical Mersenne questions are, these values appear likely to be such 
that if we knew their exact values, we would find them arbitrary.
In physics, some hold out a hope that the fundamental constants in the fundamental laws of nature may be ``nice numbers'' like
$2$, $\pi$, $\sqrt 2$ or $e$. It seems intuitively even less plausible that things would so turn out in ethics. 

And even if the parameters turned out to be such ``nice numbers'', that would itself be a very surprising fact, because while
such numbers seem very natural in physics, they seem rather less natural in ethics. Imagine that you should benefit your 
parent over a sibling just in case the ratio of benefits is no lower than $1:\sqrt{2}$. That would itself seem arbitrary.
It seems that whatever the numbers turn out to be, they will have an appearance of arbitrariness and of contingency.

\section{Continuity}
Many of the examples involve thresholds, such as the amount of intelligence needed for respect or the degree to which a government
needs to care for the common good to have auhtority. It is plausible to reject the idea that there are discrete thresholds, and instead hold 
that there are continuous functions, say a function $r(x)$ specifying the degree of respect required to be shown to a being with intelligence
of degree $x$. 

But then instead of explaining one threshold, one needs to
explain the whole complex shape of the ``respect function''. On the most naive version of this, intellectual power will be graphed along one axis
and respect on another, which will raise Mersenne questions about the slopes of the graph, the positions of the inflection points, and so on. 
But of course in reality, both intelligence and respect have many dimensions, so what we have is a complex function of many arguments and whose
values are multidimensional. 

In general, moving from thresholds to continuous functions only multiplies the degrees of freedom that call out for explanation.

\section{The human nature solution}
On our Aristotelian picture, the nature of an organism grounds norms about what the organism's structure and behavior 
should be. In particular, the nature of the organism will ground many arbitrary-seeming norms, such as those governing
the range of appropriate sizes of Indian elephants, the migratory behaviors of monarch butterflies, and the lengths of 
human femurs. Having the nature makes the organism be the kind of organism it is, and imposes on it the associated norms.

In the case of humans, the behaviors include voluntary ones, and so it is unsurprising that there are norms governing these
as well. And just as there are many parameters governing bodily structure and sub-voluntary behavior, there are many parameters
governing moral behavior, all grounded in the form. 

At the same time, Aristotelian optimism provides us with evidence as to what the parameters approximately are. The actual
bodily structures of humans give defeasible evidence as to what normative human bodily structure is and the actual behaviors
of humans give defeasible evidence of moral norms. And in both cases, we have ways of identifying healthier or more virtuous 
paradigms, using the optimistic idea??backref that the various ways of doing well tend to hang together with some
degree of unity, and the structure and behavior of such paradigms gives us further evidence as to the norms.

Admittedly, there appears to be a disanalogy between health and virtue. We might use a Mahatma Ghandi or a Mother Teresa to 
figure out moral norms, but we wouldn't use an Usain Bolt or a Serena Williams to figure out physical norms. One explanation 
of the difference is that Bolt and Williams have highly-developed traits that are specialized to a forms of life quite
different from that of the typical human---namely, the life of a professional athlete---while Ghandi and Teresa's excellences
in justice, fortitude and mercy are as important to our life as to theirs.

All this raises the question of why the form includes these norms and not others. Here there is an easy 
answer available. The form is at least partly defined by the norms it includes. Thus, Mersenne's question about the lion and
the ant when reformulated into normative terms, as the question of why the lion's strength \textit{ought to} be greater
than the ant's, is easily answered: this follows fron defining features of what make lions be lions and ants be ants. 

The appearance of arbitrariness and of contingency in the ethical Mersenne problems is somewhat misleading: it is like the appearance of arbitrariness and
contingency in the fact that water is H$_2$O or that carbon atoms have six protons. Water couldn't have a different chemical
structure and carbon atoms couldn't have a different number of protons. But it is also an important truth  here that there 
could be other substances that could have a different chemical structure or a different number of protons. Similarly, \textit{we}
couldn't have other norms of preferential treatment than the ones written on our nature, but there could be---and perhaps in
this vast universe are---other intelligent animals with other such norms.

\section{Other solutions}
We thus have many Mersenne questions pointing to arbitrary-seeming parameters in ethical rules.
I will now argue that a broad spectrum of ethical theories and solultions are unlikely to yield good answers to the Mersenne questions
or else raise new Mersenne questions of their own.

\subsection{Kantianism}
Kantianism is an attempt to derive moral rules from the very concept of objective rationality. Famously, this leads to difficulties in
accounting for the substantive content of rules. For instance, from the point of view of objective rationality, it is difficult
to generate a presumption in favor of causing pleasure and against causing pain. The more tightly connected a moral rule is to the
specifics of the human condition and of the circumstances, the more difficult it will be for the Kantian to account for it. But the Mersenne questions above
thrive precisely on such detail. Consider, for instance, the improbability of a good Kantian account of how much we 
should, other things being equal, favor siblings over cousins, or of why proxy consent is sufficient for surgery but insufficient for
sex. The ``logical distance'' between the high level principles, like the categorical imperative to treat others as ends and never
as mere means or to act according to universalizable rules, and such specific moral content appears unlikely to be bridgeable.
Thus, precisely those cases that we have seen to raise compelling Mersenne problems make Kantianism an implausible ethical theory.

Of course, such appearances can be deceiving. One might well have antecedently thought that the relatively simple axioms of set 
theory are unlikely to generate the richness of mathematical theorems that we have seen to come from them. So it would be good
to go beyond an intuition of ``distance''.

There are at least four ways to do that. First, proceed by intuitions regarding a specific example. Consider two different moral rules regarding to the relative treatment
of siblings and cousins. One rule says that benefits to siblings are to be slightly preferred to benefits to first cousins and the
second says that first cousins and siblings are to be treated on par. Neither rule requires us to treat anyone as a mere means or 
takes away from treating people as ends. Both rules are universalizable. So we are not going to be able to derive one rule rather
than the other from Kantianism as originally formulated by Kant. 

Second, we can makes use of a heuristic as to the validity of arguments. One heuristic I employ in checking whether a numbered argument 
given by undergraduate students is valid, i.e., whether its conclusion logically follows from its premises, is to see if the conclusion of the
argument contains any substantive terms that do not appear in any of the premises. If it does, it is in practice unlikely that the 
argument is valid, though of course there are possible exceptions. If the premises are contradictory, then the logical
rule of explosion makes every conclusion a valid consequence. And it could also be that the conclusion is disjunctive and the
substantive term that did not occur in the premises occurs in one disjunct while another disjunct follows from the premises (though 
I have yet to see this happen in a student paper).  An argument from premises about the nature of rationality as such
with a conclusion about specific familial relationships or about specific human activities such as sex or surgery fails the heuristic,
and hence is unlikely to be valid. And the cases do not seem to be like the most common exceptions---the premises are not contradictory
and the conclusion is not disjunctive.

Third, all or most of the examples that raised Mersenne questions have an appearance of contingency to them, in a way that does not
fit with the hypothesis that they derive from necessary principles about the nature of rationality. One way to formulate this
contingency is to note that many of the rules are ones that we would not expect to apply to other intelligent species. If we
came across an alien species that regarded familial ties as somewhat more or somewhat less important than we think permissible
for humans, we should not judge them immoral. It would not surprise us if other intelligent animals---perhaps ones occupying
other niches---were rationally or morally required to take greater or smaller risks than we.\footnote{One thinks, for instance,
of the Klingons and Kelpians from the Star Trek universe, respectively.}

Finally, we have an epistemological argument. While clearly we do not know the exact values of the parameters in the Mersenne
questions, we have some approximate knowledge, as already indicated above in a number of the cases. We clearly did not come
to this approximate knowledge by logically deriving it from Kantian first principles. Nor did we even do so by means of an
intuition that they follow from these principles. For I take it that we do not in fact have an intuition that, say, the
preference for siblings over cousins follows from Kantian principles. If anything, we have an intuition that it does not.
So, it seems that if these rules in fact follow from Kantian principles, it's just a coincidence that our beliefs about
the parameters are correct, a coincidence that makes the beliefs be mere justified true belief rather than
knowledge. But the beliefs are knowledge. So, the Kantian explanation does not work. 

The epistemological argument has some force, but not that much. First, the argument is related to the highly controverted
literature on evolutionary debunking arguments.??refs,add?? Second, a theistic reader has an easy way out of the argument:
God knows what values of parameters in fact logically follow from Kantian principles and could either directly instil in
us correct beliefs about them or ensure that we evolve in a way that yields such true beliefs.

\subsection{Act utilitarianism}
The main problem with act utilitarianism is that it generates incorrect moral claims. It says that a healthy patient
whose organs can save three others can be killed when doing so doesn't have any other countervailing consequences such
as making others more callous. It says that if you and I are loners who make no contribution to society, but I own
a dog and you don't have any pets, then you have a duty to sacrifice your life for mine, to save my dog from being
ownerless; and if neither of us has a pet, but you enjoy chocolate a little more than I do while everything else is
equal, then I have a duty to sacrifice my life for you, since your life would include slightly more utility.

Moreover, as we saw in ??back, for utilitarianism to be plausible and not swinish requires a hierarchy of goods,
and there will be Mersenne questions regarding that.

Finally, even hard-nosed desire-fulfillment or hedonistic utilitarianism will be unlikely to be exempt from Mersenne
questions. There are multiple mental state concepts that could be argued to correspond to the words ``desire'' and 
``pleasure''. 

When the psychotherapist tells Jones that she always unconsciously wanted to kill her mother, is that a ``desire''
in the sense of desire-fulfillment utilitarianism or not? A case can be made either way, and this decision point generates
a degree of freedom for the theory, and hence a Mersenne question as to why it is one sort of ``desire'' or the other that counts
as defining the good. In fact, reflection the complexity of human life as seen in literature??ref:ColinAllen? shows that there are likely
to be many ``desire''-type concepts, differing along multiple dimensions, and hence generating a multiplicity of Mersenne question.
And there will be multiple ways of quantifying the strength of a desire.

And as for pleasure and pain, we will again have a broad variety of concepts and a multiplicity of ways of quantifying them.
This can perhaps best be seen if we think about the mental life of possible and actual non-human sentients. Does a particular
state of an earthworm count as a pleasure? It is unlikely to be exactly like a state of ours. There will likely be many ways
of classifying mental states across species, and on some the worm's state will be a pleasure and on others it won't. So we have
a degree of freedom in our act utilitarianism as to what we count as pleasure or pain in non-humans. And even within humans 
there are complex questions. Consider for instance masochism or the subtle morose ``satisfaction'' of the pessimist who sees 
everything going downhill. There are likely to be different ways of classifying states as pleasures or pains, and the hedonistic
utilitarian will have a Mersenne question as to why one rather than another classification is the one that defines ethics.

\subsection{Rule utilitarianism}
On rule utilitarianism, instead of requiring that each action optimize total utility, it is required that each action
follow rules that are themselves optimized for total utility. Rule utilitarianism's main advantage is held to be that
its escape from the counterintuitive consequences of act utilitarianism. The rule not to kill the innocent may well be the
optimal rule for us, even if in a lifeboat situation it would maximize utility for the two stronger people to kill
and eat the weaker third.

Rule utilitarianism could not only neatly explain the
apparently arbitrary specifics of the moral rules, but could also explain the appearance of arbitrariness and contingency
in a way that, say, Kantianism is unlikely to. For the optimization procedure that would define the moral rules would be a vast
and complex one, taking into account the impact of the actions falling under the rules both in the short and the long run, 
both on humans and on non-humans. It is unsurprising if a complex optimization procedure produces results that seem
arbitrary but are in fact carefully chosen to their end. A computer-optimized airplane wing will have precise angles and
bends that cannot really be explained without running through the whole computation.

Moreover, rule utilitarianism is less prone than Kantianism to make our limited but true beliefs about the moral rules be 
merely coincidental. For we have evolved biologically and mimetically in the service of survival and reproduction, and 
because of the contingent connections between these goods and other aspects of utility, evolution put pressures on us 
that directed our moral beliefs in a truthful direction. There are deep and difficult questions whether this is enough
to make the connection between our beliefs and the truth be sufficient for knowledge??refs, but there is more hope here
than on the Kantian side.

However, famously, rule utilitarianism divides into two varieties, depending on exactly what the rules are optimized for.
On ideal rule utilitarianism, the rules are such that everyone's successfully following them would be optimal, even if in
fact they are too difficult for us to follow. Ideal rule utilitarianism, however, is widely held to reduce to act
utilitarianism, since if everyone were to actually follow the rule of maximizing utility, that would be optimal
with respect to maximizing utility. But act utilitarianism has already been put aside.??backref

Non-ideal rule utilitarianisms, on the other hand, inject a note of realism into the optimization procedures. For instance,
what might render a set of rules correct is that if everyone were to \textit{try} to follow them, optimal results would
result. This already raises a Mersenne question. For trying is something that comes in degrees, and it is very likely that
different rules will be generated when we optimize for the utility resulting from everyone's trying hard to follow them
than if if we optimize for the utility resulting from everyone's trying with minimal effort. And there will be a vast
number of intermediate cases, so there will be a Mersenne question of what grounds the fact that $\alpha$, say, is
the right degree of effort for defining the optimization procedure that generates the moral rules.

Furthermore, specifying the degree to which the hypothetical agents try to follow the moral rules is not enough to specify the
optimization procedure. For instance, one has to specify the level of intelligence of the hypothetical agents, their non-moral 
interests and the environment, which yields multiple Mersenne questions as to what the requisite levels of these for the 
hypothetical optimization procedure are. 

The only way to avoid such questions is to  simply require the counterfactual world to match our
world in the respects, but this runs into two problems.
First, we would normally expect a world where all agents try to follow the moral rules to have agents that have different non-moral
interests, higher levels of intelligence since such a world would have a much more just educational system
than ours and hence would nurture children into greater intelligence, and a rather different natural environment. If we try
to keep the three factors fixed while having the hypothetical agents try to follow the moral rules, we are likely to get
some very unlikely counterfactual results, just as keeping too much of our world fixed in a counterfactual situation
results in the odd claim that if Oswald did not kill Kennedy, Kennedy would have been buried alive. 
Second, we have to say that if our history had gone slightly differently, so that (say) the distribution of intelligence in the
general population were slightly different, the optimization procedure would have generated different rules, and hence different
moral rules would have been true. Indeed, on this view we would get the very strange idea that what we morally do can affect morality
itself.

Besides this, there are other non-ideal aspects that we should probably introduce. Some of our important moral rules discuss how
we should deal with culpable malefactors. But in a world where everyone tries to do the right thing, depending on the strength
of trying, there might well be \textit{no} culpable malefactors, or at least very few. And it is unlikely that moral rules optimized
for such a very different situation would be likely to be the right ones for us. So we probably need to optimize the rules with
respect to a hypothetical situation where not everyone tries to follow them. And that raises Mersenne questions as to how many
people in the hypothetical case follow these rules, and what the others do with their lives. 

In short, ideal rule utilitarianism is implausible, while developing the non-ideal rule utilitarian project raises multiple
Mersenne questions as to the details of what is to be fixed in the hypothetical situation.

\subsection{Social contract}
Social contract theories ground ethical rules in agreement between agents.  We can divide this based on whether the agreement 
is actual or hypothetical. Actual agreement theories face obvious problems. First, it is highly implausible to think of the
typical agent in society as having \textit{actually} agreed to live by moral rules, apart from special cases such as a pious
person vowing to God to sin no more. Second, actual agents can agree to live by unjust rules, even rules unjust to themselves,
and such rules would not constitute morality. 

Contemporary social contract theories tend instead to be based on duties grounded in hypothetical agreement between agents in situations of 
ignorance.??refs Anyone who has been in a long committee meeting knows that actual agreement between agents can result in 
complex rules with much apparent arbitrariness, and it would be unsurprising if hypothetical agreement were similar. Thus far, 
social contract fits our data well. 

But the hypothetical agreement condition involves multiple parameters such as how smart the hypothetical agreers are (and there are
multiple dimensions of intelligence), what exactly are they ignorant of, how many of them are there, what are their attitudes towards
risk and uncertainty, etc. 
We have here an explanation of the Mersenne parameters in terms of other Mersenne parameters, and the
problem remains fully entrenched. 

The risk and uncertainty point is worth emphasizing. Some hypothetical agreement theorists think that 
rational agents would only agree to rules that do not treat anyone inhumanly.??refs But a rational agent who is more accepting of
risk will be willing to tolerate rules that create a minority group that is treated inhumanly if the risk of being a member of 
that group is sufficiently small---i.e., if the group is a small enough fraction of the general population---and the the benefits 
to the majority are sufficiently large. There will be types of inhuman treatment and levels of risk that it would not be 
rational to accept for the sake of a high probability of a large benefit, but the lines between these and the ones that it would 
be rational to accept do not seem derivable from any plausible set of basic principles of rationality. 

Granted, typical Kantian constructivists will insist that certain kinds of inhuman treatment would never be rationally acceptable. But 
now consider the Mersenne questions about these kinds of treatment. For instance, destroying the autonomy of another person might
be taken never to be rationally acceptable. But a minor limitation on another's autonomy clearly is acceptable for a sufficiently
great good: if the only way to save a country from nuclear destruction by evil enemy would be to acquiesce in the enemy's demand 
that everyone wear jeans on Friday, then this limitation on sartorial autonomy should be enforced. Somewhere there is a line between
minor limitations of autonomy and such deep destruction of autonomy that could not be tolerated no matter the price. The only
``natural'' place to draw this line would be at \textit{complete} destruction of autonomy. But if it is only complete destruction
of autonomy that is prohibited by the Kantian, then this does not place a sufficient constraint on the rules that could
be accepted. For instance, the enslavement of persons would not be prohibited, as long as the enslaved persons were still capable of 
some autonomous agency, no matter how minor. 

Furthermore, even prohibiting treatment that completely annuls someone's autonomy will
not avoid Mersenne questions in the vicinity. For we will have probabilistic questions. Is it permissible to perform an action that
has a 99.9% chance of destroying someone's autonomy, a 99% chance, a 90% chance, a 51% chance, a 10% chance, etc.? The only natural
lines to draw seem to be at 100%, 50% and 0%, but none of these will do. A prohibition on an action that has a 100% chance of destroying
autonomy prohibits nothing: any action we perform can fail. A prohibition on an action that has a 0% chance of destroying autonomy
prohibits everything: I scratch my head, and there is a tiny chance that due to some weird sequence of events this causes an earthquake
that leads to you getting hit on the head by a beam that results in your life being reduced to a vegetative level. And while 50% is 
much more reasonable, in some difficult cases it is excessively restrictive. If a child is certain to die within a day, and is suffering
from horrific pain that can only be relieved by a drug that has a 50% chance of reducing the child to a vegetative state for the rest
of the day, administering the drug can be permissible.

\subsection{Virtue ethics}
Aquinas himself invoked the virtuous agent as providing at least the epistemic path to an answer to the preferential treatment
question. We could also take virtue ethics to provide an answer to the Mersenne question: What makes these parameters, rather
than others, hold is that the virtuous agent's patterns of behavior are thus and so parameterized.

But this of course simply shifts the problem to that of why the virtuous agent's patterns of behavior are parameterized as
they are. The best answer to that question appears to be the one given in the Aristotelian tradition which grounds this in
the agent's nature.

\subsection{Divine command}
On divine command ethics, the right is what is commanded by God.
Divine command ethics, like social contract and rule utilitarianism, carries with it significant hope for explaining the apparent
arbitrariness in ethical parameters. We would not be surprised if the laws coming from an infinitely intelligent and good legislator 
had significant complexity that to us would look like arbitrariness. 

It may initially seem the divine command ethics runs into the same problem of pushing the Mersenne questions back to the
question of why God legislated these parameters and not others. But notice that the Mersenne problems I have been discussing
are \textit{grounding} questions. Even if God's legislation were completely arbitrary in a way that ultimately violated the
Principle of Sufficient Reason, on divine command ethics we would have a \textit{ground} for the parameters in preferential
treatment and other ethical rules being what they are. To say that we should prefer siblings over first cousins in a ratio
of $1.7:1$ because God commanded so is to give a ground for the obligation, even if that ground itself needs an explanation.
Compare the moral prohibition on adding cyanide to friends' drinks. There would be something absurd if that prohibition were
ungrounded. But it has a ground, or at least a partial ground: cyanide is fatal to humans. Imagine now that there was in fact 
no possible explanation of why cyanide is fatal to humans. Nonetheless, the grounding problem for the moral prohibition would
have been solved by citing the danger of cyanide.

In this way, our ethical grounding Mersenne problem is quite different from Mersenne's merely explanatory problem. In Mersenne's case
to explain why the distance between the earth and the moon is what it is in terms of other parameters of earlier states of the
solar system does not make significant progress. But when we have given a plausible ground to the moral obligation, we have
indeed made progress. Mersenne's original argument depends for its plausibility on a fairly general Principle of Sufficient
Reason.??ref-on-PSRr Here we just use a heuristic principle that moral truths with
an appearance of arbitrariness need a deeper ground.

Moreover, the divine command theorist has nice answers available to the question of why God chose these rules. For instance,
God could be an act consequentialist and could have optimized the rules to produce the best consequences, including perhaps
such consequences as the value of following and disvalue of breaking moral rules in addition to first order values and 
disvalues like pleasure and pain.  We would expect a complex optimization to produce results with an appearance of arbitrariness.
A sailboat hull computer-optimized to minimize drag is likely to have many parameters that look arbitrary to those who do not
know how it was generated.

At the same time, we still have some serious Mersenne grounding problems. The plausibility of divine command ethics rests in
the idea that God is a legitimate authority and legitimate authorities need to be obeyed. This suggests that logically prior
to divine command ethics there is some sort of a proto-ethical general rule about obedience to legitimate authority. That 
rule itself will have to have parameters specifying which authorities are legitimate and what the scope of their authority is. 
And we will have the Mersenne problem of grounding these parameters.

Moreover, even if we do not have such a general rule about all authority, but a specific rule about divine authority, this
will still raise some Mersenne problems. For, as Aquinas noted??ref, legislation only has a claim on our obedience when it 
is appropriately promulgated. And promulgation is a complex concept involving thresholds and parameters. It is not necessary
for promulgation that all those subject to the legislation have heard of it. But it is not enough for the legislators to
meet secretly, and write the legislation on a stone  buried on public land. Intuitively, we need the legislation to be 
reasonably accessible to those governed by it, but there are many parameters hidden behind the word ``reasonably'', and 
we need grounds for them all.

Nor is it even the case that the promulgation condition on God's commands is met in a really clear way, so that all that
would suffice is some proto-rule that has a really strict and non-arbitrary promulgation condition like that everyone 
governed knows of the rules. For any such strict condition is likely to have in fact been violated by God's commands, since
there is no agreement on what God's commands are---or even on there being a God.

What is worse, when we focus on the Mersenne cases in ethics, it unclear that divine commands instituting the parameters
would even satisfy a fairly modest promulgation that requires those who try really hard to be able to find what the legislation 
is when it is relevant to life. There surely are cases where we have tried really hard to figure out what is the right thing
to do and we didn't succeed. Perhaps it could be argued that we didn't try ``hard enough'', but now we are the true
Scotsman territory.??more?

\section{Other attempts at escape}
\subsection{Particularism}
One might try to escape the Mersenne questions by opting for particularism. On particularism, while there may be general rules
like ``Other things being equal, don't torture people'', the application of these general rules to specific situations is not
rule-governed. Hence, there won't be a rule specifying when one, say, favors a sibling over a cousin. Instead, there are 
particular facts about what to do in particular situations. 

However, particularism only multiplies the Mersenne questions. For whereas on rule-based systems we had Mersenne questions
about why the parameters in the rules had the values they do, now we will have Mersenne questions about why in particular actual
circumstances $C_1$ we should act one way while in slightly different particular actual circumstances $C_2$ we should act a
different way.

Furthermore, plausibly, there will still abstractly speaking be a function that assigns to each circumstances a hypothetical determination
of how one would be obligated to act in that circumstance. There may, of course, be no formula specifying the function, but that does not
affect the Mersenne question of why this function rather than another, perhaps similar one, is correct.

\subsection{Brute necessity}
Perhaps we could say that it is a brute, unexplained but necessary truth that the answers to the ethical Mersenne questions are as they are.
The boundaries lie where they do, but there is no special ontology behind them: it's just a necessary truth that we should prefer parents to cousins,
that an armed up-rising up against a regime responsible for Nazi-style atrocities is permissible while only non-violent protest against the
faults of modern-day Canada is permitted, and so on. 

Of course, brute necessities should never be a first resort in theorizing, but sometimes they might be acceptable as a final resort. Consider
Mersenne-type questions one could ask about set theory. If the Zermelo-Fraenkel with Choice (ZFC) Axioms for set theory are consistent, then for every natural number
$n$ they are compatible with the hypothesis $CH_n$ that there are exactly $n$ cardinalities strictly between the cardinality of the natural numbers and
the cardinality of the real numbers (the hypothesis $CH_0$ is the famous Continuum Hypothesis).??ref:check Suppose it turns out that in fact $CH_{15}$ is true.
We would have an excellent Mersenne question as to why it is $CH_{15}$ that is true, but the mind boggles as to what could be a satisfactory answer to that
question, much as it does in the ethical questions. Perhaps the truth of $CH_{15}$ could be a brute fact, albeit a necessary one since it seems implausible
that mathematical truths be contingent (though see ??Pruss for an Aristotelian metaphysical story on which they might be). 

Some brute necessities can perhaps be admitted in ethics. For instance, if $CH_{15}$ is necessarily true, then it is necessarily impermissible 
for us to punish someone for falsely informing us that $CH_{15}$ is true. This impermissibility would derive from the impossibility of $CH_{15}$ being
false (and hence the impossibility of falsely informing someone of that it's true) and the impermissibility of punishing people for actions that they did not 
do. (It is possible, of course, to insincerely inform someone of a necessary truth. But that's a different wrong action, even if equally bad.) 

But truly ethical brute necessities are deeply implausible. Here is one way to see this. Suppose there is a sequence $s$ of one or more English sentences expressing your 
favorite set of fundamental and necessarily ture ethical norms. For instance $s$ might be the single injunctions ``Love your neighbor as yourself'' or ``Maximize total pleasure minus 
pain of all sentients'', or it might be a longer list. Encode $s$ into a sequence of decimal numbers in some natural way, for instance by encoding each symbol in
$s$ into a three decimal digit ASCII number. It is widely believed---though it has not been proved---that $\pi$ is a normal number, so every possible sequence of
digits occurs in it. If so, then the decimal encoding of $s$ occurs somewhere inside $\pi$---and even if not, it may well still do so. Suppose that the decimal
encoding of $s$ occurs in $\pi$ as the $n$th through $(n+m)$th digits. Now consider this metaethical theory: 
\dref{pi-metaethics}{To do the right thing is to follow the English injunctions in three decimal-digit ASCII encoding between the $n$th and $(n+m)$th digits of $\pi$.}
Call this $\pi$-metaethics. On the hypothesis that the fundamental ethical injunctions are necessary and can be expressed in English, some version of $\pi$-ethics 
has the correct normative content. But, nonetheless, no version of $\pi$-metaethics 
has any plausibility. For there is no plausible normative connection between an injunction being found inside $\pi$ and its being binding on us.

Admittedly, if we in fact found a sequence of English injunctions near the beginning of $\pi$ (say, starting with the tenth digit), we would have some reason 
to follow them. But the reason would be something like this: The best explanation for why these injunctions are found in $\pi$ is found in a being or beings that
in some way incomprehensible to us can control mathematical truths or, more plausibly, the evolution of our linguistic systems, and there is good pragmatic reason to follow the commands of such beings. Perhaps they have
our good in mind, perhaps they will get mad if we don't follow their commands, or perhaps they are trying to inform us of the true ethics. But nonetheless
$\pi$-metaethics would be false. The reason these injunctions would apply to us wouldn't be that they are found in $\pi$, but something else, such as that
a being with practical authority commanded them to us or a being with epistemic authority informed us of them.

In other words, a metaethics where the ethical claims are grounded in something intuitively of no relevant to our moral activity, such as the content of the
digits of $\pi$, is not plausible. To be a candidate for a grounds of ethical claims, a thing needs to be ethically compelling. For a more controversial
illustration of this point, consider that no collection of the traditional attributes of God (omnibenevolence, creation, omniscience, omnipotence, etc.) is such 
as to make it plausible that the commands of a being with those attributes are what ethics is (??ref:MacIntyre??), and this is a strong reason to doubt divine
command metaethics. 

But now take some attempt at founding an arbitrary-seeming ethical principle on a non-compelling ground, say the digits of $\pi$, and remove the ground altogether. Removal
of the ground surely does not make the story any better. Someone who said that what explained why we should favor siblings over cousins by a margin of
twenty percent by saying that it is thus written starting with the $n$th digit of $\pi$ would be ethically ridiculous (though if $n$ is small, finding
the injunction might be some evidence for its correctness). But suppose we drop the spurious $\pi$-based ground: surely the ungrounded ethical claim is
no better off than the spuriously grounded one. 

There may be ethical truths that are not themselves grounded. But these truths should be compelling ethically---perhaps the Golden Rule is like that---and not 
have an appearance of arbitrariness. And there may be arbitrary-seeming truths in ethics, but they are not fundamentally ethical.

\subsection{A two-step vagueness strategy}
It is very tempting to dismiss the Mersenne questions above with a two-step strategy. In each case, we first give non-arbitrary grounds for 
an approximate and vague determination of the parameters involved. Thus, while it is implausible to think that, say, social contract theory
will generate a precise answer to the preferential treatment question, it is reasonable to think it will generate claims like: ``Benefits to 
siblings are to be \textit{somewhat} preferred to benefits to cousins.'' And, then, we simply note that the Mersenne question as to the grounds
of the exact dividing line has the false presupposition that there is an exact dividing line---instead, we have insuperable vagueness.

An initial concern with the two-step strategy is to worry whether other ethical theories can actually generate sufficient non-arbitrary grounds
that have the degree of precision that we think really is there. 
This concern has two variants. One involves cases where we know what the facts generating the Mersenne questions are.
Kantianism, for instance, is unlikely to generate even a vague morally-relevant 
distinction favoring siblings over cousins, and yet we know there is such a distinction. The problem of ranking types of goods generates difficult 
Mersenne questions as to what grounds comparisons that we know are there, such as that fundamental philosophical truths are more valuable than the 
pleasures of chocolate. The second variant of the concern involves cases where we agonize over what to do. Our agonizing is a sign of our intuition
that there is an answer to a moral problem, albeit one we cannot discern. While we may not be seeking for absolute precision, and may be willing
to accept some level of vagueness, in a number of cases we seek for more precision than the various alternatives to the form-based theory can ground. 

Suppose the initial concern can be allayed in both of its forms, perhaps by clever development of a theory that does generate
the vague moral claims and by biting the bullet and admitting that moral agonizing is out of place in these vagueness cases. 
There is still another question: how do we account for the vagueness here. There are three main contemporary accounts of vagueness:
(a)~non-classical logic, (b)~supervaluationism and (c)~epistemicism. 

On non-classical logic approaches to vagueness, one typically increases the number of truth values beyond two. 
Consider an ethical Sorites
series, where we fix some circumstances $C$ and then say:
\begin{itemize}
\item[($A_0$)] Giving $ \$1000$ to a stranger is better than giving $ \$0$ to one's parent.
\end{itemize}

Now for each positive integer $n$, the following material conditional sounds plausible:
\begin{itemize}
\item[($A_n$)] If giving $ \$1000$ to a stranger is better than giving $ \$n$ to one's parent, then giving  $ \$1000$ to a stranger is 
    better than giving $ \$(n+1)$ to one's parent.
\end{itemize}

From $A_0$ and $A_1$, one concludes by \textit{modus ponens} that giving $ \$1000$ to a stranger is better than giving $ \$1$ to one's parent.
From this and $A_2$, by \textit{modus ponens} one concludes that this is true even if what one gives one's parent is $ \$2$. Continuing onward,
once we get to $A_{2000}$, we conclude that it's better to give $ \$1000$ to a stranger than $ \$2000$ to one's parent, which is false. Thus,
we need to reject one of the premises $A_n$. Presumably it's one with $n>0$, since $A_0$ is clearly true. But a material conditional
$p\rightarrow q$ is false just in case $p$ is true and $q$ is false. Hence, if $A_n$ is false for $n>0$, we have:
\ditem{2-trans}{Giving $ \$1000$ to a stranger is better than giving $ \$n$ to one's parent and giving $ \$1000$ to a stranger is not better than giving
    $ \$(n+1)$ to one's parent.}
And that is exactly the kind of sharp transition that the vagueness theorist wishes to deny.

The non-classical approach to vagueness typically involves a logic with many truth values, e.g., a truth value for every number between
$0$ (fully false) and $1$ (fully true). Then the statement:
\begin{itemize}
\item[($B_n$)] Giving $ \$1000$ to a stranger is better than giving $ \$n$ to one's parent 
\end{itemize}
is true for $n=0$ (note that $B_0$ is just $A_0$), but becomes less and less true as $n$ increases. If we have a large enough number of
truth values, we can accept this at face value.

But, surely, $B_1$ and $B_2$ are simply true, too. On the other hand, $B_{999}$ and $B_{1000}$ are simply false. So it does not seem to be the case
that we always have \textit{strict} decrease of truth value with increasing $n$. And hence whereas in the
classical logic reading we had one transition to be explained, from true to false, now we have at least two: from truth to truth values intermediate
between true and false, and from intermediate truth values to falsity. And the transitions appear to be just as arbitrary as before. Thus we have doubled 
the number of Mersenne questions. And if we say that the second-order questions are also taken into account with multivalent logic---say, it's being
the case for some $n$ that $B_n$ is neither true nor false that---then the multiplication of questions increases even more.

Perhaps, though, one can dig in one's heels and insist on strict decrease of truth value. But the precise assignment of 
intermediate truth values---say, $B_{505}$ getting a truth value of $T_{0.51}$---also calls for an explanation. Thus it seems we have a vast multiplication
of Mersenne questions. But there is a response to this argument: ??refs argues that 
the exact truth values are a mere feature of the logical model and all that has reality is their ordering. And
the ordering of the truth values is, perhaps, quite non-arbitrary in that $B_{m}$ is truer than $B_n$ precisely when $m<n$. But the insistence that the ordinal
properties of truth values is what has reality still does not escape the multiplication of Mersenne questions. For consider a different set of ethical
questions involving a threshold. For instance, let $C_x$ say that one has a duty to obey the orders of a government that cares to degree $x$ about the common
good, for some method of $x$ of quantifying care about the common good, where, say, $x=-1$ corresponds to the Nazi German state and $x=1$ corresponds to modern
Finland. Then $C_{-1}$ is pretty false $C_1$ is pretty true. But even if all we insist on is the ordering of truth values, then we will still have a vast,
perhaps infinite, number of Mersenne  questions like: 
\ditem{2-Mers-BC}{At what value $n$ does $B_n$ become less true than $C_{0.24}$?}
For clearly $B_0$ is truer than $C_{0.24}$ while $B_{2000}$ is falser.

?? higher levels of multivalued logic

The most common response to vagueness these days is supervaluation. The terms of a sentence can have multiple precisifications,
with a different truth value corresponding to a different choice of precisifications. ``Bob is bald'' may be
true if we precisify ``bald'' as having less than half a cubic centimeter of scalp hair and false if we precisify it as
having fewer than a meter of hair. Then we have vagueness. When, on the other hand, a sentence is true (respectively, false) under all precisifications,
we say it is super-true (super-false).

In the ethical examples, such as whether it is better to give $ \$200$ to a stranger or $ \$100$ to a parent in circumstances $C$, presumably the 
supervaluationist escape from Mersenne questions will be that no matter how far we precisify $C$, the statement will be vague due a vagueness in 
ethical terms such as ``better'' or ``right'' or ``wrong'' which have multiple precisifications yielding different truth values for the ethical
claim. For instance, it may be better$_{17}$ to give the double amount to the stranger but not better$_{40}$. Indeed, on a view like this, we will have cases 
(precisely specified by means of the monetary amounts and the circumstances $C$) where for some precisifications of ``better'' it will be better to 
give to the parent and for others it will be better to give to the stranger.??explain-better  

Just as in the multivalued logic case, this multiplies Mersenne questions. For where previously it looked like we have a transition from its being true that it's better
to favor the stranger to its being not true, now we have two transitions: from its being super-true that it's better to favor the stranger (say, when the amount of
benefit to the stranger is extremely large) to its being vague whether it's better to favor the stranger to its being super-false that it's better to favor the stranger.
And supervaluating at the next level up---say, supervaluating ``super-true''---only multiplies the Mersenne questions more.

But there are some additional problems for the supervaluationist response. A standard objection to supervaluationism in general is that it implies that it is
super-true that there is a sharp boundary of ``bald'': for, given any precisification ``bald$_i$'', there is a sharp boundary for it. In doing this, supervaluationism
explicitly forces the denial of its governing intuition that there are no sharp boundaries. 

Finally, the application of supervaluationism to ethics is itself deeply problematic. It is truism that we have reason to do what is better. Truisms had better be 
super-true. 
This implies two possibilities with regard to the truism.
Either for every precisification ``better$_i$'' we have reason to do what is better$_i$, or else we need to precisify ``reason'' and ``better'' in lockstep
when we precisify the truism, so that for every $i$ it will be true that we have reason$_i$ to do what is better$_i$. Neither option is satisfactory.

If for every $i$ we have reason to do what is better$_i$, given the existence of infinitely many precisifications here, it seems that the choice whether to
favor the stranger and the parent is governed by infinitely many reasons on both sides. This infinite multiplication of reasons is implausible. Moreover, there is no overall winner here---no reason all things considered---for if
there were, then we could raise our Mersenne question with regard to the overall winner, and we would be no further ahead. But saying that there is no on-balance reason
here denies the intuition that cases near the boundary are hard cases, that it is a difficult question to figure out whether to favor the parent or the stranger, since
as soon as one can see that one is in the vague region, one could just conclude that neither action is on balance required by one's reasons. 

But if there are infinitely many ways to precisify ``reason'', none of them privileged, then this undercuts the very idea of our life being governed in
a non-arbitrary way by rationality. It seems entirely arbitrary whether we follow reasons$_{17}$ or reasons$_{40}$ in our lives. Many questions of rationality
turn into purely verbal questions as to how ``reason'' is to be precisified. And the same goes for related terms like ``morality'' and ``virtue''. This 
does not seem to do justice to the non-arbitrariness that is central to a realist conception of reason, morality and virtue. The point here is similar to the
one raised in ??backref regarding $\pi$-metaethics: it would be arbitrary to require obedience to the commands that are found starting with the billionth digit of $\pi$,
rather than the commands found in some other location. 

Finally, consider an epistemicist theory of vagueness according to which there is a true semantic theory that assigns to each term the precise meaning it has in the
light of the patterns of our use of that term, but neither that theory nor the empirical data on the patterns of language use are available
to us in sufficient detail to settle the meaning of vague terms.  Thus, there is a precise fact as to how much hair one can have
and yet have ``bald'' apply to us, a fact grounded in the patterns of our use of the word ``bald'', but it is a fact that is not accessible to us. Similarly, there is
a precise meaning of ``right'', ``better'' and similar ethical terms, a fact grounded in the patterns of our linguistic usage. If the transition between bald and non-bald occurs
between 98 and 99 hairs, there is nothing mysterious about the fact that someone with 98 hairs is bald and someone with 99 is not, just as there is nothing mysterious
about the fact that a backless chair is a stool. 

But the problem here is exactly the same as the last problem with supervaluationism. Ethical questions are turned into purely verbal questions.
Just as on supervaluationism, there is a multiplicity of concepts closely corresponding to our words ``right'', ``better'' and ``reason''.
On supervaluationism, none of these concepts was privileged, which turned ethical questions into purely verbal questions, undercutting the idea
that our lives are to be governed by reasons and morals. On epistemicism, there \textit{are} privileged concepts that exactly correspond to the
words, but they are privileged purely linguistically---it just so happens that these privileged concepts better fit with our usage under the
correct semantic theory. We get an unacceptable arbitrariness on which if our linguistic practices were somewhat different, we would be using the 
word ``better'' differently, and there would be nothing less natural about that usage. If so, then our actions' being governed by the better 
or the right, rather than by some variant property, would be entirely arbitrary.

In summary, non-classical logic violates classical logic, which should only be a last resort, and further multiplies rather than resolving 
Mersenne questions. Supervaluationism likewise multiplies Mersenne questions. And, perhaps most seriously, both supervaluationism and 
epistemicism as applied to ethics turn ethical questions into purely verbal ones, undercutting a robust realism.

\subsection{Anti-realism}
Retreating from realism in ethics to error theory does, of course, remove all the Mersenne problems in ethics. But the
cost is high: it is incorrect to say that genocide is wrong. Moreover, since some of the Mersenne problems involve not
just morality but also prudential reasoning, this requires one to deny the correctness of standard prudential reasoning.
But perhaps the most serious problem with the error theoretic solution is that we will have parallel Mersenne problems
in other normative areas, such as epistemology (??forward) and semantics (??forward), and the cost of error theory 
there is very high indeed: indeed one will no longer  be able to correctly say that one \textit{ought to} accept error theory.

A more moderate solution is to opt for a form of ethical relativism. Relativism, of course, suffers from serious and
standard objections.??refs Perhaps the most obvious is that it justifies an ultra-conservative approach: for if what
I (in the case of individual relativism) or my society (in the social variant) thinks is guaranteed to be true, then
I or society has no reason to take variant views into account, since if you disagree with me or my society, you're guaranteed to be 
wrong (from my or my society's point of view).

Moreover, relativism is itself prone to
Mersenne questions. Consider individual relativism first on which a moral claim is true just in case one believes
it. The Mersenne question here will be most obvious if one opts for a view on which belief reduces to having a
credence above some probabilistic threshold, say $0.95$. For then the relativist view comes down to the thesis
that a moral claim is true just in case one assigns it a credence of at least $0.95$. But that seems arbitrary.
Why should one be obligated to do what one has credence of $0.95$ in, but not obligated what one has credence of $0.93$ in?
So we have a threshold problem.

Many, however, resist the reduction of belief to a credential threshold. But if we do not so reduce belief, we should
then see belief as just one positive doxastic state among many, such as surmising, being inclined to think, believing,
being confident that, and being sure that. Moreover, a little reflection shows that
such classifications are too coarse grained to do justice to the richness of our mental life. So we have a Mersenne
question: Why are moral claims made true by my believing them rather than my surmising them or being sure of them?

And thinking that the problem here just involves degrees of confidence is probably neglecting much complexity in the human
mind. There is likely a continuum between fully believing and merely acting as if one believes. Why does moral truth show up
in the continuum where it does? Or think of the case when the psychotherapist diagnoses one with a subconscious belief.
Either such define moral truths or they do not, and whichever it is, we have a Mersenne question as to why. And consciousness
itself may come in degree.

Furthermore, a narrow relativism that just makes those moral claims that we actually believe is very implausible. Suppose I believe that it is wrong
to eat animals, and I know that cows are animals, but I do not actually draw the conclusion that it is wrong to eat cows.
On such a narrow relativism, it would be wrong for me to eat animals but it would not be wrong for me to eat cows, even though
I know them to be animals. This is incredible. So we want to extend moral truth at least to things that clearly follow from my moral beliefs.
But probably we do not want to extend it to things that follow in ways that are far beyond our ability to know. For, first,
if do extend it thus far, then a Kantian might end up counting as a relativist, since the Kantian may think that moral truths are
necessary truths, and that necessary truths follows follow from everything. And, second, it seems that this loses sight of the 
internalist motivations of relativism. But if we restrict moral truth to things that follow \textit{sufficiently easily} from
our beliefs. And we will have a Mersenne question of grounding where the line of sufficient easiness lie.

If our relativism is of the social sort, we will have analogues to the above Mersenne questions raised by belief and consequence.  And we 
will have more Mersenne questions There is a complex and difficult literature on how to attribute doxastic states to a community. A reasonable
reading of that literature is that there is a multiplicity of concepts that can be expressed with a phrase like ``The committee believes 
that $p$.'' For instance, belief by the vast majority of the committee members is enough on the more reductive concepts, 
while on more procedural versions of the concepts the committee's belief requires some sort of a joint procedure, such as a vote. 
There will be many answers here, corresponding to a broad spectrum of takes on what a community's beliefs is. And a social relativist
will then have a Mersenne question as to why moral truth is defined by the particular take in question.

The second set of Mersenne question arises from the question of identifying what counts as one's community. I am a citizen of two countries
and a permanent resident of a third. Are the moral beliefs of all of these communities---no doubt, mutually contradictory in various ways---true
for me, or only of one? Do moral beliefs come to be true as applied to me because I am legally a member of the community, or because I identify
with it emotionally, or because I would like to identify with it emotionally? Or is it, perhaps, that every community's beliefs are true
for the community, but there is no such thing as being true for the members of the community? (That would nicely solve the problem of
contradictions between the various communities I am a part of.) How large does a community have to be to define moral truths? Is a chess
club a commmunity that defines moral truths? What if the chess club goes down to one member? Are a pair of friends a community? It is clear
that there are many degrees of freedom in a social relativistic theory, and we would have a Mersenne question corresponding to each of them.

\section{Hume's objection: Complexity, instinct and nature}
Hume saw the complexity of property, inheritance, contract and jurisdiction, and used this complexity to argue for
apparently \textit{against} a Natural Law account: 
\begin{quote}
For when a definition of \textit{property} is required, that relation is found to resolve itself into any
possession acquired by occupation, by industry, by prescription, by inheritance, by contract, \&c. Can
we think that nature, by an original instinct, instructs us in all these methods of acquisition?
(??Enquiry::Morals)
\end{quote}

To further expand on the complexity, Hume notes the vast variety between these rules in different societies,
and analogizes them to the variety of architectures found in housing across societies. A better account,
Hume insists, is that we simply engineer rules for the sake of social utility, just as we engineer houses
for various ends, and in different environments this results in different solutions, albeit ones with a
lot of commonality.\footnote{It is natural to try to solve the problem by adverting to positive law. But
Hume notes that there is much complexity with respect to the institutions by which positive laws are produced.}

Three responses are possible. 

First, we do actually have good empirical reason to think that our moral 
intuitions and instincts vary quite a bit from case to case. An account of our actual moral instincts 
is likely to have enormous complexity. Granted, some of the complexity of our actual moral instincts 
is due to failures. For instance, racist or sexist biases introduce complexity by distinguishing cases that do
not morally differ. And it would be \textit{correctly functioning} instincts that we would expect the natural
law to be expressed through. Thus, even if Hume's argument fails with respect to our actual instincts, it may
work with respect to correctly functioning instinct, since we have reason to think that this would in some
respects be simpler than our actual instinct. 

However, in fact, it is far from clear that purifying our instinct of malfunctions would make for so 
much simplicity as would be needed to support Hume's argument. Removal of some biases will indeed remove
some complexity. But enough complexity is likely to remain that Hume's argument will not be convincing.
Moreover, it is likely that our instincts sometimes fail through not making distinctions that should be made,
and hence in some respects our correctly functioning intuitions would likely be more complex.

Second, the account I am defending is one on which our forms set norms. These norms can be arbitrarily complex.
Granted, there will be a harmony between these forms and our instincts and intuitions. But this harmony is not
a one-to-one mapping. There is such a thing as normal and abnormal food for a type of organism, and we expect the organism
to have instincts that tend to direct them to normal consumption and away from abnormal consumpton. But just as correctly
functioning sight can still err, so too a correctly functioning feeding instinct can lead organisms to ingest
what is abnormal and to refrain from what is normal, especially in environments that have abundances different from the
ones present where the organisms evolved. Thus, our natural instinctive
preference for high-calorie foods leads humans in affluent Western countries to abnormal consumption, and hence to
Reflection can gain additional information as to normative consumption on the basis of
high rates of obesity.??refs By reflecting on our nutritive instincts and \textit{other data}---such as the teleology of
nutrition and medical facts about us---we can get additional information as to what is normative consumption for
an organism, including a human one. This additional information is still fallible, and may fall short of the complexity
of the norms involved. And what goes for our nutritive instincts is even more strongly applicable to our moral ones.

Third, recall Hume's own solution to the problem that the complexity of the rules is a result of our social engineering for social utility. 
Hume's solution is subject to complexity problems of its own. For
instance, what groups count as societies and how do we aggregate the benefits to individuals to get a social utility, etc.? 
But something similar to Hume's solution can be appropriated for the natural law account. We can suppose norms in our nature 
establishing the goals for certain social institutions, such as property or state authority, and perhaps establishing some 
constraints on how these goals are to be pursued, and at the same time requiring us to engineer institutions that satisfactorily
pursue these goals within the contraints. These norms will be complex, but will be less complex than the vast complexity in
our social institutions. Thus, we have a bootstrapping, from fairly complex norms setting the ends and constraints for social
institutions, to the institutions themselves.
\chaptertail 
