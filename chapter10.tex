\def\mychapter{X}
\ifdefined\book
\else
\documentclass[11pt,oneside]{amsbook}
\usepackage[backend=biber, citestyle=authoryear]{biblatex}
\usepackage{mathpazo}
\usepackage{graphicx}
\usepackage{amsmath}
\usepackage{tikz}
\usetikzlibrary{arrows}
%\usepackage{titlesec}
\addbibresource{bibliography.bib}
\newcommand\posscite[1]{\citeauthor{#1}'s (\citeyear{#1})}
\newcommand\plural[1]{#1\mathrm{s}}
%\def\posscitewithextra[#1]#2{\citename{#2}'s (\citeyear{#2}, #1)}

%\newcounter{subsubsubsection}[subsubsection]
%\renewcommand\thesubsubsubsection{\thesubsubsection.\arabic{subsubsubsection}}
%\titleformat{\subsubsubsection}
%  {\normalfont\normalsize\bfseries}{\thesubsubsubsection}{1em}{}
%\titlespacing*{\subsubsubsection}
%{0pt}{3.25ex plus 1ex minus .2ex}{1.5ex plus .2ex}

\ifdefined\book
\renewcommand{\thechapter}{\Roman{chapter}}
\else
\renewcommand{\thechapter}{\mychapter}
\fi

\linespread{1.7}
\usepackage[margin=1.25in]{geometry}
\sloppy
\makeatletter
%% TODO: This is a cheat. It's supposed to be {paragraph}{4}, and that's 
%% what it is in amsbook.cls, but then it fails.
\def\paragraph{\@startsection{paragraph}{3}%
  \normalparindent\z@{-\fontdimen2\font}%
  \normalfont}
\def\subsubsubsection{\paragraph}
\makeatother

\def\smalltick{0.15cm}
\def\bigtick{0.3cm}
\def\pointcircle{0.08cm}
\def\causalnode{0.35cm}

\hyphenation{dia-chro-nic}

%\usepackage[utf8]{inputenc} % set input encoding (not needed with XeLaTeX)
\usepackage{amssymb}
\usepackage{mathtools}
\usepackage{enumitem}
\usepackage{amsthm}
\usepackage{physics}
%\usepackage{ntheorem}
\usepackage{chngcntr}
\counterwithin{figure}{section}

\makeatletter
% \def\@endtheorem{\endtrivlist\@endpefalse }% OLD
\def\@endtheorem{\endtrivlist}%

\setlist[description]{font=\normalfont\scshape}

\catcode`\|=\active\def|{\mid}
\DeclarePairedDelimiter{\ceil}{\lceil}{\rceil}
\DeclarePairedDelimiter{\floor}{\lfloor}{\rfloor}
\newcommand{\Subj}{\mathbin{\raisebox{.15ex}{$\scriptscriptstyle{\Box}$}\kern-.425em\rightarrow}}
\def\Existence{E!}
\def\Believes{\operatorname{Believes}}
\def\True{\operatorname{True}}
\def\Perfection{\operatorname{Perfection}}
\def\ext{\operatorname{Ext}}
\def\Iff{\leftrightarrow}
\def\Implies{\rightarrow}
\def\Entails{\Rightarrow}
\def\Cov{\operatorname{Cov}}
\def\Equiv{\Leftrightarrow}
\def\Form{\operatorname{Form}}
\def\Informs{\operatorname{Informs}}
\def\technical{$\star$}
\def\vtechnical{$\star\star$}
\def\power{\wp}
\def\Nec{\Box}
\def\Poss{\Diamond}
\def\Prop#1{$\langle$#1$\rangle$}
\def\R{\mathbb R}
\def\N{\mathbb N}
\def\tele{tel\={e}}
\makeatletter
\newtheoremstyle{indented}{3pt}{3pt}{\addtolength{\leftskip}{4.5em}}{-2.5em}{\sc}{.}{.5em}{}
\def\Principle#1#2#3{\theoremstyle{indented}\newtheorem*{principle}{#2}\begin{principle}\def\@currentlabel{#2}\label{#1}#3\end{principle}\let\principle\undefined}
\makeatother
\def\pref#1{{\sc\ref{#1}}}
\def\enum#1{\resume{enumerate}\item #1\end{enumerate}}
\def\ditem#1#2{\begin{enumerate}[resume]\item \label{\mychapter:#1} #2\end{enumerate}}
\def\nitem#1#2{\begin{description}\item[#1\label{\mychapter:#1}] #2\end{description}}
\def\bref#1{\ref{\mychapter:#1}}
\def\dref#1{(\ref{\mychapter:#1})}
\def\drefglobal#1{(\ref{#1})}
\usepackage{graphicx} % support the \includegraphics command and options
\usepackage{array} % for better arrays (eg matrices) in maths
\def\Not{\operatorname{\sim}}
\def\St{\operatorname{St}}
\def\num{\operatorname{num}}
\def\And{\mathrel{\&}}
\def\Or{\vee}
\def\BigOr{\bigvee}
\def\<{\langle}
\def\>{\rangle}
\def\union{\cup}
\def\nleq{\not\le}
\def\N{\mathbb N}
\def\R{\mathbb R}
\def\C{\mathbb C}
\def\Powerset{\mathcal P}
\def\star#1{{}^*#1}
\def\hN{\star{\N}}
\def\hR{\star{\R}}
\def\Z{\mathbb Z}
\def\Power{\mathcal P}
\def\Implies{\rightarrow}
\def\True{\operatorname{True}}
\def\Socrates{\mathrm{Socrates}}
\def\actual{@}
\def\Law{\operatorname{Law}}
\def\Chance{\operatorname{Chance}}
\def\Var{\operatorname{Var}}

\def\H2O{H${}_2$O}

\def\scr{\mathcal}
\def\e{\varepsilon}
\def\eps{\varepsilon}
\newtheorem{lem}{Lemma}
\newtheorem{prp}{Proposition}
\newtheorem*{theorem}{Theorem}
\newtheorem{corollary}{Corollary}
\newtheorem{cond}{Condition}

\renewcommand\thechapter{\Roman{chapter}}

\def\chaptertail{\ifdefined\book\else\end{document}\fi}
 

\title{Infinity, Causation and Paradox}
\author{Alexander R. Pruss}
%\date{} % Activate to display a given date or no date (if empty),
         % otherwise the current date is printed

\begin{document}
\setcounter{secnumdepth}{3}
\setcounter{tocdepth}{4}

\end{document}
\fi

\restartlist{enumerate}

\chapter{Evolution, Harmony and God}\label{ch:God}
\section{The origin of the forms}
\subsection{Evolution and forms}
We have good empirical reasons to think that the variety of biological structures that fills our planet 
is largely or completely the product of unguided variation together with natural selection. However, as
I have argued, there are good philosophical reasons to think that the organisms with these structures
have normatively laden forms which specify how the organisms should behave, endow them with the causal
powers that make that behavior possible, and impel them towards that behavior. 

It is implausible to think that the forms supervene on the biological structures. For instance, one theory
of the evolution of wings for gliding is that small wings are useful for heat dissipation. Larger wings allow
for more dissipation of heat, but are also more expensive for the organism to maintain. However, at around
size at which the heat-dissipation benefits are outweighed by the maintenance costs, the wings also become
useful for gliding. It is plausible that a species $A$ that has the smaller wings has them with the telos of
heat dissipation. But a species $B$ that has evolved the larger wings has them with the telos of gliding, either
instead of or in addition to heat dissipation.??ref,check But we can now suppose a member of $B$ whose wings are defective
and only good for heat dissipation. Such a member's biological structure might be largely indistinguishable from
that of a normal member of $A$, and yet it is normatively different: such wings are defective in $B$ but entirely
appropriate in $A$. If these norms are grounded in forms, it seems there is a different form in members of $B$ than
of $A$.

In general, in the evolutionary process, we expect small transitions in genetically-based biological structure 
between parents and children, with no change between the parent's form and the child's form. For if we had constant change
between the parent's form and the child's form, our best account would be that the form simply matches the
biological structure, which would not allow for genetic defects, and yet genetic defects---deviations of genetically-based
biological structure from the kind norms---are clearly possible.  Moreover, it is important to our ethics
that all human beings, despite a wide variety in physical and mental endowments---including the striking biological
difference between male and female---are beings of the same kind. 

We thus need an explanation of why it is that at certain apparently relatively rare and discrete points in the evolutionary 
sequence we have a new form on the scene. This itself yields Mersenne questions: while some transitions of form might happen
to coincide with a particularly striking genetic transition, we expect a number of them to come along with only minor
genetic transitions, seemingly at arbitrary positions. What explains these transition points?

Hitherto in this book, such questions were answered by invoking the forms themselves. And this can be done in this case
as well. We might suppose that the form of species $A$ endows the members of $A$ with a causal power to generate new
members of $A$ in some circumstances, together with new instances of the form of $A$, but also a causal power to generate
new members of $B$ in other circumstances, along with new instances of the $B$ form. The difference in circumstances could be
determined by the DNA content in the gametes joining together, so that when a descendant is going to have such-and-such DNA 
contents, the descendant gets the form of $A$, but with other DNA contents, the descendant gets the form of $B$. 

This story requires the forms to contain intricate specifications of which form is generated when. Granted, the slew 
of Mersenne questions we have already raised should make us circumspect about balk at mere complexity of form.
But now observe that the story as given above requires that the first biological organism on earth---presumably
some simple unicellular or maybe even proto-cellular?? organism---contain within it a form that codes for the causal
power to produce forms of all possible immediate descendants of it. These immediate descendant forms then would have 
to code for the causal power to produce all their possible immediate descendants, and so on. Thus, the
form of the first and simplest organism would implicitly code for all the forms of life that would ever actually be
found on earth, and indeed all the forms of life that \textit{could} ever descend from it.\footnote{It is tempting to
say that the number of possible descendant forms is infinite, but that is not clear. After all, there could be some
physical limit to the size of the genetic code fo a biological organism given our laws of nature. But in any case,
finite or not, the number of possible descendant forms is incredibly large.} We thus have here a dizzying complexity.

???few species story!??? no help, still have complexity

But the problem does not stop here. For we can now ask where that immensely sophisticated form of the first organism comes
from? If we say that it comes from the causal powers of non-living substances, such as fields or fundamental particles,
then we have to posit an even greater complexity in the forms of these non-living substances. The result would be highly
counterintuitive, by supposing non-living things to have immense sophistication of form. Further, however, we would need a 
story of where the first forms arose from. If we take the above account to its logical conclusion, then at the Big Bang
we would already have particles or fields whose forms implicitly included the vast formal complexity of all physically
possible living organisms. And this in turn yields a powerful design argument. For the idea that such complexity would
simply come about for no reason at all is utterly implausible. 

Thus, the story that forms contain the rules for the generation of future forms points towards a being whose own power is
sufficiently great to generate such forms. And to avoid a vicious regress, such a being would need to be a necessary one.

But note that once we have accepted the existence of a necessary being that is the ultimate source of the varied forms 
in our world, we can now tweak the story to avoid the implausible idea that unicellular organisms implicitly code for
the forms of elephants and unicorns. Instead of supposing that the transitions between forms corresponding to certain
selected changes in genetic structure are caused by the parent forms, we can suppose that the necessary being is directly
responsible for the transitions of forms. On such a view, the form of a unicellular organism might only endow its
possessor with the ability to generate a descendant of the same kind, and the necessary being would directly produce
any new forms when it is appropriate to do so.

\subsection{Reasons for creating forms}
Of course, this would lead to the question of \textit{why} the necessary being produces new forms when it does so.
Here, taking the necessary being to be rational can help. For there can be good value-based reasons for the transitions
to fall in some places rather than others. 

Consider, first, an odd thought experiment. A horse-like animal comes into existence with an maximally flexible form such that
whatever the animal does fulfills the norms in the form. To eat and grow is one proper function, and to starve and produce
a corpse is just as proper a function. Whatever our ``flexihorse'' does or undergoes is equally good for it. But there is something unsatisfactory about the flexihorse as a creature. If whatever the flexihorse does is equally good for it, then the fact 
that the flexihorse flourishes is just a direct and trivial consequence of its externally imposed form rather than the individual's 
\textit{accomplishment}. 

Reflection on this suggests there is a value in creating organisms that can fail to fulfill their norms. This value might be
grounded in the forms themselves: it might be that real horses, unlike flexihorses, have self-achievement of flourishing among the
proper functions in their form. And there is a value in creating organisms that have additional types of good written into their
form, including such self-achievement. Alternately, one might hold that in addition to kind-relative goods, there may be
kind-independent goods---perhaps grounded in imitation of the creator??forwardref?---and self-achievement of flourishing
could be one of these.

Either way, a rational being creating organisms has reason to create organisms that can fail to achieve their form, and hence
has reason to create beings with less flexible norms. Moreover, there appears to be a comprehensible value---again, either 
kind-relative or kind-independent---in production of beings of the same kind. As an intuition pump here, think of the
\textit{Symposium}'s idea that the yearning for eternity is exemplified in animal reproduction. Thus, we can give a value-based
explanation for why a necessary being would create beings in discrete kinds, with norms that the beings need not live up to.

\section{Harmony}
\subsection{Review}
At a number of points in the book, we have found noteworthy harmonies, many of them falling under the head of ``Aristotelian optimism'', that call out for explanation. At this point, let us collect a number of them as well as add a few more.

\subsection{Epistemology of normativity and form}\label{ch:epist-of-form}
Central to Aristotle's epistemology of normativity is the optimistic assumption that things generally go right.
This allows us to get evidence as to what is good or proper from how things are actually behaving: that behavior 
is likely to be correct.

The assumption that things tend to go right is not quite enough, however, for an epistemology of normativity. 
It won't, for instance, give us knowledge of what is right in rare cases. Furthermore, the claim that behavior
tends to go right has well-known systematic exceptions. In cases where people are under significant social pressure
to act in ways that are morally wrong, while some resist, many---perhaps a majority---will make morally inappropriate
compromises. 

In addition to optimism about behavior, it seems we additionally need optimism about intuitions, innate beliefs or 
high Bayesian priors about specific substantive propositions (I will now just say ``intuitions'' for short). We can try to derive 
this optimism about intuitions from optimism about behavior combined with the claim that cognitive behavior has truth as 
an end. However, the claim that cognitive behavior has truth as an end seems to depend on intuition about ends. One might 
try argue that the claim that cognition has truth as an end is analytic---it follows from the very conception of cognition.
But that would just shift the bump under a rug to a different place: our intuition that such-and-such behaviors are in fact
\textit{cognitive}. So it seems that some optimism about intuitions is unavoidable. 

Another area for optimism beyond behavior is with regard to our ability to identify which things have form, and which groups of
things have \textit{the same} form. There is good reason to think it is metaphysically possible to have particles structured in 
a way that is empirically indistinguishable from a horse and yet with the particles not constituting a horse or any other 
whole. One way to make this plausible is to recall the Small Beginnings Thesis??backref according to which material substances, 
like horses, start subatomic in size. Given that there is nothing specifically empirically equine about a subatomic entity
(it wouldn't of course be really tiny four-legged critter, but rather something a fundamental particle or a portion of a 
wavefunction with a horse form), it is clear that at that stage it is possible to have something that is not a horse but 
that is empirically indistinguishable from our subatomic horse. 

One could put Small Beginnings aside, and ask whether it would be metaphysically possible to have something empirically
just like a full-grown horse but without a horse form. Given that the Aristotelianism that I have been defending is a 
robust one, where the horse form is an additional entity rather than a mere arrangement of particles (as we saw, this also
followed from the Small Beginnings Thesis), it seems hard to deny that possibility. Furthermore, suppose that in a lab we 
carefully arrange particles just as they are in a full-grown horse. If a horse form comes into existence, we need an explanation 
of where it came from. There are three initially plausible options here: (a)~nowhere, (b)~from the physical constituents (particles,
fields, wavefunction, etc.)\ and (c)~from God or some other very powerful supernatural being.

If the form came from nowhere, we have a violation of the venerable principle that nothing comes from nothing.
But such violations, if they happen at all (which is dubious??ref), are brute contingent facts. Hence they need
not happen, and it is metaphysically posisble that the particles come to be shaped like in a horse without a horse
form arising.

If the form came from God or some other powerful supernatural being, then it is very plausible that this being would also 
have the power to withhold the form.

If the form came from the physical constituents, which seems the best of three options, then we have to suppose that the 
fundamental physical entities in our universe have preprogrammed into them the forms of all possible material substances,
which they necessarily produce as needed. But this hypothesis also calls for some optimism. It surely would be metaphysically
possible to have a world with physical constituents that behave empirically like ours, but where these constituents are not 
such as to necessarily produce horse (or other organic) forms---say, because it is random whether they produce a horse form 
in any given chunk of horse-like matter. To deny this hypothesis would take an optimism that goes beyond supposing that 
substances tend to behave correctly, since these hypothetical physical constituents would be behaving correctly when they 
would fail to produce a horse form in horse-like matter.

But now if it is metaphysically possible to have something empirically indistinguishable from a horse but that is a mere
heap of physical stuff, the same would be true of humans. And thus to know the normative structure of the human world 
we would need to rule out the hypothesis that many of the human-like chunks of matter around us are form-zombies, formless
beings that behave like we do. 

Perhaps, though, we can conclude that a given horse-shaped pile of matter is a horse from the fact that its 
parents are horses, together with the fact that it is normal behavior of horses to give birth to horses rather
than merely horse-shaped piles of matter. However, this regularity is one that we cannot non-circularly get to 
empirically by observing horse behavior. As far as empirical observation goes, we see horses give birth to horse-shaped 
piles of matter. That these piles are horses requires an intuition. It seems metaphysically possible to have 
animals that are horse-like in physical structure and empirical behavior but that give birth to something other than horses.

One way to be optimistic about getting right the match between forms and piles of matter will be by supposing that 
our intuitions about the differences in structure and behavior that give rise to differences of kind, when subjected 
to a due process and brought to a rational (by human lights) equilibrium will tend to get things right. Another way 
would be to suppose a broad uniformity of nature thesis that makes invisible joints (say, though between forms) tend to 
correspond to visible joints. But a uniformity of nature thesis may not be sufficient for the epistemology of rare-case 
norms, so the intuition-focused approach may be better. In any case, in order to have a satisfactory epistemology of form,
our optimism needs to go beyond the thesis that typically things go right---important as that thesis is.

Here we can simply gesture at a process of gaining knowledge of forms and norms on the basis of the behavior of 
substances (or apparent substances) and our intuitions (or innate beliefs or substantive Bayesian priors), together 
with a rational equilibrium, doubtless in a communal setting since we are social epistemic agents. We can optimistically
suppose that such a rational equilibrium brings about a belief that all the human-like beings that are biologically 
within the \textit{homo sapiens} species, despite biological variation (due to age, sex, or genetic subcommunity), share 
in the same form, while the currently existing organisms outside that biological species do not.\footnote{Things are less
clear with regard to other biological species of genus Homo, such as Neanderthals. And so they should be, since we cannot
observe the regularities in their behavior, except by archaeological means.} ??zetafish

\subsection{Nomic coordination}
On an Aristotelian picture of reality, the orderly behavior of the universe is grounded in the causal powers of 
disparate substances. This, however, is puzzling on two counts. First, we have the puzzle that there is a 
realtively small number of forms found in the universe. Why, for instance, is there one kind of human form shared by all 
the billions of humans, rather than a different human-like form had by each individual? 

Note that the problem of the large ratio of individuals to kinds is one that every theorist faces. There are at least $10^{80}$ 
particles in the universe, but only about seventeen fundamental particle types (??ref). Still it is worth noting that the 
Aristotelian does not have any obvious distinctive answer to this problem. 

There is, however, a second problem that not everyone has. Some Aristotelians (including the author of this book??backref) 
enjoy beating up on Humeans for making 
nomic explanations bottom out in an unexplained harmony of nature, where the vast numbers of particles all 
happen to behave the same way, and the laws of nature merely describe this behavior. But for Aristotelians the same problem returns 
at the level of kinds. There is a significant number of different kinds of 
substances in nature: humans, dogs, oak trees, electrons, etc. Yet their behavior is nomically coordinated. For instance,
in classical mechanics, each object, regardless of its type, exerts a gravitational force proportional to inertial mass
and inversely proportional to the square of the distance.\footnote{Or, more precisely, distance to its microscopic parts: the overall
gravitational force of a spatially extended object that is not spherically symmetric only follows the inverse square law 
asymptotically.}  Thus on an Aristotelian account of laws each object---or at 
least each object with mass---will have to have a form with a gravitational causal power with the same dependencies on 
mass and distance. Perhaps even more strikingly, the causal powers of all the forms in nature are so coordinated that 
mass-energy is always conserved. Energy lost by one object is gained by another, regardless of what kind each object 
is. 

Granted, the problem is in a sense many orders of magnitude different than for Humeans. While the Humean leaves 
unexplained the coordination between $10^{80}$ individual particles, the Aristotelian has a problem about the 
unexplained coordination between $17$ individual particle types and a large number, but likely not far above $10^{10}$
if earth is the only planet with life, of types of biological kinds. But while a coincidence between $10^{80}$ things
may be vastly more improbable than that between $10^{10}$, neither coincidence makes for a satisfactory theory.
And this is not a problem for everyone: views on which there are global ``pushy'' laws do not have this problem. 

Moreover, just as the Humean faces the sceptical worry that hitherto unobserved particles might behave differently, 
the Aristotelian faces the sceptical worriy that hitherto unobserved \textit{kinds} of substances, say undiscovered 
types of bacteria on the bottom of the ocean or on other planets, might behave differently even at the level of physics, 
since their physical behavior is grounded in their form. 

One might make some contribution towards a solution by supposing a richer structure in the forms that we have hitherto
done, going back to Aristotle himself. For while Aristotle's metaphysics puts a significant emphasis on the \textit{species},
it also has multiple levels of genera that he appears to take metaphysically seriously. Suppose now that we say that 
the species form has a structure where it is composed of a multiplicity of levels of generality, and at each level 
it has a metaphysical component specifying the level. Thus, perhaps skipping metaphysical levels or including extraneous 
ones, we might suppose that an oak tree form has a broadleaf tree generic form as a component, which in turn contains 
a plant generic form, in which there is a DNA-based organism generic form, and in which there is a generic form shared 
by all the material substances in our universe. 

We might then call the last generic form ``the physics form'', and suppose that it is responsible for the common 
physical behaviors---the gravitational pulls, the mass-energy conservation, and so on. With respect to fundamental
physical behavior, then, we have one kind across all the substances. But the more specific levels of form in the 
oak tree still have a role to play. Minimally, they play a normative role: they specify how the oak tree \textit{should}
behave in virtue of being an oak tree, or in virtue of being a broadleaf, or in virtue of being a DNA-based organism.
But they likely have some sort of a causal role. That causal role could fill in the gaps in fundamental physical causal laws
if these have gaps---which they may well have if they are indeterministic---but at least it will help to explain the 
passing on of form to offspring: the offspring of broadleafs tend to be broadleafs, and so on. 

This solution, however, still leaves something unexplained: Why do all the substances in our universe have the 
same generic ``physics form''? We can, however, reduce the amount of what is unexplained. For instance, we might 
suppose that in substantial causation where one substance causes another, the physics form of the causing substance
causes the effected substance to have the same physics form. On this account, all we need to explain is why 
\textit{initially} there was only one physics form among all the initial kinds of entities. Since the initial 
kinds of entities are presumably only fundamental physical constituents like particles, we have moved from a 
coincidence between $10^{80}$ things in Humeanism, to $10^{10}$ kinds in our less structured Aristotelianism,
to now somewhere around $17$ kinds. But that's still something that calls out for an explanation.

Perhaps, though, explaining the harmony between a relatively small number of kinds is not very different from 
a non-Aristotelian advocate of pushy global laws having the explain the harmony between these laws---say, that 
all the laws governing fundamental forces are compatible with mass-energy conservation. Maybe we have a difference
between a coincidence involving four forces and one involving $17$ particle types, but this kind of a difference 
may not be that decisive. Nonetheless, it would be good if the Aristotelian had something further to say here.

Furthermore, global pushy laws involve an interesting harmony or fine-tuning issue recently identified by ??ref. A version 
of the problem arises from the fact that laws are supposed to move objects in accordance with the properties that 
material objects have. For instance, the Coulomb law induces a force of $-k q_1 q_2/r^2$ between particles with charges 
$q_1$ and $q_2$ (measured in some natural physical system of units), with the force being measured in the direction from 
one object to the other. Now consider the infinitely many possible 
fundamental properties quantifiable by real numbers, which I will call ``charge-like'' properties. For each one of these
charge-like properties, say $Q$, there is a possible 
Coulomb-like law $L_Q$ that says that two objects have a force of $-k q_1 q_2/r^2$ between them when the particles have 
$Q$-values $q_1$ and $q_2$ respectively. Presumably, only one Coulomb-like law $L_Q$ is exemplified in our world, namely
$L_{\text{charge}}$. But now consider this vast coincidence: our world is a world with particles that have a small
number of fundamental physical properties including charge and our world also has the law $L_{\text{charge}}$ which is 
but one law from among the infinitely many $L_Q$ laws, the vast majority of which would be irrelevant to our world's 
particles. In other words, we have a harmony between the properties and the laws: our world has laws that are relevant 
to the properties.

One might object that for all we know our world has laws involving all the infinitely many other charge-like properties,
but we will never know about any of these laws, because the properties of particles governed by them are unexemplified
in our world. But on this view, the full collection of our world's laws of nature is immensely complex in a way that 
appears to be a massive violation of Ockham's razor. 

Alternately, the advocate of pushy global laws might say that there are no laws governing unexemplified properties. 
Thus, only those $L_Q$ can exist where $Q$ is exemplified. However, this has the odd consequence that what laws we 
have in the world depends on what nomically contingent events happened. Consider a universe initially consisting of 
two photons, but then at some point the two photons happen to collide and produce an electron-positron pair, thereby
ensuring that charge is exemplified, and suppose that this universe has $L_{\text{charge}}$ just as ours does. But 
then the existence of $L_{\text{charge}}$ in this universe depends on a nomically contingent event: the collision of 
the photons. This seems wrong. The laws of nature do not seem to be piece-meal generated by physical events. Even more 
seriously, that an electron-positron pair can be generated from a collision of two photons is itself a law of nature.
But if the existence of this law depends on the existence of electrons and positrons, there being no laws governing 
unexemplified properties (in this case, electronicity and positronicity), then we have a vicious circularity: the law
depends on the electron-positron pair coming from the collision, but the law explains why the collision results in such 
a pair.

One might worry that a similar problem shows up for Aristotelians. If, say, the Coulomb law is grounded in the forms of the electron
and positron, then it seems that if the photons do not collide, there is no such law. But that's not quite right. For 
in the universe 
where photons do not collide, it is still true that if there \textit{were} electrons and positrons, they \textit{would} 
behave in accordance with Coulomb's law, a fact grounded in the forms that electrons and positrons \textit{would} 
by definition have (namely the forms that would constitute them as electrons and positrons, respectively). The 
advocate of pushy laws that do not exist for uninstantiated properties has a much harder time saying what could 
ground a similar counterfactual.

Perhaps the advocate of pushy global laws could say that for every set of fundamental physical properties that can be 
instantiated in the same world, there are laws about how objects with those properties would behave, even though 
the vast majority of these laws do not apply to anything actual in our universe. We could then say that while Ockham's 
razor is not globally applicable to the laws as a whole---there is a vast infinity of these laws---for any fixed and 
relative small set $U$ of fundamental physical properties instantiable in the same world, Ockham's razor applies to 
the subset of the laws governing objects with properties from $U$---we should opt for simpler candidates for these 
laws. This will be enough to recover our preference for simpler scientific theories. 

Nonetheless, this much is clear: if we attempt to solve the coordination between the exemplified properties and 
the laws by supposing a vast set of laws, there does not seem to be a significant simplicity advantage to 
global pushy laws over Aristotelian form-grounded laws.

\subsection{Harmony between types of normativity}
\subsubsection{Introduction}
Consider three central aspects of our normative life: flourishing, morality, and epistemic rationality. At times 
they conflict. But not that often. I will now argue that there is a surprising amount of harmony between the three,
beginning by considering all three possible pairings.

\subsubsection{Flourishing and morality}
One of the great discoveries of Western philosophy was Socrates' realization that virtue is central to our flourishing
or happiness. 

We should not, however, go along with Socrates' more extreme claim that virtue is all there is to our flourishing. 
First, we have Aristotle's observation that a person can have perfect virtue but remain unconscious all their life,
and that is not flourishing. It is not just the possession but the exercise of virtue that contributs to flourishing.
Second, we have Vlastos' decisive counterexample of the vomity bed??ref: it clearly contributes to our flourishing that 
we not have to sleep in a bed covered in vomit, even though this good is not virtue (nor the exercise of virtue).

Nonetheless, the possession and exercise of virtue is the main contributor to our happiness. The person who
has pleasure, reputation, wealth and health but is vicious leads an unenviable life, and a person who has
the possession and exercise of virtue while lacking much in the way of other goods leads a flourishing life,
except perhaps in cases where the shortfall of other goods is extreme, as in Aristotle's example of being tortured
on the rack??ref. 

Aristotle rightly observed that normally there is a pleasure in acting virtuously. Additionally, a virtuous 
person takes pleasure in the flourishing of others, especially their friends??ref:Aristotle, while at the same time 
contributing to this flourishing. The life of virtue tends to be pleasant.

And while there are obvious notable exceptions, as a rule virtuous people tend to have a good reputation and be liked
by others. And an important part of our moral goodness is benefiting human beings in a broad variety of ways, and we 
are human beings ourselves. Moreover, we are often particularly well-positioned to benefit ourselves, so virtuous people
have a tendency to do so. Furthermore, executive virtues are among the virtues, and without the executive virtues, we are 
unlikely to be happy.

There are, of course, many conflicts between morality and happiness. The most obvious is that morality often requires
us to forego various other goods. To a significant degree, however, the sacrifice is compensated for by the innate value
of the life of virtue and the pleasures thereof. 

But we should not be too Pollyannish here. As noted before, there may 
be harms other than vice that can destroy one's flourishing. Torture, both due to physical pain and the subsequent 
psychological damage, is a glaring example, and in some cases morality requires one to stand up to torture. 

Second, it is worth noting that we can imagine hypothetical cases where doing what is morally right makes one less 
well off as a moral agent. One might imagine an ethics class that is so inspiring that it tends to 
transform students morally for the better, and a student who can only afford the cost of the class of the class by 
committing theft. We can then imagine that the class is so transformative that its benefits to one's character outweigh
the harms of the theft. Nonetheless, the theft is wrong. Or we can imagine a case where one is a morally harmful 
environment and the only way to escape that environment is by doing something morally wrong. For instance one might 
live in a soul-destroying totalitarian state and to escape it one would need to steal money in order to pay for airplane 
tickets or to bribe officials for permission to leave.\footnote{The bribery may not be wrong, but the theft may well be.}

Furthermore, as Stohr has noted, there are times when acting virtuously is not innately pleasant: consider her 
example of the pediatric oncologist having to break tragic news to parents??ref. 

Nonetheless, the exceptions to the confluence of virtue and happiness, while sometimes tragic, tend to be rare. 
Torture is, fortunately, relatively rare, and not all cases of torture in the world are ones where morality requires 
one to stand up to the suffering. Some cases are, for instance, ones where terrorists or other criminals are tortured
to reveal their wrongful plans---in those cases, where inflicting the torture is morally wrong, there is no duty to 
withstand it, and there may even be a duty to yield to it. Other cases are ones where the torture is inflicted 
as an unjust punishment for real or imagined crimes, and in those cases there is no question of withstanding or 
yielding, just of surviving.

\subsubsection{Flourishing and epistemic rationality}
Epistemic rationality aims at epistemic flourishing, which is an important aspect of our flourishing as persons.
Epistemic flourishing is not just a function of how many facts one knows. More important knowledge, and especially
wisdom, as well as an understanding of how our knowledge hangs to gather contribute more to epistemic flourishing 
than knowledge of trivia, and hence epistemic rationality aims more at this more important kind of knowledge. But 
this kind of knowledge, in turn, is particularly important to human flourishing.

At the same time, the life of virtue is central to our flourishing, and knowledge---and especially wisdom---is 
crucial to the life of virtue. To engage in morally excellent action, we need to understand ethical norms, the 
human beings that we affect by our actions, including especially what is good for them, our own selves, and our 
relationships with others. Furthermore, for this morally excellent action to be successful, we often need empirical
knowledge of the means by which good may be accomplished---and here we need to include both common-sense knowledge 
as well as the sciences.

The sciences are not just instrumentally good to us. Knowing the causal structure of the world not only helps us 
live healthier lives, but is an important part of our flourishing as human beings, and gives us delight, because
the world's knowable explanatory structure is delightfully beautiful.

We can imagine beings whose natures impose on them standards of epistemic rationality that are too high for these beings to 
be able to know what is needed for flourishing. For instance, these could be Cartesian beings whose standards of rationality 
require infallibility, and yet which are not capable of infallibly cognizing what is needed for flourishing. But fortunately 
we are not such beings. We can know that friendship is good for us and that physical suffering is bad for us. 

\subsubsection{Morality and epistemic rationality}
The morally good tend to be wise, and the wise tend to be morally good. Moral norms impel us to investigate what is 
right and good. There are beings none of whose behavioral norms are knowable to them. After all, many organisms---bacteria, almost surely fungi and plants, and likely lower animals---are incapable of knowing anything, and yet they all have behavioral norms.
Now, moral norms are not just any behavior norms, but norms of \textit{voluntary} behavior, and the organisms incapable
of knowledge are presumably also incapable of voluntary behavior. 

One might think that it is partly constitutive of being a moral agent that one knows some of one's moral obligations, if 
only in the most general terms---say, that one knows that the good is to be pursued and evil to be avoided, as per 
Aquinas's first principle of Natural Law.??ref This is not clear. Error theorists about morality claim not to believe
that anything is to be pursued or avoided. If they do not believe it, neither do they know it. Yet they are moral agents.
We might be suspicious that error theorists really do believe what they claim to believe. If so, then consider that knowledge
tends to be defeasible, and being confident that one has a decisive argument against something tends to provide a defeater 
for it. Some error theorists are confident that they have a decisive argument against there being moral truths??refs, and it 
seems plausible that even if humans are unable to fail to believe in morality, in the presence of such confidence humans do 
not \textit{know} moral truths. 

I think we can, further, imagine a whole species of no members of which are capable of knowing any moral rules. 
One option might be a species where fallible and correct moral belief is widespread, but the epistemic standards of 
their nature are so high that infallibility would be required for justification. Another option would be a species 
where the moral norms are conditional on the agent having some beliefs about moral rules, but the members of this 
species have no epistemic access to moral facts. Nonetheless, due to the evolutionary advantages of morality for 
social animals, they have evolved to have largely correct moral beliefs---though unjustified ones. 

Finally, we could suppose Bayesian moral agents whose epistemically rational nature is such that they ought to have 
very low priors for all positive moral claims, so low that in fact the evidence they are able to gather is 
insufficient to raise the posteriors for positive moral claims above $1/2$. 

The moral life need not \textit{require} knowledge of moral norms. Mere belief may be sufficient. But beings incapable
of knowing moral norms are obviously lacking in harmony between the moral and epistemic lives. 

Furthermore, not only our knowledge but our belief on occasion falls short of the truth about specific moral norms.
How much is lacking varies from person to person. It is very unlikely that there is some conceptual limit to how much
ignorance of specific moral norms is possible. Thus, it is possible to have beings whose epistemic norms are so stringent
that they rule out true belief about the moral norms that are in play for daily life. If belief in moral norms is needed 
to live the moral life, such beings would have a direct conflict between epistemic and moral normativity: they could only 
live the moral life by failing in epistemic rationality. Again, an example of such beings would be Bayesian agents whose
priors for specific moral norms are so low that rational posteriors would always remain low, perhaps even far lower than 
is needed for belief. This last thought would take care of the objection that one does not even need belief in moral norms for the moral life, that all one needs is a moderate credence. 

Besides knowledge of specifically moral truths, morally excellent action requires knowledge or at least true belief about 
other people, both in general and in particular. Meaningful relationships often require deep understanding of the other 
person, and hence require standards of epistemic rationality that permit such. If, for instance, our epistemic standards
were such that rationality requires us to remain agnostic about whether there are other minds, our moral lives would 
make little sense. 

Furthermore, it is infamously true that lack of understanding other persons---whether as individuals or participants in 
a culture other than one's own---has a tendency to lead to morally deficient treatment of these persons. Conversely, a 
gain in understand is apt to lead to better treatment. At the same time, however, there are exceptions. Sometimes if we 
knew a person better, we would know more about their faults, and this would lead us to treat them poorly. But typically 
even this exception does not produce a major conflict between moral and epistemic rationality for two reasons. First, 
except in the case of rare and exceptional secret misdeeds---such as someone's being secretly a child 
abuser---epistemic rationality will lead us to honestly investigate our own lives and will typically find parallels in 
our own lives to the faults of others. This should lead to a more forgiving attitude towards others, as well as motivation
to improve the faults in ourselves. Second, an understanding of the faults of others can be necessary for us to help them 
become better.

Finally, a number of philosophers think there is an irreducible phenomenon of I-Thou second-person knowledge. And there is good
reason to think that second-person knowledge requires an empathy that can only be developed by a life of virtue, and 
tends to both come from and deepen a morally good relationship. One might, of course, imagine cases where second-person 
knowledge enables one to hurt the other in an especially painful way. However, such betrayals are relatively rare.

We could imagine beings whose moral norms prohibit acting in epistemically excellent ways. One could imagine
beings that are morally required to engage in practices productive of self-deceit in order to preserve morally important
beliefs, such as the belief in other minds. Or ones that are morally forbidden to engage in scientific investigation,
in order to better preserve a sense of mystery about the natural world.

Humans are fortunate. Our epistemic standards are such as to allow us to know the moral norms needed for daily life, 
and to understand other people in ways sufficient for the moral life. And not only are we not morally prohibited from 
epistemically excellent behavior, but our morality calls on us to engage in such.

There is, however, one significant area of conflict between moral and epistemic rationality. Our knowledge of human 
biology could probably be greatly advanced by means of immoral experiments on human subjects. This observation decreases
the harmony between moral and epistemic rationality. But perhaps not by all that much. First, a practice of immoral 
experiments on human subjects is likely to result in a society with a deadened conscience, and hence with a significant
loss of \textit{wisdom}. And wisdom is, arguably, more important to human epistemic teleology than biology. Second,
the significance of the conflict may be rather specific to our present point in scientific development. Prior to 
modern medicine, the epistemic benefits of immoral human experimentation would likely have been small. And in the 
future, we will likely be able to find morally unexceptionable scientific methods that provide as much, or nearly as much, epistemic
benefit as immoral human experimentation, such as computer simulations and experiments on non-human animals genetically 
modified to be unconscious.

\subsection{Harmony within each normative aspect}
\subsubsection{Morality}
Some thinkers believe in a unity of the virtues thesis on which it is impossible to fully possess any virtue 
without possessing them all. One approach to arguing for such a thesis is to hold that there is an architectonic 
virtue which is needed for any virtue and which entails all the others. Socrates apparently believed that wisdom 
was such a virtue. On such a view, for instance, courage requires wisdom, while wisdom precludes the vices opposed 
to courage, such as cowardice or foolhardiness. Similarly, St.~Paul held that having \textit{agap\={e}} implied the 
fulfillment of all of morality, and might well have agreed that without love there was no virtue. ??modern-refs

But even without a dogmatic affimration of the unity of the virtues, it is plausible that vices tend to distort even 
virtues that they do not oppose, and the virtues tend to be mutually supporting. In fiction, one can meet a villain 
of direct speech who can be relied on to keep promises and avoid lies, while yet being selfish and violent??Sanderson:Cett. 
But in real life such an individual is rare (and in fact I personally do not find Cett a plausible character). To some degree 
the reason for the rarity is practical: honesty may make 
one get caught. But there is a deeper reason that applies in all cases: the most powerful motive for honesty is respect 
for fellow persons, and that respect is apt to prevent vice in general. Admittedly, it is possible to respect others 
under some aspect while disrespecting them in other respects, say honestly respecting others as fellow truth-seekers 
while violently disrespecting them as living beings, such a mixture is unlikely. After all, a major part of why others 
are worthy of respect as living beings is that they are truth-seekers.

There are some pairings of virtues where we are apt to see a conflict. For instance, honesty in feedback appears to 
conflict with kindness. But even in these pairings, it is plausible that the conflict is largely if not entirely
due to a shallow exhibition of one or the other virtue. But it is not necessary to insist on the point. All I need 
for my argument is that there is a \textit{tendency} to mutual reinforcement between the virtues, a tendency that 
calls for explanation. Without a principled reason to expect harmony, we would expect a random pair of virtues to be 
like gravity and electromagnetism: sometimes they pull together but just as often they pull or push in opposite 
directions.

A second type of harmony within the normative sphere is the paucity of real dilemmas: cases where one is simultaneously
required and prohibited from the same thing. There does not appear to be a principled reason why there could not be 
cases where in light of one fact, one would be defective in will in $\phi$ing, but in light of another fact, one 
would be defecting in will in failing to $\phi$, so whatever one does, one acts wrongly. Certainly, it is all too 
easy to specify an artifact with a complex teleology that guarantees that it will 
always be defective in some respect. One might say that a spork should spear food well while at the same time holding 
liquids well, and this incompatible teleological arrangement explains the unpopularity of this implement. And a randomly
generated set of rules is one where we would expect frequent conflict.

But while there is no principled reason to avoid conflict between moral rules, real dilemmas are, indeed, sufficiently
rare that philosophers have seriously discussed whether they exist. 

On one view, there is a limited set of well-delineated famalies of real dilemmas. One proposed family is where the 
dilemma is the result of a prior wrongdoing.??ref:Aquinas Thus, a person with children who has gambled on credit may have both a duty 
to support the children and a duty to pay creditors, with the two duties being genuinely incompatible: whatever the 
unfortunate agent does, the agent does defectively. Another proposed family is cases arising from mistaken conscience:
if $\phi$ing is forbidden, but conscience requires you to $\phi$, then whether you $\phi$ or not, you do wrong.??ref:Murphy 
That all real dilemmas fit into such neat characterizations is a surprise. It is certainly not something we would expect 
in a random collection of rules.

On broader view, there may be real dilemmas that do not fit into such neat categories. But even these are likely to be 
rare, and more rare than we would expect on a randomness hypothesis.

\subsubsection{Flourishing}
While we may not have many principled conflicts between types of flourishing, limited resources induce many 
practical conflicts. Choices need to be made.
Even affluent people one must choose between leisure and money, between developing one set of skills or another, 
between avoiding danger and enjoying benefits of dangerous activity, and so on. But most people are not so affluent,
and the conflicts between types of flourishing they face are much more painful, such as between medical treatment and food security, or shelter and clothing. On its face, then, the various aspects of our flourishing are disharmonious. 

But a significant amount of harmony returns when we realize that making hard choices well is itself one of the most 
important aspects of our flourishing, and so the practical conflict between types of first order flourishing itself gives rise
to a higher harmony. More generally, there is less in the way of conflict between types of flourishing the more we look 
at highest types of flourishing, such as virtue, wisdom, and participation in beauty. 

While virtue and wisdom have already been discussed in the context of harmony, participation in beauty is worth focusing
on all to briefly. We find aesthetic practices in all human cultures, even those whose material life is harshest. Very 
few if any human communities are so impoverished that their members cannot tell stories to each other because their moral
duties leave them no time. Very few if any are incapable of some degree of decoration on their tools of daily use. But
above all, virtue and wisdom themselves can have a beauty that exceeds all the beauties produced by professional artists.
Most people can make their life into a work of art, and in doing so will possess moral and epistemic flourishing.

\subsubsection{Epistemic norms}
The unified character of our epistemic life is evident. The very goal of the epistemic life appears to be an explanatory unification of reality. Moreover, insights from one area are often helpful analogues for other areas. While there are differences in 
substantive reasoning methods, the general procedures of logical reasoning and weighing of evidence appear to be 
subject-independent.

Nonetheless, a fundamental tension in our epistemic life was noticed by William James, namely that between pursuit of truth and 
avoidance of falsehood. If one believes every proposition, one has all the truths, but at the expense of having 
all the falsehoods as well. If one believes no propositions, one has avoided all the falsehoods, at the expense of 
having no truth. 

But we probably should not consider this an actual disharmony in our epistemic norms. If we understand the epistemic
life in terms of belief, as in this Jamesian example, we can simply understand the norm of belief as: believe 
all and only the relevant propositions you have evidence for. This admittedly can be seen as a conjunction of 
two norms: (a)~believe all the relevant propositions you have evidence for, and (b)~don't believe any propositions 
you lack evidence for, where each norm is easily fulfilled on its own, while it is the conjunction of the norms that 
is challenging. However, this is no more disharmonious than it would be to formulate the task of a monochromatic pixel
artist by saying: blacken each pixel if and only if blackening it is a part of your artistic vision. One can artificially
split this into two conjoined injunctions, (a)~to blacken pixels whose blackening is a part of the artistic vision, and (b)~to 
leave unblackened those whose blackening is not a part of the vision, each of which is easy to fulfill on its own, but there
is no real disharmony here. Moreover, the split norms when fulfilled the ``easy way''---by blackening all or none of the 
image area, respectively---have little if any value. The value is in the combination. Similarly, a person who believes 
all propositions gets no wisdom, understanding or practical benefit out of it. For instance, if you believe that an ibuprofen 
will relieve a headache and that it won't relieve a headache, you get no practical guidance. And it is not clear that 
the paralysis that comes from believing nothing is any advantage to getting lots of stuff wrong.

Another potential tension is found in paradoxes, where our intuitions pull us to accept premises that end up jointly 
incompatible. However, there is good reason to see paradoxes in a more optimistic light, as tools that take us beyond
a surface-level understanding of the phenomena. 

\subsubsection{Disharmony}
The occasional conflicts between and within normative domains do raise a difficulty akin to the problem of evil for theists.
But at this point in our dialectics, the problem is not as severe. For while the theist needs to commit to the idea that 
every particular instance of evil has a theodicy, since every particular instance of evil must be compatible with the existence
of a perfect God if theism is true, the Aristotelian optimist as such is only committed to a harmonious tendency in our 
lives. Such a tendency is compatible with occasional highly significant exceptions.

Of course, if one opts for a theistic explanation of human nature, as I will eventualy argue one should, the exceptions 
will indeed cry out for a theodicy. The theodical task is a difficult one, as is well known. But it is a task that is 
beyond the scope of this book.

\subsubsection{A Darwinian evolutionary explanation?}
If a species' moral, prudential and epistemic norms were sufficiently disharmonious, members of the species whose 
behavior is driven to accord with these norms would tend not to survive and reproduce. It is tempting, thus, to think
that there is an evolutionary explanation of the harmony among our norms. 

One problem with such an evolutionary explanation is that it pushes back the problem to that of explaining the accord
between behavior and norms. Since we have argued that the norms themselves do not reduce to physical features, but 
depend on a metaphysical form, and evolutionary explanations involve physical causal stories, the explanation of the 
accord between behavior and norms is not subject to an evolutionary explanation. 

A second, and perhaps even more serious, problem is that Darwinian evolution explain phenotypic features that physically arise 
from genotypic features. An evolutionary explanation of the harmony of the norms would require the norms to supervene on 
physical phenotypic features. But we have reason to deny any such supervenience, because any such supervenience would require
a function from physical phenotype to norms with a vast number of unexplained free parameters, such as all the parameters we
discussed in Chapters~??backrefs. We solved the problem of free parameters by locating them in the form or nature of the 
animal, with it being a contingent matter of fact which forms are actualized, but on a supervenientist view, the function 
from phenotype to norm, and hence the parameters defining it, would have to be necessary. And it is implausible to think 
that such a vast number of independent and presumably messy parameters is metaphysically necessary.

\subsubsection{Non-realism?}\label{sec:non-realism}
I have to confess that in thinking about normative harmony, I find myself tempted to feel 
a certain suspicion. The harmony feels somehow suspicious. We wouldn't accept a theory on which
our norms are typically in conflict---such a theory would be ``too inconvenient''. There is something 
that feels ``too convenient'' about the harmonies posited here. It is tempting to say that we have 
the harmonies between the norms because we created the norms.

There are several ways of expanding on this. On an epistemic non-realism, there are objective, mind-independent norms, but 
our access to them is poor or non-existent, and out of wishful thinking or due to evolutionary pressures we have made 
up stories about a coherent realm of norms, whereas the reality might well be quite different. On a metaphysical non-realism 
we can keep the same debunking story about how we made up the stories about the norms, but deny that there are any 
objective, mind-independent norms, in favor of either an error theory that holds there are no norms or a subjectivism
that says there are norms but only ones created by our social practices. 

As said in the Introduction, a robust realism is assumed by this book. It is beyond the scope of the book to argue for 
this realism. At this point, however, it is worth noting that the arguments of the book do put significant pressure 
on the realism. Perhaps some readers, after realizing what controversial metaphysics is the best explanation of a realist
view will conclude that the realism is false. That is a different discussion.

At the same time, to the realist who is unwilling to abandon realism, the suspiciousness of the harmony should be 
evidence that there is something to be explained here. 

\subsection{Fit to DNA}
Imagine a world that is physically very much like ours, but the feathered flyers are defective
pigs, while the oinking mud-wallowers are defective birds. The defects go deep: it's not that the DNA fails
to be expressed correctly, but the DNA is itself defective, as happens all too often but to a much less
surprising degree in our world.

Is such a misfit between DNA and norms possible? Suppose that all we have available is Aristotelian metaphysics.
Why couldn't a feathered flyer have a porcine form? The main tool available in traditional Aristotelian metaphysics
is the principle that the matter of a thing must be properly disposed to its form, but we have seen in Section~\ref{disposition}
of Chapter~\ref{ch:VIII} that a proper disposition thesis sufficiently strong to distinguish the matter of a bird from 
that of a pig is unlikely to be true. 

Furthermore, we have to explain how a fairly continuous sequence of genotypic changes over evolutionary history 
correlates with discrete changes between forms. When birds evolved from dinosaurs, at some point the ability to 
fly became normative, with the occasional individuals incapable of flight being defective birds, whereas their flightless 
ancestors were normal (at least in this regard) dinosaurs. On the other hand, ????

To be an intellectually satisfied Aristotelian after Darwin, Franklin, Watson and Crick requires an explanation of how 
form and DNA track one another, an explanation that was not available to Aristotle.

%\subsection{Nature zombies}
%Aristotelian metaphysics allows for a curious hypothesis: nature zombies. Nature zombies are macroscopic entities that
%have the same physical structure as real organisms---say, oak trees or humans---but have no nature as whole (their
%physical constituents may have physical natures). A nature zombie would lack mind, and would have no intrinsic 
%normativity (unless there is some in their physical constituents), and above all would not even be a substance,
%but a mere heap of constituents.
%
%We can ask several questions about nature zombies. First, there is an epistemological one: how can I tell that
%all the apparent organisms around me aren't nature zombies? (I can tell I am not one, because I can tell that I have 
%a mind.) Second, assuming it is indeed so, why is it that there aren't any nature zombies around? Third, and perhaps
%most deeply, why isn't it the case that \textit{every} apparent organism is a zombie---i.e., why are there any 
%macroscopic things with natures?
%
%The epistemological question is easily handled in the same way that other skeptical hypotheses are. The Aristotelian
%can simply say that it is a part of our nature as the specific kind of reasoners we are that we should dismiss skeptical
%hypotheses. We don't need to (and couldn't) tell the real organisms from the zombies to justifiably think the General
%Sherman Tree, Seabiscuit and Biden to be (or have been) real organisms. We just need to be able to tell the organisms
%from the rocks and the like, which in these three cases is easy (though it is appropriately hard if we turn to viruses).
%
%What about the explanatory questions, however? It seems surprising, after all, if a bunch of physical constituents make
%an organic substance with a rich normatively laden form. Given the fact of abiogenesis---that life arose from non-life
%about 3.5 billion years ago---we might expect to have a nature zombie world. The zombies could be expected to evolve just 
%as well as the normatively-laden organisms that (assuming the arguments of this book are sound) we have around us.
%
%An Aristotelian move would be to suppose that the physical constituents of the universe themselves have natures, and it 
%could be that their natures have the causal power, when the physical constituents are rightly arranged, to produce a living 
%substance, and lack the causal power to produce a zombie. \textit{Prima facie}, the chemicals in a primordial soup could 
%produce one of three outcomes: a soupy mess, a zombie organism, and a real organism.  But it could be that their causal
%powers are so restricted that a zombie organism is beyond their power---it must be either a soupy mess (the typical outcome
%of mixing the chemicals) or a real organism with form and all. And the organism, in turn, could be the common ancestor of 
%all the other organism-like entities. ???

\subsection{Exoethics}
Exoethics is a hypothetical discipline studying both how we should treat intelligent agents and how they should behave.
In this section I will focus on the latter question.

We would expect that ethics for non-human rational beings---``aliens''---would be similar in norms 
where the biopsychosocial traits are similar, and by and large intelligible differences in norms where they are 
different. We would expect reproductive ethics to be different from ours for asexually reproducing beings, and the 
requirements of altruism to be different from ours for intelligent bees or intelligent sharks. And we would expect 
these differences to be intelligible: we would expect the intelligent bees to have norms requiring more altruism than
is required of humans and intelligent sharks to have less in the way of altruistic norms, and we would be amazed if 
it turned out the opposite way.

One explanation of this would be some grand function from biopsychosocial traits to kind-specific ethical norms, 
effectively meaning that there is a kind-neutral set of conditional ethical norms whose antecedents include facts
about the traits of the agent's kind. We already saw in earlier chapters how complex the details of ethical 
norms for humans would be, and \textit{a fortiori} a grand set of plausible conditional norms for all rational beings 
would have to be much more complex, likely with a vast number of seemingly arbitrary parameters that require grounding.

The Aristotelian way to account for such grand norms would be to suppose an excessively complex form of rational animal that 
includes conditional norms that specify how you should behave if you're an intelligent bee, an intelligent primate, 
an intelligent shark and perhaps a plasma-based intelligence in a star. The complexity here appears excessive. It is 
implausible to think that I contain information on what specifically I should do if I were an intelligent bee.
Moreover, assuming that the seemingly arbitrary parameters do indeed give us good reason to think that the norms 
are contingent---that another form of rational animal is \textit{possible}, one with different conditionals (perhaps requiring
slightly more altruism from the intelligent bees and slightly less from the intelligent sharks)---then there seems 
little advantage to supposing that we should expect all the rational animals in our world should have the same 
form, unless it so happens that humans are the only rational animals, in which case the excess of hypothetical norms 
inapplicable to humans but written into our form seems to violate Ockham's Razor.

It seems better, then, to suppose that if there were aliens, they would have their own form, with their own norms.
But now we come back to the puzzle. We \textit{would} expect their norms to be similar to and vary from ours in ways
that are explicable by their biopsychosocial traits. But why? Perhaps we just \textit{have to} be optimistic about 
our own forms and their harmony in order to avoid skepticism, and perhaps we are required by our own form to avoid 
skepticism. But why should we be optimistic about hypothetical aliens, especially if we accept the logical possibility
of lack of harmony between different norms as well as between norms and DNA and environment? Why does our optimism
extend beyond us?

To a lesser degree, the latter question can be posed about non-human non-intelligent life forms, which are not merely
hypothetical. How can we simply reject as crazy the hypothesis that the cottonmouth snakes are supposed to be 
intelligent but happen to have all inherited a DNA defect, present from the beginning of their species, which 
prevents the expression of this intelligence?

\subsection{Aquinas' Fourth Way and the good}
Aquinas' Fourth Way??ref puzzles the modern reader. It begins with a principle that comparisons between
degreed properties are grounded in a comparison to a maximal case: one is more $F$ when one is more like
the item that is maximally $F$. Aquinas then illustrates the principle with the case of heat and fire:
an object is hotter provided that it is more akin to the hottest thing, namely fire. He then applies
the principle to goodness, and concludes that there is a best thing, and this is God.

The fire illustration is not just unhelpful to us, since we know that fire is not the hottest thing (the sun is almost
twice as hot as the hottest flame), but it is actually a conclusive counterexample to the degree property principle,
since we can easily compare temperatures without reference to an alleged hottest object.\footnote{In any finite universe,
presumably there will be a hottest object. However, temperature comparison is not defined by that object, since 
even if Bob is in fact the hottest object, we would expect it to be physically possible to have a hotter object 
than Bob. But if degrees of heat were defined by closeness to Bob, it would not be possible to be hotter than
Bob, since nothing can be closer to Bob than Bob.}

So Aquinas' comparison principle is false. But I contend that there is still something to his argument
when applied to the good. 

Now, a form-based metaphysics gives a powerful account of the good for a being of
a particular kind---an oak, a sheep or a human, say---in terms of its match to the specifications of the
form. It also gives a ground to comparisons between the good of different instances of the same kind:
a four-legged sheep is, other things equal, better at sheepness than a three-legged sheep, because it
more completely fulfills the specification in their ovine nature. In fact, this is itself a counterexample 
to Aquinas' comparison principle, in that we can compare degrees of success at sheepness without supposing
any individual sheep to be perfect.

However, in addition to value comparisons within a kind, there are ones between kinds. When Jesus says
that we are ``worth more than many sparrows'' (Mt.\ 10:31??ref), what he says is quite uncontroversial.
Indeed, even a perfect sparrow seems to have less good than a typical human. While the nature of a sparrow
will enable value comparisons between sparrows, and that of a human between humans, we still have the question
of what grounds the value the difference between sparrows and humans. Some Aristotelians reject cross-kind 
value comparisons as nonsense.??refs But given the intuitive plausibility of many such comparisons, this rejection
is a costly one.

Aquinas' Fourth Way is not infrequently seen as more Platonic than his other arguments for the existence of
God, and Plato indeed had a solution to the problem of cross-kind comparisons, by talking of differing degrees
of imitation of the Form of the Good, which itself is perectly good. Plato, on the other hand, lacked a 
satisfactory solution to the problem of intra-kind comparisons. He may well have thought that there 
was a Form of Humanity??refs, which exemplified humanity perfectly, so that similarity to the Form of 
Humanity would define how good one is at being human. However,
we can see that this solution is clearly unsatisfactory. First, the Forms are immaterial, so the Form of Humanity is 
immaterial, and hence it lacks fingers. Thus, the fewer fingers a human has, the more they are like the Form of
Humanity, and hence, absurdly, the more perfect they are. Second, if somehow the Form of Humanity ends up having 
body parts, then the Form of Humanity either has an even number of cells or an odd one. But clearly neither option
is more perfect than the other. 

Central to Plato's solution to cross-kind value comparisons is the self-exemplification of the Form of the Good:
the Form of the Good is itself maximally good. But a similar self-exemplifying Form cannot be used to account for
intra-kind comparisons. Aristotle, on the other hand, has the non-self-exemplifying forms immanent in things. 
The Aristotelian form of humanity specifies human perfection, but does not do so by exemplifying it. It has neither
fingers nor cells, but it \textit{specifies} that humans should have ten fingers while specifying an age-dependent
normal range of cell numbers rather than a specific cell count.

Notwithstanding the ge neral falsity of Aquinas' comparison principle for degreed properties, Aquinas provides us
with a plausible extension of the Aristotelian system to allow for comparisons of degrees of good between objects
of different kinds in terms of the similarity to or degree of participation in a maximally good being, a divine being
that plays the role of a self-exemplifying Form of the Good. The human being participates in God in respect of
abstract intellectual activity, Aquinas will contend, while sparrows do not, and in that important respect, at least,
humans are more like God. On the other hand, the sparrow's movements approximate divine omnipresence better than 
the stillness of a mushroom does, and in that respect at least the sparrow is superior to the mushroom. We have,
thus, a ground for something like a great chain of being.

There are still difficulties here. While the human is superior in intellectual activity, the sparrow moves around
with greater three-dimensional freedom. How can we say that the human is superior all things considered? Where we
previously had a problem of cross-kind comparisons, we now have the problem of cross-attribute comparisons. 
Intuitively, the human's intellectual superiority to the sparrow trumps the sparrows motive superiority to the
human, and enables us to say that the human is more perfect on the whole. This higher level question is difficult
indeed. 

But there is some hope in thinking that in attributing different divine attributes we sometimes express divinity
to different degrees. It may be that there is no meaningful comparison between how well we express divinity by
saying that God is all-knowing versus by saying that God is all-powerful, saying that God knows the
multiplication table up to $10\times 10$ expresses divinity less well than saying that God can create any
possible physical reality. I suggested earlier that motion imitates divine omnipresence. Thus, the sparrow's
ability to fly imitates God's presence throughout several kilometers surrounding the surface of the earth, while
the human's more limited mobility imitates God's presence in a thin two meter shell of air surrounding that surface.
But the degreed difference between the two divine attributes---each a limited special case of omnipresence---imitated here 
might well be trumped by the fact that the sparrow does not imitate God's abstract intellectual activity \textit{at all}
while the human does imitate that activity, and does so in respect of a very wide scope of things (the human can think
abstractly about the whole universe, for instance). 

In ??backref, we gave a non-theistic Aristotelian sketch of a three-step great chain of being. The account here has
a hope of allowing one to fill in more intermediate links.
Even if the details in the comparisons between different attributes or respects do not work out, we still have an advantage 
for the theistic Aristotelian in being able to make cross-kind comparisons under specific respects, like motility or intelligence.

??ref:Jeffrey/Ward

\subsection{Modern technology and outlandish scenarios}
In ??backref, I argued that an ethics based on human form can simply ignore outlandish scenarios
that are far outside of our ecological niche, such as ones involving infinite numbers of
beneficiaries. However, there is a danger in this line of reasoning. As Arthur C. Clarke famously
said, ``Any sufficiently advanced technology is indistinguishable from magic.''??ref To human beings
50,000 years ago (or even just 500 years ago!) much of our technology would indeed be magical, and 
decisions that we routinely need to make, say in bioethics, would be predicated on outlandish assumptions. 

We might thus expect an ethics and epistemology grounded in a form possessed by hunter-gatherer primates
to be silent on dilemmas of a highly technological society, leaving us to do whatever we wish, or, even worse, 
to fail to harmonize with the shape of our lives, like that of a fish on land. Yet while there are, as there 
have always been, difficult and controversial moral and epistemological cases, we do not in fact find 
ourselves adrift without guidance in the modern world. Virtue continues to contribute to our flourishing,
and ancient texts, whether religious or philosophical, continue to point to good ways of living. 

This gives us reason to think that if our moral norms are grounded in human nature, human nature was somehow
picked out with foresight for what kinds of challenges humans would face in the distant future. Our ethics
does may not work in outlandish situations, such as those involving infinities as noted in ??backref, but it works in a
broader range of moral environments much broader than that found in early homo sapiens society. Thus, the theistic
version of our natural law theory both accounts for the apparent unsatisfactoriness of our ethics in situations that
humans apparently never find themselves with and the applicability in situations across a very wide range of situations,
wider and more technologically varied than the natural environment of other animals. This kind of foresight points to
a foreseer, indeed a designer, and hence towards a theistic version.

The move I suggested in ??backref for outlandish scenarios, namely that our ethics and epistemology simply does not apply
to them, may seem problematic given that we live in a world where many things that our not-too-distant ancestors would have
seen as outlandish are real. We fly regularly around the world and irregularly to the moon, speak with people on the other
side of the planet, move organs from one person to another, make cats glow by inserting jellyfish genes, program machines 
to have conversations with us, have bombs that can wipe out most megafauna including humans, and can clone at least embryonic 
humans. Our capabilities and the situations that we are in are quite different from those we evolved for. And further changes
may be facing us. Many think there is a serious possibility that human beings will spread through the galaxy, affecting vast 
numbers of lives, which may make actual seemingly outlandish questions about where our actions have very slight probabilistic 
effects on vast outcomes.??ref:Fanaticism,glitchy ethics??backref and discuss:ch4 on infinity

One approach to these modern questions is a principle-conservativism: the questions are settled by moral principles that we accepted
for millenia. Because the default for an action is moral permissibility, in the case of qualitatively new kinds of actions, 
principle-conservativism is apt to lead to a radical expansion of moral possibilities. If we had no principles governing human DNA
manipulation in the past, even if this was simply because we had no concept of DNA, now there are no limits, except limits coming
from traditional principles of harm and consent. Principle-conservatism in these kinds of cases would paradoxically justify a vast 
change in the shape of human lives of a sort that arguably is not compatible with human flourishing. Conversely, however, we have 
cases like the invention of effective therapeutic surgery. Prior to these, any serious degree of cutting open of the human being would be an instance of grave harm, and generalizing from
those cases to therapeutic surgery would have been unfortunate for the human race. 

A more naturalistically inclined Aristotelian could despair about ethics in quintessentially modern situations. Absent foresight from a God
or an axiarchic principle, we should not expect our natures to provide guidance in these situations, or at least ``non-glitchy''
guidance??backref. This line of thought could lead the Aristotelian to a dark view on which there just is no answer to a number
of contemporary moral questions, or on which the answer conflicts in a glitchy way with our moral intuitions, or perhaps
even one on which the true moral norms conflict, and we get hard-to-avoid moral dilemmas. 

But a theistic story can restore optimism. God can know what kinds of seemingly outlandish scenarios might
actually be relevant to the lives of his creatures, and can wisely choose the forms whose norms that fit with these. The resulting norms may 
seem \textit{ad hoc}, especially in edge cases: they won't be the elegant principles of classic utilitarianism (though of course classic utilitarianism
faces significant difficulties in out-of-our-experience situations, as we saw in ??backref:population-ethics). And an apparent
\textit{ad hoc} character in divinely-instituted rules as applying to edge cases should not surprise us---wise legislation does not eschew 
judgment calls.

An interesting question is whether a theistic Aristotelian should be surprised by having ethics glitch in some actual cases, in one of 
the three ways discussed in ??backref: (i)~real dilemmas, (ii)~conflict between moral rules and the reasons for them, and (iii)~conflict between
moral rules and our intuitions. After all, logical space contains an infinite number of possibilities for a form of a rational 
being, and a perfect being should be able to combine one with an environment in which there would be no actual glitches. 

I grant that it is very plausible that a perfect being \textit{could} do that. But would the perfect being do it? Even for a being
whose power is unlimited by anything other than logic, there can be unavoidable costs to options. Intuitively, there is a value to
the most important norms of behavior for limited beings\footnote{Why limited????}---say, norms governing killing---having a significant simplicity, so that
they can be reasoned about more easily, especially under time pressure.  At the same
time, there is a value to a diverse and rich moral environment. And there is a value to morality lacking glitches. Plausibly,
one cannot have all three values to their maximal degree at the same time---there may be logically unavoidable trade-offs. 
And there does not appear to be strong reason to think that a perfect being would be so enamoured of one of the three value that
we would expect that value to be present to the maximal degree. In particular, we should not expect a completely unglitchy ethics.

But we might have reason to hope that glitches are rare in the actual circumstances faced by humanity, or that the worst of the
glitches should only occur in the case of agents who have wrongfully produced the circumstances for this glitching.\footnote{Compare the theory that real moral
dilemmas only occur in the case of agents who have done wrong, say by making contradictory promises.}


\subsection{Avoiding radical scepticism}
There is a number of sceptical hypotheses that have the property that they cannot be ruled out either on logical grounds
or \textit{a posteriori}. These include hypotheses that the world around us is a computer simulation, that our moral
intuitions are disconnected from moral reality, that we are Boltzmann brains, i.e., short-lived brains in bubbles of 
oxygen arising from fluctuations in the vacuum of space, that we live in an infinite multiverse that undercuts all
probabilistic reasoning??ref, that simpler scientific theories are more often right other things being equal, and so 
on. Yet we think these hypotheses false. If we think them false neither on logical nor empirical grounds, it must be 
because we assign low probabilities to them prior to empirical assessment. Moreover, this assignment is required by 
our rationality: those who fail to assign low probabilities to them are irrational.

Our Aristotelian account can ground the correctness of this judgment of irrationality in human nature.??cf.backref 
But there would be something deeply problematic about us if the low \textit{epistemic} probability of the sceptical hypotheses 
were not matched by a low \textit{objective chance} for them to be true in light of the causal and stochastic structure of the world. 
If in fact the most likely way for a being with our rational nature and mental life to arise would be as a Boltzmann
brain, then even if we have lucked out and are not a Boltzmann brain, there is a disharmony between the world and our
mental life. 

In such a lucky case, the connection between our nature-required priors and the world then appears too fragile, chancy and ``unsafe''??ref
for the beliefs essentially dependent on these priors to count as knowledge. Even a reliabilist should say that if some beings
were required by their nature to assign a very high prior probability to the hypothesis that the universe formed an even number 
of years before life first arose, and it was mere chance that this hypothesis was true with the causal structure of reality
not assigning it a higher probability than the hypothesis of an odd number of years between the beginning of the universe and
the beginning of life, then that hypothesis is not knowledge. Yet it is very plausible that we \textit{know} the sceptical hypotheses
under discussion to be false. (This judgment has admittedly been disputed by a number of epistemologists who admit with G.~E. Moore that
I know that I have two hands, but will not allow the Moorean inference that I know that I am not a brain in a vat, despite the fact that I have two hands
obviously entailing that I am not a brain in a vat. ??ref) And even without considerations of knowledge, we might note that an optimism
resting on an assumption of mere luck appears paradigmatically irrational. 

The problem is perhaps most pressing in the case of the kinds of highly abstract \textit{a priori} intuitions discussed in ch4:??backref,
such as the intuition that the axioms of arithmetic are consistent or that nothing can cause itself. For if there is merely a coincidence
between the truth of the intuitions and our possession of  a nature that normatively requires us to have these intuitions and causally 
impels us to them, then even though we may be justified in following the intuitions, they are unlikely to count as knowledge.
Aristotelians thus need a theory on which there is the right kind of connection between our rational nature and the metaphysical, causal and 
stochastic structure of the world. 

%\subsection{Global aesthetic-like features}\footnote{I am grateful to Nicholas Breiner for drawing my attention, in the context of
%justice, to this form of explanation of moral features.}

\subsection{Commonalities between laws}
????????

\section{Kind-independent goods}
Aristotelianism does really well with explaining kind-dependent values. But there also appear to be values that appear
to transcend kinds, such as simplicity, diversity, flourishing, achievement, etc. Furthermore, we can compare kinds.
The Aristotelian account defended in previous chapters can ground the comparison between a flourishing and a non-flourishing 
human, or between a flourishing and a non-flourishing chantarelle mushroom. But it is also obvious that a human is a better
kind of entity than a mushroom.
????

\section{Complexity and explanation}\label{sec:hierarchy}
\subsection{A problem}
A central form of argument for Aristotelianism is based on Mersenne questions and the messy normative complexity of our
lives. But do we not normally prefer simpler theories to more complex ones, and hence should we not reject the normative
complexity in favor of a simpler theory like utilitarianism in the name of Ockham's razor?

Ockham's razor, however, has always been a defeasible criterion: entities are not to be multiplied \textit{beyond necessity}.
But sometimes there is necessity. It would be simple to suppose that all trees of a single species look exactly the same.
But that just wouldn't fit with our evidence. In biology, one does not expect individuals of the same sort to be exactly
alike, unlike in fundamental particle physics. 

Specifying what a rational animal of a particular species ought to be like
and how it should behave can be expected to involve a lot of information. How much information? Well, we might take the information contained
the DNA common to all humans to give us a lower order of magnitude bound, since the common DNA presumably encodes something
about what human bodies are supposed to be like. There are 3.2 billion base pairs in human DNA, and 
99.1% of human DNA is said to be common to all humans (??ref https://www.sciencedirect.com/topics/biochemistry-genetics-and-molecular-biology/dna-profiling).
Since each base pair is two bits of information, that means about 6.3 billion bits, or the equivalent of about 500,000
book pages.\footnote{Counting a page at 1800 characters and each character at seven bits: $6.3\times 10^9/(1800\cdot 7)=500000$
(oddly exactly, by coincidence!).} 

Imagine the task of designing the rules of behavior for a rational animal that has a significant complexity in its bodily
life, subject to the constraint that the rules lead to a life that elegantly balances moral and epistemic norms, and fits
well with the bodily nature of the animal and its niche in the ecosystem. It is plausible to think this will be several
orders of magnitude more complex a task than that of designing the rules for a well-balanced and significantly embodied game 
such as tennis. Generating a game of pleasing elegance and yet compelling complexity, especially an embodied one, takes a 
fair amount of information, and the official rules for tennis are about forty pages.??ref 

We can think of simplicity as an aesthetic criterion in theory choice. But simplicity is not the only factor contributing
to beauty! (If it were, the most beautiful art would be no art: you can't get simpler than an installation that can be 
completely described by $\sim\exists x(x=x)$.) Overall theoretical simplicity is one way of having an elegantly unified
theory. But one can also achieve elegant unification in other ways. Consider, for instance, hierarchical organization. Wittgenstein's
\textit{Tractatus}??ref achieves a unification by being summed up in seven top-level sentences, with a progressive hierarchical
amplification and justification in terms of multiple levels of sentences. Or consider the unification achieved in biology by
Linnaean and Darwinian taxonomies.

We could have an ethics that is simply simple: it has a briefly expressible rule that covers everything in full detail. But 
just as it is unlikely that we would get a compelling racquet sport with a single brief rule, even if we allowed for some
vagueness, it is unlikely that we would get a harmonious set of norms for the life of a rational animal out of such a rule---that, 
indeed, is an upshot of the enumeration of the many Mersennian issues of detail in normative phenomena that have been discussed
in this book. 

Can we have some other kind of theoretical unification? I think we can. As discussed in connection with particularism(??did I??backref),
we can suppose suppose a hierarchical structure. We can have a hierarchical ethics, at the top with one or more principles like Aquinas's ``Pursue the good
and avoid the bad'', the Kantian injunction to treat others as ends rather than mere means, or the Biblical ``Love your neighbor as yourself''.
But perhaps unlike the historical Kant, we need not take the top level principle or principles to have all of the normative informational 
content for morality. Instead, we can think of it as a unifying headline, perhaps to explaining tennis by saying: ``Hit the ball back into the
other player's side.'' There will be further rules that are not mere logical derivations, but build on the general principle expressed in 
the higher level rules by giving more specific rules. The second level rules themselves are not unlikely to need further adumbration.

Consider Hillel's famous response to the request that he explain the Jewish law while standing on one leg: 
\begin{quote}
That which is hateful to you, do not do to your fellow. That is the whole Torah, all the rest is commentary. Now, go and learn it [the commentary].
??ref:add scholarly translation
\end{quote}
Now it is clear that in fact that the primary Jewish commentaries on the five books of the Torah (i.e., the Mishnah, and the
Talmuds which are commentaries on the Mishnah??check,refs) contain normative material not found in these books. Nor should this be 
a controversial claim, since rabbinical tradition holds that the rabbis have an oral tradition going over and beyond the books of 
the Torah. Thus we should probably interpret Hillel as saying that the Golden Rule (in his negative formulation) is a kind of summary,
rather than as saying that the rest is logically derivable. Similarly, Aquinas, after giving his ``first precept''??ref that good is to be done and evil avoided, lists second level laws such as
preserving human life, respecting the reproductive life of us a rational animals, knowing the truth (especially about God), and living 
in society. It is clear that there we still have not reached the level of normative information needed to resolve all moral 
questions.

In both the Hillel and Aquinas cases, we have a unification of ethics under one or more general principles that are insufficient
for deriving all the specifics. This is akin to the explanatory unification that modern biology receives from evolutionary theory. 
Besides generalities like that species tend to mutate towards inclusively fitter forms, the basic principles of evolutionary theory---random 
variation and the survival of the fittest---do not generate specific predictions. However, they do organize the vast sphere of modern
biological knowledge. 

Famously, Aristotle has observed that in ethics, unlike in geometry, one can only speak in ways that are true for the most 
part.\footnote{??ref. Of course, this claim itself needs to be carefully understood. Aristotle himself says that murder and 
adultery are always wrong. Perhaps he is thinking that murder and adultery are definitionally wrong---murder being a wrongful 
killing and adultery being sex contrary to respect for marriage?} On the account I am defending, this is not quite right. Instead,
many of the higher level ethical claims are what one might call ``generalities'' that organize ethical reasoning. These claims 
can be exceptionlessly true, but there are limits to how helpful they are in particular cases. Thus, it may always be true that we 
should respect human life, but this does not give a clear answer as to what the health care provider should do when the family of 
a particular patient requests disconnection from life support. The respect claim describes, in general terms, the shape that the 
finer-grained principles have. And it may well be that the finest grained principles which apply to certain particular cases
have a complexity beyond our practical ability to specify, and so we do not have principles that definitively settle a case.

While I have used ethics in the above discussion, the same plausibly applies to epistemic rationality, where we have a very general
principle like ``Pursue understanding (or knowlege or truth)'', with finer-grained specifications such as ``Avoiding error is more
important than getting at truth'', ``Prefer elegant theories'' and ``Direct your attention to more important matters.'' In the case
of semantics, we may, on the other hand, have a high level principle that ``Meaning follows usage'', and then a variety of finer-grained 
principles about how usage yields meaning. As we get to very fine-grained principles, we have an extremely complex account, but hierarchically
organized.

\section{Some Aristotelian tools for explaining harmonies}
\subsection{Three tools}
The classic Aristotelian has several tools for explaining the harmonies in Aristotelian optimism.\footnote{This section owes much to discussion in my mid-sized objects seminar, and especially to Christopher Tomaszewski's suggestions on the explanatory powers of forms.}

The first classical tool is the theory of the adequacy of matter to form on which for matter to receive a particular 
form, it must fit it sufficiently well, because a form's power of informing is limited to some kinds of matter. 

This might well explain the harmony between the DNA and the phenotypic features of an organism, on the one hand, 
and its norms, on the other hand. A molecule-by-molecule copy of a pig on this theory cannot have the form of a sparrow or 
a bat or the like, and hence we need not wonder at why the norms governing a pig do not require it to fly in order to flourish.

By itself, the material adequacy tool is insufficient to explain the harmony between DNA and norms, however. For while 
it can explain why a molecular duplicate of a pig cannot be specifically a sparrow or a bat, it does not explain why there 
isn't some other form that requires flying for flourishing but also has the power to inform a molecular duplicate of a pig.

The second classical tool consists in the causal powers embedded in the forms. The form of a sparrow not only includes the 
norm of flying, but a tendency to develop a power to fly in the organism. The forms thus have tendencies to make the organisms
with the forms behave in ways in accordance with the norm.

Again, the causal power tool has a similar limitation to the material adequacy tool. While we can explain why \textit{sparrows}
have both the norm of flying and a tendency to be physically capable of flight, we need an explanation for why there isn't some 
other form than that of a sparrow---the form of an unfortunate bird that has the same norms as a sparrow but none of the causal 
powers relevant to flight. 

A natural answer to the limitations of the material adequacy and causal power tools is found in the third tool: the sparseness 
of forms. The logical space of forms is sparse: only some combinations of norms and powers (including 
the power of informing) are candidates for being contained in a single form. Compare here Plato's insistence that there is 
no Form of Mud.??ref:Parmenides The reason why the evolved pig-like animals in our world do not have a form that both makes 
flying normative and can inform a pig replica is simply that there cannot be such forms. And similarly, nothing in our world 
has has the norms of a sparrow without the causal powers relevant to flight, because no such form exists in logical space.

\subsection{Evaluation of the tools}
First, note that the material adequacy tool is problematic in light of the arguments of ??backref that organisms begin 
to exist in bodies that are too small to have any significant structure. That said, in ??backref it was noted that even 
if no significant structure is required of matter to have a specific organic form, nonetheless some structures may be 
incompatible with some forms. Thus, perhaps, a bird form could inform something without much significant structure, but 
could not inform a pig-like body. I argued that this view is not sufficiently metaphysically motivated, but is still
compatible with the arguments. So perhaps the material adequacy advocate can keep on using the material adequacy tool.

Next, note that both the material adequacy and causal power tools in their above defense depend on the sparseness tool.
So our attention should shift to sparseness. Now, ontologists who have a sparse theory of properties typically formulate
sparseness in terms of the denial of logically complex properties (such as being both negatively charged and cubical)
and/or of properties that do not cut nature at explanatory joints (such as baldness). 

But this kind of sparseness will not help the Aristotelian. Consider a sparrow form $S$ and a less harmonious form 
$S'$ which is structurally similar to the sparrow form but has its parameters and powers arranged in a less harmonious 
ways---for instance, maybe the form is capable of informing a less sparrow-like body, or perhaps it has overt 
behavioral and epistemic norms that conflict much more than those of a sparrow. The important thing here is that 
we can suppose that the less harmonious form $S'$ simply differs from $S$ in the values of various parameters, rather 
than in terms of logical complexity. Moreover, if instantiated in some matter, $S'$ would cut nature at the joints 
just as much as $S$ would---both would specify exactly the kind of object that the matter constitutes.

Standard sparseness theories are value-neutral, and restrict forms on structural grounds. But to explain harmonies,
we need a value-based sparseness that rules out \textit{bad} forms. Given that what forms are possible is a fundamental
metaphysical feature of reality, the sparseness theory requires a fundamental metaphysics sensitive to value. The main
such metaphysical theories are theism, pantheism, and axiarchism, on all of which the fundamental features of reality 
induce a drive to the good.

Furthermore, simply ruling out \textit{bad} forms is probably insufficient for Aristotelian optimism. We could imagine 
borderline good forms, ones full of disharmony and a disunited ethical life, leading to many skeptical worries, but 
still on the whole good. Aristotelian optimism requires us to think that such forms are \textit{less likely} to be
instantiated, and hence that in our theorizing about norms we should prefer theories that show a greater unity, 
but does not and probably should not require us to rule out such forms. 

For ruling out the metaphysical possibility of borderline good forms would require fundamental metaphysics itself to 
draw a line between the forms that are good enough and those that are not. It seems difficult to find a non-arbitrary
place to draw this line---again, we have Mersenne questions. The only non-arbitrary lines appear to be those between
the bad, the neutral and the good, as well as perhaps between the optimal and the inoptimal.

Restricting forms by means of the line between the optimal and the inoptimal might well be sufficient for Aristotelian 
optimism. But how could we use optimality considerations to generate sparseness? There appear to be two options. First,
we could require the forms themselves to be optimal, and, second, we could require that the forms be such as to be found
in an optimal world. 

It initially seems that if we require the forms themselves to be optimal, we will only have one form---the very best
possible form, that of a perfect divine being---and hence only one kind of substance in reality, whereas we were searching for an 
explanation of optimism about the diverse forms in the world, or at least the human form. Moreover, we have good reason to 
think a perfect divine being is not the only substance there is. For if that is the only substance, then we must be modes or 
parts of it. But my moral failings rule out the hypothesis that I am a mode or part of a perfect substance.

However, there is another optimalist explanation of what forms there are. Instead of applying optimalism to the forms, 
apply it to worlds, like Leibniz, Leslie and Rescher did??refs, so that there is only one metaphysically possible world. 
Then we can say that the space of forms contains only ones that 
can be instantiated in the best of all worlds.\footnote{This is not exactly Leibniz's own view. Leibniz thought
that the forms of entities competed for instantiation in the divine mind.??ref This metaphor implies a winner and a loser,
and hence the existence of forms that lose the contest, and hence are not instantiated in the best of all possible worlds.} 
This gives a value-based sparseness in the logical space of forms, but of course seems to give as much optimism as anyone could
want---we can't do better than to think that we are in the best of all possible worlds. 

The downside of best-possible-world optimalism, of course, is metaphysical modal fatalism: there is only metaphysically 
possible world. We now have a dilemma regarding the class of worlds among which ours is said to be best. Either that class
contains only one world or it contains multiple worlds. If it contains only one world, then it is unclear why optimalism
implies optimism. For if optimalism implies optimism, then we would expect worlds-based pessimalism, the view that we are 
in the worst of all worlds, to imply pessimism. But if there is only one world for ``reality to choose from'', then 
both optimalism \textit{and} pessimalism hold, and it is quite unclear whether we should be optimistic or pessimistic.

Now, suppose that there are multiple worlds---likely infinitely many. Since on optimalism only the best world is 
metaphysically possible, the other worlds must be metaphysically impossible. At this point we appear to have two 
initially plausible options. First, we could have no logical restrictions on the worlds---thus, we will have worlds
with square circles, and so on---and, second, we could require the worlds to be (narrowly) \textit{logically} coherent, even if 
they are not metaphysically possible. In other words, they have no strictly logical contradictions in them, though they 
may violate fundamental metaphysical principles, like the principle of optimalism. 

If we place no logical restrictions on worlds, then it is unlikely that the best world is in fact one that is 
logically coherent. For consider the kinds of things theodicists would say in defending the compatibility of 
the goodness of God in the face of the horrendous evils of this world, which are presumably also going to have 
to be similar to what the optimalist will need to say to defend the optimality of our world in the face of these
evil. These may be claims that various great goods in the world such as courage in the face of evil, forgiveness of 
real evils, and exercises of sacrificial love logically require evil. But this defense of the optimality of our world will fail if we drop logical constraints. We can have courage in the face of evil without evil, forgiveness of real evils without
real evils, and exercises of sacrificial love with no losses. Furthermore, on optimalism, the best 
\textit{possible} world cannot include free will of the sort libertarians believe in, since that kind of freedom
requires the possibility of alternate possibilities, whereas on optimalism there is only one possible world. But an 
incoherence world can include libertarian-style free will. In fact, it can include libertarian-style free will and 
determinism, since incoherence is no bar. Thus it seems very likely that the best of all worlds will be incoherent, 
if incoherent worlds count.

One might respond that incoherence is itself necessarily a disvalue. But since we are talking of an incoherent world, we could 
suppose that at our best incoherent world incoherence is not actually disvaluable---perhaps even despite being necessarily 
disvaluable. (In an incoherent world, one may not be able to infer actuality from necessity!)

Or for a different approach, imagine a world that contains nothing but one entity, Bob, with the logically incoherent property 
of being such as to be better than any world or any plurality of one or more entities in any world (the property is incoherent
since it implies that its possessor is better than itself). In any case, the point should be made. If our world is the best
one from among all worlds, including the incoherent ones, then we should expect our world to be incoherent. But surely it's not.

Now suppose we constrain worlds to be logically, but perhaps not metaphysically, possible. A proposition is narrowly logically
possible provided that no contradiction can be derived from it by the rules of the correct logic. A serious philosophical 
difficulty here is specifying what one means by ``the correct logic''. An obvious necessary condition on the correct logic $L$
is that if $q$ follows from $p$ according to $L$, then the material conditional ``if $p$, then $q$'' has to be metaphysically
necessary. But if logical necessity is to be distinguished from metaphysical necessity, as the proposal under discussion requires,
the converse cannot always be true. In other words, not everything that follows of metaphysical necessity can follow of logical
necessity. But now it is not clear which metaphysically necessary conditions correspond to rules of ``the correct logic''.

There is, further, a serious technical problem. First note that if logical necessity is distinguished from metaphysical necessity, it is presumably distinguished by the formality of logical necessity: the fact that logical necessity is expressed by means of 
formal rules of derivation from formal axioms. Second note that the S5 axiom of modality is quite plausible for whatever
modality governs the worlds the optimalist says the best one is chosen from. The S5 axiom says that if something is possible 
in one possible world, it's possible in all of them. For if logical modality varies between the logically possible worlds that
the best one is chosen from, then what logically possible worlds there are will vary between the logically possible worlds. Hence, explanatorily prior to the choice of the best logically possible world there is no fact about what worlds are in fact logically possible, and the idea of the underlying metaphysical principles ``choosing'' between these worlds seems to make sense, since 
prior to the choice of one of them, there is no such thing as ``these worlds''. It turns out that there is a conflict between 
the formality of logical necessity and S5, however.

But before getting to the conflict, we should note that my argument does not need the full S5 axiom. It only needs 
perhaps the weakest imaginable version of it:
\ditem{weakBrouwer}{There is some statement $s$ such that it is logically necessary that is logically possible that $s$.}
The S5 axiom says this is true whenever it is logically possible that $s$, but \dref{weakBrouwer} only says this is true 
for at least one $s$. For instance, it is highly intuitive that it is logically necessary that it is logically possible 
that evil is evil, or that $2+2=4$, and so on.

But now for the conflict.??ref It is very plausible that among the things that are logically necessary are the Peano axioms of 
arithmetic. These are so intuitively implied by our concept of the natural numbers, that if we deny their logical necessity,
we might as well deny the logical necessity of such claims as that genuine forgiveness requires a wrong or that courage 
requires evil, claims that I have argued are likely to be needed for a defense of the optimality of our world. Moreover, for
the sake of the \textit{formality} of logical necessity, the axioms need to be formally recursively specifiable, and similarly
for the rules of inference. But now G\"odel's Second Incompleteness Theorem comes in and tells us that no formal system 
that includes the Peano axioms can prove its own consistency. Suppose, however, that it is logically necessary that it is 
logically possible that $s$. On the coherence account, that it is logically possible that $s$ means that a contradiction
cannot be derived from $s$. And that it is logically necessary that it is logically possible that $s$ means that there is 
a formal proof of the fact that no contradiction can be formally proved from $s$. But if we could prove that no 
contradiction can be formally proved from $s$ in the formal system of true logic, we could also prove that no 
contradiction can be formally proved in the formal system of true logic (since a proof of a contradiction in the system
woudl yield a proof of a contradiction in the system from $s$ by simply sticking $s$ in as a premise)---in other words,
we would have an in-system proof of the consistency of the system, contrary to G\"odel's Second Incompleteness Theorem.

Now, maybe there is some notion of the formality of a system that escapes this argument, but in any case we have a 
serious difficulty for an account that takes narrowly logical possibility to be essential to identifying the worlds 
from among which the best is chosen. Thus, whether or not we constrain these worlds by logic, we have a serious problem
for optimalism.

??harmony BETWEEN forms

\section{Explanation of our normative complex}






\subsection{A pattern of explanation of norms}\label{sec:moral-explanation}
Here is a familiar pattern. We have a deeply-seated moral intuition about the general prohibition, call it $g$, of some action,
such as incest. It is not clear how to derive the prohibition in its full generality from intuitively more basic principles, such as one of 
the categorical imperatives. Easy considerations, which I will call the $c$s, show that in \textit{typical} cases the action is wrong,
but our moral intuition goes beyond these typical cases. Thus, considerations of the abuse of power, distortion of 
familial dynamics, and genetic harms show that most cases of incest are wrong, but it is easy to imagine cases
of incest to which these considerations do not apply---say, elderly siblings who were raised apart---and yet moral
intuition forbids incest in those cases as well.

We can now save the moral intuition by saying that the more general prohibition $g$ is simply a fundamental moral rule,
not reducible to the $c$s that explain why the action is wrong in typical cases. But if we stop
at this, the connection between $g$ and the $c$s mere happenstance, and that seems intuitively wrong. The abuse of 
power, distortion of family dynamics, and genetic harms should be relevant to why incest is wrong.

At this point, often we are in a position to see another fact: it is quite beneficial to have a 
general moral prohibition beyond the prohibitions arising from the $c$s. 

One reason for such a benefit from a general prohition could be that our judgment as to whether the $c$s apply to a given case is fallible, especially given our capacities
for self-deceit, and the costs of violating the $c$s are so high that it would be better for us to have a 
general prohibition than to try to judge things on a case-by-case basis. 

Second, in some examples of
the pattern, serious deliberation about the forbidden action can itself harm one or more of the goods
involved in the $c$s: thus, having to weigh whether the distortion-of-family-dynamics consideration
applies against a particular instance of incest can itself distort the agent's participation in family
dynamics. 

Third, we could have a tragedy of the commons situation. It could be that the $c$s are actually insufficient
to render an instance of the action wrong, but we would be better off as a society if we had general
abstention from the action. Thus, perhaps, the genetic harm coming from one more couple's engagement in incest
would be insufficiently significant to render the incest wrong, but without a general prohibition, incest
would be sufficiently widespread as to cause serious social problems. A general prohibition that is not
logically dependent on the $c$s would help avert such social harms.

These considerations are very familiar to us in the case of positive law. Jaywalking involves harms such as
disruption of traffic flow and the danger of death of the pedestrian and of trauma to the driver, and the
considerations of these make jaywalking wrong in typical cases. There are 
obvious instances, however, where these considerations do not apply: say, crossing a road where the pedestrian
can clearly see that there are no intersections or cars on the road for a significant distance in either 
direction. However, it may be better for people simply to abstain from jaywalking than judging whether the 
disruption and safety considerations apply on a case-by-case basis, because there could be so much harm if 
the judgment were to go wrong. As a result, it is can be reasonable for a state simply to ban jaywalking
altogether (or to ban it with some clear and easily adjudicated exceptions). We similarly resolve cases of
tragedy of the commons with positive law: think, for instance, about laws against littering.

In the case of positive law we have two different explanations. First, there is an explanation of why
the forbidden action is wrong in general: this is because it has been competently forbidden by legitimate authority.
This explanation need not make reference to considerations such as disruption of traffic flow or danger
of death.\footnote{Though in some cases \textit{some} such reference may be needed in order to establish
that the matter falls within the competence of the authority in question. Thus, a government agency may
be permitted to make rules on matters where traffic flow disruption is concerned.} Second, there is an 
explanation as to why the action has been forbidden by the authority---and here all the rich considerations
are relevant.

\subsection{Theism}
A theistic version of natural law can have precisely the above pattern. An action is morally forbidden because
our nature is opposed to it, an instance of grounding explanation. This explanatory fact does not make reference 
to the $c$s. But we still have a further question to ask that it is natural to put in the form: ``Why does our nature 
includes this prohibition?'' But since our nature is essential to us, the answer to that question could 
simply be the necessary truth that we couldn't exist without this nature. However, we can put the question
in a different way: ``Why are there intelligent primates on earth with a nature that includes this prohibition
rather than some other kind of intelligent primates with a nature that does not include this prohibition?''
And here the theist can answer: Because it would be good, in light of the $c$s and the
further considerations in favor of generalizing the prohibition beyond the cases where the $c$s specifically
apply, to have intelligent primates with a nature that includes this prohibition, and God acted in light
of this good.\footnote{A divine command theorist can make the same move, but divine command theory has some
liabilities which were discussed in ??backref.}

The explanation may still appear viciously circular. On an Aristotelian metaphysics of value, what is good for us
is grounded in our possession of our form. How could the possession of our form, then, be explained by what is good 
for us? But it is difficult to see the difficulty in the context of theistic selective explanations. Whatever 
form will be exemplified will define what is good for its possessors. If God were to choose to exemplify a form
that defines one and only one state as good for its possessors but that also makes it nearly impossible to attain
that state, the result would be beings that almost universally are in a bad state. There is reason not to do that.
Instead, God has good reason to select a form that makes it much easier to attain the good state defined by the form.

Here we should make a distinction between the specific goods grounded in our form---health, friendship and the like---and
the good of fulfilling our nature. The specific goods are grounded in our nature. But it is not clear that we should say that 
the good of fulfilling our nature is itself grounded in our nature. It is plausible in the Aristotelian context to say that 
to be good for $x$ just \textit{is} to fulfill $x$'s nature. This identity is simply reductive. Given this, the circularity
in the explanation disappears. What specific things are good for us is grounded in our nature. But it is good for us to fulfill
our nature, and that fact is independent of what the nature is. It thus makes sense to explain \textit{which} nature is 
exemplified by considerations of how apt the possessors of that nature would be to fulfill that nature, and have that good.
We thus have a kind of explanation of why rabbit-like beings specifically have reproduction be good for them---having reproduction
be good for them is more apt for the fulfillment of their nature than, say, having the discovery of mathematical truth be good
for them. It is the general good of fulfillment of any nature that can explain why a nature with such-and-such specific goods is
selected.

In fact, we can have explanatory relations running both ways between goods and norms of behavior. If having norm $N$ promotes
some specific good $S$, then that could explain why a being whose form codes for $S$ being good also has norm $N$ of behavior.
But conversely, we could explain why a being whose form codes for norm $N$ also codes for $S$ being good for the being. 
If the norms are selected for exemplification by a perfectly good God, we may expect both forms of explanation to show up,
as well as a hybrid model where both $N$ and $S$ are chosen together for their fit.

This kind of divine selection explanation of both ethical norms and goods extends from ethical to prudential, epistemic and 
semantic norms. As we saw in ??backref, ethical, prudential, epistemic and semantic norms all interact in complex ways with 
what is good for us, and this interaction can provide God with reasons in favor of some and against other combinations of
norms and goods.

We would expect God to have access to the truthmakers of fundamental abstract intuitions. Indeed, on some theological accounts,
God himself is the ultimate truthmaker of many of them, including notably mathematical and modal truths.??refs If so, then 
if God designed our nature so that our intuitions might mirror his knowledge, it is plausible that our justified following of
these intuitions does indeed yield knowledge in us.

\subsection{Non-theistic alternatives}
What could such a selective cause be like? There are three main candidates for selective causes in the philosophical
literature: evolution, axiarchic principles, and intelligent designers such as God. 

Genetic descent with variation only directly governs the non-normative aspects of organisms. It is not sufficient to explain
the form that the organism has, given that the form encodes normative features as well. Nonetheless, it is worth considering
the possibility of a law of nature linking DNA to form, a law of nature specifying that when an organism with such-and-such 
DNA comes into existence, it has such-and-such a form. This law of nature would be immensely complex, with many free parameters
raising Mersenne questions. Moreover, if the law of nature is to cohere with the Aristotelian optimism that is crucial to our
Aristotelian account, there must be a fit among the various aspects of the form and between the form and the actual physical 
body plan and physical environment. It is implausible that this fit, in us and presumably in the myriad of other organisms,
is just a coincidence. Such a law of nature calls out for an explanation. On pain of vicious regress, the need to explain the 
law of nature points towards one of the other two explanatory candidates: axiarchic principles and intelligent designers.
Moreover, without a value-laden explanation of the linkage between DNA and form, the hierarchical explanations discussed 
in Section~\ref{sec:selection}, and the non-deductive hierarchical explanation of normative principles is replaced by a vast
coincidence, and hence we do not have a satisfactory answer to the complexity objection. 

Our Aristotelian account in order to be intellectually satisfactory requires an explanation that itself delves into some 
normative domain. This explanation could directly govern the imposition of forms or via some intermediary like a linking
law of nature. 

Furthermore, the need to explain our fundamental \textit{a priori} intuitions in a way that connects them with their truth
is unmet by a purely evolutionary approach. Evolution doesn't respond to the consistency of the Peano axioms of arithmetic 
and give us a corresponding intuition.

At this point, our choice appears to be between an explanation involving a non-intelligent tendency towards value and an intelligent one,
such as the theistic one that we have already discussed.

The main candidate for the non-intelligent tendency are axiarchic principles, such as those defended by Leslie and Rescher.??refs
These are fundamental metaphysical principles that require the world to be optimal. 

It is worth noting that while I am discussing axiarchism as an alternative to theism, Rescher himself takes his theory to
imply theism: it is better for there to be a God, and hence there is a God.??ref And if there is a God, then presumably this God
is sovereign and governs the selection of forms, and we can skip forward to the discussion of theism. Leslie's version also
involves supernatural beings. But Leslie thinks that what is best is not that there be one infinite all powerful, all knowing
and all good God, but infinitely many omniscient observers who enjoy the world thereby adding to its value, though without
creating it, since then there would be the possibility of conflict between them.??ref,check

There are four main problems with axiarchic principle explanations.

First, intuitively, a metaphysical principle \textit{constrains} what beings can exist and how they behave rather than somehow 
explaining the positive existence of beings. But the axiarchic principles are supposed to explain the existence of beings: the beings
in reality exist because it is for the best that they do so. 

Second, there does not appear to be a unique best world. We could take any good world and add one more happy disembodied mathematician.
This might not produce an overall better world. It might, for instance, be aesthetically inferior in some way---say, by having too many
mathematicians and thus offending against simplicity, or by having a non-prime number of mathematicians (perhaps the aesthetically best 
number of mathematicians is a prime of the form $2^n-1$ for some large $n$)---but it is superior in at least one significant way, namely 
by having an additional happy mathematician, and it is not plausible to think that it would be an overall inferior world.

Third, axiarchic principles appear to lead to modal collapse. If metaphysical principles require everything to be 
for the best, then it seems that everything must be the way it is. 

There are at least three potential ways out of the modal collapse objection. The first is Leibniz's solution who distinguished between 
moral necessity and logical necessity. A proposition is logically necessary, according to Leibniz, provided that there is a finite
proof of a contradiction from its negation. It is morally necessary provided that there is a finite \textit{or infinite} proof of a 
contradiction from its negation. There are infinitely many logically possible worlds (where as usual $p$ is possible just in case its negation
is not necessary) but only one morally possible world---the best world. It is logical modality, then, that answers to our intuitions about
the broad range of possibilities for reality.

Unfortunately, Leibniz's notion of logical necessity in terms of finite proof does not fit well with much later developments in logic and modal logic.
Consider a very weak version of Axiom S4 of modal logic. Axiom S4 says that \textit{any} necessary proposition is necessarily necessary.
Weak S4 says that \textit{some} necessary proposition is necessarily necessary. This seems utterly uncontroversial. For instance, surely, that 
everything is either green or not green is not only necessary, but necessarily necessary.\footnote{Weak S4 can be proved to follow from 
the Necessitation Rule, which says that if $p$ is a theorem, so is $\Nec p$, as long as the logical system is such as to have at least one theorem.
For if $p$ is a theorem, then $\Nec p$ is a theorem by Necessitation, and hence so is $\Nec\Nec p$.} From Weak S4 and uncontroversial axioms
of modal logic it follows that some proposition is necessarily possible.\footnote{Suppose $\Nec\Nec p$. By Axiom~T, $\Nec p \rightarrow p \rightarrow \Poss p$
is a theorem. By the Distribution Axiom, it follows that $\Nec\Nec p \rightarrow \Nec\Poss p$ is a theorem. Since we have $\Nec\Nec p$, by
modus ponens we have $\Nec\Poss p$.} But now a proposition $p$ is necessarily possible in Leibniz's sense of logical modality just in case 
there is a proof that it is possible. And $p$ is possible just in case there is no proof of $\Not p$. Thus, $p$ is necessarily possible
just in case there is a proof that there is no proof of $\Not p$. Now, in an inconsistent logical system, there is a proof of \textit{every}
proposition. Hence, a proof that there is no proof of $\Not p$ would be a proof that the logical system we are working with is consistent.
But as long as the logical system has a recursively enumerable??? set of axioms (and to deny that would not be in the spirit of Leibniz's
notion of finite proof), includes the axioms of arithmetic (the idea that the axioms of arithmetic could be false in some possible world
seems hard to buy) and is actually consistent, then by G\"odel's Second Incompleteness Theorem the system cannot prove its own consistency.
And hence it cannot prove $p$ is possible on the Leibnizian account of possibility, and thus does not make $p$ necessarily possible on that
account.\footnote{??discuss objection paper}

The second way out of modal collapse is to limit the scope of axiarchic principles to producing what one might call the best 
\textit{skeleton} for a world. Say that a skeleton for a possible world consists of all the explanatorily fundamental parts
of the world, such as the initial conditions and the laws of nature. As long as the laws of nature and/or causal powers of 
the initial beings are indeterministic, we could make only one skeleton possible, while yet having a multiplicity of possible 
worlds differing in how that skeleton evolves indeterministically into a fully fleshed out world. This will save our intuitions
about more ordinary possibilities: I might have forgotten to come to class today, you could have found my arguments more convincing
than they are, and the French could have emerged from World War II as the dominant world power. But forcing the laws of nature to 
be necessary is pretty counterintuitive. 

The third response is to allow for tied or incommensurable worlds, and say that the axiarchic principle requires \textit{a} best world,
but not \textit{the} best one. One might, if one wishes, also combine this with the skeleton move and let the principle require the world
to have \textit{an} optimal skeleton. This response would also answer our earlier objection that there is no such thing as the best world.
It is mysterious, however, how the axiarchic principle would then go about selecting which precise world or skeleton exists from among the
optimal ones. A principle is not a person who can choose between a set of incommensurable options, nor is it an indeterministic cause that
has a range of possible effects. ???

The final difficulty for axiarchic views is the problem of evil. Looking at the litany of suffering in human history, 
our world doesn't look like the best of all possible worlds. Axiarchic views can make use of many of the responses to 
the problem of evil given by theists. This vast literature is beyond the scope of this book.??refs

\subsection{Theistic choice points}
Suppose we are convinced that we need a theistic Aristotelianism.
At this point there are metaphysical and theological choice points. One metaphysical choice point is whether there are any
uninstantiated forms. If there are, as on a Platonic picture, then we have a theological question: Does God freely choose which ones
to create, or do they exist necessarily, say in the mind of God? If there are no uninstantiated forms, as on a more classically 
Aristotelian picture, then probably the most parsimonious theistic story is that God creates in the act of creating the substances 
that instantiate them. We thus have three views: Theistic Voluntarist Platonist Aristotelianism, Theistic Involuntarist Platonist Aristotelianism
and Theistic Classical Aristotelianism.

On the Platonist versions, the forms have some kind of uninstantiated mode of existence, in addition to the instantiated mode of 
existence they have in creatures. (I am assuming here that we have already decided in favor of individual forms---??backref. Perhaps,
though, on the Platonist versions that choice point should be revisited?)

The Voluntarist Platonism option may seem to have some unnecessary complexity. If God chooses which forms to create, it is 
puzzling why God would ``bother'' with the ones that aren't going to get instantiated. There is, however, a possible answer:
to open a field of possibilities to creatures. Perhaps the forms need to have some kind of Platonic existence in order for
creatures to have the power of producing their instantiations. There is a value to the earth ecosystem ``having a choice'', with
many evolutionary possibilities of what kinds of biological substances should exist, and on a more Platonic version of the 
metaphysics this could require the pre-existence of these forms in their uninstantiated mode. 

On both the Voluntarist Platonist and Classical versions, there is or can be a field of possibilities for other forms than 
the ones that actually exist. On the Voluntarist Platonist version, these are other forms that God could have created
\textit{ex nihilo} independently of instantiation.  On the Classical version, these are other forms that God could have 
instantiated and thereby brought not existence. Presumably, God knows what these possibilities are, and so they have some 
kind of existence as ideas in the mind of God. There is much room here for difficult metaphysical exploration of the 
exact status of these divine ideas.???many-refs Nonetheless, this point shows that there is a commonality between all three
versions of theistic Aristotelianism: there is a field of formal possibilities. On the Voluntarist Platonist and Classical
theories, this is a field of divine ideas. On the Involuntarist Platonism, this is a field of necessarily existing forms.

On all three views, God selects from that field of possibilities some forms that will be instantiated, and maybe also some 
forms that creatures can on their own cause to be instantiated. There are significant metaphysical differences between the
views, but all three involve a similar kind of divine selection model of which forms are instantiated.

\subsection{Participation}
\subsubsection{The account}
But there is also a different way that we could have a theistic explanation of normative features of forms. Classical theism
holds that all things are either God or participate in God. In such a setting, it is natural to think of a form as a way for 
a being to participate in God. But now while God's infinity and otherness may give a wide scope to what sorts of
arrangements of features could count as a participation, that scope is plausibly narrower than all logically non-contradictory
arrangements of features. Some candidate norms, like a requirement of causing gratuitous pain to others, just may not be included 
in any metaphysical possible way of participating in a perfectly good God. And some combinations of individually admissible features may also 
fail to be found in any metaphysically possible mode of participation in a perfectly unified God, such as having conversation with 
conspecifics as central for one's good while having the essential causal power of deterministically exploding whenever one approaches a 
conspecific within talking distance. 

Such a theistic participatory limitation on forms yields a more metaphysical explanation of some aspects of Aristotelian optimistic harmony than 
divine selection does. Moreover, this mode of explanation lends itself more easily to supernaturalist stories other than theism, 
such pantheism or a classical Platonism centered on the Form of the Good. 

Nonetheless, a mere limitation on the space of possibilities for forms is insufficient for explaining all the aspects of Aristotelian 
optimism. First, a limitation of forms by itself does nothing to rule out the possibility of a form being always instantiated in beings 
that happen to inhabit an environment completely unsuitable for flourishing according to the norm. 

Second, unless we think the limitation is really severe, our explanations of norms will be curtailed. For what we will be able to explain
is why the complex of norms is minimally acceptable---such as to be minimally capable of participating in God (or the Form of the Good,
on a classically Platonic version). But the limitation won't explain cases where norms fit particularly well together, since if they 
fit less well, the norms could still be found in some possible form. For instance, in humans living by the moral norms is central to  
flourishing. This ensures that any human that lives by the moral norms automatically has quite a bit of flourishing, and one who 
does not live by the moral norms cannot be said to flourish overall. Plausibly, a much weaker degree of unity between overall flourishing
and the norms governing the will would suffice for a form of a being that participates in God: living by moral norms could be a less
central aspect of flourishing. A divine selection explanation can advert to God's having a good reason to produce beings with the greater 
degree of unity in their normative features, and thereby explain the higher degree of unity, while a participatory limitation explanation
would only explain why the degree of integration is at least minimal.

\subsubsection{An objection}\label{sec:limited-pl}
There is, however, a serious problem with the participatory limitation account. One of the main things that led us to grounding ethics
in human form was the appearance of contingency implied by the vast number of seemingly arbitrary parameters in ethics. But does not 
the same thing difficulty apply to the participatory limitation account? For there seem to be parameters defining the boundaries between
participatable combinations of normative features and unparticipatable ones. Just how much unity between the norms is needed in a form
that participates in a God who is one? How much normative egoism can be found in a form that participates in a perfectly loving God?

We might try to shift the difficulty onto parameters in the divine nature which determine what is a possible participator in God.
These parameters can be necessary, since God is normally thought of as a necessary being. 

However, there are shortcomings of this approach. First, it becomes puzzling why we don't simply do a similar thing for the Mersenne questions that 
led us to forms. We could suppose, for instance, a Platonic account on which there is a vast number of metaphysically necessary ethical ur-norms
with metaphysically necessary but still seemingly arbitrary parameters, instead of norms found in natures of particular types of 
rational beings. Given the plausibility that different possible intelligent species would have different parameters in their norms, 
the ur-norms would presumably include many conditional ones connecting the non-normative features of a species with the norms governing
thier behavior: ``If you have such-and-such genetics, then you should prefer parents to strangers to degree $x$.'' 

One response is that shifting the difficulty from parameters found in human ethics to parameters in divine ethics is still philosophically
advantageous. First, as discussed in ??backref-and-add, there is an advantage in norms that are grounded in a nature, whether divine or human, 
in that these norms are not an alien imposition of dubious binding power, but are the requirements of one's very own nature. Second, there may
be a greater unity in a complex set of norms governing a single divine being than in abstract Platonic norms governing all possible beings
in conditional form. Third, there is some hope that the parameters governing divine nature are fewer and more unified than any plausible set 
of parameters governing humans. 

On the other hand, the idea of arbitrary parameters in God is theologically and philosophical unattractive. Such parameters fit poorly 
with classical theism's doctrine of divine simplicity. Moreover, it is the lack of arbitrary parameters that makes God an attractive 
explanatory posit. If God is to have a vast number of arbitrary parameters, is it not just as simple to explain things in terms of 
a brute necessity of a Big Bang, or of a particular set of necessarily selected forms? 

Another response would be to hope that a participatory limitation account does not involve parameters. Perhaps there is no degreed
limit on norms found in a form that participates in a perfect God, but the rather there are sharp non-arbitrary limits: 
nothing contrary to divine nature, such as hatred of persons or cruelty or injustice, can be required. If we make this move, we 
won't be able to account for all of Aristotelian optimism using participatory limitations. For optimism involves more than just belief
in the barest minimum of positivity. The optimism that is essential to give us epistemic access to the norms under the Aristotelian 
synthesis requires a significant degree of union in the form, not just the barest minimum.

One might worry that a similar difficulty applies to the divine choice account. God is more likely to actualize a more unified 
form. But what are the parameters in the function that governs the relationship between the degree of unity in a form and the chance that 
God would actualize that form? We might again insist that parameters in the divine nature are not particularly problematic.

But there is another move. The idea that there are numerical chances assigned to divine actions is itself theologically and philosophically
problematic. First, numerical chances seem to be a kind of limitation on divine power.??refs:Murphy??? Second, the idea of assigning 
numerical chances to the vast infinity of possible divine action. This infinity exceeds any infinite cardinality, since for any
cardinal number $\kappa$, God could create precisely $\kappa$ angels, while the class of cardinal numbers exceeds any particular cardinal 
in its size??refs, and so the class of divine actions creating groups of angels is beyond cardinality. Assigning numerical probabilities 
in such a setting is fraught with mathematical difficulty. 

Instead of supposing numerical chances assigned to divine actions, we might simply suppose that there is a non-numerical qualitative 
rule: what better matches the divine nature is more likely to be instantiated. This rule need not even impose a total ordering on the
space of possible divine creative actions, because there may be vast scope for incommensurability between divine actions.??ref:divine-creative-freedom,
so that there will be divine actions whose chances are themselves incommensurable.\footnote{$^*$What could that mean? Here is a mathematical
parable. The Banach-Tarski paradox has it that we can divide a solid mathematical ball into five disjoint pieces, $E_1,E_2,E_3,E_4,E_5$, and move these pieces to 
construct two balls of equal size to the original. Imagine randomly choosing a point in the original ball, and asking which of the
five pieces it lies in. Paradox ensues if we make too many comparisons between pieces, such as that the point is more likely to lie in 
$E_1$ or $E_2$ than in just $E_3$, or vice versa, but it is safe at least to say that it is more likely that the point is in $E_1$ or $E_2$
than in just $E_1$. However, it may be reasonable at least to say that if $A$ is a proper subset of $B$, then the point is more likely to lie in $B$ than
in $A$. There are still technical problems with this (cf.\ ??ref:Pruss-domination), even though lying in $E_1$ versus in $E_2\cup E_3$ have
incommensurable chances.} We might then further suppose that the qualitative ordering on the chances still yields epistemic probabilities for humans. For
the human form could prescribe specific numerical rations of epistemic probabilities where the objective chances (??explain chances vs probabilities earlier) 
have a merely comparative or even incommensurable relationship.

The divine selection account, thus, seems to have a greater chance of escaping the arbitrary parameter worry than the 
participatory limitation account.

??why not divine command theory?? alienness?

\subsection{A dual account}
Moreover, it is quite reasonable to combine the theistic choice and participatory limitation accounts. Given the considerations in 
Section~\ref{sec:limited-pl}, the participatory limitation account is unlikely to suffice on its own, after all.

On a combined account, some aspects of form, especially coarser-grained ones, can be explained by participatory 
limitation while others, especially finer-grainer ones, can be explained by divine selection. The result is an explanatorily rich account of 
normativity, which predicts a minimal coherence throughout one's normative complex, and leads one to expect higher degrees of unification
in more central aspects.

This expectation of unity in turn yields another tool to help us to actually find out what our norms are. We should prefer normative theories
that allow for significant integration and harmony between moral, epistemic and other norms. An integrated picture of human flourishing
is obviously attractive.

\section{Final remarks}
??explain how theism grounds the variety of harmonies discussed earlier in the chapter

\chaptertail

