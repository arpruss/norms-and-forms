\def\mychapter{X}
\ifdefined\book
\else
\documentclass[11pt,oneside]{amsbook}
\usepackage[backend=biber, citestyle=authoryear]{biblatex}
\usepackage{mathpazo}
\usepackage{graphicx}
\usepackage{amsmath}
\usepackage{tikz}
\usetikzlibrary{arrows}
%\usepackage{titlesec}
\addbibresource{bibliography.bib}
\newcommand\posscite[1]{\citeauthor{#1}'s (\citeyear{#1})}
\newcommand\plural[1]{#1\mathrm{s}}
%\def\posscitewithextra[#1]#2{\citename{#2}'s (\citeyear{#2}, #1)}

%\newcounter{subsubsubsection}[subsubsection]
%\renewcommand\thesubsubsubsection{\thesubsubsection.\arabic{subsubsubsection}}
%\titleformat{\subsubsubsection}
%  {\normalfont\normalsize\bfseries}{\thesubsubsubsection}{1em}{}
%\titlespacing*{\subsubsubsection}
%{0pt}{3.25ex plus 1ex minus .2ex}{1.5ex plus .2ex}

\ifdefined\book
\renewcommand{\thechapter}{\Roman{chapter}}
\else
\renewcommand{\thechapter}{\mychapter}
\fi

\linespread{1.7}
\usepackage[margin=1.25in]{geometry}
\sloppy
\makeatletter
%% TODO: This is a cheat. It's supposed to be {paragraph}{4}, and that's 
%% what it is in amsbook.cls, but then it fails.
\def\paragraph{\@startsection{paragraph}{3}%
  \normalparindent\z@{-\fontdimen2\font}%
  \normalfont}
\def\subsubsubsection{\paragraph}
\makeatother

\def\smalltick{0.15cm}
\def\bigtick{0.3cm}
\def\pointcircle{0.08cm}
\def\causalnode{0.35cm}

\hyphenation{dia-chro-nic}

%\usepackage[utf8]{inputenc} % set input encoding (not needed with XeLaTeX)
\usepackage{amssymb}
\usepackage{mathtools}
\usepackage{enumitem}
\usepackage{amsthm}
\usepackage{physics}
%\usepackage{ntheorem}
\usepackage{chngcntr}
\counterwithin{figure}{section}

\makeatletter
% \def\@endtheorem{\endtrivlist\@endpefalse }% OLD
\def\@endtheorem{\endtrivlist}%

\setlist[description]{font=\normalfont\scshape}

\catcode`\|=\active\def|{\mid}
\DeclarePairedDelimiter{\ceil}{\lceil}{\rceil}
\DeclarePairedDelimiter{\floor}{\lfloor}{\rfloor}
\newcommand{\Subj}{\mathbin{\raisebox{.15ex}{$\scriptscriptstyle{\Box}$}\kern-.425em\rightarrow}}
\def\Existence{E!}
\def\Believes{\operatorname{Believes}}
\def\True{\operatorname{True}}
\def\Perfection{\operatorname{Perfection}}
\def\ext{\operatorname{Ext}}
\def\Iff{\leftrightarrow}
\def\Implies{\rightarrow}
\def\Entails{\Rightarrow}
\def\Cov{\operatorname{Cov}}
\def\Equiv{\Leftrightarrow}
\def\Form{\operatorname{Form}}
\def\Informs{\operatorname{Informs}}
\def\technical{$\star$}
\def\vtechnical{$\star\star$}
\def\power{\wp}
\def\Nec{\Box}
\def\Poss{\Diamond}
\def\Prop#1{$\langle$#1$\rangle$}
\def\R{\mathbb R}
\def\N{\mathbb N}
\def\tele{tel\={e}}
\makeatletter
\newtheoremstyle{indented}{3pt}{3pt}{\addtolength{\leftskip}{4.5em}}{-2.5em}{\sc}{.}{.5em}{}
\def\Principle#1#2#3{\theoremstyle{indented}\newtheorem*{principle}{#2}\begin{principle}\def\@currentlabel{#2}\label{#1}#3\end{principle}\let\principle\undefined}
\makeatother
\def\pref#1{{\sc\ref{#1}}}
\def\enum#1{\resume{enumerate}\item #1\end{enumerate}}
\def\ditem#1#2{\begin{enumerate}[resume]\item \label{\mychapter:#1} #2\end{enumerate}}
\def\nitem#1#2{\begin{description}\item[#1\label{\mychapter:#1}] #2\end{description}}
\def\bref#1{\ref{\mychapter:#1}}
\def\dref#1{(\ref{\mychapter:#1})}
\def\drefglobal#1{(\ref{#1})}
\usepackage{graphicx} % support the \includegraphics command and options
\usepackage{array} % for better arrays (eg matrices) in maths
\def\Not{\operatorname{\sim}}
\def\St{\operatorname{St}}
\def\num{\operatorname{num}}
\def\And{\mathrel{\&}}
\def\Or{\vee}
\def\BigOr{\bigvee}
\def\<{\langle}
\def\>{\rangle}
\def\union{\cup}
\def\nleq{\not\le}
\def\N{\mathbb N}
\def\R{\mathbb R}
\def\C{\mathbb C}
\def\Powerset{\mathcal P}
\def\star#1{{}^*#1}
\def\hN{\star{\N}}
\def\hR{\star{\R}}
\def\Z{\mathbb Z}
\def\Power{\mathcal P}
\def\Implies{\rightarrow}
\def\True{\operatorname{True}}
\def\Socrates{\mathrm{Socrates}}
\def\actual{@}
\def\Law{\operatorname{Law}}
\def\Chance{\operatorname{Chance}}
\def\Var{\operatorname{Var}}

\def\H2O{H${}_2$O}

\def\scr{\mathcal}
\def\e{\varepsilon}
\def\eps{\varepsilon}
\newtheorem{lem}{Lemma}
\newtheorem{prp}{Proposition}
\newtheorem*{theorem}{Theorem}
\newtheorem{corollary}{Corollary}
\newtheorem{cond}{Condition}

\renewcommand\thechapter{\Roman{chapter}}

\def\chaptertail{\ifdefined\book\else\end{document}\fi}
 

\title{Infinity, Causation and Paradox}
\author{Alexander R. Pruss}
%\date{} % Activate to display a given date or no date (if empty),
         % otherwise the current date is printed

\begin{document}
\setcounter{secnumdepth}{3}
\setcounter{tocdepth}{4}

\end{document}
\fi

\restartlist{enumerate}

\chapter{Evolution, Harmony and God}\label{ch:God}
\section{The origin of the forms}
\subsection{Evolution and forms}
We have good empirical reasons to think that the variety of biological structures that fills our planet 
is largely or completely the product of unguided variation together with natural selection. However, as
I have argued, there are good philosophical reasons to think that the organisms with these structures
have normatively laden forms which specify how the organisms should behave, endow them with the causal
powers that make that behavior possible, and impel them towards that behavior. 

It is implausible to think that the forms supervene on the biological structures. For instance, one theory
of the evolution of wings for gliding is that small wings are useful for heat dissipation. Larger wings allow
for more dissipation of heat, but are also more expensive for the organism to maintain. However, at around
size at which the heat-dissipation benefits are outweighed by the maintenance costs, the wings also become
useful for gliding. It is plausible that a species $A$ that has the smaller wings has them with the telos of
heat dissipation. But a species $B$ that has evolved the larger wings has them with the telos of gliding, either
instead of or in addition to heat dissipation.??ref,check But we can now suppose a member of $B$ whose wings are defective
and only good for heat dissipation. Such a member's biological structure might be largely indistinguishable from
that of a normal member of $A$, and yet it is normatively different: such wings are defective in $B$ but entirely
appropriate in $A$. If these norms are grounded in forms, it seems there is a different form in members of $B$ than
of $A$.

In general, in the evolutionary process, we expect small transitions in genetically-based biological structure 
between parents and children, with no change between the parent's form and the child's form. For if we had constant change
between the parent's form and the child's form, our best account would be that the form simply matches the
biological structure, which would not allow for genetic defects, and yet genetic defects---deviations of genetically-based
biological structure from the kind norms---are clearly possible.  Moreover, it is important to our ethics
that all human beings, despite a wide variety in physical and mental endowments---including the striking biological
difference between male and female---are beings of the same kind. 

We thus need an explanation of why it is that at certain apparently relatively rare and discrete points in the evolutionary 
sequence we have a new form on the scene. This itself yields Mersenne questions: while some transitions of form might happen
to coincide with a particularly striking genetic transition, we expect a number of them to come along with only minor
genetic transitions, seemingly at arbitrary positions. What explains these transition points?

Hitherto in this book, such questions were answered by invoking the forms themselves. And this can be done in this case
as well. We might suppose that the form of species $A$ endows the members of $A$ with a causal power to generate new
members of $A$ in some circumstances, together with new instances of the form of $A$, but also a causal power to generate
new members of $B$ in other circumstances, along with new instances of the $B$ form. The difference in circumstances could be
determined by the DNA content in the gametes joining together, so that when a descendant is going to have such-and-such DNA 
contents, the descendant gets the form of $A$, but with other DNA contents, the descendant gets the form of $B$. 

This story requires the forms to contain intricate specifications of which form is generated when. Granted, the slew 
of Mersenne questions we have already raised should make us circumspect about balk at mere complexity of form.
But now observe that the story as given above requires that the first biological organism on earth---presumably
some simple unicellular or maybe even proto-cellular?? organism---contain within it a form that codes for the causal
power to produce forms of all possible immediate descendants of it. These immediate descendant forms then would have 
to code for the causal power to produce all their possible immediate descendants, and so on. Thus, the
form of the first and simplest organism would implicitly code for all the forms of life that would ever actually be
found on earth, and indeed all the forms of life that \textit{could} ever descend from it.\footnote{It is tempting to
say that the number of possible descendant forms is infinite, but that is not clear. After all, there could be some
physical limit to the size of the genetic code fo a biological organism given our laws of nature. But in any case,
finite or not, the number of possible descendant forms is incredibly large.} We thus have here a dizzying complexity.

???few species story!??? no help, still have complexity

But the problem does not stop here. For we can now ask where that immensely sophisticated form of the first organism comes
from? If we say that it comes from the causal powers of non-living substances, such as fields or fundamental particles,
then we have to posit an even greater complexity in the forms of these non-living substances. The result would be highly
counterintuitive, by supposing non-living things to have immense sophistication of form. Further, however, we would need a 
story of where the first forms arose from. If we take the above account to its logical conclusion, then at the Big Bang
we would already have particles or fields whose forms implicitly included the vast formal complexity of all physically
possible living organisms. And this in turn yields a powerful design argument. For the idea that such complexity would
simply come about for no reason at all is utterly implausible. 

Thus, the story that forms contain the rules for the generation of future forms points towards a being whose own power is
sufficiently great to generate such forms. And to avoid a vicious regress, such a being would need to be a necessary one.

But note that once we have accepted the existence of a necessary being that is the ultimate source of the varied forms 
in our world, we can now tweak the story to avoid the implausible idea that unicellular organisms implicitly code for
the forms of elephants and unicorns. Instead of supposing that the transitions between forms corresponding to certain
selected changes in genetic structure are caused by the parent forms, we can suppose that the necessary being is directly
responsible for the transitions of forms. On such a view, the form of a unicellular organism might only endow its
possessor with the ability to generate a descendant of the same kind, and the necessary being would directly produce
any new forms when it is appropriate to do so.

\subsection{Reasons for creating forms}
Of course, this would lead to the question of \textit{why} the necessary being produces new forms when it does so.
Here, taking the necessary being to be rational can help. For there can be good value-based reasons for the transitions
to fall in some places rather than others. 

Consider, first, an odd thought experiment. A horse-like animal comes into existence with an maximally flexible form such that
whatever the animal does fulfills the norms in the form. To eat and grow is one proper function, and to starve and produce
a corpse is just as proper a function. Whatever our ``flexihorse'' does or undergoes is equally good for it. But there is something unsatisfactory about the flexihorse as a creature. If whatever the flexihorse does is equally good for it, then the fact 
that the flexihorse flourishes is just a direct and trivial consequence of its externally imposed form rather than the individual's 
\textit{accomplishment}. 

Reflection on this suggests there is a value in creating organisms that can fail to fulfill their norms. This value might be
grounded in the forms themselves: it might be that real horses, unlike flexihorses, have self-achievement of flourishing among the
proper functions in their form. And there is a value in creating organisms that have additional types of good written into their
form, including such self-achievement. Alternately, one might hold that in addition to kind-relative goods, there may be
kind-independent goods---perhaps grounded in imitation of the creator??forwardref?---and self-achievement of flourishing
could be one of these.

Either way, a rational being creating organisms has reason to create organisms that can fail to achieve their form, and hence
has reason to create beings with less flexible norms. Moreover, there appears to be a comprehensible value---again, either 
kind-relative or kind-independent---in production of beings of the same kind. As an intuition pump here, think of the
\textit{Symposium}'s idea that the yearning for eternity is exemplified in animal reproduction. Thus, we can give a value-based
explanation for why a necessary being would create beings in discrete kinds, with norms that the beings need not live up to.

\section{Explaining harmony by natures and evolution}
\footnote{This section owes much to discussion in my mid-sized objects seminar, and especially to Christopher Tomaszewski's suggestions on the explanatory powers of forms.}
\subsection{Number of natures}
\subsection{Nomic coordination}
\subsection{Aristotelian optimism revisited}
\subsection{Fit to DNA}
\subsection{Fit to niche}
\subsection{Nature zombies}
\subsection{Exoethics}
\subsection{Aquinas' Fourth Way and the good}
Aquinas' Fourth Way??ref puzzles the modern reader. It begins with a principle that comparisons between
degreed properties are grounded in a comparison to a maximal case: one is more $F$ when one is more like
the item that is maximally $F$. Aquinas then illustrates the principle with the case of heat and fire:
an object is hotter provided that it is more akin to the hottest thing, namely fire. He then applies
the principle to goodness, and concludes that there is a best thing, and this is God.

The fire illustration is not just unhelpful to us, since we know that fire is not the hottest thing (the sun is almost
twice as hot as the hottest flame), but it is actually a conclusive counterexample to the degree property principle,
since we can easily compare temperatures without reference to an alleged hottest object.\footnote{In any finite universe,
presumably there will be a hottest object. However, temperature comparison is not defined by that object, since 
even if Bob is in fact the hottest object, we would expect it to be physically possible to have a hotter object 
than Bob. But if degrees of heat were defined by closeness to Bob, it would not be possible to be hotter than
Bob, since nothing can be closer to Bob than Bob.}

So Aquinas' comparison principle is false. But I contend that there is still something to his argument
when applied to the good. 

Now, a form-based metaphysics gives a powerful account of the good for a being of
a particular kind---an oak, a sheep or a human, say---in terms of its match to the specifications of the
form. It also gives a ground to comparisons between the good of different instances of the same kind:
a four-legged sheep is, other things equal, better at sheepness than a three-legged sheep, because it
more completely fulfills the specification in their ovine nature. In fact, this is itself a counterexample 
to Aquinas' comparison principle, in that we can compare degrees of success at sheepness without supposing
any individual sheep to be perfect.

However, in addition to value comparisons within a kind, there are ones between kinds. When Jesus says
that we are ``worth more than many sparrows'' (Mt.\ 10:31??ref), what he says is quite uncontroversial.
Indeed, even a perfect sparrow seems to have less good than a typical human. While the nature of a sparrow
will enable value comparisons between sparrows, and that of a human between humans, we still have the question
of what grounds the value the difference between sparrows and humans. Some Aristotelians reject cross-kind 
value comparisons as nonsense.??refs But given the intuitive plausibility of many such comparisons, this rejection
is a costly one.

Aquinas' Fourth Way is not infrequently seen as more Platonic than his other arguments for the existence of
God, and Plato indeed had a solution to the problem of cross-kind comparisons, by talking of differing degrees
of imitation of the Form of the Good, which itself is perectly good. Plato, on the other hand, lacked a 
satisfactory solution to the problem of intra-kind comparisons. He may well have thought that there 
was a Form of Humanity??refs, which exemplified humanity perfectly, so that similarity to the Form of 
Humanity would define how good one is at being human. However,
we can see that this solution is clearly unsatisfactory. First, the Forms are immaterial, so the Form of Humanity is 
immaterial, and hence it lacks fingers. Thus, the fewer fingers a human has, the more they are like the Form of
Humanity, and hence, absurdly, the more perfect they are. Second, if somehow the Form of Humanity ends up having 
body parts, then the Form of Humanity either has an even number of cells or an odd one. But clearly neither option
is more perfect than the other. 

Central to Plato's solution to cross-kind value comparisons is the self-exemplification of the Form of the Good:
the Form of the Good is itself maximally good. But a similar self-exemplifying Form cannot be used to account for
intra-kind comparisons. Aristotle, on the other hand, has the non-self-exemplifying forms immanent in things. 
The Aristotelian form of humanity specifies human perfection, but does not do so by exemplifying it. It has neither
fingers nor cells, but it \textit{specifies} that humans should have ten fingers while specifying an age-dependent
normal range of cell numbers rather than a specific cell count.

Notwithstanding the general falsity of Aquinas' comparison principle for degreed properties, Aquinas provides us
with a plausible extension of the Aristotelian system to allow for comparisons of degrees of good between objects
of different kinds in terms of the similarity to or degree of participation in a maximally good being, a divine being
that plays the role of a self-exemplifying Form of the Good. The human being participates in God in respect of
abstract intellectual activity, Aquinas will contend, while sparrows do not, and in that important respect, at least,
humans are more like God. On the other hand, the sparrow's movements approximate divine omnipresence better than 
the stillness of a mushroom does, and in that respect at least the sparrow is superior to the mushroom. We have,
thus, a ground for something like a great chain of being.

There are still difficulties here. While the human is superior in intellectual activity, the sparrow moves around
with greater three-dimensional freedom. How can we say that the human is superior all things considered? Where we
previously had a problem of cross-kind comparisons, we now have the problem of cross-attribute comparisons. 
Intuitively, the human's intellectual superiority to the sparrow trumps the sparrows motive superiority to the
human, and enables us to say that the human is more perfect on the whole. This higher level question is difficult
indeed. 

But there is some hope in thinking that in attributing different divine attributes we sometimes express divinity
to different degrees. It may be that there is no meaningful comparison between how well we express divinity by
saying that God is all-knowing versus by saying that God is all-powerful, saying that God knows the
multiplication table up to $10\times 10$ expresses divinity less well than saying that God can create any
possible physical reality. I suggested earlier that motion imitates divine omnipresence. Thus, the sparrow's
ability to fly imitates God's presence throughout several kilometers surrounding the surface of the earth, while
the human's more limited mobility imitates God's presence in a thin two meter shell of air surrounding that surface.
But the degreed difference between the two divine attributes---each a limited special case of omnipresence---imitated here 
might well be trumped by the fact that the sparrow does not imitate God's abstract intellectual activity \textit{at all}
while the human does imitate that activity, and does so in respect of a very wide scope of things (the human can think
abstractly about the whole universe, for instance). 

In ??backref, we gave a non-theistic Aristotelian sketch of a three-step great chain of being. The account here has
a hope of allowing one to fill in more intermediate links.
Even if the details in the comparisons between different attributes or respects do not work out, we still have an advantage 
for the theistic Aristotelian in being able to make cross-kind comparisons under specific respects, like motility or intelligence.

??ref:Jeffrey/Ward

\subsection{The complexity objection}
A central form of argument for Aristotelianism is based on Mersenne questions and the messy normative complexity of our
lives. But do we not normally prefer simpler theories to more complex ones, and hence should we not reject the normative
complexity in favor of a simpler theory like utilitarianism in the name of Ockham's razor?

Ockham's razor, however, has always been a defeasible criterion: entities are not to be multiplied \textit{beyond necessity}.
But sometimes there is necessity. It would be simple to suppose that all trees of a single species look exactly the same.
But that just wouldn't fit with our evidence. In biology, one does not expect individuals of the same sort to be exactly
alike, unlike in fundamental particle physics.???

\subsection{Epistemology of normativity and form}\label{ch:epist-of-form}
[Argument: If a guided missile has form, it's alive by the Ch?? account of life. But it's not alive. So it lacks form. Is this a bad argument???]
\subsection{Ethics and happiness}

\subsection{Norms that fit with modern technology and any real but outlandish scenarios}
In ??backref, I argued that an ethics based on human form can simply ignore outlandish scenarios
that are far outside of our ecological niche, such as ones involving infinite numbers of
beneficiaries. However, there is a danger in this line of reasoning. As Arthur C. Clarke famously
said, ``Any sufficiently advanced technology is indistinguishable from magic.''??ref To human beings
50,000 years ago (or even just 500 years ago!) much of our technology would indeed be magical, and 
decisions that we routinely need to make, say in bioethics, would be predicated on outlandish assumptions. 

We might thus expect an ethics and epistemology grounded in a form possessed by hunter-gatherer primates
to be silent on dilemmas of a highly technological society, leaving us to do whatever we wish, or, even worse, 
to fail to harmonize with the shape of our lives, like that of a fish on land. Yet while there are, as there 
have always been, difficult and controversial moral and epistemological cases, we do not in fact find 
ourselves adrift without guidance in the modern world. Virtue continues to contribute to our flourishing,
and ancient texts, whether religious or philosophical, continue to point to good ways of living. 

This gives us reason to think that if our moral norms are grounded in human nature, human nature was somehow
picked out with foresight for what kinds of challenges humans would face in the distant future. Our ethics
does may not work in outlandish situations, such as those involving infinities as noted in ??backref, but it works in a
broader range of moral environments much broader than that found in early homo sapiens society. Thus, the theistic
version of our natural law theory both accounts for the apparent unsatisfactoriness of our ethics in situations that
humans apparently never find themselves with and the applicability in situations across a very wide range of situations,
wider and more technologically varied than the natural environment of other animals. This kind of foresight points to
a foreseer, indeed a designer, and hence towards a theistic version.

The move I suggested in ??backref for outlandish scenarios, namely that our ethics and epistemology simply does not apply
to them, may be problematic, however. For \textit{some} seemingly outlandish scenarios could turn out to be real. Many religious people think that some or all
individual humans will live forever. And many people, religious or not, think there is a serious possibility that human beings will
spread through the galaxy, affecting vast numbers of lives, which may make actual seemingly outlandish questions about where 
our actions have very slight probabilistic effects on vast outcomes.??ref:Fanaticism  Here, a theistic story might also
be attractive. God can know what kinds of seemingly outlandish scenarios are actually relevant to the lives of his creatures,
and can wisely choose the forms whose norms that fit with these. The resulting norms may seem \textit{ad hoc} at times: they
won't be the elegant principles of classic utilitarianism, for instance, but those kinds of principles face great difficulties
in outlandish scenarios. But a wise rule does make judgments that can be \textit{ad hoc}.


\subsection{Avoiding radical scepticism}
%% The threat of saturated nonmeasurable sets and brute facts can be avoided if we OUGHT to have low priors for them.
%% But being such that we ought to have low priors for something that is a not unlikely possibility given the causal structure
%% of the world undercuts knowledge. So we need a cooperative causal structure of the world.
\section{Explaining harmony theistically}
\section{Explanations of moral norms}
\subsection{A pattern of ethical explanation}
Here is a familiar pattern. We have a deeply-seated moral intuition about the general prohibition, call it $g$, of some action,
such as incest. It is not clear how to derive the prohibition from intuitively more basic principles, such as one of 
the categorical imperatives. Easy considerations, which I will call the $c$s, show that in \textit{typical} cases the action is wrong,
but our moral intuition goes beyond these typical cases. Thus, considerations of the abuse of power, distortion of 
familial dynamics, and genetic harms show that most cases of incest are wrong, but it is easy to imagine cases
of incest to which these considerations do not apply---say, elderly siblings who were raised apart---and yet moral
intuition forbids incest in those cases as well.

We can now save the moral intuition by saying that the more general prohibition $g$ is simply a fundamental moral rule,
not reducible to the $c$s that explain why the action is wrong in typical cases. But if we stop
at this, the connection between $g$ and the $c$s mere happenstance, and that seems intuitively wrong. The abuse of 
power, distortion of family dynamics, and genetic harms should be relevant to why incest is wrong.

At this point, often we are in a position to see another fact: it is quite beneficial to have a 
general moral prohibition beyond the prohibitions arising from the $c$s. 

One reason for such a benefit from a general prohition could be that our judgment as to whether the $c$s apply to a given case is fallible, especially given our capacities
for self-deceit, and the costs of violating the $c$s are so high that it would be better for us to have a 
general prohibition than to try to judge things on a case-by-case basis. 

Second, in some examples of
the pattern, serious deliberation about the forbidden action can itself harm one or more of the goods
involved in the $c$s: thus, having to weigh whether the distortion-of-family-dynamics consideration
applies against a particular instance of incest can itself distort the agent's participation in family
dynamics. 

Third, we could have a tragedy of the commons situation. It could be that the $c$s are actually insufficient
to render an instance of the action wrong, but we would be better off as a society if we had general
abstention from the action. Thus, perhaps, the genetic harm coming from one more couple's engagement in incest
would be insufficiently significant to render the incest wrong, but without a general prohibition, incest
would be sufficiently widespread as to cause serious social problems. A general prohibition that is not
logically dependent on the $c$s would help avert such social harms.

These considerations are very familiar to us in the case of positive law. Jaywalking involves harms such as
disruption of traffic flow and the danger of death of the pedestrian and of trauma to the driver, and the
considerations of these make jaywalking wrong in typical cases. There are 
obvious instances, however, where these considerations do not apply: say, crossing a road where the pedestrian
can clearly see that there are no intersections or cars on the road for a significant distance in either 
direction. However, it may be better for people simply to abstain from jaywalking than judging whether the 
disruption and safety considerations apply on a case-by-case basis, because there could be so much harm if 
the judgment were to go wrong. As a result, it is can be reasonable for a state simply to ban jaywalking
altogether (or to ban it with some clear and easily adjudicated exceptions). We similarly resolve cases of
tragedy of the commons with positive law: think, for instance, about laws against littering.

In the case of positive law we have two different explanations. First, there is an explanation of why
the forbidden action is wrong in general: this is because it has been competently forbidden by legitimate authority.
This explanation need not make reference to considerations such as disruption of traffic flow or danger
of death.\footnote{Though in some cases \textit{some} such reference may be needed in order to establish
that the matter falls within the competence of the authority in question. Thus, a government agency may
be permitted to make rules on matters where traffic flow disruption is concerned.} Second, there is an 
explanation as to why the action has been forbidden by the authority---and here all the rich considerations
are relevant.

A theistic version of natural law can have precisely this pattern. An action is morally forbidden because
our nature is opposed to it. This explanatory fact does not make reference to the $c$s. But we still
have a further question to ask. The most obvious way to ask the question is to query why our nature 
includes this prohibition. But since our nature is essential to us, the answer to that question could 
simply be the necessary truth that we couldn't exist without this nature. However, we can put the question
in a slightly different way: Why are there intelligent primates on earth with a nature that includes this prohibition
rather than some other kind of intelligent primates with a nature that does not include this prohibition?
And here the theist can answer: Because it would be good, in light of the $c$s and the
further considerations in favor of generalizing the prohibition beyond the cases where the $c$s specifically
apply, to have intelligent primates with a nature that includes this prohibition, and God acted in light
of this good.\footnote{A divine command theorist can make the same move, but divine command theory has some
liabilities which were discussed in ??backref.}

\subsection{Global aesthetic-like features}\footnote{I am grateful to Nicholas Breiner for drawing my attention, in the context of
justice, to this form of explanation of moral features.}
\subsection{Family}
\subsection{Retributive justice}
\subsection{Divine authority}
\section{Kind-independent goods}
Diversity, flourishing, self-achievement, etc....??? Imitation?

\chaptertail

\def\mychapter{XI}
\ifdefined\book
\else
\documentclass[11pt,oneside]{amsbook}
\usepackage[backend=biber, citestyle=authoryear]{biblatex}
\usepackage{mathpazo}
\usepackage{graphicx}
\usepackage{amsmath}
\usepackage{tikz}
\usetikzlibrary{arrows}
%\usepackage{titlesec}
\addbibresource{bibliography.bib}
\newcommand\posscite[1]{\citeauthor{#1}'s (\citeyear{#1})}
\newcommand\plural[1]{#1\mathrm{s}}
%\def\posscitewithextra[#1]#2{\citename{#2}'s (\citeyear{#2}, #1)}

%\newcounter{subsubsubsection}[subsubsection]
%\renewcommand\thesubsubsubsection{\thesubsubsection.\arabic{subsubsubsection}}
%\titleformat{\subsubsubsection}
%  {\normalfont\normalsize\bfseries}{\thesubsubsubsection}{1em}{}
%\titlespacing*{\subsubsubsection}
%{0pt}{3.25ex plus 1ex minus .2ex}{1.5ex plus .2ex}

\ifdefined\book
\renewcommand{\thechapter}{\Roman{chapter}}
\else
\renewcommand{\thechapter}{\mychapter}
\fi

\linespread{1.7}
\usepackage[margin=1.25in]{geometry}
\sloppy
\makeatletter
%% TODO: This is a cheat. It's supposed to be {paragraph}{4}, and that's 
%% what it is in amsbook.cls, but then it fails.
\def\paragraph{\@startsection{paragraph}{3}%
  \normalparindent\z@{-\fontdimen2\font}%
  \normalfont}
\def\subsubsubsection{\paragraph}
\makeatother

\def\smalltick{0.15cm}
\def\bigtick{0.3cm}
\def\pointcircle{0.08cm}
\def\causalnode{0.35cm}

\hyphenation{dia-chro-nic}

%\usepackage[utf8]{inputenc} % set input encoding (not needed with XeLaTeX)
\usepackage{amssymb}
\usepackage{mathtools}
\usepackage{enumitem}
\usepackage{amsthm}
\usepackage{physics}
%\usepackage{ntheorem}
\usepackage{chngcntr}
\counterwithin{figure}{section}

\makeatletter
% \def\@endtheorem{\endtrivlist\@endpefalse }% OLD
\def\@endtheorem{\endtrivlist}%

\setlist[description]{font=\normalfont\scshape}

\catcode`\|=\active\def|{\mid}
\DeclarePairedDelimiter{\ceil}{\lceil}{\rceil}
\DeclarePairedDelimiter{\floor}{\lfloor}{\rfloor}
\newcommand{\Subj}{\mathbin{\raisebox{.15ex}{$\scriptscriptstyle{\Box}$}\kern-.425em\rightarrow}}
\def\Existence{E!}
\def\Believes{\operatorname{Believes}}
\def\True{\operatorname{True}}
\def\Perfection{\operatorname{Perfection}}
\def\ext{\operatorname{Ext}}
\def\Iff{\leftrightarrow}
\def\Implies{\rightarrow}
\def\Entails{\Rightarrow}
\def\Cov{\operatorname{Cov}}
\def\Equiv{\Leftrightarrow}
\def\Form{\operatorname{Form}}
\def\Informs{\operatorname{Informs}}
\def\technical{$\star$}
\def\vtechnical{$\star\star$}
\def\power{\wp}
\def\Nec{\Box}
\def\Poss{\Diamond}
\def\Prop#1{$\langle$#1$\rangle$}
\def\R{\mathbb R}
\def\N{\mathbb N}
\def\tele{tel\={e}}
\makeatletter
\newtheoremstyle{indented}{3pt}{3pt}{\addtolength{\leftskip}{4.5em}}{-2.5em}{\sc}{.}{.5em}{}
\def\Principle#1#2#3{\theoremstyle{indented}\newtheorem*{principle}{#2}\begin{principle}\def\@currentlabel{#2}\label{#1}#3\end{principle}\let\principle\undefined}
\makeatother
\def\pref#1{{\sc\ref{#1}}}
\def\enum#1{\resume{enumerate}\item #1\end{enumerate}}
\def\ditem#1#2{\begin{enumerate}[resume]\item \label{\mychapter:#1} #2\end{enumerate}}
\def\nitem#1#2{\begin{description}\item[#1\label{\mychapter:#1}] #2\end{description}}
\def\bref#1{\ref{\mychapter:#1}}
\def\dref#1{(\ref{\mychapter:#1})}
\def\drefglobal#1{(\ref{#1})}
\usepackage{graphicx} % support the \includegraphics command and options
\usepackage{array} % for better arrays (eg matrices) in maths
\def\Not{\operatorname{\sim}}
\def\St{\operatorname{St}}
\def\num{\operatorname{num}}
\def\And{\mathrel{\&}}
\def\Or{\vee}
\def\BigOr{\bigvee}
\def\<{\langle}
\def\>{\rangle}
\def\union{\cup}
\def\nleq{\not\le}
\def\N{\mathbb N}
\def\R{\mathbb R}
\def\C{\mathbb C}
\def\Powerset{\mathcal P}
\def\star#1{{}^*#1}
\def\hN{\star{\N}}
\def\hR{\star{\R}}
\def\Z{\mathbb Z}
\def\Power{\mathcal P}
\def\Implies{\rightarrow}
\def\True{\operatorname{True}}
\def\Socrates{\mathrm{Socrates}}
\def\actual{@}
\def\Law{\operatorname{Law}}
\def\Chance{\operatorname{Chance}}
\def\Var{\operatorname{Var}}

\def\H2O{H${}_2$O}

\def\scr{\mathcal}
\def\e{\varepsilon}
\def\eps{\varepsilon}
\newtheorem{lem}{Lemma}
\newtheorem{prp}{Proposition}
\newtheorem*{theorem}{Theorem}
\newtheorem{corollary}{Corollary}
\newtheorem{cond}{Condition}

\renewcommand\thechapter{\Roman{chapter}}

\def\chaptertail{\ifdefined\book\else\end{document}\fi}
 

\title{Infinity, Causation and Paradox}
\author{Alexander R. Pruss}
%\date{} % Activate to display a given date or no date (if empty),
         % otherwise the current date is printed

\begin{document}
\setcounter{secnumdepth}{3}
\setcounter{tocdepth}{4}

\end{document}
\fi

\restartlist{enumerate}

\chapter{Eternal Life and Fulfillment}\label{ch:eternal-life}
\chaptertail

??delete?

??interact with Oderberg on suffering and pain

