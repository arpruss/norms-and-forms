\def\mychapter{X}

\ifdefined\book
\else
\documentclass[11pt,oneside]{amsbook}
\usepackage[backend=biber, citestyle=authoryear]{biblatex}
\usepackage{mathpazo}
\usepackage{graphicx}
\usepackage{amsmath}
\usepackage{tikz}
\usetikzlibrary{arrows}
%\usepackage{titlesec}
\addbibresource{bibliography.bib}
\newcommand\posscite[1]{\citeauthor{#1}'s (\citeyear{#1})}
\newcommand\plural[1]{#1\mathrm{s}}
%\def\posscitewithextra[#1]#2{\citename{#2}'s (\citeyear{#2}, #1)}

%\newcounter{subsubsubsection}[subsubsection]
%\renewcommand\thesubsubsubsection{\thesubsubsection.\arabic{subsubsubsection}}
%\titleformat{\subsubsubsection}
%  {\normalfont\normalsize\bfseries}{\thesubsubsubsection}{1em}{}
%\titlespacing*{\subsubsubsection}
%{0pt}{3.25ex plus 1ex minus .2ex}{1.5ex plus .2ex}

\ifdefined\book
\renewcommand{\thechapter}{\Roman{chapter}}
\else
\renewcommand{\thechapter}{\mychapter}
\fi

\linespread{1.7}
\usepackage[margin=1.25in]{geometry}
\sloppy
\makeatletter
%% TODO: This is a cheat. It's supposed to be {paragraph}{4}, and that's 
%% what it is in amsbook.cls, but then it fails.
\def\paragraph{\@startsection{paragraph}{3}%
  \normalparindent\z@{-\fontdimen2\font}%
  \normalfont}
\def\subsubsubsection{\paragraph}
\makeatother

\def\smalltick{0.15cm}
\def\bigtick{0.3cm}
\def\pointcircle{0.08cm}
\def\causalnode{0.35cm}

\hyphenation{dia-chro-nic}

%\usepackage[utf8]{inputenc} % set input encoding (not needed with XeLaTeX)
\usepackage{amssymb}
\usepackage{mathtools}
\usepackage{enumitem}
\usepackage{amsthm}
\usepackage{physics}
%\usepackage{ntheorem}
\usepackage{chngcntr}
\counterwithin{figure}{section}

\makeatletter
% \def\@endtheorem{\endtrivlist\@endpefalse }% OLD
\def\@endtheorem{\endtrivlist}%

\setlist[description]{font=\normalfont\scshape}

\catcode`\|=\active\def|{\mid}
\DeclarePairedDelimiter{\ceil}{\lceil}{\rceil}
\DeclarePairedDelimiter{\floor}{\lfloor}{\rfloor}
\newcommand{\Subj}{\mathbin{\raisebox{.15ex}{$\scriptscriptstyle{\Box}$}\kern-.425em\rightarrow}}
\def\Existence{E!}
\def\Believes{\operatorname{Believes}}
\def\True{\operatorname{True}}
\def\Perfection{\operatorname{Perfection}}
\def\ext{\operatorname{Ext}}
\def\Iff{\leftrightarrow}
\def\Implies{\rightarrow}
\def\Entails{\Rightarrow}
\def\Cov{\operatorname{Cov}}
\def\Equiv{\Leftrightarrow}
\def\Form{\operatorname{Form}}
\def\Informs{\operatorname{Informs}}
\def\technical{$\star$}
\def\vtechnical{$\star\star$}
\def\power{\wp}
\def\Nec{\Box}
\def\Poss{\Diamond}
\def\Prop#1{$\langle$#1$\rangle$}
\def\R{\mathbb R}
\def\N{\mathbb N}
\def\tele{tel\={e}}
\makeatletter
\newtheoremstyle{indented}{3pt}{3pt}{\addtolength{\leftskip}{4.5em}}{-2.5em}{\sc}{.}{.5em}{}
\def\Principle#1#2#3{\theoremstyle{indented}\newtheorem*{principle}{#2}\begin{principle}\def\@currentlabel{#2}\label{#1}#3\end{principle}\let\principle\undefined}
\makeatother
\def\pref#1{{\sc\ref{#1}}}
\def\enum#1{\resume{enumerate}\item #1\end{enumerate}}
\def\ditem#1#2{\begin{enumerate}[resume]\item \label{\mychapter:#1} #2\end{enumerate}}
\def\nitem#1#2{\begin{description}\item[#1\label{\mychapter:#1}] #2\end{description}}
\def\bref#1{\ref{\mychapter:#1}}
\def\dref#1{(\ref{\mychapter:#1})}
\def\drefglobal#1{(\ref{#1})}
\usepackage{graphicx} % support the \includegraphics command and options
\usepackage{array} % for better arrays (eg matrices) in maths
\def\Not{\operatorname{\sim}}
\def\St{\operatorname{St}}
\def\num{\operatorname{num}}
\def\And{\mathrel{\&}}
\def\Or{\vee}
\def\BigOr{\bigvee}
\def\<{\langle}
\def\>{\rangle}
\def\union{\cup}
\def\nleq{\not\le}
\def\N{\mathbb N}
\def\R{\mathbb R}
\def\C{\mathbb C}
\def\Powerset{\mathcal P}
\def\star#1{{}^*#1}
\def\hN{\star{\N}}
\def\hR{\star{\R}}
\def\Z{\mathbb Z}
\def\Power{\mathcal P}
\def\Implies{\rightarrow}
\def\True{\operatorname{True}}
\def\Socrates{\mathrm{Socrates}}
\def\actual{@}
\def\Law{\operatorname{Law}}
\def\Chance{\operatorname{Chance}}
\def\Var{\operatorname{Var}}

\def\H2O{H${}_2$O}

\def\scr{\mathcal}
\def\e{\varepsilon}
\def\eps{\varepsilon}
\newtheorem{lem}{Lemma}
\newtheorem{prp}{Proposition}
\newtheorem*{theorem}{Theorem}
\newtheorem{corollary}{Corollary}
\newtheorem{cond}{Condition}

\renewcommand\thechapter{\Roman{chapter}}

\def\chaptertail{\ifdefined\book\else\end{document}\fi}
 

\title{Infinity, Causation and Paradox}
\author{Alexander R. Pruss}
%\date{} % Activate to display a given date or no date (if empty),
         % otherwise the current date is printed

\begin{document}
\setcounter{secnumdepth}{3}
\setcounter{tocdepth}{4}

\end{document}
\fi

\restartlist{enumerate}

\chapter{Details and open questions}\label{ch:details}
\section{Introduction}
I have argued that in order to explain a large multitude of normative and metaphysical features 
of reality, especially of the human being, we would do well to posit a metaphysically robust 
entity, a nature or form, in each substance, and especially in human beings. This form is extremely rich 
in informational content, 
as it has to answer a very large number of Mersenne questions in ethics, epistemology, semantics,
and metaphysics.

But the theory so far is highly underspecified. The point of this chapter is to discuss some issues
resolving which would specify the theory a little further in terms of both teleology and fundamental metaphysics. 

\section{Teleology}
\subsection{Parts and aspects}
When my arm is functioning poorly due to an injury, I am functioning poorly insofar
as I have an arm. It it tempting to reduce evaluations of the function or flourishing
of a part of a substance to the function or flourishing of the whole in respect of the
part. But while this is tempting, there is some reason to resist this reduction.

Of course, everyone agrees that there are cases where the flourishing or languishing of a part or aspect is
instrumental to the opposite state of the whole. Xenophon has Socrates give the example of a 
person who is harmed by their wisdom because a tyrant hears about the wisdom and has the
wise person kidnapped to serve as an adviser.??ref And many a person would have escaped a broken
leg sustained if they had a minor sprain that kept them from skiing. But such cases can be
reconciled with a reduction of the flourishing of the part to the flourishing of the whole by
noting that there is nothing particularly strange about an aspect of languishing being a means to 
flourishing (e.g., one might get a cash settlement for an injury). 

But there are more interesting cases where the flourishing state of a part seems not merely instrumental to 
the opposite flourishing state of the whole, but constitutive of it. Muscles are torn down by exercise
and regrow stronger. The process of tearing down is harmful to the muscles themselves, but is a constitutive 
part of the proper functioning of the body's system of adaptation to particular activities. And, more
generally, the death of cells is part and parcel of the normal self-renewing persistence of a multi-cellular
organism.

These cases are perhaps not entirely convincing. One might insist that when cells die as part of the self-renewal
of the organism, the death is itself a part of the proper functioning of the cell, and hence both the cell and the
whole are flourishing. Historically, Aristotelians have tended to insist that a thing's destruction is bad
for it.???refs However, this may be mistaken. If we think of substances as four-dimensional objects---spatiotemporal
entities---then a thing is destroyed just in case it has an upper temporal boundary. Now, having \textit{spatial} 
boundaries is not bad for a thing---indeed, having spatial boundaries of the right sort is constitutive of a thing's
having the correct shape and size. A dog so bloated as to be boundless, taking up the whole universe, would not be a 
healthy dog. Similarly, a substance of a specific sort could have proper temporal boundaries, and if so, it might be harmed not just by
living too short a time, but also by living too long a time.\footnote{
Whether it is a part of human flourishing to have an upper temporal boundary or whether humans better flourish
in living forever is a difficult
question whose thorough discussion is beyond the scope of this book. We can note that human beings 
pursue projects that  are at least typically cut short by death, and their pursuit of such projects 
is a part of their flourishing. Moreover, there is reason to doubt that a satisfactory degrees of virtue, wisdom, 
and understanding of the world, and a sufficiently broad array of deep friendships are practically 
attainable in a typical human lifespan, which points to the idea that it is not good for us that our upper
temporal boundary be at death (on the
other hand, see ??refs), and hence points towards life after death given Aristotelian optimism.} It is plausible that typical human cells have 
proper temporal boundaries as well, and hence their living too long may be bad for the organism as a whole
\textit{and} for the cell as a part.

Cases of redundancy may provide a more convincing example of where the flourishing of a part comes apart from
the partial flourishing of the whole. Suppose that an organism for its basic functioning needs $n_1$ functioning parts of some sort---say,
cells of a particular kind, or legs, or teeth---and for the sake of redundancy it needs some larger number $n_2$.
Suppose that having more than $n_2$ is supernormal for the organism (as per our discussion in ??backref), until we
reach some large number $n_3$ at which point the organism has too many.  

For the sake of definiteness, suppose that the parts are teeth of some hypothetical organism that does not
have a precisely prescribed number of teeth, and that $n_1$ is 30, $n_2$ is 35, and $n_3$ is 40. With fewer
than 30 teeth, the organism fails to chew well. With 35 to 39, it has a healthy level of redundancy. And at 40 or more,
it has too many teeth. Suppose now that Sally is an organism of this sort and she has 38 teeth. One of the teeth, however,
is getting worn down quite a bit. That tooth is no longer fully functional, and hence that tooth is failing to flourish.
However, this does not constitute any failure of flourishing in Sally. Even if the tooth stopped functioning entirely,
Sally would still have sufficient dentation both for first-order purposes and for redundancy. Sally can still be fully
functioning as a whole with respect to her teeth, even though that tooth is not itself fully functioning, and even though
she would be fully functioning to a higher degree if that tooth were to fully function. In this case, the direction of
flourishing of the part and of the whole seems to be the same, but nonetheless one can have full flourishing of the whole
without every part fully flourishing. In such a case, it would not be correct to say that Sally is languishing with respect
to that tooth. For that tooth doesn't make her languish---it just makes her flourish less.

Interestingly, redundancy can be between parts or aspects of very different sorts. We might suppose that a flourishing human 
being has a sufficient number of abilities of various sorts. These abilities can be moral, intellectual, emotional, or physical, and
within each category, they can differ quite significantly from each other. Then full flourishing could turn out to be compatible with a severe
impairment within certain abilities---for instance, one's mathematical aptitudes might be dysfunctional, and yet the \textit{person} 
might fully flourish. This would allow an Aristotelian to accept the thesis that some persons with significant disabilities can nonetheless
be fully flourishing.??refs For the disability can constitute the failure of a part or aspect of the person to flourish, without
thereby compromising flourishing of the whole. 

We could, in principle, suppose that there are some cases where the failure of flourishing of a part might not even make the 
whole flourish less. Going back to Sally, perhaps 37 teeth is better than 38 (maybe 37 makes for better fit within the jaw),
even though any number from 36 to 39 is fully normal. In that case, if Sally's 38th tooth is starting to deteriorate, this could
be moving her to an even better state. 

At the same time, there are proper parts and aspects such that the flourishing of the part or aspect always lines up with the
flourishing of the whole. In the case of a person, to have one's will flourish in an action is to flourish with respect to the 
will, and makes one better off with that respect, and hence we do not need to correct looser discussion earlier in this book where
no distinction was made between flourishing with respect to the will and having one's will flourish. Plausibly, rationality is similar. 
One is always better off for being more moral and more rational, and a failure of flourishing of one's morality is a failure of the 
person as a whole. 

One way to handle the distinction between flourishing of the part and flourishing of the whole is to suppose,
contrary to classical Aristotelianism, that the parts are themselves substances, with their own forms. Another
way, however, is to suppose that there is only one substantial form, that of the whole, and this substantial
form in addition to specifying the conditions for the flourishing of the whole specifies the conditions for the
flourishing of the parts which are not reducible to the flourishing of the whole. The second option appears
somewhat simpler, but our account can work with both.

\subsection{Teleological reduction}
\subsubsection{A multiplicity of concepts}
Let us return to the norms for the whole, however.
The applications given in this book make use of various normative concepts that are said to be grounded in forms,
such as proper function, teleology, and flourishing. It would be good to investigate if these 
can be further unified, under either one of these concepts, or some further unifying concept.

First, as we have seen in the discussion of the supererogatory and supernormal, we have both binary and comparative normative 
concepts, often in the same context. Suppose a stranger is about to be hit by a train, and the only four options are:

\ditem{kick}{give them a kick to ensure that they have no chance of survival}
\ditem{fun}{make fun of them}
\ditem{idle}{stand by idly}
\ditem{jump}{jump in and push them out of the way likely at the cost of one's own life.} 

The binary distinction is that
the first two options are impermissible, and the other two are permissible. But there are comparative distinctions
with the two binary categories: \dref{kick} is worse than \dref{fun}, and \dref{jump} is better than \dref{idle}. 
Furthermore, the binary distinction does 
not reduce to ``big differences''---we cannot just say that the impermissible is whatever is much worse than 
something permissible. For while \dref{fun} is much worse than \dref{idle}, \dref{idle} is also much worse than \dref{jump} (though 
it is more natural to say that \dref{jump} is much better than \dref{idle}). 

Proper function and teleology appear to be primarily binary concepts: a thing functions properly or improperly, and a thing either 
does or does not achieve its end. Flourishing, on the other hand, seems to comprise both the binary and the comparative. The mildly vicious person suffering undeserved horrendous pain appears not to be flourishing \textit{simpliciter}, but if the pain 
were increased,
they would languish even more. Flourishing at this point appears the best candidate for a foundational concept for our norms: it supports a binary distinction \textit{and} comparisons.

However, the claim that teleology is a primarily binary concept---a thing's achieving or not achieving its end---is
an oversimplification. To see this we need an extended digression on ends, going beyond traditional Aristotelian
thinking.

\subsubsection{Ends}
Many activities seem to occur for an end. The activity then counts as successful provided that the end occurs and occurs as a
fulfillment of the activity. An organism produces gametes in order to reproduce. A cat chases birds in order to catch them,
and eats them for nourishment. And I put on shoes to keep my feet comfortable when I walk. The end-directedness of much voluntary
activity is obvious, though whether there really is teleology in the involuntary cases is a more controversial.

The Aristotelian tradition tends to analyze voluntary action as always end-directed, but also tends to see involuntary activity as 
often, if not always, directed at an end. I will argue, starting with the case of voluntary action, that many interesting phenomena 
would be misclassified as end-directed. The actual structure can be more complex, and while it has a directional structure, it is
misleading to think of that structure as teleological in the sense of possessing a ``target \textit{telos}'', an aim-to-be-achieved,
an end that fulfills it.

Consider a sprinter who is running a hundred meters all out against a clock, rather than against other opponents. The runner has an end, 
namely to sprint 100 meters. But sprinting 100 meters does not explain the intense effort the runner puts in. Less than half of the effort 
could have been put in, and 100 meters would still have been sprinted. The bulk of the runner's effort is explained by being directed 
not at completing the sprint but at completing it in good time.

But what exact state of affairs does the runner's speed-directed effort have as its end? One might have a particular target time in mind when running, but in our example we are imagining a sprinter who runs all out. A runner who is just aiming at a particular time would be reasonably expected to slow down if it became obvious
that a slower (and hence less exhausting) run would still achieve the target time, but not so our all-out sprinter. Our sprinter may have some specific time in mind
to motivate themselves (say, 9.9~seconds), but interpreting their action as merely aiming at that time does not capture all of the directional structure 
of the performance. Any shortening of the time of the run is welcome given the sprinter's aims, even if it goes
below the time they have in mind.

We would normally describe the sprinter as trying to run 100 meters ``as fast as possible'', and that seems to be a coherent description
of an end. However, the language of ``as fast possible'' should not be taken literally. First, we have the question of what the relevant
comparison class is. Is the runner trying to run as fast as any human being can on any track? This seems an excessively
grandiose aim. As fast as they themselves can run on this track on this occasion? But then what aspects of ``this occasion'' are 
fixed (e.g., the level of motivation)? And what no minimum time exists, but only a lower bound that cannot be reached, much like
an object with mass can come arbitrarily close to the speed of light but cannot achieve that speed.

Second, suppose we fix a particular sense of ``as fast as possible'', and then after ten meters the runner realizes that they have
been running slower than was possible. At this point, it is no longer possible for the runner to achieve the goal of literally running
the hundred meters in as short a time as possible. But we do not expect the runner to stop. We expect the runner to resume running all-out, as part of the
same directed activity.

There is obviously a teleological structure to the sprinter's run. But it is not a structure of aiming to \textit{achieve} a 
target \textit{telos}. 
We can think of this structure as a kind of preference structure: a faster time is always preferred to a lower time. 
Such preference structures are common in games, where we ourselves have defined the teleology, but we can also find them in the 
case of teleologies that are not our own invention. For instance, suppose the human mind aims at understanding (cf.~??backref). If we understand
this in terms of aim at a target \textit{telos}, the understanding of everything or of everything humanly understandable, then as soon as 
we realize that we cannot have this (for there are humanly understandable things that I cannot understand because they have faded to much
into the past---say, facts about Napoleon's motivations on some day unrecorded in history), the pursuit would become pointless. 
It is better to say that our teleological orientation set a preference ordering on our understanding, where understanding a set
$B$ of items is better for us than understanding a set $A$ of items whenever $A$ is a proper subset of $B$ (every item in $A$ is
in $B$, but not \textit{vice versa}) and when the levels of understanding for each item are kept fixed. This ordering then obviously
affects our epistemic lives, but also our practical ones (since one needs to make sacrifices in order to understand).

Additionally, a teleology concerned with specific aims has a difficulty capturing the directedness at lesser
degrees of failure often but not always present in our actions. In chess, if it becomes clearly inevitable that
you will lose, you are expected to resign. In many sports, on the other hand, such resignation is poor sportsmanship.
Instead, you are expected to aim to minimize the amount you lost by, whether by trying to minimize the difference
between your and your opponent's score, or by trying to make your opponent work hard for their victory.

There also seem to be cases outside of rational activity with a teleological directedness not understood in terms of the pursuit of a specific \textit{telos}. 
A pecan tree produces pollen in order to have offspring. The more offspring, the better! Again, the 
\textit{telos} is not to have infinitely many offspring\footnote{One might suppose that there is a limit here based on how much offspring
it is possible for one pecan tree to have. But it is not clear that there is such a limit in principle. Wouldn't a pecan tree always be better
off if it lived an extra year and had more offspring (perhaps transported to another planet, to avoid overcrowding)?}, but rather the teleology
seems best understood in terms of a preference ordering: more offspring, keeping the health of the offspring constant, is better.

As an alternative we might suppose that one can have infinitely many target \textit{tel\^e}. Thus, to account
for the teleology towards understanding, we might say that humans have a \textit{telos} to understand
phenomenon $\phi_1$, and another \textit{telos} to understand $\phi_2$, and so on, for all the phenomena that can be understood by humans, or we might say that a pecan tree has a \textit{telos} have
at least one offspring, and a \textit{telos} to have at least two offspring, and so on. But while I have argued that our form is 
complex, to suppose such infinite complexity seems excessive. Furthermore, it is plausible that each target \textit{telos} impels the organism
causally---e.g., pulling on us in our deliberation. But infinitely many such things causally influencing a single aspect of us would violate 
the principle of causal finitism defended recently by multiple authors??refs. Furthermore, each organism would always have infinitely 
many unfulfilled target \textit{tel\^e}, and that may seem just too pessimistic to believe.

It may, of course, sound oxymoronic to talk of a teleological structure without a target \textit{telos}. But that's a merely verbal point. We can
think of teleology as about pointing and directedness, without a \textit{target} as such. Imagine for instance that I am competing in an 
odd archery competition, where the winner is the one whose arrow hits closer to the center---except that if you hit the exact center, you
automatically lose. Then there is an obvious sense in which I am aiming. Indeed, since the chance of hitting the center is infinitesimal,
I should even set my bow's sight on the center, but it is not correct to say that I am aiming to \textit{hit} the center. 

Terminologically, we in these cases we might either adopt the seeming oxymoron of ``teleology without a \textit{telos}'', but it seems neater to distinguish between a target \textit{telos} and a directional \textit{telos}. A target \textit{telos} is
one that fulfills the striving when it is achieved. A directional \textit{telos} specifies a direction, but there need
not be a target to be hit. Sometimes a directional \textit{telos} can be formulated in terms of an unachievable 
``ideal target'', such as ``perfection'' or ``getting to the end of the track instantly''.  Imagine a golf course 
where the distance between a hole and the starting tee is so long that  a hole-in-one is humanly impossible. In the
first shot, the player may aim \textit{in the direction of} or \textit{towards} the hole, but it wouldn't be correct
to call the hole the target of the shot. The hole is an ``ideal target''. And sometimes there may even be no ideal target, just a direction, as in the case of the odd archery competition where one aims in the direction of closeness to the 
center while avoiding the center itself. 

It is, parenthetically, interesting to note that this way of thinking of teleology complicates a well-known 
Kantian argument for why we should believe in 
an afterlife and God.??ref The argument holds that we are unable to achieve complete virtue in this life. But since we cannot aim at
the impossible, and need to aim at virtue, we should suppose a life beyond death, and a being capable of ensuring that in that after-life virtue
is achievable. But a critic might use the above understanding of teleology to defend the idea that we have an unlimited 
directedness in the \textit{direction} of perfect virtue without a directedness at the \textit{achievement} of perfect virtue. To respond to this criticism, the Kantian needs to  argue that either perfect virtue is a genuine (not merely 
ideal) \textit{target telos}, or we have as a target end some ``adequate'' degree of virtue that is not
practically achievable in this life.

\subsubsection{Teleology \textit{vs.}\ flourishing}
Given that flourishing is itself not just a binary distinction, it is plausible that we can reduce these more
complex teleological phenomena to flourishing. We can suppose, for instance, that pecan trees flourish 
\textit{simpliciter} when they have \textit{an} offspring, but flourish more when they have more. And we can
suppose that the fulfillment of our practical intentional aims---or perhaps with a restriction to
cases where we aim at a good or at least a non-bad---is a part of our flourishing even when the aims are
directional rather than targeted, and that we flourish to the degree to which we fulfill these aims. 

But now that we see the non-binary complexity in teleology, we might ask whether there isn't a reduction
in the opposite direction. For we can capture the binary concept of flourishing as opposed to languishing
in terms of the fulfillment of a target \textit{telos}, while capturing comparative flourishing in terms
of how well a directional \textit{telos} is fulfilled. Alternately, we might ask whether there is even 
a non-verbal distinction between flourishing and the generalized notion of teleology. 

There is, however, some reason to think that the phenomenon of flourishing is broader and more fundamental
than that of teleology. Teleology describes an entity's \textit{striving} if not always towards a target,
at least in a direction. But we can imagine cases of flourishing that are a kind of externally imposed
gift. Normally, we appreciate our existence as a kind of good. But it does not seem right to think of us
as having a striving for existence---after all, our existence is explanatorily prior to all our strivings.
We might strive to \textit{persist}, but that is not the same as striving to exist  in the first place.
Or imagine there is a God who creates beings whose good consists in union with God, but where this
union with God is something that the creature can do nothing to either promote or hamper---the union must 
be a pure gift. This seems intelligible, and there may even be some theologians tempted to think it true
in our own case.??refs

\subsection{Interdependence}
There is a children's picture book with the
jarring line that pigs give us ham. In an Aristotelian setting, we can ask: Is it a \textit{telos} of a pig
to provide ham to predators or of a butterfly larva to be a host for the offspring of a parasitoid wasp? 

A positive answer fits well with one aspect of Aristotelian optimism and poorly with another. For it makes
more things in nature be instances of proper function, but at the same time it makes the \textit{tel\^e} of
the prey be less harmonious, since the prey will presumably properly avoid being preyed on. 

Intuitively,
being eaten is clearly not good for one. However this line of thought begs the question. 
If giving one's life for a member of another species is one's \textit{telos}, it is thus far good for one.
But one can do justice to the intuition being eaten isn't good for one by noting that even if providing
food for another could be good for one, it might not be good \textit{overall}. Aristotelian optimism 
implies a \textit{tendency} towards harmony within the teleology of a substance, but is compatible with
there being some \textit{tel\^e} that are in tension with one another, say to feed another and to survive.
In such a case, survival may take priority, and so its loss is an overall harm---but a lesser harm than if
the organism lost its life without benefiting another. 

The same multiple-\textit{tel\^e} line of thought responds to the objection that organisms tend 
to engage in behavior clearly aimed at preventing their becoming
food for another---they try to escape predation. Considered without regard for teleology, a pig engages
in activities that promote the predator's nutritive good---the pig eats and grows---and it engages
in other activities that oppose the predator's nutritive good---the pig runs off. Aristotelian optimism
sees a tie between activity and teleology. If so, then it is not clear that we should privilege higher
level behavior in figuring out teleology. Perhaps the pig's becoming nutritionally beneficial is indeed
a pursuit of one porcine \textit{telos} just as much as its running away is a pursuit of a different
porcine \textit{telos}.\footnote{One might also argue that the a prey animal's running off prevents a population
implosion that would be harmful to the predators as well. (I am grateful to Dominic Pruss for this 
observation.)}

It is very plausible that sometimes the good of another individual of one's own kind is a \textit{telos}.
Parenthood is an obvious example: the existence and flourishing of one's offspring is a part of one's
own flourishing. But even in fairly primitive organisms, there seem to be cases that go beyond parenthood.
In some slime mold colonies, a fruiting body in a stalk is produced, and the organisms forming the stalk of the fruiting body do
not themselves reproduce, but instead sacrifice their reproductive potential to elevate other organisms
and make them better able to reproduce. 

In cross-species symbiotic relationships, we may think that each organism only promotes the good of the 
other as a means to its own good, but we need not think this. We might think that each organism has the
other's flourishing as a part of its own \textit{telos}. We might speculate that, after all, that if we
met up with intelligent aliens, promoting their flourishing would be an expression of virtue in
humans, even if there were no mutuality in the promotion.

That said, I do feel the intuitive force of the idea that it is implausible to think of the prey as having the
nourishment of a predator as a \textit{telos}. But even if we do not wish to go as far as to say that pigs
have a ham-production teleology, we have room for a substantive and robust picture of teleologically supported
interdependence between species. And applied to humans this would give us an Aristotelian reason for why the good of the earth ecosystem 
could be not merely a means to the human good, but partly constitutive of it, which would make a positive
contribution to environmental ethics. 

\section{Individual or shared forms?}
\subsection{Platonism and two Aristotelian views}
There is a long-standing scholarly debate whether Aristotle thinks forms are individual---numerically different ones for different members of 
the same kind---or jointly shared by all members of the same kind.??refs The shared form view at first sight seems rather 
more Platonic than Aristotelian, but as long as the shared form is immanent in the individuals and governs their functioning, the view can count as Aristotelian. In any case, it is time to compare an individual-forms Aristotelianism,
a shared-form Aristotelianism and Platonism with regard to the applications in this book. 

All three views have a robust view of human nature, whether this be a Platonic universal, a shared immanent form, 
or an individual form. All three views can give a grounding to answers to Mersenne questions about normative 
domains, and thus secure many of the normative applications of our theory of a robust human nature. But there
are applications where the tie is broken.

One might also wish to consider hybrid views on which there are both shared universals and individual immanent 
forms.\footnote{I am grateful to Bryan Reece for this suggestion.} Such views were held by some medieval thinkers??refs. Since all the applications in this book are neatly
handled by the individual immanent forms, the question of whether there \textit{additionally} are shared universals
can be bracketed. I will thus stipulate the individual forms view to be that there are \textit{at least} individual
immanent forms, and the shared universals and Platonist views to include a denial of such individual forms.

\subsection{Mind}
If we follow Aristotelian tradition and identify the mind with the form (or some aspects of the form), then a shared form 
view leads to Averroes' theory that we all have the same mind, and the Platonist view is a non-starter. If the will is part of the mind, then we get absurd consequences
about the lack of personal responsibility for our action, and in any case, the view is famously counterintuitive. However, 
there are two reasons this argument may not fully convince a contemporary Aristotelian metaphysician to embrace an individual
forms view. 

First, one might completely reject the identification of the mind with the form. While identifying the mind with 
form offers some attractive theoretical simplicity, one might instead suppose the form defines the mind's teleology, and the mind is a physical or non-physical system 
structured by that teleology, perhaps along the lines of the teleological functionalism sketched in ??backref. Or, second,
one might take hold that the human mind has two parts: one part is the non-physical form and the other part is the brain.
On this view, the mind is only partially identified with the form. This would allow different people to have numerically different
minds on a shared form view, because we have numerically different brains---your and my mind would have a part in common
and a part not in common. 

However, there are some other considerations in favor of an individual forms view.

\subsection{Diachronic identity and synchronic unity}
In ??backref, the Aristotelian account was used to give a simple view of the diachronic identity of substances,
including ourselves: they persist because of having the numerically same form. This, of course, does not work
if we all share the same form.

There is also a synchronic unity problem.
My nose and my heart are parts of the same human, while my nose and your stomach are not. What constitutes the 
difference? On an individual form view, there is an elegant Aristotelian answer parallel to the diachronic
identity answer: my nose and my heart are informed by the same
form, while my nose and your heart are not. This answer will obviously not do on either the Aristotelian or
the Platonist shared form view.

Nor will it do to say that the difference is constituted by the fact that there is continuous human matter joing my nose with my
heart, since if you and I were conjoined twins, but conjoined neither by heart nor nose, my nose would be connected by continuous
matter with your heart, even though those would still not be the nose and heart of one human. 

One might try for a teleological account of the unity of the human being: my nose and my heart function together (the nose 
allows the entry of oxygen which is distributed by the heart). However, we are social animals, and teleological cooperation
crosses individual boundaries.

It could be that some other account friendly to shared forms is available. But the simplicity and elegance of the
individual-form account of individual human unity is a reason to opt against the shared form view.

\subsection{Ethical intuitions}
Our form is arguably the most important part of us. Thus, you and I have our most important
part in common on a shared form view. This is intuitively both attractive and unattractive ethically. 

It is attractive because it makes our commonality deeper, and hence seems to make it easier to see
why egoism makes little sense, since we are only partly different from each other, and only in
respect of the less important parts. It is unattractive, however, for the same reason. If 
you and I have so much in common, then why should I go to the trouble of benefiting you when 
I can just benefit me, and you overlap with me anyway in our most important part? So far, thus,
it seems that ethically it's a wash between the shared and individual form views.

On the other hand, considerations of autonomy push one a little towards an individual form view.
If you choose for your life to be that of a solitary artist, then it seems like your action 
primarily affects you and not me, and I have no right to complain, and should cede you your
autonomy. But on a shared form view, your going off to sculpt rocks on a distant island directly affects
what happens to my most important part, my form---my form comes to inform a human being separated
from much of humankind, and engaging in artistic pursuits that I may happen to disagree with. Using 
a central part of me for ends I disagree with is problematic. On a shared form view, just as conjoined
twins perforce need to deliberate together about what to do with their lives, we are \textit{all} stuck with
one form, and it seems that we need to deliberate together about the shapes of our lives, to a degree
that seems excessive. For instance, suppose that you and I are complete strangers, and I am enamoured of 
a life of simplicity, while you want
your life to be a rich tapestry. Since my own form---indeed, perhaps, my very soul---is involved in your
tapestry, your choice detracts from my life of simplicity, and so on the shared form view, you have a significant 
reason to avoid complexity. That reason may be defeasible, but it is still a reason, and a reason too many. 

\subsection{Distant reproduction}
For evolution to
work with substantial forms, sometimes organisms of one metaphysical
kind must produce organisms of another kind. For instance, supposing
that wolves are a different metaphysical kind from dogs, and dogs
evolved from wolves, it must have happened that two wolves reproduced
and made a dog. (One may think wolves and dogs are metaphysically the same
kind, but let's suppose not for the sake of the argument.) If we
are to avoid occasionalism about this, we have to suppose that the two
wolves had a causal power to produce a dog-form under those
circumstances.

Plausibly dogs evolved from Pleistocene wolves in Siberia, but there was also a
Pleistocene wolf population in Japan, and so imagine that the causal power
to produce a dog was found in both wolf populations. Suppose,
counterfactually, that a short period of time after a pair of wolves
produced a dog in Siberia, a pair of Japanese wolves also produced a
dog. 

On an Aristotelian shared-form view, when the Siberian wolves produced a dog,
they did two things: they produced a dog-form and they made a dog
composed of the dog-form and matter. But when the Japanese wolves
produced a dog, the dog-form already existed, so they only thing they
could do is make a dog composed of matter and that dog-form.

The first oddity here is this. Our (perhaps imaginary) Japanese
wolves didn't know that there was already a dog in Siberia, so when they
produced a dog, they exercised exactly the same causal powers that their
Siberian cousins did. But their exercise of these causal powers had a
different effect, because it did not produce a new form, since the form
already existed, and instead it made the form get exemplified in some
matter in Japan. It is odd that the exercise of the same causal power
works differently in the same local circumstances.

Second, there is an odd action-at-a-distance here. The dog-form was
available in Siberia, and somehow the Japanese wolves in the story made
matter get affected by it thousands of kilometers away.

In fact, to make things worse, we can suppose the Japanese wolves
only lagged a two or three milliseconds after their Siberian cousins. In
that case, the Siberian wolves caused the existence of the dog-form,
which then affected the coming into existence of a dog in Japan in a
faster-than-light way. Indeed, in some reference frames, the Japanese
dog came into existence shortly \textit{before} the dog-form came into
existence in Siberia. In those reference frames we have backwards
causation: the Siberian wolves make a dog-form and that dog form
organizes matter in Japan \textit{earlier}. (Compare the causal argument
for the Small Beginnings Thesis in ??backref.)

On the other hand, if there is an absolute simultaneity, notwithstanding
Special Relativity, we can imagine that the Japanese wolves reproduced 
at \textit{exactly} the same time as the Siberian ones. Then we either
have puzzling overdetermination-at-a-distance, where each pair of wolves
engages in an activity that is a sufficient cause of the dog-form, or else
we have an odd degree of freedom where nature has to decide which pair is
the one that gets to cause the dog-form and which one merely gets to 
imprint the form on matter.

Third, we can suppose a multiverse made of causally isolated\footnote{Apart from
their common causal origin in God or some other first cause if there is
one.} and spatiotemporally disconnected universes, and suppose that the Siberian
wolves are in one universe while the Japanese ones are in another. On this version,
either we have overdetermination of dog-form across causally isolated universes, or we 
have some sort of a coordination across causally isolated universes and events that
have no common time sequence to determine which pair of wolves gets to make the
dog-form. 

If, on the other hand, every dog has a numerically distinct form,
there is no difficulty: the Japanese wolves' activity can be entirely
causally independent of the Siberian ones', whether the two are in
one universe or two.

The above is a reason to prefer the Aristotelian individual-forms view or the
Platonist view (since presumably the Platonist universals are not produced by
any evolutionary activities) to the Aristotelian shared-forms view.

\subsection{Normative knowledge}
There is also a perhaps reason to prefer either Aristotelian view to the Platonist one
insofar as the Aristotelian view might give a more direct account of at least some of 
our knowledge of normative matters. Since the Aristotelian forms are causally efficacious, we can
suppose that the teleological features that ground norms also impel the substances
that have the features to pursue their \textit{tel\=e}. And from these pursuits
we get some knowledge of the norms, by means of the optimistic principle that things
tend to go right. Granted, as we will see in ??forwardref there are limitations to this 
approach, but at least \textit{some} knowledge of normative matters seems to be neatly
grounded in the very norms known. 

There does seem to be something right about the intuition that one can often just \textit{see} the 
norms in the norm-governed behaviors, and taking the grounds of the norms to be causally efficacious makes
this possible. Furthermore, the phenomenology of moral motivation is often that we feel ``the norms'' pulling
on us. An Aristotelian view not only lets us take this ``pull'' quite seriously---the norms are aspects of our
causally efficacious nature which impel us to act teleologically---and to include it among thegrounds for our knowledge
of our teleology.

Indeed, worries about knowledge of Platonic abstracta is one of the main reasons for a widespread
rejection of Platonic views in our time.??ref Theism might be able to help here---God perhaps
grounds the abstracta and thereby knows them, and can give us innate knowledge of them. But perhaps
even a theist will have the intuition that some of our ordinary humdrum knowledge of such normative
facts as that one should proportion belief to evidence does not involve divine illumination. 
After all, a plausible theological insight is that if God exists, then God likes to work through creatures
as far as possible. This insight seems central to the most plausible (partial) responses to the
problem of evil we have. God wants us to grow in virtue and doesn't just zap us into a final state.
God wants nature to have a hand in creation and hence allows a long often violent evolutionary process.
And God lets things happen according to the laws of nature rather than directly micromanaging things.??refs
Admittedly, in ??forwardref I will argue that something like theism is still needed to defend full Aristotelian 
optimism. But methodologically it seems plausible that we should minimize invocations of God. And we
should try to save the phenomenology of seeing norms in the behavior of others and ourselves.


\section{Accidental normative forms}
If you have promised to $\phi$, you should $\phi$. Consider two Aristotelian metaphysical explanations of what is going on here.
On the conditional-norm explanation, the human form contains the conditional norm that you should $\phi$ if you have promised 
to $\phi$, which when combined with the fact that you have promised to $\phi$ grounds an obligation to $\phi$. But there is
another possible explanation. Perhaps promising to $\phi$ causes, in virtue of a power contained in the human form, 
the normative accident of being such that you ought to $\phi$ to come into existence. One might think of the second story as 
a very robust normative power theory: a theory on which normative powers are a type of causal power that brings into existence
an irreducible and new normative entity. (Most normative power theories do not consider normative powers to be a type of 
causal power.??refs) 

The conditional-norm account posits fewer entities, and insofar as this is the case, Ockham's razor favors it. And intuitively
it seems to be the right account of promissory obligations. If an analogous account holds for all norms, then the human substantial form could be 
the only normative property of the human being---there would be no normative accidents.

Now, while the conditional-norm account for humans
posits fewer entities (i.e., forms), it makes the human form contain more information in the form of conditionals.
For instance, on the conditional-norm view, both humans with and without a Y chromosome have a human nature that specifies what
range of physical developments that are normal expressions of the genes that are unique to the Y chromosome. On the 
normative accident view, it may be that humans with a Y chromosome also have an additional non-physical property governing
the expression of Y-based genes, but humans without a Y chromosome do not have any accidents or substantial forms
governing such expression.\footnote{It's worth noting that both views are compatible with a broad variety of views on gender and 
transsexuality, since choosing between the metaphysics of the two views does not settle the question of what the range
of normal physical developments is. One might think that the normative accident view allows for a more conservative
theory on which there is a metaphysical component---an accidental form---that determines whether one is really male or female, 
an accident of maleness or an accident of femaleness.  But at the same time, the normative accident view could enable one to have 
a metaphysical basis for the claim that one's real sex and/or gender fails to match one's biological constitution at birth.
It is also worth noting that for norms relevant to sex and/or gender, the conditional norm view could make the antecedents 
of the conditionals be facts about DNA (such as whether one has a Y chromosome) but could also make them involve facts about 
psychology and society. The metaphysics does not by itself settle the normative questions here.}

At the same time, the normative accident view of human beings involves significantly more in the way of \textit{causal} laws
presumably grounded in human causal powers, such as that when one makes a promise, a promissory-obligation
accident comes into existence, or when one has such-and-such DNA, then such-and-such an accident governing norms of gene expression
comes into existence. Moreover, since normative accidents are in large part defined by the norms they embody, the informational
complexity of the normative accident is presumably present in the causal power for its production---it is a power to produce an
accident of, say, being required to $\phi$.  Thus while we have reduced the \textit{normative} complexity in human nature, we have done 
so at the cost of \textit{causal} complexity, \textit{and} a multiplication of entities.

This last point does not apply to normative accidents that have a cause outside of the human being. But it is difficult to think
of examples, with the exception of one theological possibility. According to many Christians, by God's grace human beings
can be directed at the ``beatific vision'', a direct vision of God. This beatific vision exceeds the power of human
nature, and it is usually taken that natural human fulfillment does not require it. One way to make sense of this is to suppose
that God by grace gives all or some humans a normative accident of a teleological directedness at the beatific vision, together
with the supernatural powers that promote the fulfillment of this telos. 

\section{$^*$Substances and quantum mechanics}
What has form, i.e., what are the substances? 
It is essential to our meta-normative applications that \textit{humans} be substances and hence have forms. And given a view of 
human beings as having an organic teleology, it would be implausible to deny form to other organisms. Thus, all organisms
have forms. But what about inorganic reality? It appears to be central to Aristotelian metaphysics that all of reality be 
grounded in substances and their properties and relations. Inorganic aspects of reality---say, particles---that are 
part of an organism will be grounded in the organism \textit{if} we accept the classic Aristotelian thesis that substances 
cannot be proper parts of substances. 

But in any case, that would still leave the vast amount of inorganic reality outside of any organisms, including particles, 
fields, rocks, mountains, planets, stars, galaxies, and so on. There has to be at least one inorganic substance to ground these.

On most interpretations of quantum mechanics there is no precise number of particles
in existence, as the universe is in a superposition of different states corresponding to different particle counts. But whether 
a substance exists, and what identity relations hold between substances, must be an objective fact. We
thus cannot have superpositions of different substance counts in reality. Thus, on most interpretations of quantum mechanics, particles
are not good candidates for substances. The one exception is the de Broglie-Bohm pilot wave interpretation (Bohmianism), on 
which there are particles or ``corpuscles'' with definite positions. If we have the intuition that particles are substances, 
that will support Bohmianism. 

The view radically opposite to a corpuscularian insistence on inorganic reality being partly grounded in substantial 
particles is to suppose that inorganic reality is grounded in some small number of global substances. This is an 
active area of research among Aristotelian philosophers of science, and a number of options are available.??refs 
These options differ depending on how the information contained in the mathematical wavefunction of the universe
is encoded in the substances of physical reality.

First, there 
could be a single global ``wavefunction substance'' that grounds the values of the global quantum wavefunction, existing
over and beyond the organic substances, and interacting with them.

Second, one might suppose that a part of the information in the 
wavefunction of the universe is encoded in states of the organic substances, and the rest (i.e., vast majority) of 
the information is encoded in a global ``Swiss cheese substance'' that has holes corresponding to the organic substances. 

Third, one might try to ground the wavefunction's data in a number of small local point-like substances.
These could be points of three-dimensional space or of four-dimensional spacetime, and the wavefunction could be 
grounded in their relations.\footnote{$^*$In a non-relativistic setting, the wavefunction for $n$ particles that 
have position as their only property is a complex-valued function $\psi$ the phase space $\R^{3n}$. A point in $\R^{3n}$ can 
be identified as a sequence $(x_1,...,x_n)$ of $n$ points each in $\R^3$. We can then suppose that there is an 
$n$-ary determinable relation (``being co-wavefunctioned''?) between any $n$ points in $\R^3$, a determinate of which is 
quantified with a complex number, which complex number yields $\psi(x_1,...,x_n)$. If the points $x_1,...,x_n$ are 
substances, then these values are indeed grounded in relations between substances.} Or they could be points in a higher-dimensional
phase space, with the wavefunction grounded in their properties. 

A Bohmian view does not escape the need for a grounding of the wavefunction, because the corpuscles are 
governed in their behavior by the wavefunction, and their postulated properties like position are not sufficient to ground the values of the wavefunction. Thus, an Aristotelian Bohmianism does not appear particularly economical: it requires organic
substances, corpuscles, and a ground for the global wavefunction.

Interpretations of quantum mechanics can be by and large seen as responses to the measurement problem: the problem of 
what to do with the fact that naively applying the Schr\"{o}dinger equation to the initial conditions in many experimental 
setups generates the prediction of a wavefunction that contains a superposition of different macroscopic results, such as 
an instrument display showing one number and its showing a different number---something we never observe. Thus, in order 
to generate predictions compatible with what we actually see in the lab, something must be done. Responses 
divide into collapse and no-collapse interpretations. On both families,
we have a baseline deterministic evolution of the wavefunction over time according to the Schr\"{o}dinger equation.

On no-collapse interpretations, the baseline is all that happens, and as a result physical reality gets more and more 
superimposed. For instance, given the likely dependence of our solar system on quantum events in the early universe,
reality is in a superposition of our solar system having formed and our solar system not having formed. But that is 
not what we observe: we simply observe our solar system. One way to think about this is that the global wavefunction 
on a no-collapse interpretation describes something like a multiverse, in some branches of it there being a solar system and 
in others there not being one. We inhabit a branch where there is a solar system. 

If you toss a quantum mechanical coin, on a no-collapse interpretation our branch splits into a branch with heads and a branch with tails. What happens
to you? There are six options:
\ditem{branch-neither}{You cease to exist.}
\ditem{branch-both}{You live in both a heads and a tails branch.}
\ditem{branch-heads-1}{You go to the heads branch and no one inhabits ``your'' position in the tails branch.}
\ditem{branch-heads-2}{You go to the heads branch and someone else\footnote{Perhaps a pre-existing person who had 
been co-located with you prior to the branch split.??ref:many-minds} inhabits ``your'' position in the tails branch.}
\ditem{branch-tails-1}{You go to the tails branch and no one inhabits ``your'' position in the heads branch.}
\ditem{branch-tails-2}{You go to the tails branch and someone else inhabits ``your'' position in the heads branch.}

Both \dref{branch-neither} and \dref{branch-both} have a pleasing symmetry. However, both have a serious problem.
Suppose the coin is very unfair, so it has a $7/8$ chance of heads and a $1/8$ chance of tails. Now, on both 
\dref{branch-neither} and \dref{branch-both}, the two new branches are equally related to you---either neither is 
\textit{your} branch or both are \textit{yours}. But obviously prior to the toss, you should plan your future 
asymmetrically---you should expect to get heads. This asymmetry in planning does not fit with the metaphysics 
of the situation.\footnote{It may occurs to the reader to say that there are eight branches, one of them with 
tails and seven with heads. Then if you live in all eight branches, you might think it makes sense to plan 
``more'' for the more common branch. There is, however, a simple technical reason why this can't be the solution.
This solution only works when the chance of heads is a rational numbers, i.e., of the form $p/q$ with $p$ and $q$ integers.
But if the chance of heads is not a rational number, say because it is $1/\sqrt{2}$, there is no way to divide 
the branches in a ratio corresponding to the chances. And the continuous evolution of the wavefunction ensures
that most of the time the chances will be irrational numbers.} There is, of course, a large literature on this argument, with significant dispute??refs, but I will take the argument to provide significant support for the asymmetric metaphysical options. 

On the remaining four options, the wavefunction does not contain all the information about the universe, since the wavefunction
only says that there is a superposition between heads and tails, and does not tell you whether the individual who tossed
the coin goes into the heads branch (as on \dref{branch-heads-1} and \dref{branch-heads-2}) or into the tails branch (as on
the remaining two options). On these options, there is no problem with uneven chances: the chance of heads can determine 
how likely it is that one will get \dref{branch-heads-1} rather than \dref{branch-tails-1}, or that one will get 
\dref{branch-heads-2} rather than \dref{branch-tails-2}.

Thus, on a no-collapse view we have good reason to think there has to be more to reality 
than the wavefunction. Indeed, a no-collapse view points towards a metaphysics where some of the facts about 
reality are encoded in the wavefunction but others are encoded by the ``location'' of the agents in the multiverse.
This fits neatly with an ontology on which we have a global substance grounding the wavefunction and agents that 
move around in the multiverse.\footnote{For interpretations like this, see ??refs.}

On collapse interpretations, superposed states, such as a superposition between an electron being in one place and its 
being in another, are resolved indeterministically into pure states, say, the electron being in a specific place. The 
value of the wavefunction specifies the chance that a given superposed state will resolve into a given pure state. 
This collapse is an exception to the deterministic Schr\"{o}dinger equation.

On these interpretations, as far as the physics goes, it is possible to think the wavefunction describes all of physical 
reality if the situations that trigger collapse are themselves described by the wavefunction. The historically first collapse interpretations??Ref had collapse triggered by observation. This appears to require the prior existence of an observer. But if we think about 
the evolution of organisms capable of observations, this evolution itself depends on a vast number of quantum events (such 
as DNA mutations), and hence prior to the appearance of the first physical observer, all we have was a superposition 
of a state where an observer-like being had already evolved and a state where an observer-like being had not yet evolved. 
Thus, unless the existence of an observer is a fact over and beyond a wavefunction that codes for a superposition of 
observer-like states, observation-triggered collapse seems unable to get off the ground. Observation-triggered collapse,
thus, offers some support for an Aristotelian metaphysics where in additional to one or more substances grounding the 
wavefunction, we have some additional macroscopic substances that are observers.\footnote{Of course, an interesting open question for such theories is what kind of interaction with a macroscopic substance counts as an observation. Do plants observe when they interact
with their environment, for instance?}

More recent collapse interpretations coming from physicists (e.g., Ghirardi–Rimini–Weber theory??ref) have collapse get triggered discretely or continuously by physical facts about the wavefunction. This is compatible with the wavefunction being a 
complete description of physical reality. These theories are empirically testable, but have not yet been sufficiently
tested.??refs If we find such a theory to be correct, an Aristotelian view that posits additional macroscopic substances 
beyond the substance or substances grounding the wavefunction will appear to be \textit{physically} otiose, though the
normative and semantic considerations of the earlier chapters will continue to be support it. Furthermore, standard 
arguments against physicalism in the theory of mind??ref-and-backref lend support to a richer ontology.

Anyway, other things being equal, no-collapse interpretations of quantum mechanics are preferable: they take the wavefunction 
to evolve under the elegant (and deterministic) Schr\"{o}dinger equation. The physically-based collapse theories modify this
evolution in a mathematically complicated way with one or more additional free parameters (such as a collapse rate and a
critical distance in the Ghirardi-Rimini-Weber theory). Theories on which collapse is triggered by interaction with something
not at the level of physics, like organisms, at least seem to have the slight advantage of preserving the idea that when physics alone is in play,
we have the elegant Schr\"{o}dinger equation---collapse is triggered by interaction with something complex that is properly 
studied by other sciences than physics (e.g., biology or psychology). 

Thus it seems that when we think about quantum mechanics, a model of reality on which there are macroscopic substances like
oaks, sharks and people, and a further substance that (alone or in cooperation with the macroscopic substances) is needed
for grounding the wavefunction is not at all unnatural. On no-collapse models and non-physical collapse models, this theory 
answers a need that arises from the physics itself---a need to generate experimental measurements.



\chaptertail
