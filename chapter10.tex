\def\mychapter{X}
\ifdefined\book
\else
\documentclass[11pt,oneside]{amsbook}
\usepackage[backend=biber, citestyle=authoryear]{biblatex}
\usepackage{mathpazo}
\usepackage{graphicx}
\usepackage{amsmath}
\usepackage{tikz}
\usetikzlibrary{arrows}
%\usepackage{titlesec}
\addbibresource{bibliography.bib}
\newcommand\posscite[1]{\citeauthor{#1}'s (\citeyear{#1})}
\newcommand\plural[1]{#1\mathrm{s}}
%\def\posscitewithextra[#1]#2{\citename{#2}'s (\citeyear{#2}, #1)}

%\newcounter{subsubsubsection}[subsubsection]
%\renewcommand\thesubsubsubsection{\thesubsubsection.\arabic{subsubsubsection}}
%\titleformat{\subsubsubsection}
%  {\normalfont\normalsize\bfseries}{\thesubsubsubsection}{1em}{}
%\titlespacing*{\subsubsubsection}
%{0pt}{3.25ex plus 1ex minus .2ex}{1.5ex plus .2ex}

\ifdefined\book
\renewcommand{\thechapter}{\Roman{chapter}}
\else
\renewcommand{\thechapter}{\mychapter}
\fi

\linespread{1.7}
\usepackage[margin=1.25in]{geometry}
\sloppy
\makeatletter
%% TODO: This is a cheat. It's supposed to be {paragraph}{4}, and that's 
%% what it is in amsbook.cls, but then it fails.
\def\paragraph{\@startsection{paragraph}{3}%
  \normalparindent\z@{-\fontdimen2\font}%
  \normalfont}
\def\subsubsubsection{\paragraph}
\makeatother

\def\smalltick{0.15cm}
\def\bigtick{0.3cm}
\def\pointcircle{0.08cm}
\def\causalnode{0.35cm}

\hyphenation{dia-chro-nic}

%\usepackage[utf8]{inputenc} % set input encoding (not needed with XeLaTeX)
\usepackage{amssymb}
\usepackage{mathtools}
\usepackage{enumitem}
\usepackage{amsthm}
\usepackage{physics}
%\usepackage{ntheorem}
\usepackage{chngcntr}
\counterwithin{figure}{section}

\makeatletter
% \def\@endtheorem{\endtrivlist\@endpefalse }% OLD
\def\@endtheorem{\endtrivlist}%

\setlist[description]{font=\normalfont\scshape}

\catcode`\|=\active\def|{\mid}
\DeclarePairedDelimiter{\ceil}{\lceil}{\rceil}
\DeclarePairedDelimiter{\floor}{\lfloor}{\rfloor}
\newcommand{\Subj}{\mathbin{\raisebox{.15ex}{$\scriptscriptstyle{\Box}$}\kern-.425em\rightarrow}}
\def\Existence{E!}
\def\Believes{\operatorname{Believes}}
\def\True{\operatorname{True}}
\def\Perfection{\operatorname{Perfection}}
\def\ext{\operatorname{Ext}}
\def\Iff{\leftrightarrow}
\def\Implies{\rightarrow}
\def\Entails{\Rightarrow}
\def\Cov{\operatorname{Cov}}
\def\Equiv{\Leftrightarrow}
\def\Form{\operatorname{Form}}
\def\Informs{\operatorname{Informs}}
\def\technical{$\star$}
\def\vtechnical{$\star\star$}
\def\power{\wp}
\def\Nec{\Box}
\def\Poss{\Diamond}
\def\Prop#1{$\langle$#1$\rangle$}
\def\R{\mathbb R}
\def\N{\mathbb N}
\def\tele{tel\={e}}
\makeatletter
\newtheoremstyle{indented}{3pt}{3pt}{\addtolength{\leftskip}{4.5em}}{-2.5em}{\sc}{.}{.5em}{}
\def\Principle#1#2#3{\theoremstyle{indented}\newtheorem*{principle}{#2}\begin{principle}\def\@currentlabel{#2}\label{#1}#3\end{principle}\let\principle\undefined}
\makeatother
\def\pref#1{{\sc\ref{#1}}}
\def\enum#1{\resume{enumerate}\item #1\end{enumerate}}
\def\ditem#1#2{\begin{enumerate}[resume]\item \label{\mychapter:#1} #2\end{enumerate}}
\def\nitem#1#2{\begin{description}\item[#1\label{\mychapter:#1}] #2\end{description}}
\def\bref#1{\ref{\mychapter:#1}}
\def\dref#1{(\ref{\mychapter:#1})}
\def\drefglobal#1{(\ref{#1})}
\usepackage{graphicx} % support the \includegraphics command and options
\usepackage{array} % for better arrays (eg matrices) in maths
\def\Not{\operatorname{\sim}}
\def\St{\operatorname{St}}
\def\num{\operatorname{num}}
\def\And{\mathrel{\&}}
\def\Or{\vee}
\def\BigOr{\bigvee}
\def\<{\langle}
\def\>{\rangle}
\def\union{\cup}
\def\nleq{\not\le}
\def\N{\mathbb N}
\def\R{\mathbb R}
\def\C{\mathbb C}
\def\Powerset{\mathcal P}
\def\star#1{{}^*#1}
\def\hN{\star{\N}}
\def\hR{\star{\R}}
\def\Z{\mathbb Z}
\def\Power{\mathcal P}
\def\Implies{\rightarrow}
\def\True{\operatorname{True}}
\def\Socrates{\mathrm{Socrates}}
\def\actual{@}
\def\Law{\operatorname{Law}}
\def\Chance{\operatorname{Chance}}
\def\Var{\operatorname{Var}}

\def\H2O{H${}_2$O}

\def\scr{\mathcal}
\def\e{\varepsilon}
\def\eps{\varepsilon}
\newtheorem{lem}{Lemma}
\newtheorem{prp}{Proposition}
\newtheorem*{theorem}{Theorem}
\newtheorem{corollary}{Corollary}
\newtheorem{cond}{Condition}

\renewcommand\thechapter{\Roman{chapter}}

\def\chaptertail{\ifdefined\book\else\end{document}\fi}
 

\title{Infinity, Causation and Paradox}
\author{Alexander R. Pruss}
%\date{} % Activate to display a given date or no date (if empty),
         % otherwise the current date is printed

\begin{document}
\setcounter{secnumdepth}{3}
\setcounter{tocdepth}{4}

\end{document}
\fi

\restartlist{enumerate}

\chapter{Evolution, Harmony and God}\label{ch:God}
\section{The origin of the forms}
\subsection{Evolution and forms}
We have good empirical reasons to think that the variety of biological structures that fills our planet 
is largely or completely the product of unguided variation together with natural selection. However, as
I have argued, there are good philosophical reasons to think that the organisms with these structures
have normatively laden forms which specify how the organisms should behave, endow them with the causal
powers that make that behavior possible, and impel them towards that behavior. 

It is implausible to think that the forms supervene on the biological structures. For instance, one theory
of the evolution of wings for gliding is that small wings are useful for heat dissipation. Larger wings allow
for more dissipation of heat, but are also more expensive for the organism to maintain. However, at around
size at which the heat-dissipation benefits are outweighed by the maintenance costs, the wings also become
useful for gliding. It is plausible that a species $A$ that has the smaller wings has them with the telos of
heat dissipation. But a species $B$ that has evolved the larger wings has them with the telos of gliding, either
instead of or in addition to heat dissipation.??ref,check But we can now suppose a member of $B$ whose wings are defective
and only good for heat dissipation. Such a member's biological structure might be largely indistinguishable from
that of a normal member of $A$, and yet it is normatively different: such wings are defective in $B$ but entirely
appropriate in $A$. If these norms are grounded in forms, it seems there is a different form in members of $B$ than
of $A$.

In general, in the evolutionary process, we expect small transitions in genetically-based biological structure 
between parents and children, with no change between the parent's form and the child's form. For if we had constant change
between the parent's form and the child's form, our best account would be that the form simply matches the
biological structure, which would not allow for genetic defects, and yet genetic defects---deviations of genetically-based
biological structure from the kind norms---are clearly possible.  Moreover, it is important to our ethics
that all human beings, despite a wide variety in physical and mental endowments---including the striking biological
difference between male and female---are beings of the same kind. 

We thus need an explanation of why it is that at certain apparently relatively rare and discrete points in the evolutionary 
sequence we have a new form on the scene. This itself yields Mersenne questions: while some transitions of form might happen
to coincide with a particularly striking genetic transition, we expect a number of them to come along with only minor
genetic transitions, seemingly at arbitrary positions. What explains these transition points?

Hitherto in this book, such questions were answered by invoking the forms themselves. And this can be done in this case
as well. We might suppose that the form of species $A$ endows the members of $A$ with a causal power to generate new
members of $A$ in some circumstances, together with new instances of the form of $A$, but also a causal power to generate
new members of $B$ in other circumstances, along with new instances of the $B$ form. The difference in circumstances could be
determined by the DNA content in the gametes joining together, so that when a descendant is going to have such-and-such DNA 
contents, the descendant gets the form of $A$, but with other DNA contents, the descendant gets the form of $B$. 

This story requires the forms to contain intricate specifications of which form is generated when. Granted, the slew 
of Mersenne questions we have already raised should make us circumspect about balk at mere complexity of form.
But now observe that the story as given above requires that the first biological organism on earth---presumably
some simple unicellular or maybe even proto-cellular?? organism---contain within it a form that codes for the causal
power to produce forms of all possible immediate descendants of it. These immediate descendant forms then would have 
to code for the causal power to produce all their possible immediate descendants, and so on. Thus, the
form of the first and simplest organism would implicitly code for all the forms of life that would ever actually be
found on earth, and indeed all the forms of life that \textit{could} ever descend from it.\footnote{It is tempting to
say that the number of possible descendant forms is infinite, but that is not clear. After all, there could be some
physical limit to the size of the genetic code fo a biological organism given our laws of nature. But in any case,
finite or not, the number of possible descendant forms is incredibly large.} We thus have here a dizzying complexity.

???few species story!??? no help, still have complexity

But the problem does not stop here. For we can now ask where that immensely sophisticated form of the first organism comes
from? If we say that it comes from the causal powers of non-living substances, such as fields or fundamental particles,
then we have to posit an even greater complexity in the forms of these non-living substances. The result would be highly
counterintuitive, by supposing non-living things to have immense sophistication of form. Further, however, we would need a 
story of where the first forms arose from. If we take the above account to its logical conclusion, then at the Big Bang
we would already have particles or fields whose forms implicitly included the vast formal complexity of all physically
possible living organisms. And this in turn yields a powerful design argument. For the idea that such complexity would
simply come about for no reason at all is utterly implausible. 

Thus, the story that forms contain the rules for the generation of future forms points towards a being whose own power is
sufficiently great to generate such forms. And to avoid a vicious regress, such a being would need to be a necessary one.

But note that once we have accepted the existence of a necessary being that is the ultimate source of the varied forms 
in our world, we can now tweak the story to avoid the implausible idea that unicellular organisms implicitly code for
the forms of elephants and unicorns. Instead of supposing that the transitions between forms corresponding to certain
selected changes in genetic structure are caused by the parent forms, we can suppose that the necessary being is directly
responsible for the transitions of forms. On such a view, the form of a unicellular organism might only endow its
possessor with the ability to generate a descendant of the same kind, and the necessary being would directly produce
any new forms when it is appropriate to do so.

\subsection{Reasons for creating forms}
Of course, this would lead to the question of \textit{why} the necessary being produces new forms when it does so.
Here, taking the necessary being to be rational can help. For there can be good value-based reasons for the transitions
to fall in some places rather than others. 

Consider, first, an odd thought experiment. A horse-like animal comes into existence with an maximally flexible form such that
whatever the animal does fulfills the norms in the form. To eat and grow is one proper function, and to starve and produce
a corpse is just as proper a function. Whatever our ``flexihorse'' does or undergoes is equally good for it. But there is something unsatisfactory about the flexihorse as a creature. If whatever the flexihorse does is equally good for it, then the fact 
that the flexihorse flourishes is just a direct and trivial consequence of its externally imposed form rather than the individual's 
\textit{accomplishment}. 

Reflection on this suggests there is a value in creating organisms that can fail to fulfill their norms. This value might be
grounded in the forms themselves: it might be that real horses, unlike flexihorses, have self-achievement of flourishing among the
proper functions in their form. And there is a value in creating organisms that have additional types of good written into their
form, including such self-achievement. Alternately, one might hold that in addition to kind-relative goods, there may be
kind-independent goods---perhaps grounded in imitation of the creator??forwardref?---and self-achievement of flourishing
could be one of these.

Either way, a rational being creating organisms has reason to create organisms that can fail to achieve their form, and hence
has reason to create beings with less flexible norms. Moreover, there appears to be a comprehensible value---again, either 
kind-relative or kind-independent---in production of beings of the same kind. As an intuition pump here, think of the
\textit{Symposium}'s idea that the yearning for eternity is exemplified in animal reproduction. Thus, we can give a value-based
explanation for why a necessary being would create beings in discrete kinds, with norms that the beings need not live up to.

\section{Explaining harmony by natures and evolution}
\footnote{??This section owes much to discussion in my mid-sized objects seminar, and especially to Christopher Tomaszewski's suggestions on the explanatory powers of forms.}
\subsection{Number of natures}
\subsection{Nomic coordination}
\subsection{Fit to DNA and niche}
??ethical, organic and epistemological

\subsection{Nature zombies}
Aristotelian metaphysics allows for a curious hypothesis: nature zombies. Nature zombies are macroscopic entities that
have the same physical structure as real organisms---say, oak trees or humans---but have no nature as whole (their
physical constituents may have physical natures). A nature zombie would lack mind, and would have no intrinsic 
normativity (unless there is some in their physical constituents), and above all would not even be a substance,
but a mere heap of constituents.

We can ask several questions about nature zombies. First, there is an epistemological one: how can I tell that
all the apparent organisms around me aren't nature zombies? (I can tell I am not one, because I can tell that I have 
a mind.) Second, assuming it is indeed so, why is it that there aren't any nature zombies around? Third, and perhaps
most deeply, why isn't it the case that \textit{every} apparent organism is a zombie---i.e., why are there any 
macroscopic things with natures?

The epistemological question is easily handled in the same way that other skeptical hypotheses are. The Aristotelian
can simply say that it is a part of our nature as the specific kind of reasoners we are that we should dismiss skeptical
hypotheses. We don't need to (and couldn't) tell the real organisms from the zombies to justifiably think the General
Sherman Tree, Seabiscuit and Biden to be (or have been) real organisms. We just need to be able to tell the organisms
from the rocks and the like, which in these three cases is easy (though it is appropriately hard if we turn to viruses).

What about the explanatory questions, however? It seems surprising, after all, if a bunch of physical constituents make
an organic substance with a rich normatively laden form. Given the fact of abiogenesis---that life arose from non-life
about 3.5 billion years ago---we might expect to have a nature zombie world. The zombies could be expected to evolve just 
as well as the normatively-laden organisms that (assuming the arguments of this book are sound) we have around us.

An Aristotelian move would be to suppose that the physical constituents of the universe themselves have natures, and it 
could be that their natures have the causal power, when the physical constituents are rightly arranged, to produce a living 
substance, and lack the causal power to produce a zombie. \textit{Prima facie}, the chemicals in a primordial soup could 
produce one of three outcomes: a soupy mess, a zombie organism, and a real organism.  But it could be that their causal
powers are so restricted that a zombie organism is beyond their power---it must be either a soupy mess (the typical outcome
of mixing the chemicals) or a real organism with form and all. And the organism, in turn, could be the common ancestor of 
all the other organism-like entities. ???

\subsection{Exoethics}
\subsection{Aquinas' Fourth Way and the good}
Aquinas' Fourth Way??ref puzzles the modern reader. It begins with a principle that comparisons between
degreed properties are grounded in a comparison to a maximal case: one is more $F$ when one is more like
the item that is maximally $F$. Aquinas then illustrates the principle with the case of heat and fire:
an object is hotter provided that it is more akin to the hottest thing, namely fire. He then applies
the principle to goodness, and concludes that there is a best thing, and this is God.

The fire illustration is not just unhelpful to us, since we know that fire is not the hottest thing (the sun is almost
twice as hot as the hottest flame), but it is actually a conclusive counterexample to the degree property principle,
since we can easily compare temperatures without reference to an alleged hottest object.\footnote{In any finite universe,
presumably there will be a hottest object. However, temperature comparison is not defined by that object, since 
even if Bob is in fact the hottest object, we would expect it to be physically possible to have a hotter object 
than Bob. But if degrees of heat were defined by closeness to Bob, it would not be possible to be hotter than
Bob, since nothing can be closer to Bob than Bob.}

So Aquinas' comparison principle is false. But I contend that there is still something to his argument
when applied to the good. 

Now, a form-based metaphysics gives a powerful account of the good for a being of
a particular kind---an oak, a sheep or a human, say---in terms of its match to the specifications of the
form. It also gives a ground to comparisons between the good of different instances of the same kind:
a four-legged sheep is, other things equal, better at sheepness than a three-legged sheep, because it
more completely fulfills the specification in their ovine nature. In fact, this is itself a counterexample 
to Aquinas' comparison principle, in that we can compare degrees of success at sheepness without supposing
any individual sheep to be perfect.

However, in addition to value comparisons within a kind, there are ones between kinds. When Jesus says
that we are ``worth more than many sparrows'' (Mt.\ 10:31??ref), what he says is quite uncontroversial.
Indeed, even a perfect sparrow seems to have less good than a typical human. While the nature of a sparrow
will enable value comparisons between sparrows, and that of a human between humans, we still have the question
of what grounds the value the difference between sparrows and humans. Some Aristotelians reject cross-kind 
value comparisons as nonsense.??refs But given the intuitive plausibility of many such comparisons, this rejection
is a costly one.

Aquinas' Fourth Way is not infrequently seen as more Platonic than his other arguments for the existence of
God, and Plato indeed had a solution to the problem of cross-kind comparisons, by talking of differing degrees
of imitation of the Form of the Good, which itself is perectly good. Plato, on the other hand, lacked a 
satisfactory solution to the problem of intra-kind comparisons. He may well have thought that there 
was a Form of Humanity??refs, which exemplified humanity perfectly, so that similarity to the Form of 
Humanity would define how good one is at being human. However,
we can see that this solution is clearly unsatisfactory. First, the Forms are immaterial, so the Form of Humanity is 
immaterial, and hence it lacks fingers. Thus, the fewer fingers a human has, the more they are like the Form of
Humanity, and hence, absurdly, the more perfect they are. Second, if somehow the Form of Humanity ends up having 
body parts, then the Form of Humanity either has an even number of cells or an odd one. But clearly neither option
is more perfect than the other. 

Central to Plato's solution to cross-kind value comparisons is the self-exemplification of the Form of the Good:
the Form of the Good is itself maximally good. But a similar self-exemplifying Form cannot be used to account for
intra-kind comparisons. Aristotle, on the other hand, has the non-self-exemplifying forms immanent in things. 
The Aristotelian form of humanity specifies human perfection, but does not do so by exemplifying it. It has neither
fingers nor cells, but it \textit{specifies} that humans should have ten fingers while specifying an age-dependent
normal range of cell numbers rather than a specific cell count.

Notwithstanding the general falsity of Aquinas' comparison principle for degreed properties, Aquinas provides us
with a plausible extension of the Aristotelian system to allow for comparisons of degrees of good between objects
of different kinds in terms of the similarity to or degree of participation in a maximally good being, a divine being
that plays the role of a self-exemplifying Form of the Good. The human being participates in God in respect of
abstract intellectual activity, Aquinas will contend, while sparrows do not, and in that important respect, at least,
humans are more like God. On the other hand, the sparrow's movements approximate divine omnipresence better than 
the stillness of a mushroom does, and in that respect at least the sparrow is superior to the mushroom. We have,
thus, a ground for something like a great chain of being.

There are still difficulties here. While the human is superior in intellectual activity, the sparrow moves around
with greater three-dimensional freedom. How can we say that the human is superior all things considered? Where we
previously had a problem of cross-kind comparisons, we now have the problem of cross-attribute comparisons. 
Intuitively, the human's intellectual superiority to the sparrow trumps the sparrows motive superiority to the
human, and enables us to say that the human is more perfect on the whole. This higher level question is difficult
indeed. 

But there is some hope in thinking that in attributing different divine attributes we sometimes express divinity
to different degrees. It may be that there is no meaningful comparison between how well we express divinity by
saying that God is all-knowing versus by saying that God is all-powerful, saying that God knows the
multiplication table up to $10\times 10$ expresses divinity less well than saying that God can create any
possible physical reality. I suggested earlier that motion imitates divine omnipresence. Thus, the sparrow's
ability to fly imitates God's presence throughout several kilometers surrounding the surface of the earth, while
the human's more limited mobility imitates God's presence in a thin two meter shell of air surrounding that surface.
But the degreed difference between the two divine attributes---each a limited special case of omnipresence---imitated here 
might well be trumped by the fact that the sparrow does not imitate God's abstract intellectual activity \textit{at all}
while the human does imitate that activity, and does so in respect of a very wide scope of things (the human can think
abstractly about the whole universe, for instance). 

In ??backref, we gave a non-theistic Aristotelian sketch of a three-step great chain of being. The account here has
a hope of allowing one to fill in more intermediate links.
Even if the details in the comparisons between different attributes or respects do not work out, we still have an advantage 
for the theistic Aristotelian in being able to make cross-kind comparisons under specific respects, like motility or intelligence.

??ref:Jeffrey/Ward


\subsection{Epistemology of normativity and form}\label{ch:epist-of-form}
[Argument: If a guided missile has form, it's alive by the Ch?? account of life. But it's not alive. So it lacks form. Is this a bad argument???]

\subsection{Ethics and happiness}

\subsection{Modern technology and outlandish scenarios}
In ??backref, I argued that an ethics based on human form can simply ignore outlandish scenarios
that are far outside of our ecological niche, such as ones involving infinite numbers of
beneficiaries. However, there is a danger in this line of reasoning. As Arthur C. Clarke famously
said, ``Any sufficiently advanced technology is indistinguishable from magic.''??ref To human beings
50,000 years ago (or even just 500 years ago!) much of our technology would indeed be magical, and 
decisions that we routinely need to make, say in bioethics, would be predicated on outlandish assumptions. 

We might thus expect an ethics and epistemology grounded in a form possessed by hunter-gatherer primates
to be silent on dilemmas of a highly technological society, leaving us to do whatever we wish, or, even worse, 
to fail to harmonize with the shape of our lives, like that of a fish on land. Yet while there are, as there 
have always been, difficult and controversial moral and epistemological cases, we do not in fact find 
ourselves adrift without guidance in the modern world. Virtue continues to contribute to our flourishing,
and ancient texts, whether religious or philosophical, continue to point to good ways of living. 

This gives us reason to think that if our moral norms are grounded in human nature, human nature was somehow
picked out with foresight for what kinds of challenges humans would face in the distant future. Our ethics
does may not work in outlandish situations, such as those involving infinities as noted in ??backref, but it works in a
broader range of moral environments much broader than that found in early homo sapiens society. Thus, the theistic
version of our natural law theory both accounts for the apparent unsatisfactoriness of our ethics in situations that
humans apparently never find themselves with and the applicability in situations across a very wide range of situations,
wider and more technologically varied than the natural environment of other animals. This kind of foresight points to
a foreseer, indeed a designer, and hence towards a theistic version.

The move I suggested in ??backref for outlandish scenarios, namely that our ethics and epistemology simply does not apply
to them, may seem problematic given that we live in a world where many things that our not-too-distant ancestors would have
seen as outlandish are real. We fly regularly around the world and irregularly to the moon, speak with people on the other
side of the planet, move organs from one person to another, make cats glow by inserting jellyfish genes, program machines 
to have conversations with us, have bombs that can wipe out most megafauna including humans, and can clone at least embryonic 
humans. Our capabilities and the situations that we are in are quite different from those we evolved for. And further changes
may be facing us. Many think there is a serious possibility that human beings will spread through the galaxy, affecting vast 
numbers of lives, which may make actual seemingly outlandish questions about where our actions have very slight probabilistic 
effects on vast outcomes.??ref:Fanaticism,glitchy ethics

One approach to these modern questions is a principle-conservativism: the questions are settled by moral principles that we accepted
for millenia. Because the default for an action is moral permissibility, in the case of qualitatively new kinds of actions, 
principle-conservativism is apt to lead to a radical expansion of moral possibilities. If we had no principles governing human DNA
manipulation in the past, even if this was simply because we had no concept of DNA, now there are no limits, except limits coming
from traditional principles of harm and consent. Principle-conservatism in these kinds of cases would paradoxically justify a vast 
change in the shape of human lives of a sort that arguably is not compatible with human flourishing. Conversely, however, we have 
cases like the invention of effective therapeutic surgery. Prior to these, any serious degree of cutting open of the human being would be an instance of grave harm, and generalizing from
those cases to therapeutic surgery would have been unfortunate for the human race. 

A more naturalistically inclined Aristotelian could despair about ethics in quintessentially modern situations. Absent foresight from a God
or an axiarchic principle, we should not expect our natures to provide guidance in these situations, or at least ``non-glitchy''
guidance??backref. This line of thought could lead the Aristotelian to a dark view on which there just is no answer to a number
of contemporary moral questions, or on which the answer conflicts in a glitchy way with our moral intuitions, or perhaps
even one on which the true moral norms conflict, and we get hard-to-avoid moral dilemmas. 

But a theistic story can restore optimism. God can know what kinds of seemingly outlandish scenarios might
actually be relevant to the lives of his creatures, and can wisely choose the forms whose norms that fit with these. The resulting norms may 
seem \textit{ad hoc}, especially in edge cases: they won't be the elegant principles of classic utilitarianism (though of course classic utilitarianism
faces significant difficulties in out-of-our-experience situations, as we saw in ??backref:population-ethics). And an apparent
\textit{ad hoc} character in divinely-instituted rules as applying to edge cases should not surprise us---wise legislation does not eschew 
judgment calls.

An interesting question is whether a theistic Aristotelian should be surprised by having ethics glitch in some actual cases, in one of 
the three ways discussed in ??backref: (i)~real dilemmas, (ii)~conflict between moral rules and the reasons for them, and (iii)~conflict between
moral rules and our intuitions. After all, logical space contains an infinite number of possibilities for a form of a rational 
being, and a perfect being should be able to combine one with an environment in which there would be no actual glitches. 

I grant that it is very plausible that a perfect being \textit{could} do that. But would the perfect being do it? Even for a being
whose power is unlimited by anything other than logic, there can be unavoidable costs to options. Intuitively, there is a value to
the most important norms of behavior for limited beings\footnote{Why limited????}---say, norms governing killing---having a significant simplicity, so that
they can be reasoned about more easily, especially under time pressure.  At the same
time, there is a value to a diverse and rich moral environment. And there is a value to morality lacking glitches. Plausibly,
one cannot have all three values to their maximal degree at the same time---there may be logically unavoidable trade-offs. 
And there does not appear to be strong reason to think that a perfect being would be so enamoured of one of the three value that
we would expect that value to be present to the maximal degree. In particular, we should not expect a completely unglitchy ethics.

But we might have reason to hope that glitches are rare in the actual circumstances faced by humanity, or that the worst of the
glitches should only occur in the case of agents who have wrongfully produced the circumstances for this glitching.\footnote{Compare the theory that real moral
dilemmas only occur in the case of agents who have done wrong, say by making contradictory promises.}

??hypothetical judgments



\subsection{Avoiding radical scepticism}
There is a number of sceptical hypotheses that have the property that they cannot be ruled out either on logical grounds
or \textit{a posteriori}. These include hypotheses that the world around us is a computer simulation, that our moral
intuitions are disconnected from moral reality, that we are Boltzmann brains, i.e., short-lived brains in bubbles of 
oxygen arising from fluctuations in the vacuum of space, that we live in an infinite multiverse that undercuts all
probabilistic reasoning??ref, that simpler scientific theories are more often right other things being equal, and so 
on. Yet we think these hypotheses false. If we think them false neither on logical nor empirical grounds, it must be 
because we assign low probabilities to them prior to empirical assessment. Moreover, this assignment is required by 
our rationality: those who fail to assign low probabilities to them are irrational.

Our Aristotelian account can ground the correctness of this judgment of irrationality in human nature.??cf.backref 
But there would be something deeply problematic about us if the low \textit{epistemic} probability of the sceptical hypotheses 
were not matched by a low \textit{objective chance} for them to be true in light of the causal and stochastic structure of the world. 
If in fact the most likely way for a being with our rational nature and mental life to arise would be as a Boltzmann
brain, then even if we have lucked out and are not a Boltzmann brain, there is a disharmony between the world and our
mental life. 

In such a lucky case, the connection between our nature-required priors and the world then appears too fragile, chancy and ``unsafe''??ref
for the beliefs essentially dependent on these priors to count as knowledge. Even a reliabilist should say that if some beings
were required by their nature to assign a very high prior probability to the hypothesis that the universe formed an even number 
of years before life first arose, and it was mere chance that this hypothesis was true with the causal structure of reality
not assigning it a higher probability than the hypothesis of an odd number of years between the beginning of the universe and
the beginning of life, then that hypothesis is not knowledge. Yet it is very plausible that we \textit{know} the sceptical hypotheses
under discussion to be false. (This judgment has admittedly been disputed by a number of epistemologists who admit with G.~E. Moore that
I know that I have two hands, but will not allow the Moorean inference that I know that I am not a brain in a vat, despite the fact that I have two hands
obviously entailing that I am not a brain in a vat. ??ref) And even without considerations of knowledge, we might note that an optimism
resting on an assumption of mere luck appears paradigmatically irrational. 

Aristotelian thus need a theory on which there is the right kind of connection between our rational nature and the causal and 
stochastic structure of the world.

\subsection{Global aesthetic-like features}\footnote{I am grateful to Nicholas Breiner for drawing my attention, in the context of
justice, to this form of explanation of moral features.}
\subsection{Family}
\subsection{Retributive justice}
\subsection{Divine authority}
\section{Kind-independent goods}
Aristotelianism does really well with explaining kind-dependent values. But there also appear to be values that appear
to transcend kinds, such as simplicity, diversity, flourishing, achievement, etc. Furthermore, we can compare kinds.
The Aristotelian account defended in previous chapters can ground the comparison between a flourishing and a non-flourishing 
human, or between a flourishing and a non-flourishing chantarelle mushroom. But it is also obvious that a human is a better
kind of entity than a mushroom.
????

\section{Complexity and explanation}\label{sec:hierarchy}
\subsection{A problem}
A central form of argument for Aristotelianism is based on Mersenne questions and the messy normative complexity of our
lives. But do we not normally prefer simpler theories to more complex ones, and hence should we not reject the normative
complexity in favor of a simpler theory like utilitarianism in the name of Ockham's razor?

Ockham's razor, however, has always been a defeasible criterion: entities are not to be multiplied \textit{beyond necessity}.
But sometimes there is necessity. It would be simple to suppose that all trees of a single species look exactly the same.
But that just wouldn't fit with our evidence. In biology, one does not expect individuals of the same sort to be exactly
alike, unlike in fundamental particle physics. 

Specifying what a rational animal of a particular species ought to be like
and how it should behave can be expected to involve a lot of information. How much information? Well, we might take the information contained
the DNA common to all humans to give us a lower order of magnitude bound, since the common DNA presumably encodes something
about what human bodies are supposed to be like. There are 3.2 billion base pairs in human DNA, and 
99.1% of human DNA is said to be common to all humans (??ref https://www.sciencedirect.com/topics/biochemistry-genetics-and-molecular-biology/dna-profiling).
Since each base pair is two bits of information, that means about 6.3 billion bits, or the equivalent of about 500,000
book pages.\footnote{Counting a page at 1800 characters and each character at seven bits: $6.3\times 10^9/(1800\cdot 7)=500000$
(oddly exactly, by coincidence!).} 

Imagine the task of designing the rules of behavior for a rational animal that has a significant complexity in its bodily
life, subject to the constraint that the rules lead to a life that elegantly balances moral and epistemic norms, and fits
well with the bodily nature of the animal and its niche in the ecosystem. It is plausible to think this will be several
orders of magnitude more complex a task than that of designing the rules for a well-balanced and significantly embodied game 
such as tennis. Generating a game of pleasing elegance and yet compelling complexity, especially an embodied one, takes a 
fair amount of information, and the official rules for tennis are about forty pages.??ref 

We can think of simplicity as an aesthetic criterion in theory choice. But simplicity is not the only factor contributing
to beauty! (If it were, the most beautiful art would be no art: you can't get simpler than an installation that can be 
completely described by $\sim\exists x(x=x)$.) Overall theoretical simplicity is one way of having an elegantly unified
theory. But one can also achieve elegant unification in other ways. Consider, for instance, hierarchical organization. Wittgenstein's
\textit{Tractatus}??ref achieves a unification by being summed up in seven top-level sentences, with a progressive hierarchical
amplification and justification in terms of multiple levels of sentences. Or consider the unification achieved in biology by
Linnaean and Darwinian taxonomies.

We could have an ethics that is simply simple: it has a briefly expressible rule that covers everything in full detail. But 
just as it is unlikely that we would get a compelling racquet sport with a single brief rule, even if we allowed for some
vagueness, it is unlikely that we would get a harmonious set of norms for the life of a rational animal out of such a rule---that, 
indeed, is an upshot of the enumeration of the many Mersennian issues of detail in normative phenomena that have been discussed
in this book. 

Can we have some other kind of theoretical unification? I think we can. As discussed in connection with particularism(??did I??backref),
we can suppose suppose a hierarchical structure. We can have a hierarchical ethics, at the top with one or more principles like Aquinas's ``Pursue the good
and avoid the bad'', the Kantian injunction to treat others as ends rather than mere means, or the Biblical ``Love your neighbor as yourself''.
But perhaps unlike the historical Kant, we need not take the top level principle or principles to have all of the normative informational 
content for morality. Instead, we can think of it as a unifying headline, perhaps to explaining tennis by saying: ``Hit the ball back into the
other player's side.'' There will be further rules that are not mere logical derivations, but build on the general principle expressed in 
the higher level rules by giving more specific rules. The second level rules themselves are not unlikely to need further adumbration.

Consider Hillel's famous response to the request that he explain the Jewish law while standing on one leg: 
\begin{quote}
That which is hateful to you, do not do to your fellow. That is the whole Torah, all the rest is commentary. Now, go and learn it [the commentary].
??ref:add scholarly translation
\end{quote}
Now it is clear that in fact that the primary Jewish commentaries on the five books of the Torah (i.e., the Mishnah, and the
Talmuds which are commentaries on the Mishnah??check,refs) contain normative material not found in these books. Nor should this be 
a controversial claim, since rabbinical tradition holds that the rabbis have an oral tradition going over and beyond the books of 
the Torah. Thus we should probably interpret Hillel as saying that the Golden Rule (in his negative formulation) is a kind of summary,
rather than as saying that the rest is logically derivable. Similarly, Aquinas, after giving his ``first precept''??ref that good is to be done and evil avoided, lists second level laws such as
preserving human life, respecting the reproductive life of us a rational animals, knowing the truth (especially about God), and living 
in society. It is clear that there we still have not reached the level of normative information needed to resolve all moral 
questions.

In both the Hillel and Aquinas cases, we have a unification of ethics under one or more general principles that are insufficient
for deriving all the specifics. This is akin to the explanatory unification that modern biology receives from evolutionary theory. 
Besides generalities like that species tend to mutate towards inclusively fitter forms, the basic principles of evolutionary theory---random 
variation and the survival of the fittest---do not generate specific predictions. However, they do organize the vast sphere of modern
biological knowledge. 

Famously, Aristotle has observed that in ethics, unlike in geometry, one can only speak in ways that are true for the most 
part.\footnote{??ref. Of course, this claim itself needs to be carefully understood. Aristotle himself says that murder and 
adultery are always wrong. Perhaps he is thinking that murder and adultery are definitionally wrong---murder being a wrongful 
killing and adultery being sex contrary to respect for marriage?} On the account I am defending, this is not quite right. Instead,
many of the higher level ethical claims are what one might call ``generalities'' that organize ethical reasoning. These claims 
can be exceptionlessly true, but there are limits to how helpful they are in particular cases. Thus, it may always be true that we 
should respect human life, but this does not give a clear answer as to what the health care provider should do when the family of 
a particular patient requests disconnection from life support. The respect claim describes, in general terms, the shape that the 
finer-grained principles have. And it may well be that the finest grained principles which apply to certain particular cases
have a complexity beyond our practical ability to specify, and so we do not have principles that definitively settle a case.

While I have used ethics in the above discussion, the same plausibly applies to epistemic rationality, where we have a very general
principle like ``Pursue understanding (or knowlege or truth)'', with finer-grained specifications such as ``Avoiding error is more
important than getting at truth'', ``Prefer elegant theories'' and ``Direct your attention to more important matters.'' In the case
of semantics, we may, on the other hand, have a high level principle that ``Meaning follows usage'', and then a variety of finer-grained 
principles about how usage yields meaning. As we get to very fine-grained principles, we have an extremely complex account, but hierarchically
organized.

\section{Explanation of our normative complex}
\subsection{A pattern of explanation of norms}\label{sec:moral-explanation}
Here is a familiar pattern. We have a deeply-seated moral intuition about the general prohibition, call it $g$, of some action,
such as incest. It is not clear how to derive the prohibition in its full generality from intuitively more basic principles, such as one of 
the categorical imperatives. Easy considerations, which I will call the $c$s, show that in \textit{typical} cases the action is wrong,
but our moral intuition goes beyond these typical cases. Thus, considerations of the abuse of power, distortion of 
familial dynamics, and genetic harms show that most cases of incest are wrong, but it is easy to imagine cases
of incest to which these considerations do not apply---say, elderly siblings who were raised apart---and yet moral
intuition forbids incest in those cases as well.

We can now save the moral intuition by saying that the more general prohibition $g$ is simply a fundamental moral rule,
not reducible to the $c$s that explain why the action is wrong in typical cases. But if we stop
at this, the connection between $g$ and the $c$s mere happenstance, and that seems intuitively wrong. The abuse of 
power, distortion of family dynamics, and genetic harms should be relevant to why incest is wrong.

At this point, often we are in a position to see another fact: it is quite beneficial to have a 
general moral prohibition beyond the prohibitions arising from the $c$s. 

One reason for such a benefit from a general prohition could be that our judgment as to whether the $c$s apply to a given case is fallible, especially given our capacities
for self-deceit, and the costs of violating the $c$s are so high that it would be better for us to have a 
general prohibition than to try to judge things on a case-by-case basis. 

Second, in some examples of
the pattern, serious deliberation about the forbidden action can itself harm one or more of the goods
involved in the $c$s: thus, having to weigh whether the distortion-of-family-dynamics consideration
applies against a particular instance of incest can itself distort the agent's participation in family
dynamics. 

Third, we could have a tragedy of the commons situation. It could be that the $c$s are actually insufficient
to render an instance of the action wrong, but we would be better off as a society if we had general
abstention from the action. Thus, perhaps, the genetic harm coming from one more couple's engagement in incest
would be insufficiently significant to render the incest wrong, but without a general prohibition, incest
would be sufficiently widespread as to cause serious social problems. A general prohibition that is not
logically dependent on the $c$s would help avert such social harms.

These considerations are very familiar to us in the case of positive law. Jaywalking involves harms such as
disruption of traffic flow and the danger of death of the pedestrian and of trauma to the driver, and the
considerations of these make jaywalking wrong in typical cases. There are 
obvious instances, however, where these considerations do not apply: say, crossing a road where the pedestrian
can clearly see that there are no intersections or cars on the road for a significant distance in either 
direction. However, it may be better for people simply to abstain from jaywalking than judging whether the 
disruption and safety considerations apply on a case-by-case basis, because there could be so much harm if 
the judgment were to go wrong. As a result, it is can be reasonable for a state simply to ban jaywalking
altogether (or to ban it with some clear and easily adjudicated exceptions). We similarly resolve cases of
tragedy of the commons with positive law: think, for instance, about laws against littering.

In the case of positive law we have two different explanations. First, there is an explanation of why
the forbidden action is wrong in general: this is because it has been competently forbidden by legitimate authority.
This explanation need not make reference to considerations such as disruption of traffic flow or danger
of death.\footnote{Though in some cases \textit{some} such reference may be needed in order to establish
that the matter falls within the competence of the authority in question. Thus, a government agency may
be permitted to make rules on matters where traffic flow disruption is concerned.} Second, there is an 
explanation as to why the action has been forbidden by the authority---and here all the rich considerations
are relevant.

\subsection{Theism}
A theistic version of natural law can have precisely the above pattern. An action is morally forbidden because
our nature is opposed to it, an instance of grounding explanation. This explanatory fact does not make reference 
to the $c$s. But we still have a further question to ask that it is natural to put in the form: ``Why does our nature 
includes this prohibition?'' But since our nature is essential to us, the answer to that question could 
simply be the necessary truth that we couldn't exist without this nature. However, we can put the question
in a different way: ``Why are there intelligent primates on earth with a nature that includes this prohibition
rather than some other kind of intelligent primates with a nature that does not include this prohibition?''
And here the theist can answer: Because it would be good, in light of the $c$s and the
further considerations in favor of generalizing the prohibition beyond the cases where the $c$s specifically
apply, to have intelligent primates with a nature that includes this prohibition, and God acted in light
of this good.\footnote{A divine command theorist can make the same move, but divine command theory has some
liabilities which were discussed in ??backref.}

The explanation may still appear viciously circular. On an Aristotelian metaphysics of value, what is good for us
is grounded in our possession of our form. How could the possession of our form, then, be explained by what is good 
for us? But it is difficult to see the difficulty in the context of theistic selective explanations. Whatever 
form will be exemplified will define what is good for its possessors. If God were to choose to exemplify a form
that defines one and only one state as good for its possessors but that also makes it nearly impossible to attain
that state, the result would be beings that almost universally are in a bad state. There is reason not to do that.
Instead, God has good reason to select a form that makes it much easier to attain the good state defined by the form.

Here we should make a distinction between the specific goods grounded in our form---health, friendship and the like---and
the good of fulfilling our nature. The specific goods are grounded in our nature. But it is not clear that we should say that 
the good of fulfilling our nature is itself grounded in our nature. It is plausible in the Aristotelian context to say that 
to be good for $x$ just \textit{is} to fulfill $x$'s nature. This identity is simply reductive. Given this, the circularity
in the explanation disappears. What specific things are good for us is grounded in our nature. But it is good for us to fulfill
our nature, and that fact is independent of what the nature is. It thus makes sense to explain \textit{which} nature is 
exemplified by considerations of how apt the possessors of that nature would be to fulfill that nature, and have that good.
We thus have a kind of explanation of why rabbit-like beings specifically have reproduction be good for them---having reproduction
be good for them is more apt for the fulfillment of their nature than, say, having the discovery of mathematical truth be good
for them. It is the general good of fulfillment of any nature that can explain why a nature with such-and-such specific goods is
selected.

In fact, we can have explanatory relations running both ways between goods and norms of behavior. If having norm $N$ promotes
some specific good $S$, then that could explain why a being whose form codes for $S$ being good also has norm $N$ of behavior.
But conversely, we could explain why a being whose form codes for norm $N$ also codes for $S$ being good for the being. 
If the norms are selected for exemplification by a perfectly good God, we may expect both forms of explanation to show up,
as well as a hybrid model where both $N$ and $S$ are chosen together for their fit.

This kind of divine selection explanation of both ethical norms and goods extends from ethical to prudential, epistemic and 
semantic norms. As we saw in ??backref, ethical, prudential, epistemic and semantic norms all interact in complex ways with 
what is good for us, and this interaction can provide God with reasons in favor of some and against other combinations of
norms and goods.

\subsection{Non-theistic alternatives}
What could such a selective cause be like? There are three main candidates for selective causes in the philosophical
literature: evolution, axiarchic principles, and intelligent designers such as God. 

Genetic descent with variation only directly governs the non-normative aspects of organisms. It is not sufficient to explain
the form that the organism has, given that the form encodes normative features as well. Nonetheless, it is worth considering
the possibility of a law of nature linking DNA to form, a law of nature specifying that when an organism with such-and-such 
DNA comes into existence, it has such-and-such a form. This law of nature would be immensely complex, with many free parameters
raising Mersenne questions. Moreover, if the law of nature is to cohere with the Aristotelian optimism that is crucial to our
Aristotelian account, there must be a fit among the various aspects of the form and between the form and the actual physical 
body plan and physical environment. It is implausible that this fit, in us and presumably in the myriad of other organisms,
is just a coincidence. Such a law of nature calls out for an explanation. On pain of vicious regress, the need to explain the 
law of nature points towards one of the other two explanatory candidates: axiarchic principles and intelligent designers.
Moreover, without a value-laden explanation of the linkage between DNA and form, the hierarchical explanations discussed 
in Section~\ref{sec:selection}, and the non-deductive hierarchical explanation of normative principles is replaced by a vast
coincidence, and hence we do not have a satisfactory answer to the complexity objection. 

Our Aristotelian account in order to be intellectually satisfactory requires an explanation that itself delves into some 
normative domain. This explanation could directly govern the imposition of forms or via some intermediary like a linkage
law of nature. 

At this point, our choice is between an explanation involving a non-intelligent tendency towards value and an intelligent one.

The main candidate for the non-intelligent tendency are axiarchic principles, such as those defended by Leslie and Rescher.??refs
These are fundamental metaphysical principles that require the world to be optimal. 

It is worth noting that while I am discussing axiarchism as an alternative to theism, Rescher himself takes his theory to
imply theism: it is better for there to be a God, and hence there is a God.??ref And if there is a God, then presumably this God
is sovereign and governs the selection of forms, and we can skip forward to the discussion of theism. Leslie's version also
involves supernatural beings. But Leslie thinks that what is best is not that there be one infinite all powerful, all knowing
and all good God, but infinitely many omniscient observers who enjoy the world thereby adding to its value, though without
creating it, since then there would be the possibility of conflict between them.??ref,check

There are four main problems with axiarchic principle explanations.

First, intuitively, a metaphysical principle \textit{constrains} what beings can exist and how they behave rather than somehow 
explaining the positive existence of beings. But the axiarchic principles are supposed to explain the existence of beings: the beings
in reality exist because it is for the best that they do so. 

Second, there does not appear to be a unique best world. We could take any good world and add one more happy disembodied mathematician.
This might not produce an overall better world. It might, for instance, be aesthetically inferior in some way---say, by having too many
mathematicians and thus offending against simplicity, or by having a non-prime number of mathematicians (perhaps the aesthetically best 
number of mathematicians is a prime of the form $2^n-1$ for some large $n$)---but it is superior in at least one significant way, namely 
by having an additional happy mathematician, and it is not plausible to think that it would be an overall inferior world.

Third, axiarchic principles appear to lead to modal collapse. If metaphysical principles require everything to be 
for the best, then it seems that everything must be the way it is. 

There are at least three potential ways out of the modal collapse objection. The first is Leibniz's solution who distinguished between 
moral necessity and logical necessity. A proposition is logically necessary, according to Leibniz, provided that there is a finite
proof of a contradiction from its negation. It is morally necessary provided that there is a finite \textit{or infinite} proof of a 
contradiction from its negation. There are infinitely many logically possible worlds (where as usual $p$ is possible just in case its negation
is not necessary) but only one morally possible world---the best world. It is logical modality, then, that answers to our intuitions about
the broad range of possibilities for reality.

Unfortunately, Leibniz's notion of logical necessity in terms of finite proof does not fit well with much later developments in logic and modal logic.
Consider a very weak version of Axiom S4 of modal logic. Axiom S4 says that \textit{any} necessary proposition is necessarily necessary.
Weak S4 says that \textit{some} necessary proposition is necessarily necessary. This seems utterly uncontroversial. For instance, surely, that 
everything is either green or not green is not only necessary, but necessarily necessary.\footnote{Weak S4 can be proved to follow from 
the Necessitation Rule, which says that if $p$ is a theorem, so is $\Nec p$, as long as the logical system is such as to have at least one theorem.
For if $p$ is a theorem, then $\Nec p$ is a theorem by Necessitation, and hence so is $\Nec\Nec p$.} From Weak S4 and uncontroversial axioms
of modal logic it follows that some proposition is necessarily possible.\footnote{Suppose $\Nec\Nec p$. By Axiom~T, $\Nec p \rightarrow p \rightarrow \Poss p$
is a theorem. By the Distribution Axiom, it follows that $\Nec\Nec p \rightarrow \Nec\Poss p$ is a theorem. Since we have $\Nec\Nec p$, by
modus ponens we have $\Nec\Poss p$.} But now a proposition $p$ is necessarily possible in Leibniz's sense of logical modality just in case 
there is a proof that it is possible. And $p$ is possible just in case there is no proof of $\Not p$. Thus, $p$ is necessarily possible
just in case there is a proof that there is no proof of $\Not p$. Now, in an inconsistent logical system, there is a proof of \textit{every}
proposition. Hence, a proof that there is no proof of $\Not p$ would be a proof that the logical system we are working with is consistent.
But as long as the logical system has a recursively enumerable??? set of axioms (and to deny that would not be in the spirit of Leibniz's
notion of finite proof), includes the axioms of arithmetic (the idea that the axioms of arithmetic could be false in some possible world
seems hard to buy) and is actually consistent, then by G\"odel's Second Incompleteness Theorem the system cannot prove its own consistency.
And hence it cannot prove $p$ is possible on the Leibnizian account of possibility, and thus does not make $p$ necessarily possible on that
account.\footnote{??discuss objection paper}

The second way out of modal collapse is to limit the scope of axiarchic principles to producing what one might call the best 
\textit{skeleton} for a world. Say that a skeleton for a possible world consists of all the explanatorily fundamental parts
of the world, such as the initial conditions and the laws of nature. As long as the laws of nature and/or causal powers of 
the initial beings are indeterministic, we could make only one skeleton possible, while yet having a multiplicity of possible 
worlds differing in how that skeleton evolves indeterministically into a fully fleshed out world. This will save our intuitions
about more ordinary possibilities: I might have forgotten to come to class today, you could have found my arguments more convincing
than they are, and the French could have emerged from World War II as the dominant world power. But forcing the laws of nature to 
be necessary is pretty counterintuitive. 

The third response is to allow for tied or incommensurable worlds, and say that the axiarchic principle requires \textit{a} best world,
but not \textit{the} best one. One might, if one wishes, also combine this with the skeleton move and let the principle require the world
to have \textit{an} optimal skeleton. This response would also answer our earlier objection that there is no such thing as the best world.
It is mysterious, however, how the axiarchic principle would then go about selecting which precise world or skeleton exists from among the
optimal ones. A principle is not a person who can choose between a set of incommensurable options, nor is it an indeterministic cause that
has a range of possible effects. ???

The final difficulty for axiarchic views is the problem of evil. Looking at the litany of suffering in human history, 
our world doesn't look like the best of all possible worlds. Axiarchic views can make use of many of the responses to 
the problem of evil given by theists. This vast literature is beyond the scope of this book.??refs

\subsection{Theistic choice points}
Suppose we are convinced that we need a theistic Aristotelianism.
At this point there are metaphysical and theological choice points. One metaphysical choice point is whether there are any
uninstantiated forms. If there are, as on a Platonic picture, then we have a theological question: Does God freely choose which ones
to create, or do they exist necessarily, say in the mind of God? If there are no uninstantiated forms, as on a more classically 
Aristotelian picture, then probably the most parsimonious theistic story is that God creates in the act of creating the substances 
that instantiate them. We thus have three views: Theistic Voluntarist Platonist Aristotelianism, Theistic Involuntarist Platonist Aristotelianism
and Theistic Classical Aristotelianism.

On the Platonist versions, the forms have some kind of uninstantiated mode of existence, in addition to the instantiated mode of 
existence they have in creatures. (I am assuming here that we have already decided in favor of individual forms---??backref. Perhaps,
though, on the Platonist versions that choice point should be revisited?)

The Voluntarist Platonism option may seem to have some unnecessary complexity. If God chooses which forms to create, it is 
puzzling why God would ``bother'' with the ones that aren't going to get instantiated. There is, however, a possible answer:
to open a field of possibilities to creatures. Perhaps the forms need to have some kind of Platonic existence in order for
creatures to have the power of producing their instantiations. There is a value to the earth ecosystem ``having a choice'', with
many evolutionary possibilities of what kinds of biological substances should exist, and on a more Platonic version of the 
metaphysics this could require the pre-existence of these forms in their uninstantiated mode. 

On both the Voluntarist Platonist and Classical versions, there is or can be a field of possibilities for other forms than 
the ones that actually exist. On the Voluntarist Platonist version, these are other forms that God could have created
\textit{ex nihilo} independently of instantiation.  On the Classical version, these are other forms that God could have 
instantiated and thereby brought not existence. Presumably, God knows what these possibilities are, and so they have some 
kind of existence as ideas in the mind of God. There is much room here for difficult metaphysical exploration of the 
exact status of these divine ideas.???many-refs Nonetheless, this point shows that there is a commonality between all three
versions of theistic Aristotelianism: there is a field of formal possibilities. On the Voluntarist Platonist and Classical
theories, this is a field of divine ideas. On the Involuntarist Platonism, this is a field of necessarily existing forms.

On all three views, God selects from that field of possibilities some forms that will be instantiated, and maybe also some 
forms that creatures can on their own cause to be instantiated. There are significant metaphysical differences between the
views, but all three involve a similar kind of divine selection model of which forms are instantiated.

\subsection{Participation}
But there is also a different way that we could have a theistic explanation of normative features of forms. Classical theism
holds that all things are either God or participate in God. In such a setting, it is natural to think of a form as a way for 
a being to participate in God. But now while God's infinity and otherness may give a wide scope to what sorts of
arrangements of features could count as a participation, that scope is plausibly narrower than all logically non-contradictory
arrangements of features. Some candidate norms, like a requirement of causing gratuitous pain to others, just may not be included 
in any metaphysical possible way of participating in a perfectly good God. And some combinations of individually admissible features may also 
fail to be found in any metaphysically possible mode of participation in a perfectly unified God, such as having conversation with 
conspecifics as central for one's good while having the essential causal power of deterministically exploding whenever one approaches a 
conspecific within talking distance. 

Such a theistic participatory limitation on forms yields a more metaphysical explanation of some aspects of Aristotelian optimistic harmony than 
divine selection does. Moreover, this mode of explanation lends itself more easily to supernaturalist stories other than theism, 
such pantheism or a classical Platonism centered on the Form of the Good. 

Nonetheless, a mere limitation on the space of possibilities for forms is insufficient for explaining all the aspects of Aristotelian 
optimism. First, a limitation of forms by itself does nothing to rule out the possibility of a form being always instantiated in beings 
that happen to inhabit an environment completely unsuitable for flourishing according to the norm. 

Second, unless we think the limitation is really severe, our explanations of norms will be curtailed. For what we will be able to explain
is why the complex of norms is minimally acceptable---such as to be minimally capable of participating in God (or the Form of the Good,
on a classically Platonic version). But the limitation won't explain cases where norms fit particularly well together, since if they 
fit less well, the norms could still be found in some possible form. For instance, in humans living by the moral norms is central to  
flourishing. This ensures that any human that lives by the moral norms automatically has quite a bit of flourishing, and one who 
does not live by the moral norms cannot be said to flourish overall. Plausibly, a much weaker degree of unity between overall flourishing
and the norms governing the will would suffice for a form of a being that participates in God: living by moral norms could be a less
central aspect of flourishing. A divine selection explanation can advert to God's having a good reason to produce beings with the greater 
degree of unity in their normative features, and thereby explain the higher degree of unity, while a participatory limitation explanation
would only explain why the degree of integration is at least minimal.

\subsection{A dual account}
It is in fact quite reasonable to combine the acocunts. Some aspects of form, especially coarser-grained ones, can be explained by participatory 
limitation while others, especially finer-grainer ones, can be explained by divine selection. The result is an explanatorily rich account of 
normativity, which predicts a minimal coherence throughout one's normative complex, and leads one to expect higher degrees of unification
in more central aspects.

This expectation of unity in turn yields another tool to help us to actually find out what our norms are. We should prefer normative theories
that allow for significant integration and harmony between moral, epistemic and other norms. An integrated picture of human flourishing
is obviously attractive.




\section{Final remarks}
??explain how theism grounds the variety of harmonies discussed earlier in the chapter

\chaptertail

\def\mychapter{XI}
\ifdefined\book
\else
\documentclass[11pt,oneside]{amsbook}
\usepackage[backend=biber, citestyle=authoryear]{biblatex}
\usepackage{mathpazo}
\usepackage{graphicx}
\usepackage{amsmath}
\usepackage{tikz}
\usetikzlibrary{arrows}
%\usepackage{titlesec}
\addbibresource{bibliography.bib}
\newcommand\posscite[1]{\citeauthor{#1}'s (\citeyear{#1})}
\newcommand\plural[1]{#1\mathrm{s}}
%\def\posscitewithextra[#1]#2{\citename{#2}'s (\citeyear{#2}, #1)}

%\newcounter{subsubsubsection}[subsubsection]
%\renewcommand\thesubsubsubsection{\thesubsubsection.\arabic{subsubsubsection}}
%\titleformat{\subsubsubsection}
%  {\normalfont\normalsize\bfseries}{\thesubsubsubsection}{1em}{}
%\titlespacing*{\subsubsubsection}
%{0pt}{3.25ex plus 1ex minus .2ex}{1.5ex plus .2ex}

\ifdefined\book
\renewcommand{\thechapter}{\Roman{chapter}}
\else
\renewcommand{\thechapter}{\mychapter}
\fi

\linespread{1.7}
\usepackage[margin=1.25in]{geometry}
\sloppy
\makeatletter
%% TODO: This is a cheat. It's supposed to be {paragraph}{4}, and that's 
%% what it is in amsbook.cls, but then it fails.
\def\paragraph{\@startsection{paragraph}{3}%
  \normalparindent\z@{-\fontdimen2\font}%
  \normalfont}
\def\subsubsubsection{\paragraph}
\makeatother

\def\smalltick{0.15cm}
\def\bigtick{0.3cm}
\def\pointcircle{0.08cm}
\def\causalnode{0.35cm}

\hyphenation{dia-chro-nic}

%\usepackage[utf8]{inputenc} % set input encoding (not needed with XeLaTeX)
\usepackage{amssymb}
\usepackage{mathtools}
\usepackage{enumitem}
\usepackage{amsthm}
\usepackage{physics}
%\usepackage{ntheorem}
\usepackage{chngcntr}
\counterwithin{figure}{section}

\makeatletter
% \def\@endtheorem{\endtrivlist\@endpefalse }% OLD
\def\@endtheorem{\endtrivlist}%

\setlist[description]{font=\normalfont\scshape}

\catcode`\|=\active\def|{\mid}
\DeclarePairedDelimiter{\ceil}{\lceil}{\rceil}
\DeclarePairedDelimiter{\floor}{\lfloor}{\rfloor}
\newcommand{\Subj}{\mathbin{\raisebox{.15ex}{$\scriptscriptstyle{\Box}$}\kern-.425em\rightarrow}}
\def\Existence{E!}
\def\Believes{\operatorname{Believes}}
\def\True{\operatorname{True}}
\def\Perfection{\operatorname{Perfection}}
\def\ext{\operatorname{Ext}}
\def\Iff{\leftrightarrow}
\def\Implies{\rightarrow}
\def\Entails{\Rightarrow}
\def\Cov{\operatorname{Cov}}
\def\Equiv{\Leftrightarrow}
\def\Form{\operatorname{Form}}
\def\Informs{\operatorname{Informs}}
\def\technical{$\star$}
\def\vtechnical{$\star\star$}
\def\power{\wp}
\def\Nec{\Box}
\def\Poss{\Diamond}
\def\Prop#1{$\langle$#1$\rangle$}
\def\R{\mathbb R}
\def\N{\mathbb N}
\def\tele{tel\={e}}
\makeatletter
\newtheoremstyle{indented}{3pt}{3pt}{\addtolength{\leftskip}{4.5em}}{-2.5em}{\sc}{.}{.5em}{}
\def\Principle#1#2#3{\theoremstyle{indented}\newtheorem*{principle}{#2}\begin{principle}\def\@currentlabel{#2}\label{#1}#3\end{principle}\let\principle\undefined}
\makeatother
\def\pref#1{{\sc\ref{#1}}}
\def\enum#1{\resume{enumerate}\item #1\end{enumerate}}
\def\ditem#1#2{\begin{enumerate}[resume]\item \label{\mychapter:#1} #2\end{enumerate}}
\def\nitem#1#2{\begin{description}\item[#1\label{\mychapter:#1}] #2\end{description}}
\def\bref#1{\ref{\mychapter:#1}}
\def\dref#1{(\ref{\mychapter:#1})}
\def\drefglobal#1{(\ref{#1})}
\usepackage{graphicx} % support the \includegraphics command and options
\usepackage{array} % for better arrays (eg matrices) in maths
\def\Not{\operatorname{\sim}}
\def\St{\operatorname{St}}
\def\num{\operatorname{num}}
\def\And{\mathrel{\&}}
\def\Or{\vee}
\def\BigOr{\bigvee}
\def\<{\langle}
\def\>{\rangle}
\def\union{\cup}
\def\nleq{\not\le}
\def\N{\mathbb N}
\def\R{\mathbb R}
\def\C{\mathbb C}
\def\Powerset{\mathcal P}
\def\star#1{{}^*#1}
\def\hN{\star{\N}}
\def\hR{\star{\R}}
\def\Z{\mathbb Z}
\def\Power{\mathcal P}
\def\Implies{\rightarrow}
\def\True{\operatorname{True}}
\def\Socrates{\mathrm{Socrates}}
\def\actual{@}
\def\Law{\operatorname{Law}}
\def\Chance{\operatorname{Chance}}
\def\Var{\operatorname{Var}}

\def\H2O{H${}_2$O}

\def\scr{\mathcal}
\def\e{\varepsilon}
\def\eps{\varepsilon}
\newtheorem{lem}{Lemma}
\newtheorem{prp}{Proposition}
\newtheorem*{theorem}{Theorem}
\newtheorem{corollary}{Corollary}
\newtheorem{cond}{Condition}

\renewcommand\thechapter{\Roman{chapter}}

\def\chaptertail{\ifdefined\book\else\end{document}\fi}
 

\title{Infinity, Causation and Paradox}
\author{Alexander R. Pruss}
%\date{} % Activate to display a given date or no date (if empty),
         % otherwise the current date is printed

\begin{document}
\setcounter{secnumdepth}{3}
\setcounter{tocdepth}{4}

\end{document}
\fi

\restartlist{enumerate}

\chapter{Eternal Life and Fulfillment}\label{ch:eternal-life}
\chaptertail

??delete?

??interact with Oderberg on suffering and pain

