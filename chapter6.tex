\def\mychapter{VI}
\ifdefined\book
\else
\documentclass[11pt,oneside]{amsbook}
\usepackage[backend=biber, citestyle=authoryear]{biblatex}
\usepackage{mathpazo}
\usepackage{graphicx}
\usepackage{amsmath}
\usepackage{tikz}
\usetikzlibrary{arrows}
%\usepackage{titlesec}
\addbibresource{bibliography.bib}
\newcommand\posscite[1]{\citeauthor{#1}'s (\citeyear{#1})}
\newcommand\plural[1]{#1\mathrm{s}}
%\def\posscitewithextra[#1]#2{\citename{#2}'s (\citeyear{#2}, #1)}

%\newcounter{subsubsubsection}[subsubsection]
%\renewcommand\thesubsubsubsection{\thesubsubsection.\arabic{subsubsubsection}}
%\titleformat{\subsubsubsection}
%  {\normalfont\normalsize\bfseries}{\thesubsubsubsection}{1em}{}
%\titlespacing*{\subsubsubsection}
%{0pt}{3.25ex plus 1ex minus .2ex}{1.5ex plus .2ex}

\ifdefined\book
\renewcommand{\thechapter}{\Roman{chapter}}
\else
\renewcommand{\thechapter}{\mychapter}
\fi

\linespread{1.7}
\usepackage[margin=1.25in]{geometry}
\sloppy
\makeatletter
%% TODO: This is a cheat. It's supposed to be {paragraph}{4}, and that's 
%% what it is in amsbook.cls, but then it fails.
\def\paragraph{\@startsection{paragraph}{3}%
  \normalparindent\z@{-\fontdimen2\font}%
  \normalfont}
\def\subsubsubsection{\paragraph}
\makeatother

\def\smalltick{0.15cm}
\def\bigtick{0.3cm}
\def\pointcircle{0.08cm}
\def\causalnode{0.35cm}

\hyphenation{dia-chro-nic}

%\usepackage[utf8]{inputenc} % set input encoding (not needed with XeLaTeX)
\usepackage{amssymb}
\usepackage{mathtools}
\usepackage{enumitem}
\usepackage{amsthm}
\usepackage{physics}
%\usepackage{ntheorem}
\usepackage{chngcntr}
\counterwithin{figure}{section}

\makeatletter
% \def\@endtheorem{\endtrivlist\@endpefalse }% OLD
\def\@endtheorem{\endtrivlist}%

\setlist[description]{font=\normalfont\scshape}

\catcode`\|=\active\def|{\mid}
\DeclarePairedDelimiter{\ceil}{\lceil}{\rceil}
\DeclarePairedDelimiter{\floor}{\lfloor}{\rfloor}
\newcommand{\Subj}{\mathbin{\raisebox{.15ex}{$\scriptscriptstyle{\Box}$}\kern-.425em\rightarrow}}
\def\Existence{E!}
\def\Believes{\operatorname{Believes}}
\def\True{\operatorname{True}}
\def\Perfection{\operatorname{Perfection}}
\def\ext{\operatorname{Ext}}
\def\Iff{\leftrightarrow}
\def\Implies{\rightarrow}
\def\Entails{\Rightarrow}
\def\Cov{\operatorname{Cov}}
\def\Equiv{\Leftrightarrow}
\def\Form{\operatorname{Form}}
\def\Informs{\operatorname{Informs}}
\def\technical{$\star$}
\def\vtechnical{$\star\star$}
\def\power{\wp}
\def\Nec{\Box}
\def\Poss{\Diamond}
\def\Prop#1{$\langle$#1$\rangle$}
\def\R{\mathbb R}
\def\N{\mathbb N}
\def\tele{tel\={e}}
\makeatletter
\newtheoremstyle{indented}{3pt}{3pt}{\addtolength{\leftskip}{4.5em}}{-2.5em}{\sc}{.}{.5em}{}
\def\Principle#1#2#3{\theoremstyle{indented}\newtheorem*{principle}{#2}\begin{principle}\def\@currentlabel{#2}\label{#1}#3\end{principle}\let\principle\undefined}
\makeatother
\def\pref#1{{\sc\ref{#1}}}
\def\enum#1{\resume{enumerate}\item #1\end{enumerate}}
\def\ditem#1#2{\begin{enumerate}[resume]\item \label{\mychapter:#1} #2\end{enumerate}}
\def\nitem#1#2{\begin{description}\item[#1\label{\mychapter:#1}] #2\end{description}}
\def\bref#1{\ref{\mychapter:#1}}
\def\dref#1{(\ref{\mychapter:#1})}
\def\drefglobal#1{(\ref{#1})}
\usepackage{graphicx} % support the \includegraphics command and options
\usepackage{array} % for better arrays (eg matrices) in maths
\def\Not{\operatorname{\sim}}
\def\St{\operatorname{St}}
\def\num{\operatorname{num}}
\def\And{\mathrel{\&}}
\def\Or{\vee}
\def\BigOr{\bigvee}
\def\<{\langle}
\def\>{\rangle}
\def\union{\cup}
\def\nleq{\not\le}
\def\N{\mathbb N}
\def\R{\mathbb R}
\def\C{\mathbb C}
\def\Powerset{\mathcal P}
\def\star#1{{}^*#1}
\def\hN{\star{\N}}
\def\hR{\star{\R}}
\def\Z{\mathbb Z}
\def\Power{\mathcal P}
\def\Implies{\rightarrow}
\def\True{\operatorname{True}}
\def\Socrates{\mathrm{Socrates}}
\def\actual{@}
\def\Law{\operatorname{Law}}
\def\Chance{\operatorname{Chance}}
\def\Var{\operatorname{Var}}

\def\H2O{H${}_2$O}

\def\scr{\mathcal}
\def\e{\varepsilon}
\def\eps{\varepsilon}
\newtheorem{lem}{Lemma}
\newtheorem{prp}{Proposition}
\newtheorem*{theorem}{Theorem}
\newtheorem{corollary}{Corollary}
\newtheorem{cond}{Condition}

\renewcommand\thechapter{\Roman{chapter}}

\def\chaptertail{\ifdefined\book\else\end{document}\fi}
 

\title{Infinity, Causation and Paradox}
\author{Alexander R. Pruss}
%\date{} % Activate to display a given date or no date (if empty),
         % otherwise the current date is printed

\begin{document}
\setcounter{secnumdepth}{3}
\setcounter{tocdepth}{4}

\end{document}
\fi

\restartlist{enumerate}

\chapter{Mind}\label{ch:mind}
\section{Multiple realizability}
Some conscious beings have brains. Start with the hypothesis that it is a necessary 
truth that all conscious beings have brains. 

First, this hypothesis is just implausible: it seems quite plausible that we could have
conscious beings with a very different body plans. 

Second, observe that brains are a specific type of organ in DNA-based animals.
To have a brain, thus, you need to have DNA. To have DNA, you need to have hydrogen
atoms. To have hydrogen atoms, you need to have electrons. A particle with a different 
electric charge would not be an electron, and the charge of the electron is definable in terms of 
the fine structure constant $e^2/(2\varepsilon_0 hc)$. If the fine structure constant
were different, we wouldn't have electrons. We might have shmelectrons that behave almost 
exactly like electrons, but they wouldn't be electrons. If we didn't have electrons, we wouldn't
have hydrogen, but at best shmydrogen. And if we didn't have hydrogen, we wouldn't have DNA,
but at best shmDNA. 

But now imagine a world extremely so similar to ours that no instruments of a sort
humans ever have a hope of constructing could ever tell the difference, but where, nonetheless,
the fine structure constant has a slightly different value. In that world we have beings that
behave, as far as any of us could ever tell by external and internal examination, just as we 
do. But they not only would \textit{be} unconscious zombies, they would \textit{have to be}
zombies---no beings with shmelectrons in place of electrons could be conscious on the hypothesis 
we are considering. That such a slight difference in physical constitution would make the
difference is extremely implausible. 

Third, if the hypothesis is true, we should be quite surprised at the existence
of consciousness. The argument just given shows that consciousness requires the precise value of
the fine structure constant that we have. How likely is that? Well, there are infinitely
many possible values that agree with our world's fine structure constant to within a thousand
significant figures. Unless our fine structure constant turns out to be some very special distinguished
value (for a while, some physicists thought it was exactly $1/137$??refs, but later measurements 
disproved that, and a recent estimate is $1/137.03599921$), the chances of getting the exact value
randomly we have is zero or at best infinitesimal. Given the fact that consciousness has great
value significance (??shvalue??), if consciousness depends on brains, and hence on electrons, then
the fact of consciousness would loudly cry out for explanation. 

The line of thought above is akin to fine-tuning arguments, where narrow ranges of fundamental constants
are claimed to be needed for life, and call out for explanation, with two options being typically offered:
a multiverse (unlikely things will happen if dice are rolled enough times) and an intelligent designer. But there are
some relevant differences in our present case. 

First, our range is much narrower---only one exact
value is compatible with consciousness on the hypothesis we are exploring---which means that objections from
the rescaling of ranges do not apply as they do in the case of the fine-tuning argument.??coarse-stuff 

Second, plausibly an intelligent designer
would be conscious, and if consciousness requires brains as we are hypothesizing, a designer will be of no help
here, on pain of circularity. 

Third, because the consciousness-permitting range has only one point on it, 
and there are uncountably infinitely many possible other values of the fine structure constant, hitting this value
will not automatically be probable even given a multiverse. If you spin a continuous fair spinner once, 
your chance of hitting a particular value is zero or infinitesimal. But the same is true for any finite number of independent spins.
Moreover, in classical probability theory, this is also true for a countably infinite number of spins. And for an
uncountably infinite number of spins, the probability is simply undefined. In light of this, the multiverse hypothesis only
really solves the problem of consciousness in our context if it is a Lewisian or Tegmarkian hypothesis that \textit{every} 
possible cosmic arrangement is realized in reality. But such a hypothesis only solves the problem at the expense of introducing serious
sceptical problems, since there will be cosmoses, just as real as ours, where every coherent sceptical hypothesis hold, and it does
not appear reasonable to think that we got so lucky as to escape them all.??refs

Tying consciousness to brains thus links consciousness to the precise laws of nature we have. That is not only intuitively implausible
but leads to serious problems. We should think that there is some flexibility in what kinds of bodies conscious beings can have.

Perhaps instead of supposing that consciousness is tied to brains, we could suppose that consciousness is tied to a range of brain-like
organs. Thus, consciousness would be compatible with having somewhat different laws of nature, resulting in fundamental particles
slightly different from the ones we have, and behavior somewhat different from the one we have, but not \textit{very} different.
But now consider the Mersenne questions about the boundaries of physical constitution compatible with consciousness. These
questions cannot be settled by invoking human nature, since they are questions that transcend the nature of any one species.
Nor can they be settled the way Mersenne settled his original questions, by invoking God's creative decision, because we are
supposing that the connection between consciousness and brain-like organs is necessary. We should avoid Mersenne questions that
do not seem to have a plausible answer. 

Furthermore, the issue of worlds practically indistinguishable to our instruments but where one has consciousness and the other
does not returns on the range view. Suppose that the upper cut-off for the fine structure constant to be compatible with
consciousness is $1/100$ (recall that our world's fine structure constant is about $1/137$). Then either $1/100$ is the
highest value compatible with consciousness or the lowest value incompatible with consciousness. If it is the highest value
compatible with consciousness, there should be a world $w_c$ with consciousness and fine-structure constant $1/100$ and a 
world $w_z$ that is practically indistinguishable from $w_c$ but where the fine-structure constant is slightly more than 
$1/100$ and hence where there are only zombies. If, on the other hand, $1/100$ is the lowest value incompatible with 
consciousness, then for a value of the fine-structure constant $(1/100)-\e$ for some positive $\e$ less than one divided by a 
googolplex there will be a world $w_c$ with consciousness. Then we should expect there to be a possible world $w_z$ with fine-structure constant
$1/100$ that is practically indistinguishable from $w_c$ (a difference of one in a googolplex should not affect anything observable),
but $w_z$ will be a zombie world, since we have assumed that a fine-structure constant of $1/100$ is incompatible with
consciousness. So in either case there will be a world with consciousness and a world with zombies which are physically
indistinguishable to humans.  But it is implausible that consciousness should depend on physical features that are so insignificant.

This line of thought pushes one to a very liberal view about what kinds of physical constitutions are compatible with consciousness.
It does not appear, in particular, that consciousness should depend on having a physical constitution that includes brains or anything
similar to brains. We thus have very significant multiple realizability.??check-mr-book

\section{Functionalism}
\subsection{Introduction}
Full-blown dualism, of course, yields significant multiple realizability. Indeed, a minded being's body could be an oak tree or even a 
rock, as long as it had the right kind of non-physical mind on dualism. We will discuss the interaction of dualism with Aristotelian
forms in Section~\ref{sec:dualism}. In the meanwhile, however, let us continue to consider broadly naturalistic accounts of mind.

We have seen that there is good reason to be very liberal about the type of physical aspect that a minded thing can have.  But 
if we are to remain in a broadly naturalistic theory, we need to put some limits on the kinds of physical constitutions that
minds can be based on. We saw earlier that limits based on particular natural kinds---DNA, brains, electrons, etc.---are highly
implausible. The most plausible remaining option is functionalism: to have a mind is to have a certain kind of functional structure,
so that, necessarily, if there is a functional isomorphism between two entities with their respective functionally-specifiable 
causal histories, if one of these entities has a mind, so does the other. Moreover, the isomorphism between causal histories implies
a significant degree of identity between the mental histories. 

On what we may call strong functionalism, their purely internal mental histories
will be the same, and in particular they will have qualitatively the same states in their histories---whenever one felt hot, so did
the other, and whenever one had a perception as of red, so did the other. The restriction to purely internal mental histories allows
for some externalism. Thus, an individual on Earth may be thinking about water, while the analogous thought in an isomorphic individual
on Twin Earth, may be thinking about XYZ, where XYZ fulfills the same causal role on Twin Earth as H$_2$O does on Earth. 

On weak functionalism, the non-qualitative purely internal mental histories will be the same, and whenever one has a conscious state, 
the other has an analogous conscious state, but the exact qualitative phenomenal character of the conscious states may depend on 
the precise physical substrate underlying the two conscious states. On weak functionalism, a silicon-based isomorph of a human being,
will have some sensation in a functional state isomorphic to a human's eating sugar, but that sensation's qualitative character may
be different from the taste of sweet. 

I will now argue that functionalism, whether weak or strong, has serious problems which can be solved by combining it with an Aristotelian hylomorphism.\footnote{The arguments
based on the possibility of malfunction will be based on the ones in ??ref:Koons-Pruss.} 

\subsection{Interpretation}
Begin with the well-known observation that simple causal systems, like the electrons buzzing inside a rock, can be re-interpreted as 
emulating the functioning of our brains, simply because they have such a vast number of states.??refs If this is right, then functionalism 
appears to lead to the absurd thesis that rocks not only think, but think like we do. One version of this argument will be given in ??forward:appendix. 

One might try to get out of this difficulty by insisting that gerrymandered functional systems do not count: only simple causal
systems count as implementing the functions. However, it is very likely that complex evolved brains like ours do have some significantly
gerrymandered states. One might try to draw a distinction between more and less gerrymandered systems, however. The functional states 
that need to be attributed to a rock to re-interpret it as thinking our thoughts are doubtless many orders of magnitude more gerrymandered
than our functional states. But now we have a nasty Mersenne question again: what makes it be the case that the transition between the degrees
of gerrymandering compatible with having a mind those incompatible with having a mind lies where it does?

\subsection{Reliability}
Next, consider the question of the reliability of functional systems. Whether our universe is deterministic or not, functional systems are imperfectly reliable.
How reliable do they need to be to count as the functional systems they are? Consider a subsystem that given two inputs representing numbers
puts out their sum 99.9\% of the time, and 0.1\% of the time puts out the product. Obviously, it is more reasonable to interpret it as 
a reliable addition system than an extremely unreliable multiplication system. But suppose the system puts out sums 50.1\% of the time
and products 49.9\% of the time. What then? 

There are two natural cut-offs. We could require that a system is defined by how it behaves 100\% of the time or by how it behaves
more than half of the time. Requiring perfect functioning would have the empirically false consequence that humans don't think. 
A 50\% cut-off, however, may be problematically low. If every subsystem of a complex functional system had a 49\% failure rate, then 
the typical outputs of the system would be largely random, because any output of the system is the result of a causal chain of many 
subsystem states, and any such chain would likely contain multiple failures. Moreover, while 50\% reliability seems to be a natural
and well-defined cut-off, the actual reliability of a system cannot be captured by a single number, if only because reliability
depends on environmental conditions. If we say that a system is an adder provided that the output is the sum of the inputs more
than half of the time, we have to specify the temperature, background radiation, and other conditions under which that reliability
is to be defined. And now we lose the neat elegance of specifying the reliability with the single number $1/2$.

We might try to define the reliability with respect to the actual environmental conditions the system was in. But suppose that
Alice finds herself in extremely harmful conditions---say, great heat or toxic fumes---but by a fluke survives with what is intuitively
full brain function for a few seconds longer than we would expect, screaming seemingly in pain. Given that under those extreme conditions
the functioning is extremely unreliable, we would have to say that Alice is in fact a mindless zombie and feels no pain. This is
implausible. 

Additionally, note that the reliability of a system varies over time, perhaps increasing over an initial burn-in period,
and then eventually decreasing. We can then define the reliability of a system instantaneously or via a time-average.
If we proceed via a time-average, then we will have the highly counterintuitive consequence that an individual who died
at a hundred was actually conscious through their life, but had they lived a decade longer, they would \textit{never even have been} 
conscious, because some crucial subsystem's reliability average over a 110-year lifespan would have been below the cut-off, but over
the hundred-year lifespan was above the cut-off. So we should define the reliability instantaneously. 

In any case, the very idea of a sharp cut-off in reliability for mindedness seems counterintuitive. Imagine two humanoids whose 
brains are nearly identical, with the exception that one brain structure crucial for consciousness in one of the humanoids has 
50.000001\% reliability and the other has merely 49.999999\% reliability. Suppose that both brains \textit{in fact}
function exactly the same way, so that the sequences of internal states are exactly the same---it's just that one is slightly
more likely to fail than the other. It does not seem very plausible to think that one would be conscious and the other not.

Finally observe that a functional system can retain its function even when highly defective and unreliable. A car that starts only 
on one of three mornings is still a car. 

\subsection{Many functions}
It is not enough that a subsystem always outputs the sum of its inputs for it to be an adder. After all, if the only inputs ever
given are pairs of zeroes, then the subsystem could just as well be a multiplier. As is well-known from Wittgenstein and Kripke??refs,
no finite amount of data is sufficient to determine the function of the system, since any finite collection of inputs and outputs is
consistent with infinitely many possible functions---admittedly, perhaps messy ones.

We might try to define the function of a system or subsystem in terms of counterfactuals. Perhaps an adder is something that \textit{would}
output the sum for all inputs in the range of allowable inputs. However, Frankfurt's counterexamples to the Principle of Alternate
Possibilities??ref can be adapted to show that this is untenable. Imagine that Bob counts as thinking that $5+7=12$ in virtue of the 
fact that his thought involves the operation of an adding subsystem in his brain. But suppose that a neuroscientist has placed a neural
scanner and bomb in Bob's vicinity in such a way that if the scanner detects the adding subsystem getting any input other than 
$5$ and $7$, the bomb blows up Bob. Now counterfactuals like ``If the inputs were $4$ and $3$, the output would be $7$'' are false.
If the inputs were $4$ and $7$, there would be no output, just an explosion.

One might try to define functions by asking what the output would be given the inputs \textit{absent external interference}. 
But what counts as interference with a system, as opposed to, say, a helpful or neutral effect, depends precisely on the system's
function. If the purpose of a wood-pulp product is to preserve inscribed information, then fire is an external interference;
but if the wood-pulp product's function is to be kindling, then fire activates its the object's function. 
Furthermore, one can internalize Frankfurt-like cases. Imagine that through science-fictional genetic modification Bob's liver
comes to behave just like the scanner-and-bomb system. 

\subsection{A neo-Aristotelian solution}
If functionalists do not have the resources to define functionalism, then this is presumably the most fundamental possible flaw
in functionalism. However, the problem of defining functions is exactly one of the one that Aristotelian forms are designed to solve.  The proper function
of a subsystem in an organism is that the fulfillment of which constitutes the organism's flourishing with respect to that subsystem.
This does not require the subsystem to be reliable, though Aristotelian optimism predicts that most systems will be reliable. 

A robust view of organisms as having forms that specify normative, and hence functional, features thus solves a central
problem with functionalism. And as we saw, there is very good reason for a naturalist to adopt functionalism---it is the
best naturalist option for saving multiple realizability. Thus, we have an argument from naturalism to Aristotelianism.
Of course, whether Aristotelianism is compatible with naturalism is not clear. If naturalism is understood to say that 
there are no fundamental normative properties, then of course there is no compatibility.

???refs

\section{Supervaluationism about minds}
A tempting solution to the problem of the multiplicity of functionalist (or other) theories of mind differing with respect
to fine details is to treat each theory as a precisification of concepts like \textit{mind} or \textit{pain}, none of which
is privileged over the others. And,
if all has gone well, then typical adult humans fall under all the precisifications of ``has a mind'' and in paradigmatic
cases of being in pain they fall under all the precisifications of ``is in pain''. 

One difficulty with this approach has already been discussed in the ethical?? context??backref, namely that similar problems
arise at the meta level: What \textit{range} of, say, standards of reliability yields a functionalist concept that is in fact a 
precification of our concept of mind? 

Furthermore, recall the argument??backref that the centrality and overridingness of ethical norms to our lives makes it 
deeply implausible to think that there is a plurality of closely related concepts, none of which is privileged over the others.
But given the deep importance of the mental to ethics, the same concern applies to the mental. That an action causes severe
pain to a non-consenting individual with no significant benefit is a conclusive moral reason not to perform the action. 
But if there are many concepts of pain with none of them privileged, and similarly of consent, then whether one has a
conclusive moral reason for an action will depend on the choice of precisification as well.

???

\section{Dualism}\label{sec:dualism}
The neo-Aristotelian teleological twist on functionalism allows one to remain to a significant degree a naturalist (more on that
in ??forward). The account reduces mental properties to functional properties, but the functional properties are not 
reducible to the kinds of properties that physics studies. A more traditional Aristotelian approach, however, is to refuse
to reduce the reduction of mental properties to non-mental ones. This is certainly also compatible with the robust view of
human nature that has been defended in this book. 

Whether the more dualist or the more functionalist view is more plausible depends on how satisfying the reduction of mental
properties to teleologically laden functionalist properties is, and one way to evaluate this reduction is to consider
whether the arguments against the reduction of mental to physical properties apply to this neo-Aristotelian functionalism.
These arguments can be divided into three categories, depending on which aspect of mental life they object to the reduction
of: content, consciousness, or freedom of will. 

As has been noted??backref, the neo-Aristotelian teleology offers hope for a reductive account of mental content. Teleological properties
can be hyperintensional---it is the purpose of eyes to see rather than to see or be such that $2+2=5$, even though necessarily
everything that sees is such that it sees or is such that $2+2=5$. Thus concerns that there is no way to reach the hyperintensionality
of mental content from the extensional or at best intensional content of the physical world do not transfer to the neo-Aristotelian
account. Similarly, concerns about the need to account for purpose in the mind---beliefs are states that are there 
\textit{in order to} mirror the world---and that the only source of purpose for the physicalist, namely evolution, is inadequate
do not apply to the neo-Aristotelian theory. Thus, content-based arguments do not tell against a neo-Aristotelian functionalism.

The case of consciousness is less clear. We can try to run standard knowledge arguments transposed to the neo-Aristotelian context.
Suppose Mary is raised in a black and white environment, and knows all the physical facts as well as all the normative facts about
human beings. In particular, she knows that certain states of the human being are such that they should occur precisely when the
human being is looking at a red object, and that these states typically occur when people see a ripe tomato. She, further, knows all
the normative interconnections between these states and other states, including all the inferential connections. Will she learn anything
further by seeing the tomato? Here intuitions may differ. It is at least easier to hold out for a negative answer to the learning question
here than in the case where Mary has a purely physical state.

Similarly, we can try for imaginability arguments. These come in two versions: zombie arguments that one can have our physical
constitution without consciousness and afterlife arguments that our consciousness could survive the destruction of the body.
On these arguments, the imaginability of a scenario provides defeasible evidence of its metaphysical possibility. 

Could there be beings that have isomorphic physical and normative properties to ours
but that are unconscious zombies? Again, this is not completely clear. Among our normative properties are moral duties. Could one
have something isomorphic to moral duties without consciousness? If not, then zombie arguments against neo-Aristotelian functionalism 
do not work as they stand. 

Alternately, what can the neo-Aristotelian functionalist say about surviving the destruction of the body? Those ``fainthearted'' 
neo-Aristotelians who hold that the form is nothing but a kind of arrangement of matter, of course, will find it troubling to
suppose the form to survive the destruction of the matter. Though even there, there is a potential precedent for a view
that would allow the form to survive. Aquinas's account of transsubstantiation holds that in the Eucharist, the accidents of
bread and wine continue to exist after the cessation of the existence of the substance. Among the accidents there will be 
\textit{shape}. If a shape can exist without that of which it is the shape, then why not an arrangement as well? Similarly,
some contemporary trope theories hold that there are ``unaffiliated'' tropes, tropes that exist without their substance. If a 
trope can exist without a substance, why can't an arrangement exist without matter? 

But of course the view defended in the book is more robust: forms are not mere arrangements. The more reality the form itself
has, the more plausible that there is no metaphysical impossibility about the form existing without the matter. 

One may, however, wonder whether \textit{we} could continue to exist without a body. ???????

Finally, we have freedom of will. Those convinced that an action that comes solely from the causal powers posited by a completed
physics??ref would not be free will not be moved by being told that these causal powers have a normative organization. Such 

??Doesn't functionalism have the same problems?

??vagueness of consciousness

\section{Teleology and representation}
\section{Teleology and mental causation}
%teleosemantics??
\section{Teleological animalism}
\subsection{Animalism}
\subsection{Cerebra}
\section{Soul and body ethics}
\section*{Appendix: Functionalism gone too far}
Consider a deterministic functional system $Q$ consisting of a 
finite number of possible computational subsystems $S_1,...,S_N$, where $S_k$ is always in exactly one state from the finite set $\scr A_k$
of possible states at each of the discrete ``significant'' moments of time $t_1,...,t_m$ over its finite lifetime\footnote{A standard modern digital computer only has defined 
computational states at ticks of its internal clock. Between ticks of the internal clocks, it is in an analogue state that is not
computationally defined. A lot of careful engineering goes into ensuring that the states become properly ``digital'' at the clock ticks.}, and 
a finite number $I_1,...,I_M$ of sensors, where $I_k$ is always in exactly one state from the finite set $\scr I_k$ of possible input
states. 

We can think of $Q$ as a finite digital computer. The total state of the system at any given time can be represented as the
$(N+M)$-tuple $(a_1,...,a_N,b_1,...,b_M)$ where $a_k$ is a state in $\scr A_k$ and $b_k$ is a state in $\scr I_k$. We can designate some of
the subsystems as outputs, connected to external effectors (muscles, motors, lights, etc.)  Furthermore, we suppose there are functional laws which provide a 
mapping $f$ from the state of the system at time $t_i$ to the state at time $t_{i+1}$. Thus, $f(a_1,...,a_N,b_1,...,B_M)= (a_1,...,a_N,b_1,...,b_M)$ 
provided that the system would transition from state $(a_1,...,a_N)$ to state $(b_1,...,b_N)$.\footnote{The mapping is independent of time. But we 
can always suppose there is a finite clock, e.g., given by a subsystem $S_k$ such that the set of states $\scr A_k$ is
the set of times during the system's lifetime, and with the transition rule that the time always gets incremented. Or we can suppose an input
from an external clock.} Presumably any deterministic analog system can be approximated by such a system $Q$ to aribitrarily high precision.
We suppose for convenience that there is some fixed initial state of $Q$, with fixed initial input and computational states.

Now consider a different system $P$ consisting of a single particle moving in the $xy$-plane in space, with a constant 
(sublight) non-zero $x$-velocity $v$, starting at $x$-coordinate $0$ at time $t_1$. For ease of visualization, suppose the $x$-axis
runs left to right, and the $y$-axis runs down to up.
Let $\scr J$ be the set of all possible
single-time input state vectors $(b_1,...,b_M)$ where $b_k \in\scr I_k$ for each $k$, and let $K$ be a positive integer such
that $\scr J$ has at most $10^K$ members. 

We now recode $Q$'s inputs into $P$'s inputs as follows. Suppose our particle is at $y$-coordinate $0$ at a time $t_0<t_1$. 
For $1\le n\le N$, let $\psi_n$ be a one-to-one function from $\scr J$ to integers between $0$ and $10^K-1$, both inclusive. The analogue 
to inputting the sensor state vector $(b_1,...,b_M)$ into $Q$ at significant time $t_n$ will now be this. We take
the $y$-coordinate value $y_{n-1}$ at $t_{n-1}$, and add to it the value $10^{-Kn} \psi_n(b_1,...,b_M)$, shifting the particle
upward along the $y$-axis as needed to ensure it reaches that by time $t_n$.\footnote{We may need to ensure the units in which
these are measured are such that the particle can be shifted in the requisite time without exceeding the speed of light.} Thus, 
the first set of inputs of $Q$ will be encoded in the first $K$ digits after the decimal point, the second set of inputs
will be encoded in the second $K$ digits, and so on.

Next, suppose $Q$'s computational states begin with the fixed initial state vector $(a_{1,1},...,a_{1,N})$ at time $t_1$. Given 
the determinism of the system, there is a mathematical function $f$ that takes a sequence $s$ of length $n$ of members of $\scr I$ 
and returns the computational state $f(s)$ that $Q$ would be in at time $t_n$ if it were to have received the sequence of inputs $s$ 
at the times $t_1,...,t_n$. Let $n(x)$ be the integer $n$ such that $x=v t_n$, if there such an integer, where we recall that
$v$ is the velocity of our particle along the $x$-axis. Let $s(x,y)$ be the sequence of $n(x)$ sensor state vectors encoded 
by the $y$-coordinate value $y$, assuming $n(x)$ is defined. Thus, the first member of $s(x,y)$ will be the sensor state vector
$s_1$ such that $\psi_1(s_1)$ is equal to the decimal number given by the first $K$ digits after the decimal point in $y$, and so on.
We now deem $P$ to be in a computational state corresponding to $f(s(x,y))$ when $P$ is at $(x,y)$. This is defined at all the
significant times $t_1,...,t_n$. We have, thus, an isomorphism between the computational states and sensor inputs of $Q$ and $P$.

We now go for one final twist. Suppose that in the actual world, the
system $Q$ gets the sensor input vectors $s_1,...,s_N$ at times $t_1,...,t_N$, respectively. We can now choose the 
function $\psi_n$ such that $\psi_n(s_n)=0$ for all $i$. Then a single particle moving with constant velocity 
$v$ along the $x$-axis, with $y$-coordinate always equal to $0$ is an isomorph to the actual functioning of $Q$.
But the very same particle will be an isomorph of any other system $Q'$ with the same significant times. Thus,
the particle will think your thoughts \textit{and} my thoughts!

??do we need a clock?

\chaptertail 


