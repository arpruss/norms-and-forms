\def\mychapter{VI}
\ifdefined\book
\else
\documentclass[11pt,oneside]{amsbook}
\usepackage[backend=biber, citestyle=authoryear]{biblatex}
\usepackage{mathpazo}
\usepackage{graphicx}
\usepackage{amsmath}
\usepackage{tikz}
\usetikzlibrary{arrows}
%\usepackage{titlesec}
\addbibresource{bibliography.bib}
\newcommand\posscite[1]{\citeauthor{#1}'s (\citeyear{#1})}
\newcommand\plural[1]{#1\mathrm{s}}
%\def\posscitewithextra[#1]#2{\citename{#2}'s (\citeyear{#2}, #1)}

%\newcounter{subsubsubsection}[subsubsection]
%\renewcommand\thesubsubsubsection{\thesubsubsection.\arabic{subsubsubsection}}
%\titleformat{\subsubsubsection}
%  {\normalfont\normalsize\bfseries}{\thesubsubsubsection}{1em}{}
%\titlespacing*{\subsubsubsection}
%{0pt}{3.25ex plus 1ex minus .2ex}{1.5ex plus .2ex}

\ifdefined\book
\renewcommand{\thechapter}{\Roman{chapter}}
\else
\renewcommand{\thechapter}{\mychapter}
\fi

\linespread{1.7}
\usepackage[margin=1.25in]{geometry}
\sloppy
\makeatletter
%% TODO: This is a cheat. It's supposed to be {paragraph}{4}, and that's 
%% what it is in amsbook.cls, but then it fails.
\def\paragraph{\@startsection{paragraph}{3}%
  \normalparindent\z@{-\fontdimen2\font}%
  \normalfont}
\def\subsubsubsection{\paragraph}
\makeatother

\def\smalltick{0.15cm}
\def\bigtick{0.3cm}
\def\pointcircle{0.08cm}
\def\causalnode{0.35cm}

\hyphenation{dia-chro-nic}

%\usepackage[utf8]{inputenc} % set input encoding (not needed with XeLaTeX)
\usepackage{amssymb}
\usepackage{mathtools}
\usepackage{enumitem}
\usepackage{amsthm}
\usepackage{physics}
%\usepackage{ntheorem}
\usepackage{chngcntr}
\counterwithin{figure}{section}

\makeatletter
% \def\@endtheorem{\endtrivlist\@endpefalse }% OLD
\def\@endtheorem{\endtrivlist}%

\setlist[description]{font=\normalfont\scshape}

\catcode`\|=\active\def|{\mid}
\DeclarePairedDelimiter{\ceil}{\lceil}{\rceil}
\DeclarePairedDelimiter{\floor}{\lfloor}{\rfloor}
\newcommand{\Subj}{\mathbin{\raisebox{.15ex}{$\scriptscriptstyle{\Box}$}\kern-.425em\rightarrow}}
\def\Existence{E!}
\def\Believes{\operatorname{Believes}}
\def\True{\operatorname{True}}
\def\Perfection{\operatorname{Perfection}}
\def\ext{\operatorname{Ext}}
\def\Iff{\leftrightarrow}
\def\Implies{\rightarrow}
\def\Entails{\Rightarrow}
\def\Cov{\operatorname{Cov}}
\def\Equiv{\Leftrightarrow}
\def\Form{\operatorname{Form}}
\def\Informs{\operatorname{Informs}}
\def\technical{$\star$}
\def\vtechnical{$\star\star$}
\def\power{\wp}
\def\Nec{\Box}
\def\Poss{\Diamond}
\def\Prop#1{$\langle$#1$\rangle$}
\def\R{\mathbb R}
\def\N{\mathbb N}
\def\tele{tel\={e}}
\makeatletter
\newtheoremstyle{indented}{3pt}{3pt}{\addtolength{\leftskip}{4.5em}}{-2.5em}{\sc}{.}{.5em}{}
\def\Principle#1#2#3{\theoremstyle{indented}\newtheorem*{principle}{#2}\begin{principle}\def\@currentlabel{#2}\label{#1}#3\end{principle}\let\principle\undefined}
\makeatother
\def\pref#1{{\sc\ref{#1}}}
\def\enum#1{\resume{enumerate}\item #1\end{enumerate}}
\def\ditem#1#2{\begin{enumerate}[resume]\item \label{\mychapter:#1} #2\end{enumerate}}
\def\nitem#1#2{\begin{description}\item[#1\label{\mychapter:#1}] #2\end{description}}
\def\bref#1{\ref{\mychapter:#1}}
\def\dref#1{(\ref{\mychapter:#1})}
\def\drefglobal#1{(\ref{#1})}
\usepackage{graphicx} % support the \includegraphics command and options
\usepackage{array} % for better arrays (eg matrices) in maths
\def\Not{\operatorname{\sim}}
\def\St{\operatorname{St}}
\def\num{\operatorname{num}}
\def\And{\mathrel{\&}}
\def\Or{\vee}
\def\BigOr{\bigvee}
\def\<{\langle}
\def\>{\rangle}
\def\union{\cup}
\def\nleq{\not\le}
\def\N{\mathbb N}
\def\R{\mathbb R}
\def\C{\mathbb C}
\def\Powerset{\mathcal P}
\def\star#1{{}^*#1}
\def\hN{\star{\N}}
\def\hR{\star{\R}}
\def\Z{\mathbb Z}
\def\Power{\mathcal P}
\def\Implies{\rightarrow}
\def\True{\operatorname{True}}
\def\Socrates{\mathrm{Socrates}}
\def\actual{@}
\def\Law{\operatorname{Law}}
\def\Chance{\operatorname{Chance}}
\def\Var{\operatorname{Var}}

\def\H2O{H${}_2$O}

\def\scr{\mathcal}
\def\e{\varepsilon}
\def\eps{\varepsilon}
\newtheorem{lem}{Lemma}
\newtheorem{prp}{Proposition}
\newtheorem*{theorem}{Theorem}
\newtheorem{corollary}{Corollary}
\newtheorem{cond}{Condition}

\renewcommand\thechapter{\Roman{chapter}}

\def\chaptertail{\ifdefined\book\else\end{document}\fi}
 

\title{Infinity, Causation and Paradox}
\author{Alexander R. Pruss}
%\date{} % Activate to display a given date or no date (if empty),
         % otherwise the current date is printed

\begin{document}
\setcounter{secnumdepth}{3}
\setcounter{tocdepth}{4}

\end{document}
\fi

\restartlist{enumerate}

\chapter{Mind}\label{ch:mind}
\section{Multiple realizability}
Some conscious beings have brains. Consider first the hypothesis that it is a necessary 
truth that all conscious beings have brains. 

First, this hypothesis is just implausible: it seems quite plausible that we could have
conscious beings with a very different body plans. 

But besides the counterintuitiveness
of the hypothesis, if the hypothesis is true, we should be quite surprised at the existence
of consciousness. For brains are a specific type of organ in DNA-based
vertebrates. There are no doubt many ways of evolutionarily solving the problems
of bodily coordination that brains solve, and the probability of arriving
specifically at brains seems quite small. 

In fact, the probability of DNA-based life seems quite small. The existence of electrons is 
tightly bound to the specific laws of nature that our world has. A particle with a different 
electric charge would not be an electron, and the charge of the electron is definable in terms of 
the fine structure constant $e^2/(2\varepsilon_0 hc)$. If the fine structure constant
were different, we wouldn't have electrons. We might have shmelectrons that behave almost 
exactly like electrons, but they wouldn't be electrons. If we didn't have electrons, we wouldn't
have hydrogen, but at best shmydrogen. And if we didn't have hydrogen, we wouldn't have DNA,
but at best shmDNA. We might have critters that behave rather like our world's brainy critters,
but they would only have shbrains, not brains. If consciousness is tied to brains, then they
wouldn't be conscious.

In other words, if consciousness requires brains, then consciousness requires the precise value of
the fine structure constant that we have. How likely is that? Well, there are infinitely
many possible values that agree with our world's fine structure constant to within a thousand
significant figures. Unless our fine structure constant turns out to be some very special distinguished
value (for a while, some physicists thought it was exactly $1/137$??refs, but later measurements 
disproved that, and a recent estimate is $1/137.03599921$), the chances of getting the exact value
randomly we have is zero or at best infinitesimal. Given the fact that consciousness has great
value significance (??shvalue??), if consciousness depends on brains, and hence on electrons, then
the fact of consciousness would loudly cry out for explanation. 

The line of thought above is akin to fine-tuning arguments, where narrow ranges of fundamental constants
are claimed to be needed for life, and call out for explanation, with two options being typically offered:
a multiverse (unlikely things will happen if dice are rolled enough times) and an intelligent designer. But there are
some relevant differences in our present case. 

First, our range is much narrower---only one exact
value is compatible with consciousness on the hypothesis we are exploring---which means that objections from
the rescaling of ranges do not apply as they do in the case of the fine-tuning argument.??coarse-stuff 

Second, plausibly an intelligent designer
would be conscious, and if consciousness requires brains as we are hypothesizing, a designer will be of no help
here, on pain of circularity. 

Third, because the consciousness-permitting range has only one point on it, 
and there are uncountably infinitely many possible other values of the fine structure constant, hitting this value
will not automatically be probable even given a multiverse. If you spin a continuous fair spinner once, 
your chance of hitting a particular value is zero or infinitesimal. But the same is true for any finite number of independent spins.
Moreover, in classical probability theory, this is also true for a countably infinite number of spins. And for an
uncountably infinite number of spins, the probability is simply undefined. In light of this, the multiverse hypothesis only
really solves the problem of consciousness in our context if it is a Lewisian or Tegmarkian hypothesis that \textit{every} 
possible cosmic arrangement is realized in reality. But such a hypothesis only solves the problem at the expense of introducing serious
sceptical problems, since there will be cosmoses, just as real as ours, where every coherent sceptical hypothesis hold, and it does
not appear reasonable to think that we got so lucky as to escape them all.??refs

Tying consciousness to brains thus links consciousness to the precise laws of nature we have. That is not only intuitively implausible
but leads to serious problems. We should think that there is some flexibility in what kinds of bodies conscious beings can have.

???Mersenne on MR

\section{Teleology and representation}
\section{Teleology and mental causation}
%teleosemantics??
\section{Teleological animalism}
\subsection{Animalism}
\subsection{Cerebra}
\section{Soul and body ethics}
\chaptertail 


