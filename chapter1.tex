\def\mychapter{I}
\ifdefined\book
\else
\documentclass[11pt,oneside]{amsbook}
\usepackage[backend=biber, citestyle=authoryear]{biblatex}
\usepackage{mathpazo}
\usepackage{graphicx}
\usepackage{amsmath}
\usepackage{tikz}
\usetikzlibrary{arrows}
%\usepackage{titlesec}
\addbibresource{bibliography.bib}
\newcommand\posscite[1]{\citeauthor{#1}'s (\citeyear{#1})}
\newcommand\plural[1]{#1\mathrm{s}}
%\def\posscitewithextra[#1]#2{\citename{#2}'s (\citeyear{#2}, #1)}

%\newcounter{subsubsubsection}[subsubsection]
%\renewcommand\thesubsubsubsection{\thesubsubsection.\arabic{subsubsubsection}}
%\titleformat{\subsubsubsection}
%  {\normalfont\normalsize\bfseries}{\thesubsubsubsection}{1em}{}
%\titlespacing*{\subsubsubsection}
%{0pt}{3.25ex plus 1ex minus .2ex}{1.5ex plus .2ex}

\ifdefined\book
\renewcommand{\thechapter}{\Roman{chapter}}
\else
\renewcommand{\thechapter}{\mychapter}
\fi

\linespread{1.7}
\usepackage[margin=1.25in]{geometry}
\sloppy
\makeatletter
%% TODO: This is a cheat. It's supposed to be {paragraph}{4}, and that's 
%% what it is in amsbook.cls, but then it fails.
\def\paragraph{\@startsection{paragraph}{3}%
  \normalparindent\z@{-\fontdimen2\font}%
  \normalfont}
\def\subsubsubsection{\paragraph}
\makeatother

\def\smalltick{0.15cm}
\def\bigtick{0.3cm}
\def\pointcircle{0.08cm}
\def\causalnode{0.35cm}

\hyphenation{dia-chro-nic}

%\usepackage[utf8]{inputenc} % set input encoding (not needed with XeLaTeX)
\usepackage{amssymb}
\usepackage{mathtools}
\usepackage{enumitem}
\usepackage{amsthm}
\usepackage{physics}
%\usepackage{ntheorem}
\usepackage{chngcntr}
\counterwithin{figure}{section}

\makeatletter
% \def\@endtheorem{\endtrivlist\@endpefalse }% OLD
\def\@endtheorem{\endtrivlist}%

\setlist[description]{font=\normalfont\scshape}

\catcode`\|=\active\def|{\mid}
\DeclarePairedDelimiter{\ceil}{\lceil}{\rceil}
\DeclarePairedDelimiter{\floor}{\lfloor}{\rfloor}
\newcommand{\Subj}{\mathbin{\raisebox{.15ex}{$\scriptscriptstyle{\Box}$}\kern-.425em\rightarrow}}
\def\Existence{E!}
\def\Believes{\operatorname{Believes}}
\def\True{\operatorname{True}}
\def\Perfection{\operatorname{Perfection}}
\def\ext{\operatorname{Ext}}
\def\Iff{\leftrightarrow}
\def\Implies{\rightarrow}
\def\Entails{\Rightarrow}
\def\Cov{\operatorname{Cov}}
\def\Equiv{\Leftrightarrow}
\def\Form{\operatorname{Form}}
\def\Informs{\operatorname{Informs}}
\def\technical{$\star$}
\def\vtechnical{$\star\star$}
\def\power{\wp}
\def\Nec{\Box}
\def\Poss{\Diamond}
\def\Prop#1{$\langle$#1$\rangle$}
\def\R{\mathbb R}
\def\N{\mathbb N}
\def\tele{tel\={e}}
\makeatletter
\newtheoremstyle{indented}{3pt}{3pt}{\addtolength{\leftskip}{4.5em}}{-2.5em}{\sc}{.}{.5em}{}
\def\Principle#1#2#3{\theoremstyle{indented}\newtheorem*{principle}{#2}\begin{principle}\def\@currentlabel{#2}\label{#1}#3\end{principle}\let\principle\undefined}
\makeatother
\def\pref#1{{\sc\ref{#1}}}
\def\enum#1{\resume{enumerate}\item #1\end{enumerate}}
\def\ditem#1#2{\begin{enumerate}[resume]\item \label{\mychapter:#1} #2\end{enumerate}}
\def\nitem#1#2{\begin{description}\item[#1\label{\mychapter:#1}] #2\end{description}}
\def\bref#1{\ref{\mychapter:#1}}
\def\dref#1{(\ref{\mychapter:#1})}
\def\drefglobal#1{(\ref{#1})}
\usepackage{graphicx} % support the \includegraphics command and options
\usepackage{array} % for better arrays (eg matrices) in maths
\def\Not{\operatorname{\sim}}
\def\St{\operatorname{St}}
\def\num{\operatorname{num}}
\def\And{\mathrel{\&}}
\def\Or{\vee}
\def\BigOr{\bigvee}
\def\<{\langle}
\def\>{\rangle}
\def\union{\cup}
\def\nleq{\not\le}
\def\N{\mathbb N}
\def\R{\mathbb R}
\def\C{\mathbb C}
\def\Powerset{\mathcal P}
\def\star#1{{}^*#1}
\def\hN{\star{\N}}
\def\hR{\star{\R}}
\def\Z{\mathbb Z}
\def\Power{\mathcal P}
\def\Implies{\rightarrow}
\def\True{\operatorname{True}}
\def\Socrates{\mathrm{Socrates}}
\def\actual{@}
\def\Law{\operatorname{Law}}
\def\Chance{\operatorname{Chance}}
\def\Var{\operatorname{Var}}

\def\H2O{H${}_2$O}

\def\scr{\mathcal}
\def\e{\varepsilon}
\def\eps{\varepsilon}
\newtheorem{lem}{Lemma}
\newtheorem{prp}{Proposition}
\newtheorem*{theorem}{Theorem}
\newtheorem{corollary}{Corollary}
\newtheorem{cond}{Condition}

\renewcommand\thechapter{\Roman{chapter}}

\def\chaptertail{\ifdefined\book\else\end{document}\fi}
 

\title{Infinity, Causation and Paradox}
\author{Alexander R. Pruss}
%\date{} % Activate to display a given date or no date (if empty),
         % otherwise the current date is printed

\begin{document}
\setcounter{secnumdepth}{3}
\setcounter{tocdepth}{4}

\end{document}
\fi

\restartlist{enumerate}

\chapter*{Acknowledgments}
I would like that thank ... Nicholas Breiner ....??
\chapter{Introduction}\label{ch:intro}
\section{Introductory Remarks}
I have a human nature or human form that governs my voluntary and involuntary activity.
Much as the government governs the activity of the people \textit{both} by legislating norms and encouraging
people to follow the norms, my nature's governance also has the dual role of setting norms for me and influencing my
activity to follow these norms. This nature is something real and intrinsic to me, something that makes me be what I am, 
a human being. 

When extended to other fundamental beings besides humans, the above is the center of Aristotle's metaphysics.
I will show that this center is extremely fruitful, providing compelling solutions to problems in ethics,
epistemology, the philosophy of mind, semantics, metaphysics and philosophy of science. Many of these are prominent problems that have been
the subject of much discussion, such as the problem of priors in Bayesian epistemology or of vagueness in semantics, while others 
are problems
that have not attracted much attention, such as the problem of seemingly arbitrary detail in moral rules. 
I shall discuss these solutions in Chapters~\ref{ch:ethics}--\ref{ch:laws}.

The ability to give unified solutions to an array of problems spread through many areas of philosophy gives one
a very good reason to accept the central Aristotelian theses. However, in Chapter~\ref{ch:God}, I will also argue that this center cannot 
hold on its own, and the way to be an intellectually satisfied Aristotelian, especially after Darwin, is to be a theist 
as well.

There are several lines of thought readers attracted to the unified Aristotelian solutions may  follow. Some
may deny that the problems facing the central Aristotelian theses are as serious as I contend. Some may agree that
the problems are serious, and regretfully reject the Aristotelian apparatus, either because they take the cost of 
the theistic solution to be too great or are unconvinced that the theistic solution works on its own terms. 
Others may agree that the problems are serious but find some other solution than the theistic one. But some, I hope, 
will conclude that the Aristotelian solutions are so attractive, and the theistic solution to the problems is sufficiently 
plausible, that this book provides not only a good reason to accept the Aristotelian center but also to accept
theism.

We will be elaborating the metaphysical apparatus of what I have been calling the ``Aristotelian center'' gradually??
as we move through the problems and details of their solutions. At the same time, not every detail of the solutions needs to be
adopted by the reader to find the general Aristotelian strategy compelling. Finally, in Chapter~\ref{ch:details} we will
collect together the needed aspects of the Aristotelian metaphysics and discuss in greater detail the metaphysics needed.

??paths through the book?

In the rest of this chapter, we will do two things. First, I will sketch the central Aristotelian metaphysics in slightly
greater detail. Second, I will discuss a neglected science-based argument from the 17th century polymath Marin Mersenne for the existence 
of God. This argument does not work, I will argue. However, an important thread running through this book will be how ``Mersenne 
problems'' analogous to the problems in science raised by Mersenne arise in many areas of philosophy and provide a compelling 
case for the existence of Aristotelian natures or forms.

\section{Aristotelian Natures}

\section{Mersenne Problems}
\chaptertail 

