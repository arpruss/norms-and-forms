\def\mychapter{I}
\ifdefined\book
\else
\documentclass[11pt,oneside]{amsbook}
\usepackage[backend=biber, citestyle=authoryear]{biblatex}
\usepackage{mathpazo}
\usepackage{graphicx}
\usepackage{amsmath}
\usepackage{tikz}
\usetikzlibrary{arrows}
%\usepackage{titlesec}
\addbibresource{bibliography.bib}
\newcommand\posscite[1]{\citeauthor{#1}'s (\citeyear{#1})}
\newcommand\plural[1]{#1\mathrm{s}}
%\def\posscitewithextra[#1]#2{\citename{#2}'s (\citeyear{#2}, #1)}

%\newcounter{subsubsubsection}[subsubsection]
%\renewcommand\thesubsubsubsection{\thesubsubsection.\arabic{subsubsubsection}}
%\titleformat{\subsubsubsection}
%  {\normalfont\normalsize\bfseries}{\thesubsubsubsection}{1em}{}
%\titlespacing*{\subsubsubsection}
%{0pt}{3.25ex plus 1ex minus .2ex}{1.5ex plus .2ex}

\ifdefined\book
\renewcommand{\thechapter}{\Roman{chapter}}
\else
\renewcommand{\thechapter}{\mychapter}
\fi

\linespread{1.7}
\usepackage[margin=1.25in]{geometry}
\sloppy
\makeatletter
%% TODO: This is a cheat. It's supposed to be {paragraph}{4}, and that's 
%% what it is in amsbook.cls, but then it fails.
\def\paragraph{\@startsection{paragraph}{3}%
  \normalparindent\z@{-\fontdimen2\font}%
  \normalfont}
\def\subsubsubsection{\paragraph}
\makeatother

\def\smalltick{0.15cm}
\def\bigtick{0.3cm}
\def\pointcircle{0.08cm}
\def\causalnode{0.35cm}

\hyphenation{dia-chro-nic}

%\usepackage[utf8]{inputenc} % set input encoding (not needed with XeLaTeX)
\usepackage{amssymb}
\usepackage{mathtools}
\usepackage{enumitem}
\usepackage{amsthm}
\usepackage{physics}
%\usepackage{ntheorem}

\makeatletter
% \def\@endtheorem{\endtrivlist\@endpefalse }% OLD
\def\@endtheorem{\endtrivlist}%

\catcode`\|=\active\def|{\mid}
\DeclarePairedDelimiter{\ceil}{\lceil}{\rceil}
\DeclarePairedDelimiter{\floor}{\lfloor}{\rfloor}
\newcommand{\Subj}{\mathbin{\raisebox{.15ex}{$\scriptscriptstyle{\Box}$}\kern-.425em\rightarrow}}
\def\Existence{E!}
\def\Believes{\operatorname{Believes}}
\def\True{\operatorname{True}}
\def\Perfection{\operatorname{Perfection}}
\def\ext{\operatorname{Ext}}
\def\Iff{\leftrightarrow}
\def\Implies{\rightarrow}
\def\Entails{\Rightarrow}
\def\Equiv{\Leftrightarrow}
\def\Form{operatorname{Form}}
\def\Informs{operatorname{Informs}}
\def\technical{$\star$}
\def\vtechnical{$\star\star$}
\def\power{\wp}
\def\Nec{\Box}
\def\Poss{\Diamond}
\def\Prop#1{$\langle$#1$\rangle$}
\def\R{\mathbb R}
\def\N{\mathbb N}
\def\tele{tel\={e}}
\makeatletter
\newtheoremstyle{indented}{3pt}{3pt}{\addtolength{\leftskip}{4.5em}}{-2.5em}{\sc}{.}{.5em}{}
\def\Principle#1#2#3{\theoremstyle{indented}\newtheorem*{principle}{#2}\begin{principle}\def\@currentlabel{#2}\label{#1}#3\end{principle}\let\principle\undefined}
\makeatother
\def\pref#1{{\sc\ref{#1}}}
\def\enum#1{\resume{enumerate}\item #1\end{enumerate}}
\def\ditem#1#2{\begin{enumerate}[resume]\item \label{\mychapter:#1} #2\end{enumerate}}
\def\dref#1{(\ref{\mychapter:#1})}
\def\drefglobal#1{(\ref{#1})}
\usepackage{graphicx} % support the \includegraphics command and options
\usepackage{array} % for better arrays (eg matrices) in maths
\def\Not{\operatorname{\sim}}
\def\St{\operatorname{St}}
\def\num{\operatorname{num}}
\def\And{\mathrel{\&}}
\def\Or{\vee}
\def\BigOr{\bigvee}
\def\<{\langle}
\def\>{\rangle}
\def\union{\cup}
\def\nleq{\not\le}
\def\N{\mathbb N}
\def\R{\mathbb R}
\def\C{\mathbb C}
\def\Powerset{\mathcal P}
\def\star#1{{}^*#1}
\def\hN{\star{\N}}
\def\hR{\star{\R}}
\def\Z{\mathbb Z}
\def\Power{\mathcal P}
\def\Implies{\rightarrow}
\def\True{\operatorname{True}}
\def\Socrates{\mathrm{Socrates}}
\def\actual{@}

\def\H2O{H${}_2$O}

\def\scr{\mathcal}
\def\e{\varepsilon}
\def\eps{\varepsilon}
\newtheorem{lem}{Lemma}
\newtheorem*{theorem}{Theorem}
\newtheorem{corollary}{Corollary}
\newtheorem{cond}{Condition}

\renewcommand\thechapter{\Roman{chapter}}

\def\chaptertail{\ifdefined\book\else\end{document}\fi}
 

\title{Infinity, Causation and Paradox}
\author{Alexander R. Pruss}
%\date{} % Activate to display a given date or no date (if empty),
         % otherwise the current date is printed

\begin{document}
\setcounter{secnumdepth}{3}
\setcounter{tocdepth}{4}

\end{document}
\fi

\restartlist{enumerate}

\chapter*{Acknowledgments}
I would like that thank ... Nicholas Breiner, Sherif Girgis, Philip Rand, Robert Verrill, ....??

\chapter{Introduction}\label{ch:intro}
\section{Introductory remarks}
I have human nature or human form that governs my activity, both voluntary and not.
Much as the government governs the activity of the people \textit{both} by legislating norms and encouraging
people to follow the norms, my nature's governance also has the dual role of setting norms for me and influencing my
activity to follow these norms. This nature is something real and intrinsic to me, something that makes me be what I am, 
a human being. You have one, too, and yours is intrinsically just like mine.

When extended to other fundamental beings besides humans, the above is the center of Aristotle's metaphysics.
I will show that this center is extremely fruitful, providing compelling solutions to problems in ethics,
epistemology, the philosophy of mind, semantics, metaphysics and philosophy of science. Many of these are prominent problems that have been
the subject of much discussion, such as the problem of priors in Bayesian epistemology or of vagueness in semantics, while others 
are problems
that have not attracted much attention, such as the problem of seemingly arbitrary detail in moral rules. 
I shall discuss problems and solutions in Chapters~\ref{ch:ethics}--\ref{ch:laws}.

The metaphysical apparatus of what I have been calling the ``Aristotelian center'' will be gradually elaborated
as we move through the problems and details of their solutions. At the same time, not every detail of the solutions needs to be
adopted by the reader to find the general Aristotelian strategy compelling. However, in Chapter~\ref{ch:details} a number of 
issues of detail will be discussed more specifically.

The ability to give unified solutions to an array of problems spread through many areas of philosophy gives one
good reason to accept the central Aristotelian theses. However, in Chapter~\ref{ch:God}, I will also argue that this center does
not plausibly hold on its own, and the best way to be an intellectually satisfied Aristotelian, especially after Darwin, is to be a theist 
as well.

There are several lines of thought readers attracted to the unified Aristotelian solutions may  follow. Some
may deny that the problems facing the central Aristotelian theses are as serious as I contend. Some may agree that
the problems are serious, and regretfully reject the Aristotelian apparatus, either because they take the cost of 
the theistic solution to be too great or are unconvinced that the theistic solution works on its own terms. 
Others may agree that the problems are serious but find some other solution than the theistic one. But some, I hope, 
will conclude that the Aristotelian solutions are so attractive, and the theistic solution to the problems is sufficiently 
plausible, that this book provides not only a good reason to accept the Aristotelian center but also to accept
theism.

I will now mention a crucial assumption: this book is predicated on a robust realism about ethics, epistemology, semantics and 
metaphysics. This realism is both metaphysical, holding that there are objective facts about the states of affairs, and 
epistemological, claiming that we have epistemic access to this objective reality, and our best views about it tend to be 
right.  A number of the arguments on this book implicitly put pressure on this realist assumption, in that they show that 
there is a choice between a variety of non-realisms and a highly controversial metaphysics (see 
Section~\ref{ch:God}.\ref{sec:non-realism}
for specific engagement with this point). The methodology here, however, 
will be that realism is taken for granted, and no matter how high the metaphysical price for solving problems in a realist
way, that price is worth paying. In particular, I take normative facts to be as much a part of our objective data about 
the world as observable physical facts. 

I do not have any particularly original argument for realism about norms. Like many others, I find persuasive an argument based on particular
cases. For instance, it is as indisputable as my having two hands that torturing minority children would be wrong even if we had 
a racist society that thought we had no duties towards members of minority groups and even if we ourselves agreed with 
that society's abhorrent views. And it is as indisputable as there being other minds that one is epistemically defective if one comes 
to believe in fairies solely to thumb one's nose at the scientific establishment, and this is true even if society agrees 
that such nose-thumbing is an epistemic virtue. Likewise, eating thumbtacks does not contribute to flourishing no matter how much one wants 
to eat them or how much one enjoys this due to a rewiring of one's pleasure system.??ref:McInerny A non-realist reader will say I am just thumping the table, of course. 
Indeed, I have thumped it, and for most of the rest of the book I will simply assume realism.

In the rest of this first chapter, two things will happen. First, I will sketch the central Aristotelian metaphysics in slightly
greater detail. Second, I will discuss a neglected science-based argument from the 17th century polymath Marin Mersenne for the existence 
of God. This argument does not work very well, I will argue. However, a central thread running through this book will be how ``Mersenne 
problems'' analogous to the problems in science raised by Mersenne arise in many areas of philosophy and provide a compelling 
case for the existence of Aristotelian natures or forms.

\section{Aristotelian natures}
\subsection{A quick introduction}
According to Aristotle, reality is fundamentally built out of substances, which are real mind-independent entities that
bear properties and are not related to another thing in the way that a property is to its bearer.
These substances are not limited to microphysical entities like quarks and photons. Aristotle takes biological organisms 
like oak trees and human beings to be paradigm cases of substances??ref. In fact, not only are substances not limited to 
microphysical entities, but there is some reason to doubt that that microphysical entities are substances at all. For 
one can have a superposition of states with different numbers of particles of a given type??ref, whereas there should be a well-defined 
number of substances.\footnote{One might hope that sometimes the wavefunction will be in an eigenstate of particle
number for a particle type, and think that at such times at least some particles will be substances. However, one would expect a system
to stay long in an exact eigenstate of particle number. Consider a system in an eigenstate of electron number. Real-world
systems are not isolated. Thus pretty much always there is a non-zero probability of a positron quantum tunnelling and annihilating
that electron to produce two photons, and then, unless the wave-function is immediately collapsed, we will presumably have 
a system that's in a superposition of a state with two photons and a state with an electron and positron. While it is possible
to think that particles are extremely short-lived substances, existing only when we have an appropriate eigenstate, we do tend
to think of substances as moderately long-lived. At the same time, one might worry that this line of thought may push some 
thinkers to suppose a single universal substance---as in Schaffer's quantum holism??ref---and that if one tries to avoid
that conclusion, doing so will offer some way of restoring particle-based substances. (I am grateful to Robert Verrill??ref for
this objection.)}

Each material substance has a form or nature---I will use the terms interchangeably in this book. This form or nature performs a number of roles including unifying the matter of the 
substance into a single thing, setting norms for the structure and activity of the substance, and guiding the actual 
development and activity of the object. The nature of the oak tree is not merely an arrangement of its particles, since an
arrangement lacks normative force. But it's also not just a norm, because a norm does not guide the activity of a mindless
thing. Indeed, in living things, the form of a substance has traditionally been considered to be its life or soul: it makes the substance
be alive. 

Natures are innate to their substances. Nonetheless, this statement underdetermines an important question, namely whether 
substances of the same sort---say, red oaks---all numerically share one nature, or each individual substance has its own nature, 
albeit in relevant respects??forwardref they are all exactly alike in substances of the same kind. For two things could in
principle share something innate to them. It could be that all people have the same soul, much as two conjoined twins can
have the same stomach. Aristotle scholarship is divided on the question whether Aristotle believed in ``individual forms'',
one per substance. However, at least one of the advantages of an Aristotelian theory of form will be accentuated if we accept 
individual forms, as we shall see??forward, and on the whole I will argue in ??forward that there is good
philosophical reason to take natures to be individual.
Nonetheless, if you like shared forms, 
\textit{many} of the benefits I will draw out for a theory of forms will be ones you, too, can have. 

%%proofed
\subsection{Aristotelian optimism}
Natures not only define how a thing should function, but also actively lead the thing to function in that way.
This means there is an inherent bias in each substance towards acting well. This bias leads to Aristotle's
optimistic thought that natural states occur ``for the most part''??ref, which is quite useful for figuring out
what is in fact natural, since the frequency of the occurrence of a state is evidence of its naturalness.

There is, however, a tension in Aristotle's own thought between the above optimism and the pessimistic observation 
that most human beings are morally bad.??ref Aristotle may be empirically wrong about most people being bad??refs?,
but nonetheless exploring the tension will help us understand Aristotelian optimism more clearly as it faces the
problem of moral evil.

There are many substances with different natures in the world. The flourishing of some requires involves the languishing
of others: the lion's feeding is the gazelle's death. Moreover, a substance's nature directs it to behavior that works
well for the substance in its natural niche. But things do not always stay in their niche. Because of this, Aristotle
has many resources for explaining why there is a significant set of cases where substances find themselves in unnatural
states.\footnote{For further discussion of the harmony between substances, see ??forwardref.} But Aristotle nonetheless
thinks that misfortune will only be a minority of the cases.

Let us return to the Aristotelian optimism that things function well ``for the most part''. What is and is not
``for the most part'' depends on the reference class. Most humans have legs, but most living substances do not. 
If the reference class of the ``for the most part'' is all activities of all substances, then human moral behavior
forms such a small portion of that class---it is so outnumbered by bacterial reproduction, say---that even if 
all human moral behavior were wicked it would be unlikely to make a difference with respect to the Aristotelian
optimism. However, at the same time, with such a broad reference class, the optimism would be of little use to us
in understanding normativity for humans, for humans could simply be an outlier in all respects. 

A more optimistic reference class would be all the activities of a particular kind of substance. On this reading, 
Aristotle would lead us to expect that each kind of substance does well in most of its activities. But moral activity
is only a small proportion of the activity of a human. We also breathe, we circulate blood, we repair cells, etc. Leibniz 
estimated that three quarters??check,ref of our activity is at an animal level. Stalin was a complete moral failure, but
still he maintained homeostasis until the age of 74. Human moral activity could, thus, be mostly bad even though most
human activity is good. Again, the tension between Aristotle's general optimism and his pessimism about human morals
would be resolved.

A yet more optimistic reference class would be a particular major type of activity---say, moral activity or reproduction---of
a particular kind of substance. Now we would have the prediction that most human moral activity will be good, and this seems
to contradict Aristotle's thesis about typical human moral badness. But even this is not clear. In MacDonald's 
\textit{The Princess and Curdie}, Curdie has just expressed to the princess's great-great-grandmother a pessimistic thesis that 
unavoidably most things humans do are bad. 
\begin{quote}
`There you are much mistaken,' said the old quavering voice. `How little you must have thought! Why, you don't seem even to 
know the good of the things you are constantly doing. Now don't mistake me. I don't mean you are good for doing them. 
It is a good thing to eat your breakfast, but you don't fancy it's very good of you to do it. The thing is good, not you.'??ref
\end{quote}
The old woman makes two important points. First, we should not forget that we perform \textit{many} morally significant actions
each day. Curdie ate breakfast. He could have thrown it at his mother, or just ungratefully poured it out on the grown.
His eating breakfast was morally good. And we perform many such morally good actions each day. Second, the fact that we
perform these morally good actions does not do us much credit, the grandmother insists. I suspect that the reason for her
pessimism here is Curdie's lack of the kinds of motivations that would render breakfast-eating positively creditable.
But the mere motivation to nourish himself was already good, even if not particularly creditable.

There is a further point we may add. While on a mathematics exam, it might be enough to get 60\% to pass,
morally speaking it is not enough that 60\% of one's actions be good. If in the morning I kick a neighbor's puppy,
at lunch I charge my private meal to a research budget, in the afternoon I plagiarize something from a foreign
language journal for inclusion in my book, and in the evening I cheat in order to beat my kid at chess, I am a bad
person even if each of these actions is paired with two morally good actions of the eating-breakfast level of 
goodness. Having a majority of one's actions be good is not nearly enough to avoid being bad. 

Thus with the reference class of ``for the most part'' restricted to moral activities, Aristotle's optimism and pessimism
can be both maintained. And the above considerations also show that Aristotle's optimism is quite compatible with 
realism or pessmism about human morality.

A further optimistic ingredient that we will at times draw on is the idea that the different ways of being
well in an organism have a tendency to mutually support each other in a unified kind of way. There will be
trade-offs, sometimes tragic ones, but by and large a healthy heart supports healthy lungs, a healthy mind supports a healthy body,
courage supports justice, justice supports courtesy, and courtesy supports kindness, all of which tend to make
one live a happier life even by hedonistic standards. 

Aristotelian ethics is sometimes accused of inferring ``ought'' from ``is''. The Aristotelian optimism should plead
guilty, but with a mitigating factor. Since things only typically act how they ought, the move from ``is'' to ``ought''
is defeasible. Nonetheless, that things act a certain way is some evidence that they ought to do so. This is, in fact,
quite plausible apart from Aristotelian assumptions. For there are three possible views on the statistical correlation
between ``is'' and ``ought'':
\ditem{is-ought-0}{There is no correlation whatsoever between how things behave and how they ought to behave.}
\ditem{is-ought-n}{There is a negative correlation between how things behave and how they ought to behave.}
\ditem{is-ought-p}{There is a positive correlation between how things behave and how they ought to behave.}

Option \dref{is-ought-0} is quite implausible. That a behavior obtains is surely \textit{relevant} evidence for
the question of how a thing should behave. 

Further, take at random any two distinct binary features $A$ and $B$ of human beings, say, 
having green eyes and having won an egg-and-spoon race. Suppose we actually measure the correlation of these two features 
across all humanity, by calculating the covariance $\Cov[A,B]=P(A\And B)-P(A)P(B)$, where $P(C)$ is the probability that a random human has
$C$. We might find that the covariance is very small. But given that there are eight billion people, how likely is it that $P(A\And B)$ should
\textit{exactly} equal $P(A)P(B)$ across the population? Surely very unlikely indeed. Thus, very likely, any two distinct binary features
are correlated, positively or negatively.

So now the choice is between supposing a positive or a negative correlation between behavior and norm. The hypothesis of a negative 
correlation does not seem very plausible. Imagine that you are thrown into a situation about which you know nothing, but see a line of people
one by one doing something---say, pressing a red button. And now it's your turn. It seems to be an unjustified pessimism to think that your 
observation provides you with evidence for the hypothesis that you should \textit{not} press the red button, as it would if you thought
the correlation beween behavior and norm was negative. If you know nothing better, it's a better bet to be a sheep. And the same, surely,
goes for other kinds of things than humans. That a behavior is exhibited is not evidence against the behavior being right. It is very unlikely
to be neutral. So it is at least some evidence for the behavior being right.

This argument does not rule out the possibility that the correlation between behavior and norm is very weak, weaker than the Aristotelian
optimist needs to get significant evidence for norms from behavior. But it is a start.

Nor need it be the case that ``is'' is always evidence for ``ought''. In some cases we may have independent evidence that a particular
entity is so defective that certain sorts of its behavior are of no positive evidential value. We shouldn't learn ethics from the 
depraved. That said, I happen to be sufficiently optimistic that I doubt that there are many people who are so depraved that they exhibit
\textit{no} positive correlation between behavior and norm, even if that correlation is not mediated by right decision-making but only
by self-interest.

\subsection{Species where most organisms fail to reproduce}
%% scott: https://alexanderpruss.blogspot.com/2022/08/the-afterlife-of-humans-and-animals.html?showComment=1661888748217

\subsection{Who are the humans?}
\subsubsection{Rational animals}
It is traditional in the Western philosophical tradition to describe the human being as a rational animal. One can
take this further and \textit{define} the human being as a rational animal. A number of modern-day Aristotelians??refs do this
and hold that if rational dolphins or octopuses evolved, they would also be human in the philosophical sense, having the 
same nature as we do. While Aristotle certainly held that we were rational, he did not \textit{define} us by our rational animality.
And such a move would fit poorly with Aristotle's hierarchical
way of defining species that inspired the Porphyrian tree. For if some but not all possible primates were human and some 
but not all possible cephalopods were human, then \textit{human} couldn't be a subdivision of vertebrate or of invetebrate. Rather, 
it would be a species that cuts across multiple genera. This might not bother those of us who do not accept the hierarchical
mode of definition\footnote{However, see Koons??ref for a fascinating defense of the Porphyrian tree.}, but it would be 
highly problematic for Aristotle. 

But there is another serious related problem. The human nature specifies what is normal for human beings. For a rational
primate, having four limbs adapted to movement on land is normal; for a rational dolphin, having flippers and a tail would be 
normal; for a rational octopus, having eight tentacles would be normal. This suggests that there isn't a single human nature
that would be shared by rational primates, cetaceans and cephalopods. 

However, this argument is much too quick. After all, the variety of normalcy problem seems may well arise even within the 
biological species \textit{Homo sapiens}, though particular examples are apt to be controversial. The American National
Institutes of Health describes lactose intolerance as ``an impaired ability to digest lactose''??ref:https://ghr.nlm.nih.gov/condition/lactose-intolerance,
which suggests that lactose intolerance is abnormal (we would not talk of an impaired ability to digest cellulose in humans).
Yet this alleged impairment is found in the majority of the human adult population worldwide, especially outside of Europe.
We could say that speaking of lactose intolerance as an impairment is just a mistake. But there is another possibility: it may
be that lactose intolerance is normal for some members of our species and abnormal for others, and the description of it as
an impairment is correct when restricted to persons of certain European ancestries.
In that case, assuming that all \textit{Homo sapiens} do share the same nature, what makes it abnormal for an adult
of Irish ancestry to be lactose intolerant but normal for an adult of Armenian ancestry\footnote{There is 4\% prevalence of
lactose intolerance in Ireland and 98\% in Armenia. ??ref:https://www.thelancet.com/journals/langas/article/PIIS2468-1253(17)30154-1/fulltext }
cannot just be our shared human nature. 

Instead, it might be that our shared human nature encodes some conditional like ``If you have the complex feature $F$, then you
should be able to digest lactose through all your life.'' This conditional applies to humans worldwide, but only a minority of humans
actually exhibit $F$. This feature might, for instance, be genetic coding for life-long lactose digestion, so that those
who have this coding ought to be lactose tolerance and those who do not need not be. In that case, even among the Irish lactose intolerance
need not always be an impairment: it would only be an impairment amng those who have the genetic coding for lactose tolerance but are
nonetheless for some other reason intolerant. Or this feature could specify some other aspect of the genome that correlates with, but
does not cause, lactose digestion.

An even more controversial and politically charged example could be sex-linked traits: for some members of our species, having a uterus may be normal, and for others
it may be abnormal. Again, our human nature could encode a conditional like ``If you have feature $G$, then you should have a uterus.''
Note that while it is controversial whether there is such a conditional, holding that there is does not by itself decide important questions about gender identity.
For instance, someone who thinks that gender identity is determined by genetics could say that the conditional holds with $G$ being a genetic
feature (say, having two X chromosomes), while someone who thinks that gender identity is determined by personal self-identification could say
that the conditional holds with $G$ being self-identification as a woman. They could then both use the conditional to argue for opposite answers
to the question of whether hysterectomy for sex-reassignment purposes should be covered by health insurance. 

Much less controversial are more temporary conditions that affect what is normal. A high temperature is normal for a sick human but abnormal for a
healthy one. Yet we surely don't want to say that getting sick is a substantial change. 

The problem of the variety what is normal among logically possible biological species of rational animals could be handled analogously. 
Human nature could specify that if you are a primate you have four limbs, if you are a cetacean you have flippers and if you are an octopus you have
eight tentacles. But while this is theoretically possible, it would mean that human nature would have to encode an infinite number of such conditionals.
Indeed, taking this to its logical conclusion, we would have to include conditionals to fit not just rational animals in worlds with laws of nature like ours,
but in worlds with a physics radically different different from ours. It seems very implausible to suppose that there is such infinite normative complexity
in our nature.

Moreover, the expansive view of humanity results in a very implausible restriction of the space of possible natures. For while Aristotelian harmony may not
allow for every combination of normative features---such as an animal that ought to live all its life deep underwater but also ought to breathe atmospheric 
oxygen---we would expect there to be a broad selection of logically possible natures harmoniously combining different normative features. Thus, just as it
is possible to have an animal that is supposed to both echolocate and fly (the bat), and animal that is supposed to both echolocate and be snake-like (perhaps
this is unexemplified, but possible), it should
be possible to have an animal that is supposed to be both snake-like and rational. But such an animal would not be human, because humans are not supposed
to be snake-like (on the expansive view, they can be non-defectively snake-like, while on narrower view, being snake-like would be a defect). So it should
be possible to have non-human rational animals.

It should be noted that even if dolphin and octopus persons are not human, it is very plausible that our human nature would require us to treat them with the
respect that persons deserve. After all, even on the broad definition of humanity, disembodied beings like deities or angels would not count as human, and yet
many people who have believed in such beings have held that we should show them at least the kind of respect we do for fellow humans, for instance by keeping
promises made to them.

\subsubsection{The narrower view}
Even within the narrower view of humanity that would exclude dolphin and octopus persons from being human, we still have the difficult question of where exactly the boundaries of humanity are to be drawn. 
We can make use of our best ethical intuitions to give us good reason to draw them in a way that includes at least all 
members of Homo sapiens. There was a small study done by philosophers of the motivations of rescuers of Jews from Holocaust, and the one common
factor found in the rescuers was a tendency to identify others as fellow humans. The intuition of these rescuers, as well as of typical anti-racists and 
anti-sexists in our time, supports taking all members of Homo sapiens to be human. 

But there are two further questions. One is about other historical species closely related to us, such as Neanderthals and Denisovans. While we could
identify humanity with Homo sapiens, we need not. Probably, we do not know enough about Neanderthal and Denisovan life to see whether it is 
plausible that one form encompasses the norms for them as well as for us. And fortunately for us this question is of merely theoretical importance.

The second question is whether while we should accept all \textit{adult} members of Homo sapiens as human, we should do so likewise for those 
biological humans who do not satisfy the philosophers' criteria for developed personhood, such as language or generalized problem-solving skills.??ref:Warren
In the case of adults, there is a simple Aristotelian argument for doing so. Adult biological humans who do not fulfill such criteria are abnormal, as is made clear
by the fact that if we could treat them medically in a way that leads to the fulfillment of the criteria, we would have good reason to do so. Indeed, if these
biological humans were not of the same kind as us, then such medical treatment by transforming them into beings like us would constitute what Aristotelians
call a substantial change, a change from one kind of thing to another. But objects do not survive substantial change. Thus, such treatment would be a killing.
That is implausible.

The remaining subquestion is for immature members of Homo sapiens, such as zygotes, embryos and fetuses. Here, the questions become more controversial.
On a view that excludes such immature members, we would have to say that their ordinary course of life is that they mature and die \textit{in utero}, giving
rise to a human in the metaphysical sense. Yet death seems to be a catastrophic event for a substance, and in the growth of these immature organisms into more mature ones
there does not appear to be such a catastrophe. This suggests that these members of the our biological species are also human in the metaphysical sense, but much
more needs to be said.\footnote{??refs}

\section{Mersenne questions}
\subsection{Mersenne's argument}
Marin Mersenne was a monk, philosopher, theologian and the 17th century equivalent of the arXiv preprint archive---he was 
a crucial line of communication between a broad variety of thinkers and scientists. He drew on his broad knowledge of
the science of the time to offer an argument that begins with many pages of questions, of which the following are
representative:
\begin{quote}
Who gave more strength to the lion than to the ant?
Who made it be that earth is not in the moon's place, and that the planets aren't larger or smaller, closer or further?
Who has ordered all the parts of the world as we see them?
...
Why is the moon 56 earth-radii away from the earth? Why is the sun 1182 [earth-radii] away from us at its apogee? ... and why is its distance at perigee not other than 1101 [earth-radii]? ...
I could equally ask you about Saturn, and Jupiter, and Mars ...??refs\footnote{The moon-earth distance is approximately correct.
The earth-sun distance is an order of magnitude off.}
\end{quote}
These ``Mersenne questions'' go on and on, with a mind-numbing number of examples. And Mersenne has one answer to all
these questions, posed in a rhetorical question: ``Was it not God?''??ref

The argument sounds similar to fine-tuning arguments for theism which became popular in the late 20th century. These
arguments, too, list a variety of physical parameters and offer God as an explanation of them all.??ref But there is
a crucial ingredient that the fine-tuning arguments, namely that the parameters listed are needed for intelligent life
as we know it, or for some other valuable trait of the universe, like its amenability to scientific investigation.??ref
The basic idea behind the fine-tuning argument is, very roughly, that nature is indifferent to value but God cares 
about value, so the fact that the parameters are valuable provides evidence for theism over naturalism.

It is, thus, natural to look in Mersenne for arguments that it is particularly valuable for the moon to be 56 earth-radii
from the earth, but at least in this work, Mersenne does not supply them or even hint at them. Nor is there any argument
that it is better that lions are stronger than ants, or that it is better for the moon to orbit the earth rather than
the other way around. If Mersenne is giving a fine-tuning argument, the argument is oddly incomplete. And Mersenne's
penchant for adumbrating detail at great length makes it unlikely that he has simply omitted such a crucial part of the
argument.

Rather, it appears that Mersenne is simply looking for an explanation of the scientific details he cites, sees no
prospect of a scientific explanation, and offers theism as the alternative. And indeed it is only in the 20th century
with computer models of solar system formation that we have much in the way of plausible answers to Mersenne's questions
about the distances between solar system bodies. For instance, the leading theory of lunar formation involves the earth being hit by
another body and a large chunk being pushed into orbit. Given assumptions about the impact, one can then explain the
resulting distance between the earth and the moon. But notice that such an explanation only gives an answers to the 
Mersenne question about the earth-moon distance at the cost of raising similar Mersenne questions about the 
parameters of the impact such as the mass distribution of the pre-impact earth, the angle and location of impact, the 
mass distribution of the impacting body, etc.

But Mersenne has a fatal argumentative flaw. Even if we grant that it is very unlikely that a future science will predict
these exact numbers, there is always the possibility of a stochastic explanation, one that does not predict exact values, 
but supposes a random natural process that generates a set of values at random. Now, if Mersenne had an argument showing
that the values of the parameters are suspiciously valuable---say, necessary for intelligent life---then a stochastic
explanation might not be as good as a theistic one. From a Bayesian point of view, we might be able to argue that it is
extremely unlikely that a random selection of parameters would have such value, but not nearly so unlikely that God would
choose such parameters and hence the data supports theism over randomness. But given that Mersenne makes no case that 
the parameters have anything to recommend them to God for creation, we have no reason to think that the probability of God 
choosing is these parameters is any higher than the probability of them arising randomly, and hence we have no support for
theism.

Suppose, however, that we had a Mersenne-type case where randomness was not a satisfactory explanation. Then there would
still be one more problem with the argument. If one is willing to deny the Principle of Sufficient Reason, one could simply
say that the parameters are what they are and there is no reason why they are like this---that they are a \textit{brute} fact.
This, however, is less satisfying than the stochastic answer, for adverting to brute fact should be a last resort, to be
chosen when no explanation is available. But here there is an option, namely theism.

\subsection{Appearance of contingency}
Mersenne gives a dizzying number of examples, and he seems to relish the sheer appearance of arbitrariness of the numbers like ``56''
and ``1182''. While this has some rhetorical force, it also has argumentative force. The more arbitrary-looking parameters
the parameters are, the less epistemically likely it is that they are what they hold of necessity or that good scientific theories
will predict their exact values. And the greater the number of parameters, the less likely it is that science can provide 
an explanation of them all.

The appearance of arbitrariness is evidence of contingency, and contingency calls out for explanation.\footnote{In ??ref, I have
argued for a Principle of Sufficient Reason (PSR) that holds that all contingent facts have an explanation. But even if one rejects
the PSR, one should hold that explaining relevant contingencies is a good feature of a theory, one that provides evidence for the
theory.} But at the 
same time, we have to be careful here. For instance,
it might seem arbitrary that protons have (approximately) 1836 times the mass of electrons, but the massses of protons and
electrons could well be essential properties of them, so that a pair of particles whose mass ratio were different from 1836 could 
not be a proton-electron pair.  So in some cases, the arbitrary-seeming parameter does in fact hold of necessity. But that does not
mean that the Mersenne question disappears. For while the parameter itself is not contingent in these cases, there is contingency
``nearby''. Even if the masses of protons and electrons are essential properties, it is possible to have particles with similar
behavior but other masses, and it will be contingent that the world contains a pair of opposite-charge particles with mass ratio 
(approximately) 1836 that form atom-like entities similar to the atoms of our world. 

The point generalizes: Sometimes the apparently arbitrary parameters can be explained by the necessary features in the essences of
things, but in those cases it will often be the case that it is contigent that these essences, rather than other similar ones, are
exemplified. In those cases, the appearance of arbitrariness yields an appearance of contingency, and the true contingency is 
nearby.

There is, however, a further worry here. Consider the apparent arbitrariness of the fact that the ratio of the circumference of
a circle to its diameter in decimal notation has $1$ and $4$ as its second and third digits, respectively. Yet this fact can 
be wholly mathematically explained by necessary mathematical truths such as that $\pi=4-\frac43+\frac45-\frac47+\cdots$.\footnote{This 
point is very similar to an argument Hume makes in Part~IX of his Dialogues??ref.} Thus the appearance of arbitrariness of a parameter is merely
\textit{defeasible} evidence of contigency in the parameter or even nearby. 

We thus have to be cautious: moving from apparent arbitrariness to contingency, whether of the parameter itself or of something ``nearby'', is
always going to be a defeasible and non-deductive move. This is why there is a value in Mersenne's giving as many examples as he does, since 
non-deductive arguments tend to stack up. But in any case, a number of Mersenne's particular examples, such as the astronomical distance examples, 
are ones where it would be difficult to believe in a necesity-based explanation without any contingency involved. 

In the rest of the book we will find that if we turn our attention away from science and towards philosophy, we will
find a myriad of cases like Mersenne's where there are seemingly arbitrary parameters. But these will be cases where
a randomness explanation is implausible, bruteness is not satisfactory and the appearance of contingency is undefeated. However, 
unlike in Mersenne's cases, I won't be arguing---at least not in the first instance---that theism provides the solution. Rather, the 
solution will be Aristotelian metaphysics of form.
\chaptertail 

