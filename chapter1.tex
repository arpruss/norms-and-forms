\def\mychapter{I}
\ifdefined\book
\else
\documentclass[11pt,oneside]{amsbook}
\usepackage[backend=biber, citestyle=authoryear]{biblatex}
\usepackage{mathpazo}
\usepackage{graphicx}
\usepackage{amsmath}
\usepackage{tikz}
\usetikzlibrary{arrows}
%\usepackage{titlesec}
\addbibresource{bibliography.bib}
\newcommand\posscite[1]{\citeauthor{#1}'s (\citeyear{#1})}
\newcommand\plural[1]{#1\mathrm{s}}
%\def\posscitewithextra[#1]#2{\citename{#2}'s (\citeyear{#2}, #1)}

%\newcounter{subsubsubsection}[subsubsection]
%\renewcommand\thesubsubsubsection{\thesubsubsection.\arabic{subsubsubsection}}
%\titleformat{\subsubsubsection}
%  {\normalfont\normalsize\bfseries}{\thesubsubsubsection}{1em}{}
%\titlespacing*{\subsubsubsection}
%{0pt}{3.25ex plus 1ex minus .2ex}{1.5ex plus .2ex}

\ifdefined\book
\renewcommand{\thechapter}{\Roman{chapter}}
\else
\renewcommand{\thechapter}{\mychapter}
\fi

\linespread{1.7}
\usepackage[margin=1.25in]{geometry}
\sloppy
\makeatletter
%% TODO: This is a cheat. It's supposed to be {paragraph}{4}, and that's 
%% what it is in amsbook.cls, but then it fails.
\def\paragraph{\@startsection{paragraph}{3}%
  \normalparindent\z@{-\fontdimen2\font}%
  \normalfont}
\def\subsubsubsection{\paragraph}
\makeatother

\def\smalltick{0.15cm}
\def\bigtick{0.3cm}
\def\pointcircle{0.08cm}
\def\causalnode{0.35cm}

\hyphenation{dia-chro-nic}

%\usepackage[utf8]{inputenc} % set input encoding (not needed with XeLaTeX)
\usepackage{amssymb}
\usepackage{mathtools}
\usepackage{enumitem}
\usepackage{amsthm}
\usepackage{physics}
%\usepackage{ntheorem}
\usepackage{chngcntr}
\counterwithin{figure}{section}

\makeatletter
% \def\@endtheorem{\endtrivlist\@endpefalse }% OLD
\def\@endtheorem{\endtrivlist}%

\setlist[description]{font=\normalfont\scshape}

\catcode`\|=\active\def|{\mid}
\DeclarePairedDelimiter{\ceil}{\lceil}{\rceil}
\DeclarePairedDelimiter{\floor}{\lfloor}{\rfloor}
\newcommand{\Subj}{\mathbin{\raisebox{.15ex}{$\scriptscriptstyle{\Box}$}\kern-.425em\rightarrow}}
\def\Existence{E!}
\def\Believes{\operatorname{Believes}}
\def\True{\operatorname{True}}
\def\Perfection{\operatorname{Perfection}}
\def\ext{\operatorname{Ext}}
\def\Iff{\leftrightarrow}
\def\Implies{\rightarrow}
\def\Entails{\Rightarrow}
\def\Cov{\operatorname{Cov}}
\def\Equiv{\Leftrightarrow}
\def\Form{\operatorname{Form}}
\def\Informs{\operatorname{Informs}}
\def\technical{$\star$}
\def\vtechnical{$\star\star$}
\def\power{\wp}
\def\Nec{\Box}
\def\Poss{\Diamond}
\def\Prop#1{$\langle$#1$\rangle$}
\def\R{\mathbb R}
\def\N{\mathbb N}
\def\tele{tel\={e}}
\makeatletter
\newtheoremstyle{indented}{3pt}{3pt}{\addtolength{\leftskip}{4.5em}}{-2.5em}{\sc}{.}{.5em}{}
\def\Principle#1#2#3{\theoremstyle{indented}\newtheorem*{principle}{#2}\begin{principle}\def\@currentlabel{#2}\label{#1}#3\end{principle}\let\principle\undefined}
\makeatother
\def\pref#1{{\sc\ref{#1}}}
\def\enum#1{\resume{enumerate}\item #1\end{enumerate}}
\def\ditem#1#2{\begin{enumerate}[resume]\item \label{\mychapter:#1} #2\end{enumerate}}
\def\nitem#1#2{\begin{description}\item[#1\label{\mychapter:#1}] #2\end{description}}
\def\bref#1{\ref{\mychapter:#1}}
\def\dref#1{(\ref{\mychapter:#1})}
\def\drefglobal#1{(\ref{#1})}
\usepackage{graphicx} % support the \includegraphics command and options
\usepackage{array} % for better arrays (eg matrices) in maths
\def\Not{\operatorname{\sim}}
\def\St{\operatorname{St}}
\def\num{\operatorname{num}}
\def\And{\mathrel{\&}}
\def\Or{\vee}
\def\BigOr{\bigvee}
\def\<{\langle}
\def\>{\rangle}
\def\union{\cup}
\def\nleq{\not\le}
\def\N{\mathbb N}
\def\R{\mathbb R}
\def\C{\mathbb C}
\def\Powerset{\mathcal P}
\def\star#1{{}^*#1}
\def\hN{\star{\N}}
\def\hR{\star{\R}}
\def\Z{\mathbb Z}
\def\Power{\mathcal P}
\def\Implies{\rightarrow}
\def\True{\operatorname{True}}
\def\Socrates{\mathrm{Socrates}}
\def\actual{@}
\def\Law{\operatorname{Law}}
\def\Chance{\operatorname{Chance}}
\def\Var{\operatorname{Var}}

\def\H2O{H${}_2$O}

\def\scr{\mathcal}
\def\e{\varepsilon}
\def\eps{\varepsilon}
\newtheorem{lem}{Lemma}
\newtheorem{prp}{Proposition}
\newtheorem*{theorem}{Theorem}
\newtheorem{corollary}{Corollary}
\newtheorem{cond}{Condition}

\renewcommand\thechapter{\Roman{chapter}}

\def\chaptertail{\ifdefined\book\else\end{document}\fi}
 

\title{Infinity, Causation and Paradox}
\author{Alexander R. Pruss}
%\date{} % Activate to display a given date or no date (if empty),
         % otherwise the current date is printed

\begin{document}
\setcounter{secnumdepth}{3}
\setcounter{tocdepth}{4}

\end{document}
\fi

\restartlist{enumerate}

\chapter*{Acknowledgments}
I would like that thank ... Nicholas Breiner ....??
\chapter{Introduction}\label{ch:intro}
\section{Introductory Remarks}
I have a human nature or human form that governs my voluntary and involuntary activity.
Much as the government governs the activity of the people \textit{both} by legislating norms and encouraging
people to follow the norms, my nature's governance also has the dual role of setting norms for me and influencing my
activity to follow these norms. This nature is something real and intrinsic to me, something that makes me be what I am, 
a human being. 

When extended to other fundamental beings besides humans, the above is the center of Aristotle's metaphysics.
I will show that this center is extremely fruitful, providing compelling solutions to problems in ethics,
epistemology, the philosophy of mind, semantics, metaphysics and philosophy of science. Many of these are prominent problems that have been
the subject of much discussion, such as the problem of priors in Bayesian epistemology or of vagueness in semantics, while others 
are problems
that have not attracted much attention, such as the problem of seemingly arbitrary detail in moral rules. 
I shall discuss these solutions in Chapters~\ref{ch:ethics}--\ref{ch:laws}.

The ability to give unified solutions to an array of problems spread through many areas of philosophy gives one
a very good reason to accept the central Aristotelian theses. However, in Chapter~\ref{ch:God}, I will also argue that this center cannot 
hold on its own, and the way to be an intellectually satisfied Aristotelian, especially after Darwin, is to be a theist 
as well.

There are several lines of thought readers attracted to the unified Aristotelian solutions may  follow. Some
may deny that the problems facing the central Aristotelian theses are as serious as I contend. Some may agree that
the problems are serious, and regretfully reject the Aristotelian apparatus, either because they take the cost of 
the theistic solution to be too great or are unconvinced that the theistic solution works on its own terms. 
Others may agree that the problems are serious but find some other solution than the theistic one. But some, I hope, 
will conclude that the Aristotelian solutions are so attractive, and the theistic solution to the problems is sufficiently 
plausible, that this book provides not only a good reason to accept the Aristotelian center but also to accept
theism.

We will be elaborating the metaphysical apparatus of what I have been calling the ``Aristotelian center'' gradually??
as we move through the problems and details of their solutions. At the same time, not every detail of the solutions needs to be
adopted by the reader to find the general Aristotelian strategy compelling. Finally, in Chapter~\ref{ch:details} we will
collect together the needed aspects of the Aristotelian metaphysics and discuss in greater detail the metaphysics needed.

??paths through the book?

In the rest of this chapter, we will do two things. First, I will sketch the central Aristotelian metaphysics in slightly
greater detail. Second, I will discuss a neglected science-based argument from the 17th century polymath Marin Mersenne for the existence 
of God. This argument does not work, I will argue. However, an important thread running through this book will be how ``Mersenne 
problems'' analogous to the problems in science raised by Mersenne arise in many areas of philosophy and provide a compelling 
case for the existence of Aristotelian natures or forms.

\section{Aristotelian Natures}
According to Aristotle, reality is fundamentally built out of substances, which are real mind-independent entities.
These substances are not limited to microphysical entities like quarks and photons---indeed,
it is not even clear that the microphysical entities are substances at all\footnote{The fact that in quantum mechanics, 
one can have a superposition of states with different numbers of particles is evidence that particles are not substances.??}---and
indeed Aristotle takes biological organisms like oak trees and human beings to be paradigm cases of substances.??ref

Each material substance has a form or nature---I will use the terms interchangeably in this book. This form or nature performs a number of roles including unifying the matter of the 
substance into a single thing, setting norms for the structure and activity of the substance, and guiding the actual 
development and activity of the object. The nature of the oak tree is not merely an arrangement of its particles, since an
arrangement lacks normative force. In living things, the form of the substance is its life or soul: it makes the substance
be alive. 

Natures are innate to their substances. Nonetheless, this statement underdetermines an important question, namely whether 
substances of the same sort---say, red oaks---all numerically share one nature or each individual substance has its own nature, 
albeit in relevant respects??forwardref they are all exactly alike in substances of the same kind. For two things could in
principle share something innate to them. It could be that all people have the same soul, much as two conjoined twins could
have the same stomach. Aristotle scholarship is divided on the question whether Aristotle believed in ``individual forms'',
one per substance. However, at least one of the advantages of an Aristotelian theory of form will be accentuated if we accept individual
forms, as we shall see.??forward  Further, there is good
philosophical reason to take natures to be individual, as we shall see in ??forward. 
Thus, I shall take natures to be individual. 
Nonetheless, if you like shared forms, 
\textit{many} of the benefits I will draw out for a theory of forms will be ones you, too, can have. 

\section{Mersenne Problems}
Marin Mersenne was a monk, philosopher, theologian and the 17th century equivalent of the arXiv preprint archive---he was 
a crucial line of communication between a broad variety of thinkers and scientists. He drew on his broad knowledge of
the science of the time to offer an argument that begins with many pages of questions, of which the following are
representative:
\begin{quote}
Who gave more strength to the lion than to the ant?
Who made it be that earth is not in the moon's place, and that the planets aren't larger or smaller, closer or further?
Who has ordered all the parts of the world as we see them?
...
Why is the moon 56 earth-radii away from the earth? Why is the sun 1182 [earth-radii] away from us at its apogee? ... and why is its distance at perigee not other than 1101 [earth-radii]? ...
I could equally ask you about Saturn, and Jupiter, and Mars ...??refs\footnote{The moon-earth distance is approximately correct.
The earth-sun distance is an order of magnitude off.}
\end{quote}
These ``Mersenne questions'' go on and on, with a mind-numbing number of examples. And Mersenne has one answer to all
these questions, posed in a rhetorical question: ``Was it not God?''??ref

The argument sounds similar to fine-tuning arguments for theism which became popular in the late 20th century. These
arguments, too, list a variety of physical parameters and offer God as an explanation of them all.??ref But there is
a crucial ingredient that the fine-tuning arguments, namely that the parameters listed are needed for intelligent life
as we know it, or for some other valuable trait of the universe, like its amenability to scientific investigation.??ref
The basic idea behind the fine-tuning argument is, very roughly, that nature is indifferent to value but God cares 
about value, so the fact that the parameters are valuable provides evidence for theism over naturalism.

It is, thus, natural to look in Mersenne for arguments that it is particularly valuable for the moon to be 56 earth-radii
from the earth, but at least in this work, Mersenne does not supply them or even hint at them. Nor is there any argument
that it is better that lions are stronger than ants, or that it is better for the moon to orbit the earth rather than
the other way around. If Mersenne is giving a fine-tuning argument, the argument is oddly incomplete. And Mersenne's
penchant for adumbrating detail at great length makes it unlikely that he has simply omitted such a crucial part of the
argument.

Rather, it appears that Mersenne is simply looking for an explanation of the scientific details he cites, sees no
prospect of a scientific explanation, and offers theism as the alternative. And indeed it is only in the 20th century
with computer models of solar system formation that we have much in the way of plausible answers to Mersenne's questions
about the distances between solar system bodies. For instance, the leading theory of lunar formation involves the earth being hit by
another body and a large chunk being pushed into orbit. Given assumptions about the impact, one can then explain the
resulting distance between the earth and the moon. But notice that such an explanation only gives an answers to the 
Mersenne question about the earth-moon distance at the cost of raising similar Mersenne questions about the 
parameters of the impact such as the mass distribution of the pre-impact earth, the angle and location of impact, the 
mass distribution of the impacting body, etc.

Mersenne gives a dizzying number of examples, and he seems to relish the sheer arbitrariness of the numbers like ``56''
and ``1182''. While this has some rhetorical force, it also has argumentative force. The more arbitrary-looking parameters
the parameters are, the less epistemically likely it is that they are what they of necessity or that good scientific theories
will predict their exact values. And the greater the number
of parameters, the less likely it is that science can provide an explanation of them all.

But Mersenne has a fatal argumentative flaw. Even if we grant that it is very unlikely that a future science will predict
these exact numbers, there is always the possibility of a stochastic explanation, one that does not predict exact values, 
but supposes a random natural process that generates a set of values at random. Now, if Mersenne had an argument showing
that the values of the parameters are suspiciously valuable---say, necessary for intelligent life---then a stochastic
explanation might not be as good as a theistic one. From a Bayesian point of view, we might be able to argue that it is
extremely unlikely that a random selection of parameters would have such value, but not nearly so unlikely that God would
choose such parameters and hence the data supports theism over randomness. But given that Mersenne makes no case that 
the parameters have anything to recommend them to God for creation, we have no reason to think that the probability of God 
choosing is these parameters is any higher than the probability of them arising randomly, and hence we have no support for
theism.

Suppose, however, that we had a Mersenne-type case where randomness was not a satisfactory explanation. Then there would
still be one more problem with the argument. If one is willing to deny the Principle of Sufficient Reason, one could simply
say that the parameters are what they are and there is no reason why they are like this---that they are a \textit{brute} fact.
This, however, is less satisfying than the stochastic answer, for adverting to brute fact should be a last resort, to be
chosen when no explanation is available. But here there is an option, namely theism.

In the rest of the book we will find that if we turn our attention away from science and towards philosophy, we will
find a myriad of cases like Mersenne's where there are seemingly arbitrary parameters. But these will be cases where
a randomness explanation is implausible and bruteness is not satisfactory. However, unlike in Mersenne's case, I won't
be arguing for theism as providing the solution. Rather, the solution will be Aristotelian metaphysics of form.

\chaptertail 

