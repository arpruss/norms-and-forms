\def\mychapter{III}
\chapter{Ethics and metaethics}\label{ch:meta}
\section{Metaethics}
Metaethics is an account of why the most fundamental ethical truths are true. If we were to make a wish-list
for metaethics, it would arguably include the following desiderata for what should follow from the theory:
\ditem{iii-obj}{Ethical truths are objective}
\ditem{iii-know}{Ethical truths are knowable}
\ditem{iii-compel}{The explanation of fundamental ethical truths makes them morally compelling to us}
\ditem{iii-plausibility}{The normative implications are plausible.}

Recall our $\pi$-metaethics on which what made ethical claims true is that they were encoded at some
specific position in the digits of $\pi$. This gave us objectivity and knowability (at least given
the specific position and encoding system).

An individual relativism
that says that the right is what agrees with one's belief as to what is right,  on the
other hand, gives us knowability, and insofar as we find morally compelling the idea that we should obey
our conscience it gives us some compellingness, but it lacks objectivity. Furthermore, its normative implications as to
what I ought to do are very plausible to me, since obviously I find my own moral views plausible, but the theory's implications
for what Hitler should do---namely, that precisely those actions that he believes are right are the ones he ought to do---are implausible
to me (and you) in light of the odiousness of his beliefs.

Utilitarianism, on the other hand, considered as a  metaethical theory about the nature of the right, yields objectivity, knowability
and moral compellingness (the idea that what we should do is maximize the good is among the \textit{prima facie} most plausible of moral ideas), but
it yields a lot of very implausible normative consequences.

A Natural Law metaethics on which for an action to be right is for it to be a proper exercise of the will according to our nature
yields a limited objectivity: it makes the right be relative to our kind. But as we saw in Chapter~\ref{ch:ethics}, this degree
of relativity is highly plausible: it is plausible that ethical requirements do vary between different kinds of intelligent
beings.

The Natural Law metaethics yields knowability when we accept the Aristotelian harmony theses that things generally function
correctly and that the various norms for a thing tend not to conflict. For instance, given such a thesis, the norms for our
emotions---including emotions such as moral repugnance or moral admiration or the feeling of obligation---are likely to cohere
with our norms for our actions, and by and large our emotions and actions are apt to be correct. This enables us to evaluate
normative ethical theories according to the constraint of whether their requirements fit sufficiently with our emotions and require
actions that are not too distant from those that people actually perform, especially in the case of people whose lives appear to
be harmoniously flourishing.  We thus have a rational equilibrium epistemology for our ethics.

The basic idea here is that we ought exercise our will correctly. This is so compelling that it smacks of triviality. Nonetheless,
the claim is not trivial, since it provides an analysis of the moral ought in terms of the functional correctness of our wills.
We find compelling the idea that we should be true to ourselves. But to be true to ourselves is not just, as is popularly supposed,
being true to our changing beliefs and values, but it is to be true to that which makes us be the kinds of things we are: our nature.

The metaethics of right action as the proper functioning of the will is \textit{prima facie} compatible with a very broad variety of normative theories.
It seems we can imagine a being whose will's proper function is to will maximal total utility{Thus, Aristotelian metaethics is compatible with
utilitarian normative ethics, but not with metaethical utilitarianism on which the right is \textit{defined} as what maximizes utility.}, or to will in accordance with God's commands, or
to will what is universalizable, or to will one's flourishing, or even to cause maximal harm to self. Some of these views will, however, be less
plausible given other Aristotelian commitments, such as harmony theses. The harmony theses make it unlikely, for instance, that the right thing
be maximal self-harm. Indeed, harmony theses ensure that the normative consequences of the ethical theory be, by and large, fairly intuitive.
At the same time, there is a real possibility of error, and of correction of that error.

Natural Law metaethics does justice to the idea that the source of our obligations is in us, rather than in some external fact---such as
a divine command---whose moral relevance is questionable. We are our own moral legislators, but because our nature is metaphysically not
up to us, we do not have a choice as to what we legislate and we can be wrong about what we have in fact legislated. Natural Law
metaethics will thus accept with modifications both the relativist's and the Kantian's insistence on autonomy, but without the
ultra-conservative consequences of relativism on which we are always guaranteed to be right and hence never have reason
to change our views, and while avoiding the merely formal character of Kantianism which makes it unlikely to yield sufficient
normative consequences to guide our lives.

Other metaethical theories may satisfy the four desiderata as well.

\section{What are moral or rational norms?}
The idea that norms are species relative suggests that in the space of possibilities---and perhaps in extraterrestrial reality
as well---there will be species of beings that are intelligent enough to have advanced science and technology, but whose natural
behavior will be quite different from us. This raises a difficult question as to what makes a particular set of
natural norms count as a set of moral or even rational norms, and hence makes the species that possesses them a species of moral 
or rational agents.

Not all natural norms are moral or rational norms. The natural norm behind a properly functioning horse shedding in the spring
is neither moral nor rational. Plausibly, a necessary condition for a moral norm is that it govern voluntary behavior.
But the question of what behavior counts as voluntary is difficult. It is tempting to say that the behavior of an entity is
voluntary if it is subject to reasons. But reasons live in a space made possible by rational norms, and so it seems we need
an account of rational norms, at least, to make sense of what behavior is voluntary.

Here is one highly speculative Aristotelian functionalist way to answer the questions. First, we connect reasons and norms with goods
considered as such. An (internal) reason 
for a behavior is a representation of the behavior as \textit{good}. A mouse may represent cheese as yummy, or maybe even as nutritious, 
but not as \textit{good}. Of course, being yummy or being nutritious is thereby good, but to represent as yummy or nutritious is not 
to represent as good. 

Then a necessary condition for a behavior to be voluntary is that it 
comes from such a reason. This, of course, raises the infamous problem of in-the-right way. A behavior can be caused by a reason
without being voluntary. The famous case is the belayer who intends to murder a climber by dropping the rope, and then
his hands start shaking at what he has intended to do, which results in an involuntary dropping. Whatever reason he had
for the murder is the cause of the dropping, but the dropping is involuntary. Aristotelian metaphysics does, however, seem
to have a tool for solving this problem. Causation can be seen to be teleological in nature, and we might say that it is 
a primitive fact that sometimes the effect \textit{is} a fulfilment the teleology of the cause, in which case we can say that the effect 
is caused in the right way. A voluntary behavior is one which is a fulfillment of the teleology of the cause.

Finally, the will can then be functionally defined as the system by which reasons lead to behavior. A rational norm is a norm of behavior
of the will favoring some or all reasons, and a moral norm is a norm of behavior of the will favoring some or all reasons that themselves 
are focused on  a good not considered primarily as a good to self. 

This is not, of course, the only way to define which norms are moral or rational. And it is quite possible that the question of which
norms are moral or rational is largely a verbal question. Go back to the characterization of reasons in terms of goods. The mouse takes
the cheese to be yummy, and that is not taking the cheese to be good. But humans often represent good things in thicker ways: as
beautiful, courageous, or even divine. Couldn't we imagine a continuum of animals where at one end the cheese is represented as 
yummy and on the other as having gustatory beauty? Somewhere in the continuum we have moved to representing the cheese as having a 
thick form of goodness. But where this happens is unclear. 

We have a certain set of norms for the will. We can call these rational, and a subset of them moral. What hangs on whether a different
set of norms governing the behavior of a different class of beings counts as rational or moral? Couldn't this be like the question of
which animals' hard projections count as horns? 

But perhaps the question of which things are moral agents matters for first order moral questions about interspecies relations, such 
as which organisms we are permitted to eat, whose lives we should save, etc. However, it is not clear that the answers to these questions will neatly line up with
determinations of the boundaries of moral agency. Consider intelligent whales that are fine philosophers and scientists in their epistemic life, 
and that even enjoy contemplating the good, but do so purely non-practically, without the good being any motivator for their 
actions, which are all instinctive. It would seem wrong to each such beings, even if they turn out not to be moral agents.
On the other hand, imagine a shrimp that has the normal behavioral complement of a shrimp, with one exception: it represents
the ingestion of algae as good, and voluntarily pursues this good as such. And its intellectual abilities are the minimum needed
for an ordinary shrimp's life as combined with the most minimal possible concept of the good. It may well be wrong to eat such shrimp,
but given a choice whether to save the life of one of these minimally moral shrimp or the life of one of the intelligent but amoral 
whales (of course, we should not be biased here by size!)


\section{Flourishing}
A substance flourishes to the extent that it functions in accordance with its norms.  Acting morally rightly is
a case of functioning in accordance with the norms for the functioning of the will. Thus, acting rightly morally is an aspect of flourishing
for those substances that have a will. At the same time, unless we should deal with a substance that consists of nothing but will, there will be other
aspects of flourishing. Because of this, conflict between moral rightness and self-interest is in principle possible.

Admittedly, Aristotelian harmony tends to limit such conflict. Right action tends to promote other aspects of a substance's well-being. But nonetheless
just as jogging sometimes promotes cardiac wellbeing at the expense of joint wellbeing, so too right action promotes volitional or moral wellbeing at the expense
of life and other goods.

Starting with Socrates, Western ethical reflection has often insisted on moral wellbeing being the most important aspect of a human's
well-being. This may seem to be necessary for preserving the idea that one should do the right thing even when this costs one heavily,
by allowing one to insist that the cost of doing wrong is always greater than the benefits. But the thesis that moral wellbeing trumps
other forms of wellbeing is neither necessary nor sufficient for preserving the need to act rightly.

It's not sufficient since even if moral wellbeing trumps other forms of wellbeing, there are imaginable situations where doing the right thing will on balance
very likely harm one's wellbeing. For instance, suppose I am a bank employee of mediocre morals and the best empirical evidence available to
me shows that taking an evening ethics class from Professor Kowalska  would be deeply inspiring and turn me into a vastly better person.
Unfortunately, the only way I could afford the tuition is to embezzle a thousand dollars from a billionaire's account. This embezzlement is
wrong, but I can reasonably expect to be a much better off morally from it.

And it's not necessary that moral wellbeing trump other forms of wellbeing, because if morally right action just is action in accordance with
the norms for the will, then it is clear apart from any trumping thesis why morally wrong action is defective: it is defective because it
fails to be an instance of the proper functioning of the will. One may be the better off if one does the wrong action, but one will still
have acted defectively.

Moreover, it is unlikely to be true that moral wellbeing always trumps other forms of wellbeing. Suppose the best science shows that on average
there is on the whole a moral improvement---perhaps in the area of compassion---from suffering severe headaches, but this improvement is tiny.
A parent who knew this should still relieve a child's severe headache, and wouldn't be acting contrary to benevolence in relieving it.

Nonetheless, it is plausible that moral wellbeing is typically the most important aspect of our wellbeing and that typically other forms
of our wellbeing are appropriately sacrificed to it. This gradation is itself encoded in the human form which specifies what is good for
us and the ordering between the goods.

Traditionally, Aristotelian action theory has insisted that we always act for our happiness. This happiness thesis is compatible with a metaethics on
which right action is the proper functioning of the will, but is neither entailed by it nor particularly plausible. While proper functioning is always
good for a substance, a substance when functioning properly in some way need not be doing so \textit{in order to} function properly in that respect.
When a flower opens up in the right season, its opening up plausibly has as its end the good of reproduction rather than the good of opening up.
Similarly, when you make dinner for your child, your right action is good for you, but you are doing it for the sake of your child and not for
the sake of the action itself.

\section{Supererogation}
??cut?

\section{Supervenience}
It is widely held that moral facts supervene on non-moral facts: if two possible worlds differ
with respect to moral facts, they must differ with respect to at least one non-moral fact. Similarly,
it is held that normative facts in general supervene on non-normative facts. The difficulty is then
to explain the supervenience relations.

The nature-first theorist has a complex relationship to both supervenience claims. Let us begin
with the supervenience of the normative on the non-normative, and first consider
general normative claims such as that every sheep should have four legs and every human should refrain
from torturing the innocent. Such normative claims are necessary truths, since their truth is a part of
what makes a sheep a sheep and a human and a human. Necessary truths vacuously supervene on any basis
we might choose, since there are no possible worlds that differ with respect to necessary truths.

But what about \textit{particular} normative claims, such as that Sally ought to have four legs
or that Biden ought to discharge the duties of the President of the United States? If we are
interested in the supervenience of the normative on the non-normative, we face a serious problem on
Aristotelian metaphysics: there are very few non-normative facts. It seems that every natural kind
is defined in part by normative properties. That Sally is a sheep is itself a normative thesis,
since a part of what it is to be a sheep is to be such that one ought to have four legs. That Sally
is an animal is also normative. And Sally is \textit{essentially} a sheep, and hence being a sheep
is central to Sally's identity in such a way that it may even be the case that even the claim that
Sally exists may count as a normative claim. Aristotelian metaphysics likely reaches even down to the fundamental
particles. Electrons not only do but should repel other electrons. A non-normative fact, thus, will
not make reference to any natural kinds, and not even to any particulars falling under natural
kinds.

On Aristotelianism, every particular, with the exception of God if there is a God, falls under a natural kind.
But facts about God are through-and-through normative, since God is essentially perfectly good. Thus,
on Aristotelianism, all facts about particulars are normative. 
Moreover, on Aristotelianism, a non-normative fact cannot include anything existential.
For to be is to be a substance or to be appropriately related to a substance (say, by being its accident).
And a part of what it is to be a substance is to have a form that specifies how one should behave. Thus,
what it is to be is in part to have norms or to be related to something that has norms. If so, then every
existentially quantified claim is normative. The denial of a normative claim is normative as well, and
since a universally quantified claim is the denial of an existentially quantified claim (everything is
$F$ if and only if there does not exist an object that is not $F$), universally quantified claims 
will be normative as well. 

But if all facts about particulars and all quantified facts are normative, it seems that \textit{all} facts
are normative on Aristotelianism.  If this is right, then to say that two worlds are the same in non-normative terms 
is to say literally nothing about them. And if all facts are normative, then any two worlds that are the same
in normative terms are altogether the same. Thus, the thesis of the supervenience of the normative on the
non-normative becomes the thesis of modal fatalism: that there is only one possible world! And we have good
reason to reject this thesis.

But while this goes against mainstream views of normativity, it is arguably an advantage of the
view. For by eliminating non-normative facts, we no longer have any puzzling phenomenon of the relationship 
between the normative and the non-normative to be explained. 

What about the moral supervening on the non-moral? Moral facts are normative facts about the
will. Again, general moral facts, such as that humans should refrain from torturing the innocent or should
discharge the duties of the President of the United States if they have voluntarily sworn the
relevant oath of office,
are necessary truths and hence trivially supervene on whatever facts we want, including
non-moral or even non-normative ones. But there are many particular normative facts, such as
that Biden should discharge presidential duties, that depend on facts about human wills,
such as that Biden \textit{voluntarily} swore the oath of office, and since it is the very nature
of the
human will to be such that various moral facts about it hold, facts about human wills are not going
to be among the non-moral facts. Thus the Aristotelian will also reject the supervenience of the
moral on the non-moral.

But what the intuition underlying the supervenience claims? We can put the intuition as follows.
Imagine a world that \textit{looks like} ours. It has bipeds that look just like us. There is,
for instance, a biped that is empirically indistinguishable from Biden. The history of these bipeds
is empirically indistinguishable from our history. Could it really be that the moral facts about
these bipeds be other than about us? Could it be, for instance, that although these bipeds behave just
as we do towards those who torture people at random, among them there is nothing morally wrong with
such torture?

\section{Outlandish paradoxes}
It is easy to generate paradoxes in ethics and decision theory by invoking outlandish situations.
Many such situations involves infinities. I will give two representative examples.

First, we have the Satan's Apple paradox about infinite sequences of choices on which something
further depends:
\begin{quote}
Satan has cut a delicious apple into infinitely many pieces, labeled by the natural numbers. Eve
may take whichever pieces she chooses. If she takes merely finitely many of the pieces, then she
suffers no penalty. But if she takes infinitely many of the pieces, then she is expelled from the
Garden for her greed. Either way, she gets to eat whatever pieces she has taken. ??ref
\end{quote}
The puzzle is that for each piece, Eve has conclusive reason to take the piece, but if she acts on
all these reasons, something terrible happens. As presented, this is a paradox about self-interest, 
but we can turn it into an ethical one by supposing that the rewards and penalties of Eve's choices 
devolve on someone else, say Adam. In that case, we can say that Eve should accept each piece and yet
that's the worst option.

Another kind of paradox involves infinite numbers of beneficiaries. Imagine that there is an infinite
number of complete strangers, numbered with the integers (negative, zero and positive),
as well as two cats, all facing a deadly danger, and you have a choice between one of three equally convenient options:
\ditem{iii-saveA}{Save the strangers numbered $0,1,2,...$.}
\ditem{iii-saveB}{Save the strangers numbered $-1,-2,-3,...$ and one cat.}
\ditem{iii-saveC}{Save the strangers numbered $1,2,3,...$ and two cats.}

Now, you have no reason to prefer
the stranger numbered $0$ over the stranger numbered $-1$, the stranger numbered $1$ over the
stranger numbered $-2$, and so on. So as far as the saving of people, \dref{iii-saveA} and \dref{iii-saveB}
are a wash, but it's better to save a cat than not to, so \dref{iii-saveB} is morally preferable.\footnote{If the
reader thinks that cats do not fall in our moral purview, just replace the saving of a cat with saving a human
from some minor harm.}

But likewise there is no reason to prefer saving the people numbered with negative integers over the
people numbered with positive integers, so as far as the saving of people goes, \dref{iii-saveB} and
\dref{iii-saveC} are balanced. However, saving two cats is better than saving one, so \dref{iii-saveC}
is better than \dref{iii-saveB}. 

But now, \dref{iii-saveA} is clearly better than \dref{iii-saveC}: for in \dref{iii-saveA}, you save stranger
$0$ instead of the two cats, and wonderful as cats are, it is much better to save that one human over two cats. 

So we have a moral preferability circle, and whatever you do, there is something better you could have
done at no greater cost. It seems plausible that you have a duty better if you can do so at no greater
cost, and yet whatever you do, you violate that duty. And so it seems that you cannot act as you ought,
thereby violating the plausible maxim that ought implies can.

There have been various attempts to defuse such paradoxes,???refs and a defender of human nature as the
foundation of ethics can accept any of them. However, there is also a simple and highly intuitive 
alternative to these defusions. A horse's nature may ground facts about the appropriate gait when
browsing on grass and the appropriate gait when fleeing a predator through water. But equine nature
is simply silent on a horse's gait when fleeing aliens in a zero-gravity environment. Similarly,
our human nature could be silent on how we should act in outlandish situations, and our
principles just need not extend to such cases. This fits very well with the ordinary person's
disdain for philosophers (like me) who spend a lot of time thinking about such cases. 

There is another similar solution. It could be that our ordinary moral rules \textit{do} extend to outlandish
cases. Thus, the moral reasoning by which we generated the moral paradoxes in Satan's Apple and the 
infinite saving case may be correctly grounded in norms in our nature. It may well be that, say,
\dref{iii-saveB} is morally better than \dref{iii-saveA}, that \dref{iii-saveC} is morally better than \dref{iii-saveB}, 
that \dref{iii-saveA} is morally better than \dref{iii-saveC}, and that you ought to do the morally best (or one
of the morally best, if there is a tie) between the three options. It's just that these specifications of 
our nature are impossible to fulfill under these circumstances. In other words, it is very plausible to
say that ought implies can in situations that are a part of humans' natural environment, but there may be
logically possible outlandish situations that go far beyond this environment where ought no longer implies can. 
Insofar as our nature gives us norms fitted to our human environment, we should not be surprised if these norms
have counterintuitive implications, such as violating ought implies can, in situations far outside that environment.

Similar solutions will be available on at least two other ethical theories where the laws may be customized
to humanity: contractarianism and divine command theory. But, on the other hand, such solutions will be
implausible on theories that purport to apply to any kind of rational being at all, theories such as utilitarianism
or Kantianism. 

A similar point can be made about outlandish epistemological paradoxes. ??refs-and-examples  On a natural law epistemology, we should not expect our nature to give us guidance, or at least satisfactory guidance, 
in situations too far out of the human environment. And while in ethics there are at least two common anthropocentric alternatives to natural law, contractarianism and divine command, in epistemology anthropocentric alternatives are harder to find.??Hawthorne?
Thus, we have perhaps an even stronger consideration in favor of a normativity based on human nature on the epistemological
side. 

More will be said in ??forward about outlandish scenarios.

\section{Agent-centrism}
\subsection{The egoism objection}
According to Natural Law metaethics, an action is right provided that its performance constitutes the will's flourishing,
and is wrong provided that its performance constitutes the will's languishing. This seems objectionably egoistic. Paradigm
cases of moral wrongness involve harm to others, and are wrong because of that harm. The thought that the action makes the 
agent languish is a thought too many.??refs Therefore, the argument goes, we should opt for an other-centered metaethics.

My response will be two-pronged. First, I will argue that the very features criticized in Natural Law provide a significant advantage
in a number of cases. Second, however, I will argue that the argument against Natural Law's agent-centric character only works
against some normative developments of the Natural Law metaethics, rather than against the metaethics. 

\subsection{The normative advantages of agent-centrism}
Other-centered theories nicely account for what is wrong with murder: it gravely harms the victim. They account slightly less well 
for what is wrong with typical cases of attempted murder: it is an attempt to harm to harm the victim. But they do not account for
atypical cases of attempted murder where the victim simply does not exist. Suppose Alice thinks that she has an identical twin 
living somewhere in Toronto, and sets out to kill her, to avoid the twin's claiming an inheritance. Alice has an extremely rare
genetic disorder which an identical twin would share, but which is very unlikely to be otherwise exhibited even in a city as large as
Toronto. She adds a poison to Toronto's water supply that targets only people with this genetic disorder, and then takes care to avoid
drinking Toronto's tap water. But in fact Alice never had a twin. 

There is thus no one that Alice is attempting to kill. Yet morally speaking, she is just as guilty as in an attempted murder case 
where her twin exists, and depending on one's views on moral luck maybe even as guilty as in the case where she succeeds in killing 
her twin. It is worth noting that in Anglo American jurisprudence, Alice might get away under the doctrine of impossible attempts,
on which an attempt has to have some feasibility, and trying to kill a non-existent person has none. (Of course, Alice is likely to be
convincted for pollution and for reckless endangerment of people with this disorder, but these are lesser evils.) However, it is clear 
that notwithstanding the law of impossible attempts, it makes no difference to Alice's guilt whether in fact she has a twin or not.

We may, of course, try to save the doctrine that wrongs are always wrongs to another by trying to identify other victims, such as 
society or God. We can try to tweak the Alice case to exclude the society solution. Perhaps Alice is trying to kill her twin sister
in a world where she thinks they are the only survivors of a disaster, but in fact Alice is the only survivor and she's never had a
twin. That won't help with the God case, at least not within classical theism, since God is traditionally thought of as a necessary 
being??Refs. Furthermore, the intuition that I am responding to is that there is something particularly centered
on the ordinary direct human victim of a wrongdoing that contributes much of the wrong. And if God or society is what we count as 
the victim in the Alice case, then it seems that we have to say that in the Alice case there is less wrong than in the more ordinary
case of attempted murder where the victim actually exists. And yet it seems that how wrong Alice's action is does not depend on whether
she has a twin.

The above focused on patient-centric wrongs. We can also think about patient-centric duties. Again, it seems that our duties go 
beyond these. Plausibly, we have a duty not to deliberately produce a human being who is so genetically constituted as to be 
practically guaranteed to have a life of unremitting suffering. Now suppose that Bob and Carl both know that they and their 
spouses have genes such that if they reproduce, the child will have a life of unremitting suffering. In light of this knowledge,
Bob refrains from reproduction, while Carl's sadistic tendencies impel him to reproduce in light of this fact. Bob and Carl both
have a duty. In Carl's case, we can identify the individual to whom he has this duty, an individual that that he has wronged.
But in Bob's case, the analogous individual does not exist---precisely because Bob has fulfilled his duty. Again, we can try to
identify others to whom Bob owes not having a child---society, God and likely Bob's wife. But since there is one less individual
here than in Carl's case, Carl has a somewhat more stringent duty, since the unfortunate child exists. However, it does not seem
that Carl has a more stringent duty than Bob.

Finally, moving away from cases, it is obvious that some degree of agent-centrism is needed for any plausible story about wrongs 
and duties. It is \textit{agents} that do wrongs and have duties. We have a duty not to eat humans. Lions, pigs and horses have no 
such  duty. Therefore, an account of what makes it wrong for me to eat other humans has to involve some facts about \textit{me}.
It cannot be wholly other-based. 

One might respond that on the Natural Law account, what makes it be wrong for me to eat other humans involves \textit{only} facts
about me, while it should also involve facts about the prospective victims. But how we understand this objection depends on how 
we read questio of ``what makes it wrong for me to eat other humans''. 

First, we can understand it as a question about the grounds of 
the general moral rule that it is wrong for humans to eat other humans. On Natural Law the grounds of that general moral rule 
are entirely within the agent. However, that is how it should be. For the general moral rule would also hold even if there were
no other human beings in the world. And in fact the general moral rule would hold \textit{non-trivially} even if there were no
other humans, since even if there were in fact no other human beings, I should avoid actions that are likely to constitute the eating
of a fellow human being (e.g., shooting and eating an animal that has a significant epistemic probability of being human). 
An account of the wrongness of eating other human beings that requires other humans to exist is unsatisfactory.

Second, we can understand the question as asking about particular cases: What is it that makes it wrong for me to eat, say, Carl?
But then the Natural Law story is going to include a fact about Carl: I am the sort of thing that shouldn't eat other humans 
and Carl is another human. 

On neither reading do we have an argument against Natural Law metaethics.

\subsection{Avoiding agent-centrism in normative Natural Law ethics}
Here let me start with a personal confession. For many years I objected to the eudaimonism I took to be at the heart of Natural
Law, which one might take to consist of the twin theses:
\ditem{eud-hamper}{What makes an action right is its promotion of the agent's flourishing.}
\ditem{eud-end}{An agent's right actions are aimed at the agent's flourishing.}
And these theses seemed objectionably egoistic.

However, while there are ways of pairing the Natural Law metaethics that I have been developing with a normative ethics
that embraces \dref{eud-hamper} and \dref{eud-end}, they are both dispensable. 

Indeed, \dref{eud-hamper} is so clearly wrong that it is unlikely that many Natural Law theorists accept it, given the 
well-known anti-consequentialism of the Natural Law community. It is wrong
to rob a bank in order to pay for the tuition of an ethics class even if there is strong empirical evidence that this 
class will be so transformative that on the whole one's flourishing will be promoted, even if one takes into account the
temporary harm done to it by the robbery. Similarly, one may have a duty to continue working in a job that is just barely
moral and empirically likely to be destructive of one's flourishing as a moral agent in order to pay the medical bills 
for a child. 

Now, on the metaethics that I am defending, what makes an action right is that it \textit{constitutes} the agent's 
flourishing with respect to the will. If we add to this the thesis that whatever constitutes the agent's flourishing
with respect to the will constitutes the agent's flourishing as a whole, then we get a version of \dref{eud-hamper}
with ``promotion'' replaced by ``constitution''. However we should not think that what constitutes the agent's
flourishing with respect to the will constitutes the agent's flourishing as a whole. If an agent is flourishing with
respect to the will, but is full of ignorance, in great pain, and lying in a bed of vomit??Vlastos-ref, the agent
is not flourishing on the whole. 

And the thesis that what makes an action right is its constitution of the agent as flourishing in respect of the will
seems to be simply a thesis about proper function, and does not imply any selfishness. Consider, for instance, that a 
bee's defending the hive at the expense of its life fulfills the bee with respect to whatever we call the driver of 
the bee's activity (we may not wish to call it a ``will''). But this does not make the bee in any real way selfish. 
And certainly a guided missile is not selfish just because it fulfills its nature by exploding.

It is tempting to think that in the case of an agent who is driven by a rational will, if what makes the action 
right is its constituting the agent as flourishing (in one respect), then by willing the action under the
description ``right action'', the agent aims at flourishing, in a way in which neither the bee nor the guided missile's
actions are aimed at flourishing. If this line of thought is correct, then the characterization of rightness in terms of 
flourishing implies \dref{eud-end}, and that seems more objectionably egoistic.

However, a rational agent's intentions are hyperintensional. It is possible to aim at heating up a room without aiming
at increasing the kinetic energy of the molecules in the room, even though what makes there be heat in a room is the
kinetic energy of molecules, and necessarily one is present if and only if the other is. Indeed, during the millenia 
before the relationship between heat and kinetic energy was known, no one aimed at increasing the kinetic energy of 
molecules while heating a room, and even now when the relationship is well-known, few people's intentions in turning 
a thermostat make reference to molecular motion. Similarly, even if the rightness of an action is grounded in, constituted 
by or even identical with the action's being an instance of the agent's flourishing as a willer, the rightness can be 
aimed at without aiming at the flourishing.

Furthermore, as has often been pointed out in the literature??(on moral fetishism), virtuous agents rarely aim at 
rightness as such. Instead, they aim at thicker right-making features of an action, such as its being an expression
of loyalty to a friend, its fulfilling a stranger's need, or its having been promised. It is because the action has
such thick features that its performance is an instance of volitional flourishing. 

There are, of course, times when an agent aims at rightness as such. One set of cases is provided by agents who cannot
figure out on their own what is right and have to take the rightness of an action on the authority of another, without
understanding what makes the action right. Such agents include small children, but also sometimes ordinary well-functioning
adults who find themselves in such situations of such moral complexity that they turn to a professional ethicist or a 
trustworthy friend for advice. Is this objectionably egoistic, assuming the rightness is constituted by the agent's volitional
flourishing? There are at least three reasons to doubt this. The first was already mentioned: the hyperintensionality in 
intentions.

To see the second reason, observe that we actually have a \textit{three layer} story:
\ditem{rightness}{the rightness of the action}
\ditem{flourishing}{the action's being such as to constitutive volitional flourishing}
\ditem{thick}{the thick features of the action because of which the action constitutes
volitional flourishing.}
The egoism objection in the case of agents aiming at right as such insists that the willing of \dref{rightness} inherits 
an egoistic character from the agent-centrism of \dref{flourishing}. But note that \dref{flourishing} is not the end of the story. Just as 
\dref{rightness} is grounded in 
\dref{flourishing}, so likewise \dref{flourishing} is grounded in \dref{thick}. And in paradigmatic cases, the thick features in \dref{thick} 
are other-centric features. If we think that willing the rightness of the action inherits egocentrism from the flourishing, we should even more 
think that it inherits  other-centrism from the thick features, since the thick features are a yet more ultimate ground of rightness
than the flourishing is.

Finally, recall that we already noticed that an action can be right and constitute volitional flourishing but hamper one's flourishing as 
a whole, as in the case of working a soul-destroying job to pay family medical bills or refusing to rob a bank in order to pay for a 
morally transformative class. In such cases, it is absurd to say that by aiming at volitional flourishing one is being selfish, since the
action does not, in fact, contribute to one's good overall.

Indeed, a metaethics that grounds rightness in flourishing as a willer is compatible with one's never pursuing one's own good.
We can (perhaps with some difficulty) imagine an alien species whose members pair off in such a way that the proper functioning of
each one's will is just to will the good of the other member of the pair. Perhaps this is a species so physically constituted that
they are always more effective at benefiting others than at self. In such a species, the Natural Law metaethics would require 
utter unselfishness---and yet what would \textit{ground} the rightness of an action would be that the action is proper to one's
will and hence constitutes the will as flourishing. We could imagine two versions of such aliens. They might be less reflective
than ourselves, and never act on higher-order reasons like rightness as such. Or they might be reflective, and might even come to
a Natural Law metaethics on which an action is made right by its constituting the agent as flourishing. In such a case, they might
aim at an action under the description ``right'', because they know that an action's rightness is ultimately grounded in their 
species in the action's benefiting the other member of the pair, though mediately in its constituting the agent's flourishing.

??agents without self-concept?

\chaptertail 