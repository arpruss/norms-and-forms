\def\mychapter{III}
\chapter{Ethical and metaethical advantages}\label{ch:meta}
\section{Metaethics}
Metaethics is an account of why the most fundamental ethical truths are true. If we were to make a wish-list
for metaethics, it would arguably include the following desiderata for what should follow from the theory:
\ditem{iii-obj}{Ethical truths are objective}
\ditem{iii-know}{Ethical truths are knowable}
\ditem{iii-compel}{The explanation of fundamental ethical truths makes them morally compelling to us}
\ditem{iii-plausibility}{The normative implications are plausible.}

Recall our $\pi$-metaethics on which what made ethical claims true is that they were encoded at some
specific position in the digits of $\pi$. This gave us objectivity and knowability (at least given 
the specific position and encoding system). 

An individual relativism
that says that the right is what agrees with one's belief as to what is right,  on the 
other hand, gives us knowability, and insofar as we find morally compelling the idea that we should obey 
our conscience it gives us some compellingness, but it lacks objectivity. Furthermore, its normative implications as to 
what I ought to do are very plausible to me, since obviously I find my own moral views plausible, but the theory's implications
for what Hitler should do---namely, that precisely those actions that he believes are right are the ones he ought to do---are implausible
to me (and you) in light of the odiousness of his beliefs.

Utilitarianism, on the other hand, considered as a  metaethical theory about the nature of the right, yields objectivity, knowability
and moral compellingness (the idea that what we should do is maximize the good is among the \textit{prima facie} most plausible of moral ideas), but
it yields a lot of very implausible normative consequences.

A Natural Law metaethics on which for an action to be right is for it to be a proper exercise of the will according to our nature
yields a limited objectivity: it makes the right be relative to our kind. But as we saw in Chapter~\ref{ch:ethics}, this degree
of relativity is highly plausible: it is plausible that ethical requirements do vary between different kinds of intelligent
beings. 

The Natural Law metaethics yields knowability when we accept the Aristotelian harmony theses that things generally function
correctly and that the various norms for a thing tend not to conflict. For instance, given such a thesis, the norms for our
emotions---including emotions such as moral repugnance or moral admiration or the feeling of obligation---are likely to cohere
with our norms for our actions, and by and large our emotions and actions are apt to be correct. This enables us to evaluate 
normative ethical theories according to the constraint of whether their requirements fit sufficiently with our emotions and require
actions that are not too distant from those that people actually perform, especially in the case of people whose lives appear to
be harmoniously flourishing.  We thus have a rational equilibrium epistemology for our ethics.

The basic idea here is that we ought exercise our will correctly. This is so compelling that it smacks of triviality. Nonetheless,
the claim is not trivial, since it provides an analysis of the moral ought in terms of the functional correctness of our wills. 
We find compelling the idea that we should be true to ourselves. But to be true to ourselves is not just, as is popularly supposed,
being true to our changing beliefs and values, but it is to be true to that which makes us be the kinds of things we are: our nature.

Finally, the Aristotelian harmony thesis ensure that the normative consequences of the ethical theory be, by and large, fairly intuitive.
At the same time, there is a real possibility of error, and of correction of that error.

Natural Law metaethics does justice to the idea that the source of our obligations is in us, rather than in some external fact---such as 
a divine command---whose moral relevance is questionable. We are our own moral legislators, but because our nature is metaphysically not 
up to us, we do not have a choice as to what we legislate and we can be wrong about what we have in fact legislated. Natural Law
metaethics will thus accept with modifications both the relativist's and the Kantian's insistence on autonomy, but without the
ultra-conservative consequences of relativism on which we are always guaranteed to be right and hence never have reason 
to change our views, and while avoiding the merely formal character of Kantianism which makes it unlikely to yield sufficient
normative consequences to guide our lives.


Other metaethical theories may satisfy the four desiderata as well. 

\section{Flourishing}
\section{Supererogation}
\chaptertail