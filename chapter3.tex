\def\mychapter{III}
\chapter{Ethical and metaethical advantages}\label{ch:meta}
\section{Metaethics}
Metaethics is an account of why the most fundamental ethical truths are true. If we were to make a wish-list
for metaethics, it would arguably include the following desiderata for what should follow from the theory:
\ditem{iii-obj}{Ethical truths are objective}
\ditem{iii-know}{Ethical truths are knowable}
\ditem{iii-compel}{The explanation of fundamental ethical truths makes them morally compelling to us}
\ditem{iii-plausibility}{The normative implications are plausible.}

Recall our $\pi$-metaethics on which what made ethical claims true is that they were encoded at some
specific position in the digits of $\pi$. This gave us objectivity and knowability (at least given 
the specific position and encoding system). 

An individual relativism
that says that the right is what agrees with one's belief as to what is right,  on the 
other hand, gives us knowability, and insofar as we find morally compelling the idea that we should obey 
our conscience it gives us some compellingness, but it lacks objectivity. Furthermore, its normative implications as to 
what I ought to do are very plausible to me, since obviously I find my own moral views plausible, but the theory's implications
for what Hitler should do---namely, that precisely those actions that he believes are right are the ones he ought to do---are implausible
to me (and you) in light of the odiousness of his beliefs.

Utilitarianism, on the other hand, considered as a  metaethical theory about the nature of the right, yields objectivity, knowability
and moral compellingness (the idea that what we should do is maximize the good is among the \textit{prima facie} most plausible of moral ideas), but
it yields a lot of very implausible normative consequences.

A Natural Law metaethics on which for an action to be right is for it to be a proper exercise of the will according to our nature
yields a limited objectivity: it makes the right be relative to our kind. But as we saw in Chapter~\ref{ch:ethics}, this degree
of relativity is highly plausible: it is plausible that ethical requirements do vary between different kinds of intelligent
beings. 

The Natural Law metaethics yields knowability when we accept the Aristotelian harmony theses that things generally function
correctly and that the various norms for a thing tend not to conflict. For instance, given such a thesis, the norms for our
emotions---including emotions such as moral repugnance or moral admiration or the feeling of obligation---are likely to cohere
with our norms for our actions, and by and large our emotions and actions are apt to be correct. This enables us to evaluate 
normative ethical theories according to the constraint of whether their requirements fit sufficiently with our emotions and require
actions that are not too distant from those that people actually perform, especially in the case of people whose lives appear to
be harmoniously flourishing.  We thus have a rational equilibrium epistemology for our ethics.

The basic idea here is that we ought exercise our will correctly. This is so compelling that it smacks of triviality. Nonetheless,
the claim is not trivial, since it provides an analysis of the moral ought in terms of the functional correctness of our wills. 
We find compelling the idea that we should be true to ourselves. But to be true to ourselves is not just, as is popularly supposed,
being true to our changing beliefs and values, but it is to be true to that which makes us be the kinds of things we are: our nature.

The metaethics of right action as the proper functioning of the will is \textit{prima facie} compatible with a very broad variety of normative theories.
It seems we can imagine a being whose will's proper function is to will maximal total utility{Thus, Aristotelian metaethics is compatible with
utilitarian normative ethics, but not with metaethical utilitarianism on which the right is \textit{defined} as what maximizes utility.}, or to will in accordance with God's commands, or
to will what is universalizable, or to will one's flourishing, or even to cause maximal harm to self. Some of these views will, however, be less 
plausible given other Aristotelian commitments, such as harmony theses. The harmony theses make it unlikely, for instance, that the right thing
be maximal self-harm. Indeed, harmony theses ensure that the normative consequences of the ethical theory be, by and large, fairly intuitive.
At the same time, there is a real possibility of error, and of correction of that error.

Natural Law metaethics does justice to the idea that the source of our obligations is in us, rather than in some external fact---such as 
a divine command---whose moral relevance is questionable. We are our own moral legislators, but because our nature is metaphysically not 
up to us, we do not have a choice as to what we legislate and we can be wrong about what we have in fact legislated. Natural Law
metaethics will thus accept with modifications both the relativist's and the Kantian's insistence on autonomy, but without the
ultra-conservative consequences of relativism on which we are always guaranteed to be right and hence never have reason 
to change our views, and while avoiding the merely formal character of Kantianism which makes it unlikely to yield sufficient
normative consequences to guide our lives.

Other metaethical theories may satisfy the four desiderata as well. 

\section{Flourishing}
A substance flourishes to the extent that it functions in accordance with its norms.  Acting morally rightly is
a case of functioning in accordance with the norms for the functioning of the will. Thus, acting rightly morally is an aspect of flourishing
for those substances that have a will. At the same time, unless we should deal with a substance that consists of nothing but will, there will be other
aspects of flourishing. Because of this, conflict between moral rightness and self-interest is in principle possible. 

Admittedly, Aristotelian harmony tends to limit such conflict. Right action tends to promote other aspects of a substance's well-being. But nonetheless
just as jogging sometimes promotes cardiac wellbeing at the expense of joint wellbeing, so too right action promotes volitional or moral wellbeing at the expense
of life and other goods. 

Starting with Socrates, Western ethical reflection has often insisted on moral wellbeing being the most important aspect of a human's
well-being. This may seem to be necessary for preserving the idea that one should do the right thing even when this costs one heavily,
by allowing one to insist that the cost of doing wrong is always greater than the benefits. But the thesis that moral wellbeing trumps
other forms of wellbeing is neither necessary nor sufficient for preserving the need to act rightly. 

It's not sufficient since even if moral wellbeing trumps other forms of wellbeing, there are imaginable situations where doing the right thing will on balance
very likely harm one's wellbeing. For instance, suppose I am a bank employee of mediocre morals and the best empirical evidence available to 
me shows that taking an evening ethics class from Professor Kowalska  would be deeply inspiring and turn me into a vastly better person. 
Unfortunately, the only way I could afford the tuition is to embezzle a thousand dollars from a billionaire's account. This embezzlement is 
wrong, but I can reasonably expect to be a much better off morally from it. 

And it's not necessary that moral wellbeing trump other forms of wellbeing, because if morally right action just is action in accordance with
the norms for the will, then it is clear apart from any trumping thesis why morally wrong action is defective: it is defective because it
fails to be an instance of the proper functioning of the will. One may be the better off if one does the wrong action, but one will still
have acted defectively. 

Moreover, it is unlikely to be true that moral wellbeing always trumps other forms of wellbeing. Suppose the best science shows that on average
there is on the whole a moral improvement---perhaps in the area of compassion---from suffering severe headaches, but this improvement is tiny.
A parent who knew this should still relieve a child's severe headache, and wouldn't be acting contrary to benevolence in relieving it. 

Nonetheless, it is plausible that moral wellbeing is typically the most important aspect of our wellbeing and that typically other forms
of our wellbeing are appropriately sacrificed to it. This gradation is itself encoded in the human form which specifies what is good for
us and the ordering between the goods.

Traditionally, Aristotelian action theory has insisted that we always act for our happiness. This happiness thesis is compatible with a metaethics on
which right action is the proper functioning of the will, but is neither entailed by it nor particularly plausible. While proper functioning is always
good for a substance, a substance when functioning properly in some way need not be doing so \textit{in order to} function properly in that respect.
When a flower opens up in the right season, its opening up plausibly has as its end the good of reproduction rather than the good of opening up. 
Similarly, when you make dinner for your child, your right action is good for you, but you are doing it for the sake of your child and not for 
the sake of the action itself. 

\section{Supererogation}
??cut?

\chaptertail