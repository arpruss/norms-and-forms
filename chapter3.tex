\def\mychapter{III}
\chapter{Natural Law's advantages for metaethics and ethics}\label{ch:meta}
\section{Metaethics}
\subsection{Some desiderata}
Metaethics is an account of why the most fundamental ethical truths are true. If we were to make a wish-list
for metaethics, it would arguably include the following desiderata for what should follow from the theory:
\ditem{iii-obj}{Ethical truths are objective}
\ditem{iii-know}{Ethical truths are knowable}
\ditem{iii-compel}{The explanation of fundamental ethical truths makes them morally compelling to us}
\ditem{iii-plausibility}{The normative implications are plausible.}

Recall our $(\pi,m,n)$-metaethics on which what made ethical claims true is that they were encoded at some
specific position in the digits of $\pi$. This gave us objectivity and knowability (at least given
the specific position and encoding system), and perhaps plausibility of normative implications, but not
compellingness.

An individual relativism
that says that the right is what agrees with one's belief as to what is right,  on the
other hand, gives us knowability, and insofar as we find morally compelling the idea that we should obey
our conscience it gives us some compellingness, but it lacks objectivity. Furthermore, its normative implications as to
what I ought to do are very plausible to me, since obviously I find my own moral views plausible, but the theory's implications
for what people with odious moral beleifs should do---namely, those actions they believe are right---are highly implausible.

Utilitarianism, on the other hand, considered as a  metaethical theory about the nature of the right, yields objectivity, knowability
and moral compellingness (the idea that what we should do is maximize the good is among the \textit{prima facie} most plausible of moral ideas), but
it famously has  implausible normative consequences.

A Natural Law metaethics on which for an action to be right is for it to be a proper exercise of the will according to our nature
yields a mildly circumscribed objectivity: it makes the right be relative to our kind. But as we saw in Chapter~\ref{ch:ethics}, this degree
of relativity is highly plausible: it is plausible that ethical requirements do vary between different kinds of intelligent
beings.

Natural Law metaethics yields knowability when we accept the optimistic Aristotelian harmony theses that things generally function
correctly and that the various norms for a thing tend not to conflict. For instance, given such a thesis, the norms for our
emotions---including emotions such as moral repugnance or moral admiration or the feeling of obligation---are likely to cohere
with our norms for our actions, and by and large our emotions and actions are apt to be correct. This enables us to evaluate
normative ethical theories according to the constraint of whether their requirements fit sufficiently with our emotions and require
actions that are not too distant from those typically performed by people whose lives appear to
be harmoniously flourishing.  We thus have a rational equilibrium epistemology for our ethics.

The basic in Natural Law metaethics idea here is that we ought exercise our will correctly. This is so compelling that it smacks of triviality. Nonetheless,
the claim is not trivial, since it provides an analysis of the moral ought in terms of the functional correctness of our wills.
We find compelling the idea that we should be true to ourselves. But to be true to ourselves is not just being true to 
our changing beliefs and values, but to be true to that which makes us be human: our human nature.

The metaethics of right action as the proper functioning of the will is \textit{prima facie} compatible with quite a broad variety of normative theories.
We can imagine a being whose will's proper function is to will maximal total utility\footnote{Thus, Aristotelian metaethics is compatible with
utilitarian normative ethics, but not with metaethical utilitarianism on which the right is \textit{defined} as what maximizes utility.}, or to will in accordance with God's commands, or
to will what is universalizable, or to will one's flourishing, or even to cause maximal harm to self. Some of these views will, however, be less
plausible given other Aristotelian commitments, such as harmony theses. The harmony theses make it unlikely, for instance, that the right thing
be maximal self-harm. Indeed, Aristotelian optimism pushes us to an ethical theory whose normative consequences are, 
by and large, fairly intuitive.
At the same time, there is a real possibility of error, and of correction of that error.

Natural Law metaethics does justice to the idea that the source of our obligations is in us, rather than in some external fact---such as
a divine command---whose moral relevance is questionable. 
The human form or nature is a constituent part of us, and thus we compellingly have the source of morality in ourselves.
In a sense, then, we are our own moral legislators. But because our nature is metaphysically not
up to us, we do not have a choice as to what we legislate and we can be wrong about what we have in fact legislated. Natural Law
metaethics will thus accept with modifications both the relativist's and the Kantian's insistence on autonomy, but without the
ultra-conservative consequences of relativism on which we are always guaranteed to be right and hence never have reason
to change our views, and while avoiding the merely formal character of Kantianism which makes it unlikely to yield sufficient
normative consequences to guide our lives.??ref 

Other metaethical theories may satisfy the four desiderata as well to varying degrees, though if I am right, Natural 
Law does it overall best.

\subsection{Internality of morality}
An attractive feature of a metaethics is thus that it grounds moral truths in features of the moral subjects themselves. It seems 
compelling to ask ``So what?'' about an external ground of moral truth, such as divine or social commands, or Platonic 
moral facts. 

Socrates argued that that virtue is the center of our flourishing on the plausible grounds that moral virtue is the health 
of the soul.??ref What makes this argument an appealing invitation to moral virtue is that health is an internal good of 
the person. Socrates' analogy opens us to seeing moral virtue as a good grounded internally to the person. But since virtue
is a disposition to living by the fundamental moral truths, making these truths be internally grounded makes virtue
more fully internally internally grounded.\footnote{Some non-fundamental moral truths, such as that I ought not sell
the laptop that I am writing this on, will not be fully internally grounded,
since they depend on circumstances and facts outside of us, such as that this laptop belongs to my employer.}

Additionally, an internalist intuition is probably a part of the explanation of the 
attractiveness of subjectivist metaethics: it is harder to say ``So what?''\ to what is at the center of 
one's self. 
Aristotelian ethics grounds moral truths in the agent's form, which is a central metaphysical constituent of the agent,
indeed literally the very soul on classical Aristotelianism. 
It is not exactly the same kind of ground as a subjectivist offers, but it is still an intimately \textit{internal} ground.

In fact, in an important way the agent's form provides a \textit{more} internal ground of moral facts than the 
agent's own beliefs. Typically, we do not choose our beliefs, but catch them, much as we 
catch flus and colds from those around us, as Richard Feldman is said to have quipped. It is largely an accident of our environment that they are what they are, especially 
if there is no metaphysical truth for them to reflect. One of the motivations for subjectivist theories is 
the intuition that morality is internally grounded. But when the internal ground is something that is itself
produced by the accidents of surrounding culture, as is clearly the case for some of our moral beliefs, 
then we have a betrayal of the internalist intuition. On the other hand, on an Aristotelian theory, the ground 
of moral truth is our form, which is a central essential metaphysical constituent of the human individual. 
Note that a Kantian has a similar advantage: the ground of morality is our rationality, and our rationality
is central and, very plausibly, essential to us.

Furthermore, a plausible subjectivism needs to take into account potential disagreements within the agent's 
beliefs and values. A reasonable solution is to insist that more deep-seated aspects of one's psyche take 
precedence over more accidental features. If you believe that people should get the benefit of the doubt, but 
in the heat of the moment you believe that the person you are arguing with should not get the benefit, the 
subjectivist should probably opt for defining your obligation in terms of the general belief, as it is more 
likely to be deeply planted in you.\footnote{An alternative is a Frankfurt-type??ref idea that higher-order values and
ideas should take precedence over lower-order ones. Thus, if Bob believes that Alice should not get the benefit
of the doubt, but also believes that he should believe that she should, then it is the second-order belief that
trumps. This does not settle all the questions of disagreement between first-order beliefs---there need not be a 
second-order belief that decides between them. Moreover, although it is psychologically uncommon for a higher-order
belief or value to be a mere whim, there is nothing to rule that possibility out. Suppose that I think to myself:
``To illustrate the point I am making in this footnote, it would be good if I had the really wacky belief that
it's required to insult blue-eyed friends on Friday the 13th of January in a prime-numbered year.'' Nobody
should think that such a whimsical second-order value or belief yields a first-order obligation.} We might say 
that our nature as human beings, understood in the optimistic Aristotelian way as both instituting norms and impelling us 
to believe and follow them, has a kind of depth that trumps the deepest of our contingent beliefs.

That Natural Law makes the source of morality be literally a part of each of human being also makes it
contemporary non-naturalist Platonist-style theories??ref, and divine command theory, on which the source 
of morality is found in  an objective reality outside of us---in heaven (Platonic or theistic)---that we 
merely participate in rather than have as a part of us.\footnote{Some Platonists, especially bundle-theorists, 
hold that properties are Platonic abstract objects that are nonetheless a part of the entities that exemplify
them.??refs Such Platonists can also say that the source of morality is truly a part of us. As Ockham argued??ref,
it is somewhat implausible to think that all human beings literally overlap, having these Platonic parts in common, 
but philosophers are used to biting such bullets. In ??forward, we will discuss the Aristotelian version of this theory
on which the human form is common to all humans rather than an individual constituent of each. Here it is perhaps
worth noting that an \textit{individual and unshared} form perhaps does a little bit more justice to the 
self-legislation intuition.}

%%proofed
\subsection{What are moral or rational norms?}
The species-relativity of norms makes it plausible that in the space of possibilities---and perhaps in 
extraterrestrial regions of actuality as well---there will be species of beings that are intelligent enough to have advanced science and technology but whose natural
behavior will be quite different from us. This raises a difficult question as to what makes a particular set of
natural norms count as a set of moral or even rational norms, and hence makes a species that possesses them a species of moral 
or rational agents.

Not all natural norms are moral or rational norms. The natural norm behind a properly functioning horse shedding in the spring
is neither moral nor rational. Perhaps a necessary condition for a moral norm is that it govern voluntary behavior or a disposition to voluntary behavior.
But the question of what behavior counts as voluntary is difficult. It is tempting to say that the behavior of an entity is
voluntary if it is subject to reasons. But reasons live in a space made possible by rational norms, and so it seems we need
an account of rational norms, at least, to make sense of what behavior is voluntary.

Here is one speculative Aristotelian functionalist way to answer the questions in three steps. First, we connect reasons and norms with goods
considered as such: an (internal) reason 
for a behavior is an apprehension of the behavior as \textit{good}. A mouse may apprehend cheese as yummy to eat, or maybe even as nutritious, 
but not as \textit{good}. Being yummy or being nutritious is a way of being good to eat, but to apprehend as yummy or nutritious is not to apprehend either the cheese or its consumption as good.

Second, we say that a necessary condition for a behavior to be voluntary is that it 
comes from such a good-regarding reason. This, of course, raises the infamous problem of in-the-right way. A behavior can be caused by a reason
without being voluntary. The famous case is the belayer who intends to murder a climber by dropping the rope, and then
his hands start shaking at what he has intended to do, which results in an involuntary dropping.??ref Whatever reason 
the belayer had
for the murder is the cause of the dropping, but the dropping is involuntary. Aristotelian metaphysics does, however, seem
to have a tool for solving this problem. Causation can be seen to be teleological in nature, and we might say that it is 
a primitive fact that sometimes the effect \textit{is} a fulfilment the teleology of the cause, in which case we can say that the effect 
is caused in the right way. A voluntary behavior is then one which is a fulfillment of the teleology of the cause,
where the cause is a good-regarding reason.

Third, the will is then functionally defined as the system by which reasons lead in the right way to behavior that promotes 
the goods apprehended by these good-regarding reasons. A rational norm is a norm of the will's behavior
favoring some or all reasons. We can then define a moral norm as identical to a rational norm, if we like.
Or, if we do not want to say that prudential reasons are all moral, we might say that a moral norm is a 
norm of behavior of the will favoring some or all reasons that themselves are focused on  a good not considered 
primarily as a good to self. 

This is not, of course, the only way to define which norms are moral or rational. And it is quite possible that the question of which
norms are moral or rational is largely a verbal question. Go back to the characterization of reasons in terms of goods. The mouse takes
the cheese to be yummy, and that is not taking the cheese to be good. But humans often represent good things in thicker ways: as
beautiful, courageous, or even divine. Couldn't we imagine a continuum of animals where at one end the cheese is represented as 
yummy and on the other as having gustatory beauty? Somewhere in the continuum we have moved to representing the cheese as having a 
thick form of goodness. But where this happens is unclear. 

We have a certain set of norms for the will. We can call these rational, and a subset of them moral. What hangs on whether a different
set of norms governing the behavior of a different class of beings counts as rational or moral? Couldn't this be like the merely verbal question of
which animals' hard cephalic projections count as ``horns''? 

But perhaps the question of which things are moral agents matters for first-order moral questions about interspecies relations, such 
as which organisms we are permitted to eat, whose lives we should save, etc. However, it is not clear that the answers to these questions will neatly line up with
determinations of the boundaries of moral agency. Consider intelligent whales that are fine philosophers and scientists in their epistemic life, 
and that even enjoy contemplating the good, but do so purely non-practically, without the good being any motivator for their 
actions, all of which are instinctive. It would seem wrong to eat such beings, even if they turn out not to be moral agents.
On the other hand, imagine a shrimp that has the normal behavioral complement of a shrimp, with one exception: it represents
the ingestion of algae as good, and voluntarily pursues this good as such. And its intellectual abilities are the minimum needed
for an ordinary shrimp's life as combined with the most minimal possible concept of the good. It may well be wrong to eat such shrimp,
but it is difficult to say whether one should save the life of one of these minimally moral shrimp or the life of one of the intelligent but amoral 
whales (of course, we should not be biased here by size!)\ given a choice.

\subsection{Supervenience}
It is widely held that moral facts supervene on non-moral facts: if two possible worlds differ
with respect to moral facts, they must differ with respect to at least one non-moral fact. Similarly,
it is held that normative facts in general supervene on non-normative facts. The difficulty is then
to explain the supervenience relations.

The nature-first theorist has a complex relationship to both supervenience claims. Begin
with the supervenience of the normative on the non-normative, and first consider
general normative claims such as that every sheep should have four legs and every human should refrain
from torturing the innocent. Such normative claims are necessary truths, since their truth is a part of
what makes a sheep a sheep and a human and a human. Necessary truths vacuously supervene on any basis
we might choose, since there are no possible worlds that differ with respect to necessary truths.

But what about \textit{particular} normative claims, such as that Sally ought to have four legs
or that Trump ought to discharge the duties of the President of the United States? If we are
interested in the supervenience of the normative on the non-normative, we face a serious problem on
Aristotelian metaphysics: there are very few non-normative facts. It seems that every natural kind
is defined in part by normative properties. That Sally is a sheep is itself a normative thesis,
since a part of what it is to be a sheep is to be such that one ought to have four legs. That Sally
is an animal is also normative. And Sally is \textit{essentially} a sheep, and hence being a sheep
is central to Sally's identity in such a way that it may even be the case that the very claim that
Sally exists counts as a normative claim. Aristotelian metaphysics likely reaches even down to the fundamental
particles. Electrons not only do but should repel other electrons. 
A non-normative fact, it seems, cannot 
make reference to any natural kinds, and cannot even make reference to particulars that fall under natural
kinds. But on Aristotelianism, every particular substance, with the exception of God if there is a God, falls under a natural kind.
And facts about God are through-and-through normative, since God is essentially perfectly good. Thus,
on Aristotelianism, all facts about the existence of particular substances seem to be normative. 

Nor can \textit{any} existential claims escape normativity, since to be a substance or to be appropriately 
related to a substance (say, by being its accident).
If so, then every
existentially quantified claim is normative. The denial of a normative claim is normative as well, and
since a universally quantified claim is the denial of an existentially quantified claim (everything is
$F$ if and only if there does not exist an object that is not $F$), universally quantified claims 
will be normative as well. 

But if all facts about particulars and all quantified facts are normative, it seems that \textit{all} facts
are normative on Aristotelianism.  If this is right, then every pair of worlds agrees with respect to all 
non-normative facts, because there are no such facts. And if all facts are normative, then any two worlds 
that are the same in normative terms are altogether the same. Thus, the thesis of the supervenience of the 
normative on the non-normative can only hold given the modal fatalist thesis that there is only one possible 
world. And of course we have good reason to reject this thesis (\textit{pace} ??ref).

But while this goes against mainstream views of normativity, it is perhaps an advantage of Natural Law metaethics. For by eliminating non-normative facts, we no longer have any puzzling phenomenon of the relationship 
between the normative and the non-normative to be explained. And at the same time, this elimination of
non-normative facts is not a form of reductionism or eliminativism, since the ordinary facts that are
usually taken to be non-normative, such as that the sky is blue, are still there, but turn out to be richer,
having normative aspects to them.

Next, what about specifically moral normativity supervening on the non-moral? Moral facts are normative facts about the
will. Again, general moral facts, such as that humans should refrain from torturing the innocent or should
discharge the duties of the President of the United States if they have voluntarily sworn the
relevant oath of office,
are necessary truths and hence trivially supervene on whatever facts we want, including
non-moral or even non-normative ones. What about particular normative facts, such as
that Trump should discharge presidential duties? Since having a human nature is an essential feature of
Trump, and the human nature essentially has moral norms, it does not seem that even such facts can supervene
on facts about humans that aren't morally normative.

But now imagine a world $w_1$ that \textit{looks like} our world $w_0$. It has bipeds that look  just like us. The history of these bipeds
is empirically indistinguishable from our history. 
There is, for instance, a biped that is empirically indistinguishable from Napoleon and
one empirically indistinguishable from Marie Curie. Even apart from formal statements of supervenience, the moral supervenientist is apt to have the intuition that in $w_1$ there would have to be the same moral facts as in $w_0$. 
The Aristotelian, however, appears to be committed to holding that there is a world $w_1$ as above where the
bipeds' moral norms are different from ours, although their bodies and behavior looks just like ours. 
Perhaps denying a theoretical intuition, namely that such a $w_1$ is impossible,
is not problematic. But there is something more problematic in the vicinity: 
moral skepticism appears to follow for the Aristotelian.
For these beings will have the same collective moral beliefs and intuitions as we do, but presumably some of these 
beliefs and intuitions will be wrong. If so, then how can we be confident in our collective beliefs and 
intuitions?

First, however, we should note very likely a number of our collective moral beliefs and intuitions \textit{are}
wrong. Certainly this was true for many of our ancestors---take, for instance, the acceptance of
slavery across large swathes of the world in times past---and it would be very implausible to think the 
same is not true of us. It was suggested that the Aristotelian is committed to the claim that there could be a world
$w_1$ where \textit{some} of the moral norms of the bipeds differ from our norms. Assuming they have the same 
moral beliefs as we do, it follows that either some of their collective moral beliefs or some of our collective 
moral beliefs are false (or both!). But that does not yield moral skepticism, since we already had good reason to think
that some of our collective moral beliefs are false, and surely that is no more a justification for moral skepticism than
the occasional errors of our senses is justification for skepticism about the physical world, \textit{pace} Descartes.??refs

What if we strengthen the story by supposing that \textit{for the most part} the moral norms of the bipeds 
in $w_1$ fail to match their collective beliefs and intuitions? First, however, it is not completely clear that this
strengthened story is possible. If a perfectly good God necessarily exists, perhaps that goodness would logically preclude the 
creation of a world with a wholesale mismatch between moral norms and moral beliefs. 
But even if it is logically possible to have a world with such a large-scale mismatch, the mere logical 
possibility of a world with a skeptical scenario does not undercut our beliefs or even knowledge. 
It is logically possible---and even has a non-zero probability---that next time you pour cream into coffee
it will spell out, in tiny swirls, Hamlet's most famous speech, but surely we know that this won't happen---or at 
least justifiedly believe it won't.??refs To get skepticism, one would need the skeptical scenario to be fairly
probable, and supervenience-style arguments do not yield any probability.

% \subsection{Flourishing and overridingness}
% A substance flourishes to the extent that it functions in accordance with its norms.  Acting morally rightly is
% a case of functioning in accordance with the norms for the functioning of the will. Thus, acting rightly morally is an aspect of flourishing
% for those substances that have a will. At the same time, unless we have a substance that consists of nothing but will, there will be other
% aspects of flourishing. Because of this, conflict between moral rightness and self-interest is in principle possible.

% Aristotelian harmony tends to limit such conflict. Right action tends to promote other aspects of a substance's well-being. But nonetheless
% just as jogging sometimes promotes cardiac wellbeing at the expense of the wellbeing of knees, so too right action sometimes promotes volitional or moral wellbeing at the expense
% of life and other goods.

% Starting with Socrates, Western ethical reflection has often insisted on moral wellbeing being the most important aspect of a human's
% well-being. This may seem to be necessary for preserving the idea that one should do the right thing even when this costs one heavily,
% by allowing one to insist that the cost of doing wrong is always greater than the benefits. But the thesis that moral wellbeing always trumps
% other forms of wellbeing is neither necessary nor sufficient for preserving the obligation to act rightly no matter the cost.

% The thesis is not sufficient for preserving the primacy of moral obligation since even if moral wellbeing trumps other forms of wellbeing, there are imaginable situations where doing the right thing will on balance
% very likely harm one's wellbeing. For instance, suppose I am a bank employee of mediocre morals and the best empirical evidence available to
% me shows that taking an evening ethics class from Professor Kowalska  would be deeply inspiring and turn me into a vastly better person.
% Unfortunately, the only way I could afford the tuition is to embezzle a thousand dollars from a billionaire's account. This embezzlement is
% wrong, and will make me morally worse off, but I can reasonably expect to be a much better off morally from it in the 
% long run given Kowalska's education.

% Nor is the trumping thesis necessary for the overridingness of moral obligations. For if morally right action is 
% simply action in accordance with
% the norms for the will, then apart from any trumping thesis it is clear that morally wrong action is defective: it is defective because it
% fails to be an instance of the proper functioning of the will. Even if one is on the whole the better off if one does the wrong action, one will still have acted defectively.

% Moreover, it is unlikely to be true that moral wellbeing always trumps other forms of wellbeing. Suppose the best science shows that on average
% there is on the whole a moral improvement---perhaps in the area of compassion---from suffering severe headaches, but this improvement is tiny.
% A parent who knew this wouldn't be acting contrary to benevolence in relieving a child's severe headache, and indeed would likely be obligated to relieve the headache.

% Nonetheless, it is plausible that moral wellbeing is typically the most important aspect of our wellbeing and that typically other forms
% of our wellbeing are appropriately sacrificed to it. This gradation is itself encoded in the human form which specifies what is good for
% us and the ordering between the goods.

% Traditionally, Aristotelian action theory has insisted that we always act for our happiness. This happiness thesis is compatible with a metaethics on
% which right action is the proper functioning of the will, but is neither entailed by it nor particularly plausible
% on independent grounds. While proper functioning is always
% good for a substance, a substance when functioning properly in some way need not be doing so \textit{in order to} function properly in that respect.
% When a flower opens up in the right season, its opening up plausibly has as its end the good of reproduction rather than the good of opening up.
% Similarly, when you make dinner for your child, your right action is good for you, but you are doing it for the sake of your child and not for
% the sake of the action itself.

% Natural Law metaethics understood as grounding obligation in the non-defectiveness of an act of will thus has
% the advantage of significant flexibility with regard to the ends of the will. It does not commit us to implausibly
% trying to explain away the ``one thought too many'' consideration of one's own flourishing in selfless action, but
% if it turns out that further investigation shows that that thought really needs to be there, it can accommodate
% the idea that proper functioning of the will needs to be so directed. 

\subsection{Agent-centrism}
\subsubsection{The egoism objection}
According to Natural Law metaethics, an action is right provided that its performance constitutes the will's flourishing,
and is wrong provided that its performance constitutes the will's languishing. This seems objectionably egoistic. Paradigm
cases of moral wrongness involve harm to others, to the \textit{patient}, and are wrong because of that harm. The thought that the action makes the 
\textit{agent} languish is a thought too many.??refs Therefore, the argument goes, we should opt for an other-centered metaethics.

My response will be two-pronged. First, I will argue that the very features criticized in Natural Law provide a significant advantage
in a number of cases. Second, however, I will argue that the argument against Natural Law's agent-centric character only works
against some normative developments of the Natural Law metaethics, rather than against the metaethics. 

\subsubsection{The normative advantages of agent-centrism}
Other-centered theories nicely account for what is wrong with murder: it gravely harms the victim. They account slightly less well 
for what is wrong with typical cases of attempted murder: it is an attempt to harm to harm the victim. But they do not account for
atypical cases of attempted murder where the victim simply does not exist. Suppose Alice thinks that she has an identical twin 
living somewhere in Toronto, and sets out to kill her, to avoid the twin's claiming an inheritance. Alice has an extremely rare
genetic disorder which an identical twin would share, but which is very unlikely to be otherwise exhibited even in a city as large as
Toronto. She adds a poison to Toronto's water supply that targets only people with this genetic disorder, and then takes care to avoid
drinking Toronto's tap water. But in fact Alice never had a twin. 

There is thus no one that Alice is attempting to kill. Yet morally speaking, she is just as guilty as in an attempted murder case 
where her twin exists, and depending on one's views on moral luck maybe even as guilty as in the case where she succeeds in killing 
her twin. It is worth noting that in Anglo American jurisprudence, Alice might get away under the doctrine of impossible attempts,
on which an attempt has to have some feasibility, and trying to kill a non-existent person has none. (Of course, Alice is likely to be
convicted for pollution and for reckless endangerment of people with this disorder, but these miss out on her murderousness.) However, it is clear 
that notwithstanding the law of impossible attempts, it makes no difference to Alice's guilt whether in fact she has a twin or not.

We may, of course, try to save the doctrine that wrongdoing always wrongs another by trying to identify other victims, such as 
society or God. We can try to tweak the Alice case to exclude the society solution. Perhaps Alice is trying to kill her twin sister
in a world where she thinks they are the only survivors of a disaster, but in fact Alice is the only survivor and she's never had a
twin. That won't help with the God case, at least not within classical theism, since God is traditionally thought of as a necessary 
being??Refs. Furthermore, the intuition that I am responding to is that there is something particularly centered
on the ordinary direct human victim of a wrongdoing that contributes much of the wrong. And if God or society is what we count as 
the victim in the Alice case, then it seems that we have to say that in the Alice case there is less wrong than in the more ordinary
case of attempted murder where the victim actually exists. And yet it seems that how wrong Alice's action is does not depend on whether
she has a twin.

The above focused on patient-centric wrongs. We can also think about patient-centric duties. Again, it seems that our duties go 
beyond these. Plausibly, we have a duty not to deliberately produce a human being who is so genetically constituted as to be 
practically guaranteed to have a life of unremitting suffering. Now suppose that Bob and Carl both know that they and their 
spouses have genes such that if they reproduce, the child will have a life of unremitting suffering. In light of this knowledge,
Bob refrains from reproduction, while Carl's sadistic tendencies impel him to reproduce in light of this fact. Bob and Carl both
have a duty. In Carl's case, we can identify the individual to whom he has this duty, an individual whom he has wronged.
But in Bob's case, the analogous individual does not exist---precisely because Bob has fulfilled his duty. Again, we can try to
identify others to whom Bob owes not having a child---society, God and likely Bob's wife. But since there is one less individual
here than in Carl's case, on a patient-centric theory Carl has a somewhat more stringent duty, since the unfortunate child exists. However, it does not seem
that Carl has a more stringent duty than Bob.

Finally, moving away from cases, it is obvious that some degree of agent-centrism is needed for any plausible story about wrongs 
and duties. It is \textit{agents} that do wrongs and have duties. We have a duty not to eat humans. Lions, pigs and horses have no 
such  duty. Therefore, an account of what makes it wrong for me to eat other humans has to involve some facts about \textit{me}.
It cannot be wholly other-based. 

One might respond that on the Natural Law account, what makes it be wrong for me to eat other humans involves \textit{only} facts
about me, while it should also involve facts about the prospective victims. But how we understand this objection depends on how 
we understand the question of ``what makes it wrong for me to eat other humans''. 

First, we can understand it as a question about the grounds of 
the general moral rule that it is wrong for humans to eat other humans. On Natural Law the grounds of that general moral rule 
are entirely within the agent. However, that is how it should be. For the general moral rule would also hold even if there were
no other human beings in the world. And in fact the general moral rule would hold \textit{non-trivially} even if there were no
other humans, since even if there were in fact no other human beings, I should avoid actions that are likely to constitute the eating
of a fellow human being (e.g., shooting and eating an animal that has a significant epistemic probability of being human). 
An account of the wrongness of eating other human beings that requires other humans to exist is unsatisfactory.

Second, we can understand the question as asking about particular cases: What is it that makes it wrong for me to eat, say, Carl?
But then the Natural Law story is going to include a fact about Carl: I am the sort of thing that shouldn't eat other humans 
and Carl is another human. 

On neither reading do we have an argument against Natural Law metaethics.

\subsubsection{Avoiding agent-centrism in normative Natural Law ethics}
Let me start with a personal confession. For many years I objected to the eudaimonism I took to be at the heart of Natural
Law, a eudaimonism one might take to consist of the twin theses:
\ditem{eud-hamper}{What makes an action right is its promotion of the agent's flourishing.}
\ditem{eud-end}{An agent's right actions are aimed at the agent's flourishing.}
And these theses seemed objectionably egoistic.

However, while there are ways of pairing Natural Law metaethics with a normative ethics
that embraces \dref{eud-hamper} and \dref{eud-end}, both theses are dispensable for a Natural Lawyer. 

Indeed, \dref{eud-hamper} is so clearly wrong that it is unlikely that many Natural Law theorists accept it, given the 
well-known anti-consequentialism of the Natural Law community. Let us grant the Socratic thesis that moral flourishing 
is central to our flourishing. This thesis has historically been taken to support deontology. However, it is wrong
to rob a bank in order to pay for the tuition of an ethics class even if there is strong empirical evidence that this 
class will be so morally transformative that on the whole one's flourishing will be promoted, even if one takes into account the
temporary harm done to one's moral self by the robbery. Similarly, one may have a duty to continue working in a job that is just barely
moral and empirically likely to be destructive of one's flourishing as a moral agent in order to pay for one's
child's medical treatment. 

Now, on the metaethics that I am defending, what makes an action right is that it \textit{constitutes} the agent's 
flourishing with respect to the will. If we add to this the thesis that whatever constitutes the agent's flourishing
with respect to the will constitutes the agent's flourishing as a whole, then we get a version of \dref{eud-hamper}
with ``promotion'' replaced by ``constitution''. However we should not think that what constitutes the agent's
flourishing with respect to the will constitutes the agent's flourishing as a whole. If an agent is flourishing with
respect to the will, but is full of ignorance, in great pain, and lying in a bed of vomit??Vlastos-ref, the agent
is not flourishing on the whole. 

And the thesis that what makes an action right is its constitution of the agent as flourishing in respect of the will
seems to be simply a thesis about proper function, and does not imply any selfishness. Consider, for instance, that a 
bee's defending the hive at the expense of its life fulfills the bee with respect to whatever we might call the driver of 
the bee's activity, a driver analogous to our will. But this does not make the bee in any real way selfish: the bee,
presumably, does not think about itself as it does not have a self-concept. 
And certainly a guided missile is not selfish just because it fulfills its nature by exploding.

It is tempting to think that in the case of an agent who is driven by a rational will, if what makes the action 
right is its constituting the agent as flourishing (in one respect), then by willing the action under the
description ``right action'', the agent aims at flourishing, in a way in which neither the bee nor the guided missile's
actions are aimed at flourishing. If this line of thought is correct, then the characterization of rightness in terms of 
flourishing implies \dref{eud-end}, and that seems objectionably egoistic.

However, a rational agent's intentions are hyperintensional. It is possible to aim at heating up a room without aiming
at increasing the kinetic energy of the molecules in the room, even though what makes there be heat in a room is the
kinetic energy of molecules, and necessarily one is present if and only if the other is. Indeed, during the millenia 
before the relationship between heat and kinetic energy was known, no one aimed at increasing the kinetic energy of 
molecules while heating a room, and even now when the relationship is well-known, few people's intentions in turning 
a thermostat dial make reference to molecular motion. Similarly, even if the rightness of an action is grounded in, constituted 
by or even identical with the action's being an instance of the agent's flourishing as a willer, rightness can be 
aimed at without aiming at flourishing.

Furthermore, as has often been pointed out in the literature??(on moral fetishism), virtuous agents rarely aim at 
rightness as such. Instead, they aim at thicker right-making features of an action, such as its being an expression
of loyalty to a friend, its fulfilling a stranger's need, or its having been promised. It is because the action has
such thick features that its performance is an instance of volitional flourishing. 

There are, of course, times when a human agent aims at rightness as such. One set of cases is provided by agents who cannot
figure out on their own what is right and have to take the rightness of an action on the authority of another, without
understanding what makes the action right. Such agents include small children, but also sometimes ordinary well-functioning
adults who find themselves in such situations of such moral complexity that they turn to a professional ethicist or a 
trustworthy friend for advice. Furthermore, in cases where an agent is not sure which action is right but decides on
probabilistic grounds, it seems plausible that they are acting for the sake of rightness as such.\footnote{In fact, some
authors??refs-in-https://www.jstor.org/stable/44122234 have used this to argue against deciding on probabilistic grounds,
but it is  so plausible that in cases of uncertainty one should decide on some kind of probabilistic ground that it seems
better to use this kind of a case to argue that there is no objectionable moral fetishism here.}

Is this objectionably egoistic, assuming the rightness is constituted by the agent's volitional
flourishing? There are at least four reasons to doubt this. The first was already mentioned: the hyperintensionality in 
intentions.

To see the second reason, observe that we actually have a \textit{three layer} story:
\ditem{rightness}{the rightness of the action}
\ditem{flourishing}{the action's being such as to constitutive volitional flourishing}
\ditem{thick}{the thick features of the action because of which the action constitutes
volitional flourishing.}
The egoism objection in the case of agents aiming at right as such insists that the willing of \dref{rightness} inherits 
an egoistic character from the agent-centrism of \dref{flourishing}. But note that \dref{flourishing} is not the end of the story. Just as 
\dref{rightness} is grounded in 
\dref{flourishing}, so likewise \dref{flourishing} is grounded in \dref{thick}. And in paradigmatic cases, the thick features in \dref{thick} 
are other-centric features. If we think that willing the rightness of the action inherits egocentrism from the flourishing, we should even more 
think that it inherits  other-centrism from the thick features, since the thick features are a yet more ultimate ground of rightness
than the flourishing is.

Third, recall that we already noticed that an action can be right and constitute volitional flourishing but hamper one's flourishing as 
a whole, as in the case of working a soul-destroying job to pay family medical bills or refusing to rob a bank in order to pay for a 
morally transformative class. In such cases, it is absurd to say that by aiming at volitional flourishing one is being selfish, since the
action does not, in fact, contribute to one's good overall.

Indeed, a metaethics that grounds rightness in flourishing as a willer is compatible with one's never being required to intentionally pursue one's own good.
We can (perhaps with some difficulty) imagine an alien species whose members pair off in such a way that the proper functioning of
each one's will is just to will the good of the other member of the pair. Perhaps this is a species so physically constituted that
they are always more effective at benefiting others than at self. In such a species, the Natural Law metaethics would require 
utter unselfishness---and yet what would \textit{ground} the rightness of an action would still be that the action is proper to one's
will and hence constitutes the will as flourishing. 

We could imagine two versions of such aliens. They might be less reflective
than ourselves, and never act on higher-order reasons like rightness as such. Or they might be reflective, and might even come to
a Natural Law metaethics on which an action is made right by its constituting the agent as flourishing. In such a case, they might
aim at an action under the description ``right'', but only because they know that an action's rightness is ultimately grounded in their 
species in the action's benefiting the other member of the pair, though mediately in its constituting the agent's flourishing. 
In such a case, there is no more objectionable egoism than you exhibit in a case where an eccentric rich person says that they
will donate lots of money to charity if you do something that is good for you and so you eat a healthy and delicious salad. 

For our fourth and final response, consider somewhat odd cases of intentional activity. Suppose that
I want to test a device that detects nerve signals between between the brain and the arm, and so I wiggle my fingers. My action aims
at finger movement, but in a sense I don't actually care about finger movement. My end is to trigger the nerve signal detector.
The movement of fingers not only is not my end, but it is not even a means to triggering the nerve signal detector. To make this
point clear, we might suppose that the detector is triggered before the nerve signal reaches the muscles controlling the fingers,
so that the finger movement comes after the triggering of the detector. Or, for a more common case, consider the practice of
follow-through in racquet sports. Players are counseled to follow-through on their hits, i.e., to keep their racquet moving after 
it has made contact with the ball or shuttle. This movement makes no noticeable causal difference to the flight of the ball or 
shuttle, but aiming at a longer movement makes the racquet movement at the time of impact stronger. If one weren't aiming to continue 
the motion after the hit, one would likely begin slowing down the motion before the hit. In a game, the post-impact motion itself is not 
a useful end, nor need it be a means to anything one cares about in a game.

We might say in cases like this that one aims at ``unnecessary event'', the finger wiggling or the extra movement, in order to 
better calibrate one's action with regard to the ends that one cares about. We can call this ``calibrational'' aiming or
intention. This calibrational aiming has something in common with the case of a hunter who knows how their gunsight is misaligned,
and hence aims a meter to the right of the deer, though the difference is that in the finger wiggling and follow-through
cases one really does intend the fingers to move and the follow-through to occur, while in the crooked sight case the hunter does
not intend the bullet to go to the right of the deer. But in all three cases one merely uses the aim in order to calibrate the 
aspects of the action that one really cares about. 

It is compatible with the Natural Law story that the agent aims at rightness, and hence at the flourishing of the will, only
calibrationally. And merely calibrational aiming at one's good is in no way selfish---it does not imply any care, not even instrumental,
about getting one's good. This is something that our altruistic paired aliens could do.\footnote{??cf:Howard: https://jesp.org/index.php/jesp/article/view/1249/332}

That said, we are not such aliens. The goods that our will naturally aims at include our own goods. The fact that an action 
results in or constitutes our own flourishing counts in favor of the action about as much as the fact that it results in or constitutes the
flourishing of another individual. Thus in our own case we can have two sets of reasons for doing the right thing with regard to other
people. First, we have the thick reasons that render the action right, reasons that in typical cases are other-concerning. In cases where
we act on another's counsel, we may not be aware of these reasons, and we may be aiming calibrationally at rightness. But, second, acting
rightly is good for us. One would be failing to have the proper attitude to oneself if one didn't take the fact that the action is good
for one into account. Thus there is an egocentric reason available whenever we know we act rightly. However, this is as it should be.
That acting well is good for us is one of the great philosophical discoveries of all time, going back to Socrates, Plato and Aristotle,
and every moral theory should include this fact, whether the theory uses this fact in grounding the right, as on Natural Law, or whether
this fact is just an additional observation independent of the grounding of the right. 


\section{Supererogation}
It initially appears that one should perform an action best supported by moral reasons. But this principle
leads to a very demanding ethics with no room for supererogation. Consider a heroic sacrifice of one's life for
the sake of others---say, jumping on a grenade to save innocent lives. In typical cases, this is more praiseworthy 
than refusing to make the sacrifice. If it is more praiseworthy, it is surely better supported by moral reasons. 
But nonetheless, someone who refuses to make the sacrifice is typically not to be blamed. ??refs

It is very puzzling how to make sense of the permissibility of doing an action less well supported by moral reasons. 
But a Natural Law account has a way by making a distinction between flourishing and languishing that applies very
broadly. A cheetah capable of reaching 120~km/h flourishes less than a cheetah capable of reaching 125~km/h, but 
neither languishes, while a cheetah only capable of 12~km/h does languish in respect of running. Many areas of 
functioning have norms allowing both a binary distinction between failing to meet the norm and meeting the norm,
and a degreed distinction between levels of flourishing in respect of the norm. 

We can say that an individual's state or activity $S$ is supernormal in respect $R$ provided that the individual flourishes 
in $S$ with respect to $R$ more than they would in the case of some alternative state or action $S'$ that
still would not be an instance of languishing with respect to $R$. And now we can say that an action is supererogatory
provided that it is supernormal with respect to the exercise of the will. 

The mystery of supererogation was how one can count as permissibly acting if on balance one has moral reason to do something 
better. But once we see that this is just a special case of a phenomenon that extends far beyond morality, and indeed beyond
the life of rational beings, the mystery should be significantly decreased. And the general framework of the supernormal
matches our intuitions that there is such a thing as falling short of what is normal, and such a thing as exceeding it.
For more on this distinction, see ??forwardref.

\section{Equality and human dignity}
We have the intuition that all humans, or at least all mentally functioning adult humans, are in some important sense 
equal. It is difficult, however, to figure out what the relevant sense of equality is.

A utilitarian has a neat procedural account of equality. It is not that humans have equal value, but rather
that when we maximize total utility, this total utility is the sum of
individual utilities with \textit{equal} weights. But there are serious problems with
utilitarianism.??backref And indeed, the equal weight part of the view is itself implausible. Suppose that 
Alice has kidnapped Bob and taken him to a distant uninhabited planet. On her way back to earth, Alice 
had an accident and got marooned on a different uninhabited planet. Alice and Bob have enough supplies to
survive a year. It is certain that neither Alice nor Bob will ever come in contact with other people. 
Carl knows the above, and is himself marooned on a third planet, but has enough 
supplies for two lifetimes. He can send a drone with half of his supplies to Alice or to Bob, but not to
both. No one besides him and his recipient will know. It seems clear that Carl should send supplies to
Bob rather than to Alice the kidnapper. But on an equal weight utility view, this is false.

Furthermore, a standard utilitarian view is that \textit{all} loci of utility are equally weighted, including
non-rational animals. Humans only count for more because they are capable of higher utilities---they have more
sophisticated pleasures or higher order desires.??ref:Singer If equal weighting of sutilities
is how the utilitarian counts as doing justice to the intuition that all humans are equal, then the utilitarian
fails to do justice to the equally strong intuition that humans and non-rational animals are \textit{not} equal.

Perhaps a deontologist can do better, saying that people have equal \textit{rights}. Now, rights are correlative to 
duties, so presumably the
equality of rights is correlative to equality of duties of non-infringement. But while it is 
equally true of everyone that they have the same basic rights, and it is equally true that it is
wrong to infringe on these rights, this is not actually enough for equality of rights. For instance, 
if $E$ is your right not to be deprived of your eyes and $T$ is your right not to be deprived of your toes, 
it is equally true that you have $E$ and you have $T$, and it is equally true that it is wrong to 
infringe on your $E$ as that it is wrong to infringe on your $T$. But $E$ and $T$ are not equal rights, since
$E$ is more stringent than $T$. For instance, if one attacker is about to infringe your $E$ and another is about
to infringe your $T$, we would expect the police to prioritize your right to your eyes over your right to your
toes. Moreover, while it is equally true that both attacks are morally wrong, they are not equally morally wrong 
(similarily, it is equally true that elephants and whales are large compared to mice, but whales are larger, both
absolutely and as compared to mice): both assaults deserve significant penalty, but the assault on eyes deserves
a greater. Within a single individual there are many unequal rights but that are, nonetheless, equally truly possessed
by one. Thus equal truth of possession of rights is insufficient for equality of rights.

Do we in fact have equal rights? Well if we compare rights by stringency, as we did $E$ and $T$, this is not clear.
We have stronger reason to stop the murder of a complete innocent than to stop the murder of a someone who just falls
short of deserving the death penalty, and the murder of a complete innocent deserves a greater punishment---if not in law, then in public opinion (??cf.Mill). Similarly, it seems clear that we should make a greater effort to stop the murder of a person
with a decade of life ahead of them than the murder of someone who otherwise has five minutes of life remaining.

Or perhaps equality is a political matter. Each mature adult, perhaps, should have equal 
input in social decisions. But while this may be the best practical way of running things, it is not
clear that it has an imperative beyond the practical. Suppose that we had an indisputably correct way
of measuring wisdom and virtue. It does not seem clearly wrong for a society to weigh votes according to 
the wisdom and virtue of the voter. The main reason not to do so is the pragmatic one that any way of
measuring wisdom and virtue will be contentious, unreliable and open to abuse.  Besides this, it seems 
that our intuitions about equality go beyond the sphere of democratic politics, and extend to non-mature
humans as well.

If we look at the range of actual human abilities, we see significant incommensurabilities and inequalities. But on a robust theory of human nature, we can say that possessing the human form is not a function of possessing multiple valuable properties that some have to a 
greater and some to a lesser degree. Each human has the \textit{human form} to exactly the same degree. The 
possession of this form is what makes us human, and it is the valuable thing that grounds our dignity. 

But does not everyone, Aristotelian or not, agree that there is such a thing as being human, and that we are all 
equally human? Yes, of course. But on non-Aristotelian views, we differ from one another in regard to the properties
that make us human. 

Let's say that we define humanity in terms of having DNA sufficiently close to some standard $H$.
Then, first, surely we differ in how closely they hew to that standard. Second, since human DNA varies from
person to person, even if you and I are equally close to the
standard, we are close to it for different reasons: I may be close to $H$ because I am very close to $H$ in portion $A$ of
my DNA and somewhat close to $H$ in portion $B$, while you are close to $H$ because you are very close to $H$ in portion
$B$ while being somewhat close in portion $A$. Third, if we take humanity to be the valuable thing that we are equal in,
then it seems we are taking closeness to $H$ to be valuable. But closeness to $H$ seems valuable in virtue of the fact
that $H$ is a way of coding for various valuable properties, such as a variety of aspects of intelligence (insofar as these
are heritable) and a variety of valuable physical features. It is indeed valuable to have DNA that codes for these good
things, but nonetheless we differ in this valuable thing---for some of us have DNA that better codes for, say, musicality, and others have DNA that codes less well for musicality.\footnote{Cf.??ref:\url{https://www.nature.com/articles/s41598-022-18703-w#Sec24}} Thus our being human on a DNA standard is constituted by a variety of independently valuable genetic features,
and we differ in these values. This does not seem to be a good account of human equality.

Or suppose that we define humanity in the way biologists define species, as groups with significantly more genetic interchange
within the group than outside it. We are members of the human species, then, because of patterns of genetic interchange in our
ancestry. Again, this is something that comes in degrees: there are human populations that are more or less isolated from the 
bulk of humanity. And, second, it is far from clear that being a part of this genetically interchanging population as such
has the kind of immense value that our equality qua human beings needs to have in order to do justice to our intuitions about
human equality. If a small group of people became isolated for several centuries they would no less be our equals.

The Aristotelian account is perhaps the unique account that defines humanity in terms of possession of an \textit{exactly}
similar constituent---the human nature---rather than the possession of non-exactly similar properties. 

But how does having the human form ground a deep fundamentally equal value of human beings? We can say that we differ with regard 
to many valuable features we have, but a particularly valuable, perhaps the most valuable one, feature is our form---it is 
that which makes us human, which sets the norms that define our dignity, that gives us our high calling of pursuing intellectual and moral virtue, and 
that is the ground of the possibility of our flourishing. Items used in  worship---holy books, vestments, thuribles, 
altars, and sacred locations---are considered sacred in significant part due to their ends. Humans have high ends in
virtue of their form or nature, and there is thus a
deep value we all have in virtue of these ends. Granted, there is a further value when these ends are fulfilled. 
But there is an equality in the baseline---in the having of the ends---and we can plausibly say that it is this equal
possession of the same high ends that constitutes our dignity. 

We can thus distinguish our valuable features into two classes: dignity, which we have in virtue of the teleological structure
specified by our form, and what one might call accidental value, which consists in flourishing according to that teleological structure.
We are all equal in dignity but vary in accidental value. 

At the same time, we talk of ``dignity'' in the case of roles that one can accidentally have, such as the dignity of a monarch
(which can be infringed with \textit{l\`ese majest\'e}). Here we can follow Aristotle's idea that philosophically important terms have a 
focal sense and non-focal derivative senses, so that the focal sense of ``health'' is the proper function of an organism's body, 
while derivative senses apply to that which indicates focal health (e.g., healthy urine) or produces it (e.g., healthy food) or is in
some other relevant way related to focal health. Focal dignity is the value we have in virtue of having the role \textit{human being}.
Non-focal dignity is value we have in virtue of having some accidental role, like monarch or parent. And just as we have a relationship
between focal dignity and accidental value, where the accidental value is subordinate to focal dignity and fulfills the teleology 
in it, likewise subordinate to non-focal dignity (which itself is a kind of accidental value) there will be accidental values of
fulfilling the teleological structure in the non-focal dignity (e.g., being a just monarch or a loving parent). It may not be a 
coincidence that it is natural to talk of our humanity as imposing a high \textit{calling} on us, and we call many accidental roles
\textit{callings}.

\section{Outlandish paradoxes}
It is easy to generate paradoxes in ethics and decision theory by invoking outlandish situations.
Many, but not all, such situations involves infinities. I will give three representative examples.

First, we have the Satan's Apple paradox about infinite sequences of choices on which something
further depends:
\begin{quote}
Satan has cut a delicious apple into infinitely many pieces, labeled by the natural numbers. Eve
may take whichever pieces she chooses. If she takes merely finitely many of the pieces, then she
suffers no penalty. But if she takes infinitely many of the pieces, then she is expelled from the
Garden for her greed. Either way, she gets to eat whatever pieces she has taken. ??ref
\end{quote}
The puzzle is that for each piece, Eve has conclusive reason to take the piece, but if she acts on
all these reasons, something terrible happens. As presented, this is a paradox about self-interest, 
but we can turn it into an ethical one by supposing that the rewards and penalties of Eve's choices 
devolve on someone else, say Adam. In that case, we can say that Eve should accept each piece and yet
that's the worst option.

Another kind of paradox involves infinite numbers of beneficiaries. Imagine that there is an infinite
number of complete strangers, numbered with the integers (negative, zero and positive),
as well as two cats, all facing a deadly danger, and you have a choice between one of three equally convenient options:
\ditem{iii-saveA}{Save the strangers numbered $0,1,2,...$.}
\ditem{iii-saveB}{Save the strangers numbered $-1,-2,-3,...$ and one cat.}
\ditem{iii-saveC}{Save the strangers numbered $1,2,3,...$ and two cats.}

Now, you have no reason to prefer
the stranger numbered $0$ over the stranger numbered $-1$, the stranger numbered $1$ over the
stranger numbered $-2$, and so on. So as far as the saving of people, \dref{iii-saveA} and \dref{iii-saveB}
are a wash, but it's better to save a cat than not to, so \dref{iii-saveB} is morally preferable.\footnote{If the
reader thinks that cats do not fall in our moral purview, just replace the saving of a cat with saving a human
from some harm much less than death.}

But likewise there is no reason to prefer saving the people numbered with negative integers over the
people numbered with positive integers, so as far as the saving of people goes, \dref{iii-saveB} and
\dref{iii-saveC} are balanced. However, saving two cats is better than saving one, so \dref{iii-saveC}
is better than \dref{iii-saveB}. 

But now, \dref{iii-saveA} is clearly better than \dref{iii-saveC}: for in \dref{iii-saveA}, you save the stranger
numbered $0$ instead of the two cats, and wonderful as cats are, it is much better to save that one human over two cats. 

So we have a moral preferability circle, and whatever you do, there is something better you could have
done at no greater cost. It seems plausible that you have a duty to do better if you can do so at no greater
cost, and yet whatever you do, you violate that duty. And so it seems that you cannot act as you ought,
thereby violating the plausible maxim that ought implies can.

Third, consider some particularly extreme apparent counterexample to deontology. We have the intuition that
it is wrong to kill innocent people, but even deontologists find it difficult to maintain that intuition
in the face of cases where killing an innocent person would save a vast number of lives. In those cases
we are apt to be uncomfortable both with killing the innocent, and with letting the vast number of others
die lest we get our hands dirty. Here there appears to be a conflict between the principle 
that innocent blood is not to be shed and the reason for that principle, the sacredness of life. 

There have been various attempts to defuse such paradoxes,???refs and a defender of human nature as the
foundation of ethics can accept any of them. However, there is also a simple and highly intuitive 
alternative to these defusions. A horse's nature may ground facts about the appropriate gait when
browsing on grass and the appropriate gait when fleeing a predator through water. But equine nature
is simply silent on a horse's gait when fleeing aliens in a zero-gravity environment. Similarly,
our human nature could be silent on how we should act in outlandish situations, and our
principles just need not extend to such cases. This fits very well with the ordinary person's
disdain for philosophers like me who spend a lot of time thinking about such cases. 

There is a second, and similar, solution. It could be that our ordinary moral rules \textit{do} extend to outlandish
cases. Thus, the moral reasoning by which we generated the moral paradoxes in Satan's Apple and the 
infinite saving case may be correctly grounded in norms in our nature. It may well be that, say,
\dref{iii-saveB} is morally better than \dref{iii-saveA}, that \dref{iii-saveC} is morally better than \dref{iii-saveB}, 
that \dref{iii-saveA} is morally better than \dref{iii-saveC}, and that you ought to do the morally best (or one
of the morally best, if there is a tie) between the three options. It's just that these specifications of 
our nature are impossible to fulfill under these circumstances. In other words, it is very plausible to
say that ought implies can in situations that are a part of humans' natural environment, but there may be
logically possible outlandish situations that go far beyond this environment where ought no longer implies can. 
Insofar as our nature gives us norms fitted to our human environment, we should not be surprised if these norms
have counterintuitive implications, such as violating ought implies can, in situations far outside that environment.

We might think of these cases as ones where morality ``glitches out''. This kind of glitching could have a variety
of forms. We might have genuine moral dilemmas where our moral requirements outright contradict each other. Or more
mildly we might have cases where true moral requirements conflict with some of our intuitions, or with what one might think of as the reason for the moral requirements. For moral requirements often come with a reason in terms of a good that
is typically promoted by the requirement, even if all the details of the moral rule cannot be logically derived from that reason.
(Compare how in the American constitutional order, copyright law is justified by the value of progress 
of ``Science and useful Arts''??ref.) Thus a prohibition on killing the innocent promotes respect for the sacredness of life,
but there may be other ways of respecting that sacredness as well. But in an extreme case where the human race would die out
if we refuse to kill an innocent, a prohibition on killing is in tension with the reason for it. In these cases, we would
expect to find ourselves pulled in different ways, as we indeed are. And it is unsurprising if norms for the governance of
a particular species should work strangely or poorly in situations far from what one might think of as the natural environment 
of this species. Nor is this some sort of a defect in the norm, any more than it is a defect in a laptop that it does not
function in a blast furnace.

Similar solutions might well be available on at least two other ethical theories where the laws may be customized
to humanity: contractarianism and divine command theory. But, on the other hand, such solutions will be
implausible on theories such as utilitarianism and Kantianism that purport to apply to any kind of rational being at all. 

A similar point can be made about outlandish epistemological paradoxes. ??refs-and-examples  On the natural law epistemology that we will discuss in ??forwardref, we should not expect our nature to give us guidance, or at least satisfactory guidance, 
in situations too far out of the human environment. And while in ethics there are at least two common anthropocentric alternatives to natural law, contractarianism and divine command, in epistemology anthropocentric alternatives are harder to find.??Hawthorne?
Thus, we have perhaps an even stronger consideration in favor of a normativity based on human nature on the epistemological
side. 

More will be said in ??forward about outlandish scenarios.

\section{Conclusions}
Natural Law provides an attractive metaethics that allows one to ground objective ethics in something that is truly
a constituent of ourselves, while avoiding subjectivism or formalism. 
While one can develop Natural Law along eudaimonist lines, the central thesis that rightness of actions is
the proper functioning of the rational will can be detached from eudaimonism and avoids the objections to it.
Moreover, the constituent, the form, that grounds our morality is something that we all have in common, and 
hence is an appropriate and valuable ground for our equal human dignity. Finally, Natural Law predicts that
ethics is likely to have ``glitches'' in outlandish situations, since we do not expect the nature of a living
thing to specify how it ought to function in environments far from those it is meant for, and this prediction
is indeed borne out.

\chaptertail 